\documentclass[thesis.tex]{subfiles}

\begin{document}

\chapter{Krein Bubble}

Krein bubbles can occur when eigenvalues of opposite Krein signature collide on the imaginary axis. In our case, this can only occur if a purely imaginary interaction eigenvalue collides with an essential spectrum eigenvalue. The essential spectrum eigenvalues are located at approximately $c k \pi i / X$ for integer $k$. In Theorem \ref{locateeigtheorem}, we proved as long as 
\[
\left| c \frac{k \pi i}X - \sqrt{\frac{\mu}{M}} \right| \geq \frac{r^{1/2}}{X^{1/2}}
\]
for all nonzero eigenvalues $\mu$ of $A$, Krein bubbles cannot occur. Since the essential spectrum eigenvalues are spaced apart on the imaginary axis by approximately $c \pi i / X$, for this to be meaningful, we need to take $(r X)^{1/2} < c \pi/2$. In this section, we consider the case when this condition does not hold.

For simplicity, we will only consider the periodic 2-pulse. The matrix $A$ for the 2-pulse is
\[
A = \begin{pmatrix}
-a & a \\
a & -a
\end{pmatrix}
\]
where
\[
a = a_0 + a_1 = \langle \Psi(X_0), Q'(-X_0) \rangle + \langle \Psi(X_0), Q'(-X_0) \rangle
\]
Since Krein bubbles can only occur when $X_1 >> X_0$, in this case we have $a = \mathcal{O}(e^{-2\alpha X_m}) = \mathcal{O}(r)$. The eigenvalues of $A$ are $\{0, -2a\}$, thus by Theorem \ref{locateeigtheorem}, the interaction eigenvalues are located at approximately $\m \sqrt{-2a/M}$. These are purely imaginary if $2a/M > 0$. Since this is the case we will consider, let $b = \sqrt{2a/M} > 0$. Then the two interaction eigenvalues are located at approximately $\pm b i$.

Fix $k$ such that $\frac{c \pi k}{X} < \delta$, where $\delta$ is given in Theorem \ref{locateeigtheorem}. We are interested in what happens near $\lambda = c \frac{k \pi i}{X}$ when $c \frac{k \pi i}{X}$ and $bi$ are close. Specifically, we will take $\lambda$ and $bi$ to both be within $r^{1/2}/X^{1/2}$ of $c \frac{k \pi i}{X}$. Let
\begin{align}\label{lambdah1}
\lambda = c \frac{k \pi i}{X} + \frac{c}{X}h
\end{align}
and
\begin{align}\label{lambdas1}
s = \frac{c \pi k}{X} - b
\end{align}
where
\begin{align}\label{hsbounds}
\left| \frac{c}{X}h \right|, |s| \leq r^{1/2}/X^{1/2}
\end{align}
Since $(r X)^{1/2} < c \pi/2$, we also have the bound on $h$
\begin{align}\label{hbound1}
|h| \leq \frac{\sqrt{r X}}{\pi c} \leq \frac{\pi}{2}.
\end{align}

The first step is to compute the determinant of the block matrix in Theorem \ref{blockmatrixtheorem} for $\lambda$ given by \cref{lambdah2}.

\begin{lemma}\label{2blockmatrix}
\begin{proof}
First, we write the block matrix $B$ as $B = B_1 + B_2$, where
\[
B_1 = \begin{pmatrix}
\begin{pmatrix}
e^{-\nu(\lambda)X_1} & -e^{\nu(\lambda)X_0} \\
-e^{\nu(\lambda)X_1} & e^{-\nu(\lambda)X_0} 
\end{pmatrix} &
2 \lambda \begin{pmatrix}
-e^{-\nu(\lambda)X_1} k_1 + e^{\nu(\lambda)X_0} k_0 & e^{-\nu(\lambda)X_1} k_1 - e^{\nu(\lambda)X_0} k_0 \\ e^{-\nu(\lambda)X_0} k_0 - e^{\nu(\lambda)X_1} k_1 & -e^{-\nu(\lambda)X_0} k_0 + e^{\nu(\lambda)X_1} k_1
\end{pmatrix} \\
p_1 \lambda
\begin{pmatrix}
e^{-\nu(\lambda)X_1} & e^{\nu(\lambda)X_0} \\
e^{\nu(\lambda)X_1} & e^{-\nu(\lambda)X_0} 
\end{pmatrix} &
\begin{pmatrix}
-a - \lambda^2 M & a \\
a & -a - \lambda^2 M
\end{pmatrix}
\end{pmatrix}
\]
and $B_2$ is the block matrix 
\[
B_2 = \begin{pmatrix}
G_1 & G_2 \\ G_3 & G_4
\end{pmatrix}
\]

Using \cref{lambdah} and \cref{hsbounds} and $b = \mathcal{O}{r^{1/2}}$,
\begin{align*}
|\lambda| &\leq \left| c \frac{k \pi }{X} \right| + \left|\frac{c}{X}h \right| \\
&\leq |b| + |s| + C\left|\frac{r^{1/2}}{X^{1/2}} \right| \\
&\leq C r^{1/2} 
\end{align*}
Using this with Lemma \ref{nulambdalemma}, we have the estimate
\begin{align*}
\nu(\lambda) &= \frac{1}{c}\lambda + \mathcal{O}(|\lambda|^3) \\
&= \frac{k \pi i}{X} + c \frac{h}{X} + \mathcal{O}(r^{3/2})
\end{align*}
from which it follows that
\begin{align*}
|e^{\nu(\lambda)X_j}| &\leq 
\left| e^{i \frac{k \pi X_j}{X}}\right| e^{|c h|} e^{\mathcal{O}(r^{3/2}X)} \\
&\leq
e^{c \pi/2} e^{\mathcal{O}(r^{1/2} rX)} \\
&\leq C e^{r^{1/2} c^2 \pi^2 / 4} \\
&\leq C
\end{align*}
Using this, the remainder matrices $G_i$ have bounds
\begin{align*}
|G_1| &\leq C |\lambda| (|\lambda| + e^{-\alpha X_m} )\\
|G_2| &\leq C |\lambda| (|\lambda| + e^{-\alpha X_m} )^2 \\
|G_3| &\leq C (|\lambda| + e^{-\alpha X_m} )(|\lambda| + e^{-(\alpha-\eta) X_m} )\\
|G_4| &\leq C (|\lambda| + e^{-\alpha X_m} )^2(|\lambda| + e^{-(\alpha-\eta) X_m} ) \\
\end{align*}
In terms of the scaling parameter $r$, these are
\begin{align*}
|G_1| &\leq C r \\
|G_2| &\leq C r^{3/2} \\
|G_3| &\leq C r^{(1 + \tilde{\eta})/2} \\
|G_4| &\leq C r^{(2 + \tilde{\eta})/2}  \\
\end{align*}
where $1/2 < \tilde{\eta} < 1$. (We can deal with this later if it ends up mattering).

We use Mathematica to evaluate the determinant. To do this, we write the matrices $G_j$ as
\[
G_j = \begin{pmatrix}t^j_1 & t^j_2 \\ t^j_3 & t^j_4 \end{pmatrix}
\]
where the orders of the $t^j_k$ are given in the bounds on the $G_i$ above. The remainder bound is a bound on all the terms involving $t^j_k$. Doing all of this, we get
\begin{equation*}
\det B = \det B_1 + \mathcal{O}(r^{2 + \tilde{\eta}/2})
\end{equation*}
where from Mathematica we have
\begin{align*}
\det &B_1 = -2 \lambda^2 M \Big[ (2a + \lambda^2 M) \sinh(\nu(\lambda)X) \\
&- 2 \lambda^2 p_1 k_0 \left( \sinh(\nu(\lambda)(2 X_0 + X_1)) - 3 \sinh(\nu(\lambda)X_1)  \right) \\
&- 2 \lambda^2 p_1 k_1 \left( \sinh(\nu(\lambda)(2 X_1 + X_0)) - 3 \sinh(\nu(\lambda)X_0)  \right) \Big] 
\end{align*}
Expanding the second line, we have
\begin{align*}
\sinh&(\nu(\lambda)(2 X_0 + X_1)) - 3 \sinh(\nu(\lambda)X_1) \\
&= \sinh(\nu(\lambda)(X_0 + X_1))\cosh(\nu(\lambda)X_0) 
+ \cosh(\nu(\lambda)(X_0 + X_1))\sinh(\nu(\lambda)X_0) 
- 3 \sinh(\nu(\lambda)X_1) \\
&= \sinh(\nu(\lambda)X)\cosh(\nu(\lambda)X_0) 
+ \cosh(\nu(\lambda)X)\sinh(\nu(\lambda)X_0) 
- 3 \sinh(\nu(\lambda)X_1) \\
&= \sinh(\nu(\lambda)X)\cosh(\nu(\lambda)X_0) 
+ \cosh(\nu(\lambda)X)\sinh(\nu(\lambda)X_0) 
- 3 \sinh(\nu(\lambda)(X_1 + X_0 - X_0) \\
&= \sinh(\nu(\lambda)X)\cosh(\nu(\lambda)X_0) 
+ \cosh(\nu(\lambda)X)\sinh(\nu(\lambda)X_0) \\
&- 3 ( \sinh(\nu(\lambda)(X_1 + X_0))\cosh(\nu(\lambda)X_0) - \cosh(\nu(\lambda)(X_1 + X_0))\sinh(\nu(\lambda)X_0) ) \\
&= \sinh(\nu(\lambda)X)\cosh(\nu(\lambda)X_0) 
+ \cosh(\nu(\lambda)X)\sinh(\nu(\lambda)X_0) \\
&- 3 \sinh(\nu(\lambda)X)\cosh(\nu(\lambda)X_0) + 3 \cosh(\nu(\lambda)X)\sinh(\nu(\lambda)X_0)  \\
&= -2 \sinh(\nu(\lambda)X)\cosh(\nu(\lambda)X_0) 
+ 4 \cosh(\nu(\lambda)X)\sinh(\nu(\lambda)X_0) \\
\end{align*}
Similarly, expanding the third line,
\begin{align*}
\sinh&(\nu(\lambda)(2 X_1 + X_0)) - 3 \sinh(\nu(\lambda)X_0) \\
&= -2 \sinh(\nu(\lambda)X)\cosh(\nu(\lambda)X_1) 
+ 4 \cosh(\nu(\lambda)X)\sinh(\nu(\lambda)X_1) \\
\end{align*}
Substituting these into the expression for $\det B_1$, we have
\begin{align*}
\det &B_1 = -2 \lambda^2 M \Big[ (2a + \lambda^2 M 
+ 4 \lambda^2 p_1(k_0 \cosh(\nu(\lambda)X_0) + k_1 \cosh(\nu(\lambda)X_1)  ) ) \sinh(\nu(\lambda)X)  \\
&- 8 \lambda^2 p_1 \cosh(\nu(\lambda)X) \left( k_0 \sinh(\nu(\lambda)X_0) 
+ k_1 \sinh(\nu(\lambda)X_1) \right) \Big] 
\end{align*}

****



At this point, we neglect some higher order terms. Since $\nu(\lambda)$ is purely imaginary for the cases we care about, all the $\sinh$ and $\cosh$ terms are bounded by 1. Since 
\[
\lambda^2 p_1(k_0 \cosh(\nu(\lambda)X_0) + k_1 \cosh(\nu(\lambda)X_1)  ) = \mathcal{O}(e^{-\alpha X_0}|\lambda|^2)
\]
we can neglect that term. Since $k_1 << k_0$, we can also neglect the term $k_1 \sinh(\nu(\lambda)X_1) )$. This leaves us with 
\begin{align*}
\det &B \approx -2 \lambda^2 M \Big[ (2a + \lambda^2 M  ) \sinh(\nu(\lambda)X) - 8 \lambda^2 p_1 k_0 \cosh(\nu(\lambda)X) \sinh(\nu(\lambda)X_0) \Big]
\end{align*}

\end{proof}
\end{lemma}

Now we approximate the various terms. Krein bubbles occur (numerically) when interaction eigenvalues and essential spectrum eigenvalues collide. The essential spectrum eigenvalues occur at approximately 
\[
\lambda = c \frac{n \pi i}{X}
\]
for integer $n$. With this as motivation, we first take the scaling $\lambda = c \mu / X$. Then we have from Lemma \ref{nulambdalemma}
\begin{align*}
\nu(\lambda) &= \frac{1}{c} \lambda + \mathcal{O}(|\lambda|^3) 
\\
&= \frac{\mu}{X} + \mathcal{O}(\mu^3 / X^3)
\end{align*}
Next, we want $\mu$ to be a small perturbation of $n \pi i$, so take 
\[
\mu = n \pi i + h i
\]
where the $i$ in front of $h i$ is for convenience. Substituting this into the $\sinh$ and $\cosh$ terms, we get
\begin{align*}
\sinh(\nu(\lambda)X) &= \sinh(n \pi i + h i + \mathcal{O}(\mu^2 / X^3)) = (-1)^n h i  + \mathcal{O}(h^3 + \mu^2 / X^3) \\
\cosh(\nu(\lambda)X) &= \sinh(n \pi i + h i + \mathcal{O}(\mu^2 / X^3)) = (-1)^n +  \mathcal{O}(h^2 + \mu^4 / X^6)
\end{align*}
Substituting these into the expression for $B$ and neglecting higher order terms, we have
\begin{align*}
\det &B \approx -2 \lambda^2 M \Big[ (2a + \lambda^2 M  ) (-1)^n h i - 8 \lambda^2 p_1 k_0 (-1)^n \sinh(\nu(\lambda)X_0) \Big] \\
&= -2 (-1)^n \lambda^2 M \Big[ (2a + \lambda^2 M  ) h i - 8 \lambda^2 p_1 k_0 \sinh(\nu(\lambda)X_0) \Big] \\
&= -2 (-1)^n \lambda^2 M \Big[ (2a + \lambda^2 M  ) h i - 8 \lambda^2 r_0 \Big]
\end{align*}
where $r_0 = p_1 k_0 \sinh(\nu(\lambda)X_0)$. Since we want to solve $\det B = 0$ and the stuff out front is nonzero, we can ignore the stuff out front. Plugging in $\lambda = c \mu / X$, this become
\[
\left(2a + \frac{c^2 \mu^2}{X^2} M\right) h i + r_0 \frac{c^2 \mu^2}{X^2} = 0
\]
Multiplying by $X^2/(M c^2)$, this becomes
\[
\left(\frac{2a}{M c^2} X^2 + \mu^2 \right) h i + \frac{r_0}{M}\mu^2 = 0
\]
Recall that Krein bubbles can only occur when the interaction eigenvalues are on the imaginary axis. This happens when $\frac{2a}{M} > 0$. For convenience, take
\begin{align*}
b &= \sqrt{ \frac{2a}{M} } \\
\end{align*}
where $b$ is real. Substituting this, we have
\begin{align*}
\left(\mu^2 - (bi)^2 \frac{X^2}{c^2} \right) h i + \frac{r_0}{M} \mu^2 &= 0 \\
\left(\mu + i b\frac{X}{c}\right)\left(\mu - i b\frac{X}{c}\right) h i + \frac{r_0}{M} \mu^2 &= 0
\end{align*}
The Krein bubble only occurs when the interaction eigenvalues are close to the essential spectrum eigenvalues (all of which are purely imaginary). For this to occur, we need either $i b X/c$ or $-i b X/c$ to be close to $n \pi i$ on the imaginary axis. By symmetry, it does not matter which one we do, so let
\[
n \pi - b X/c = 2 k
\]
where $k$ is real. It follows that
\begin{align*}
\mu &= n \pi i + h i = 2 k i + h i + i b X/c \\
\mu - i b X/c &= 2 k i + h i\\
\mu + i b X/c &= 2 k i + h i + 2 i b X/c
\end{align*}
Substituting these in, we have
\[
\left(2 k i + h i + 2i b\frac{X}{c}\right)\left(2k i + i h\right) h i + \frac{r_0}{M} \left(2 i k + h i + i b \frac{X}{c}\right)^2 = 0
\]
Dividing by $i^2$, this becomes
\[
\left(2 k + h + 2 b\frac{X}{c}\right)\left(2k + h\right) h i + \frac{r_0}{M} \left(2 k + h + b \frac{X}{c}\right)^2 = 0
\]
We note that everything except for $h$ is real. Since $h$ and $k$ are small compared to $b X/c \approx n \pi$, this is approximately
\[
2 b\frac{X}{c}\left(2k + h\right) h i + \frac{r_0}{M} \left( b \frac{X}{c}\right)^2 = 0
\]
which simplifies to
\[
\left(2k + h\right) h i + \frac{r_0 b X }{M c} = 0
\]
Expanding the $\sinh$ term in $r_0$ in a Taylor series about $0$, we have
\begin{align*}
r_0 &= p_1 k_0 \sinh(\nu(\lambda)X_0) \\
&= p_1 k_0 \sinh\left( \frac{n \pi i + h i}{X} X_0 + h.o.t. \right) \\
&= p_1 k_0 \sinh\left( \frac{n \pi i X_0}{X} h.o.t. \right) \\
&= i p_1 k_0 X_0 \frac{n \pi}{X} + h.o.t.
\end{align*}
Substituting $n \pi = b X/c + 2 k$, this becomes
\begin{align*}
r_0 &= i p_1 k_0 X_0 \frac{b X + 2 k c}{X c} + h.o.t. \\
&= i \frac{p_1 k_0 b X_0}{c} + h.o.t.
\end{align*}
Substituting this in, we have
\[
\left(2k + h\right) h i + i \frac{p_1 k_0 b^2 X X_0 }{M c^2} = 0
\]
Dividing by $i$, this simplies to
\[
h^2 + 2kh + R = 0
\]
where 
\[
R = \frac{p_1 k_0 b^2 X X_0 }{M c^2}
\]
which is real. This has solution
\[
h = -k \pm \sqrt{k^2 - R}
\]
Then for $|k| \leq \sqrt{R}$,
\[
h = -k \pm i \sqrt{R - k^2}
\]
and so $h$ lies on a circle of radius $\sqrt{R}$ centered around the origin in the complex plane. Substituting back for all the things
\[
\sqrt{R} = \frac{b}{c} \sqrt{\frac{p_1 k_0 X X_0 }{M} }
\]
This is the radius in the perturbation $h$. Since $\lambda = (n \pi i + h i)/X$, to get the radius in $\lambda$, we divide by $X$ to get the Krein bubble radius
\[
\frac{\sqrt{R}}{X} = \frac{b}{c} \sqrt{\frac{p_1 k_0 X_0 }{M X} } = \mathcal{O}\left(\frac{ X_0^{1/2} e^{-(5/2) \alpha X_0} }{X^{1/2}} \right)
\] 
Numerics confirms the scaling in $X$ to high accuracy. There is no easy way to confirm the scaling in $X_0$ numerically, but questionably crude numerics confirms the scaling in $X_0$ to reasonable accuracy.

\end{document}
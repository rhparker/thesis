\documentclass[thesis.tex]{subfiles}

\begin{document}

\chapter{Krein Bubble}

\section{Rewriting the System}

First, we prove a simple lemma from finite dimensional linear algebra.

\begin{lemma}
Let $A$ be an $n \times n$ diagonalizable matrix with simple kernel spanned by $v$, and let $\ker A^* = \text{span}(w)$. Then 
\begin{enumerate}[(i)]
\item We can scale $v$ and/or $w$ so that $\langle v, w \rangle = 1$.
\item Assuming we have chosen the scaling in part (i), the projection on the kernel of $A$ is given by
\[
P_{\ker A} = \langle w, \cdot \rangle v
\]
\end{enumerate}
\begin{proof}
Let $0, \lambda_2, \dots, \lambda_n$ be the eigenvalues of $A$ with corresponding eigenvectors $v, v_2, \dots, v_n$. If $\langle v, w \rangle = 0$, then $v \perp \ker A^*$, thus by the Fredholm alternative we can solve $A u = v$. This is not possible since $A$ has simple kernel. Since $\langle v, w \rangle \neq 0$ we can scale $v$ and/or $w$ so that $\langle v, w \rangle = 1$.

For part (ii), let $P u = \langle w, u \rangle v$.
First, let $u \in \ker A$. Then $u = \alpha v$ for $\alpha \in \R$, thus 
\[
P u = \langle w, \alpha v \rangle v = \alpha \langle w, v \rangle v = \alpha v = u
\]
Since $A$ is diagonalizable, its eigenfunctions form a basis for $\R^n$. Let $u \in R^n$ and write $u$ in the eigenbasis as
\[
u = \alpha v + \sum_{k=2}^n \alpha_k v_k
\]
For $k = 2, \dots, n$, since $\lambda_k \neq 0$
\begin{align*}
\langle w, v_k \rangle &= \langle w, \frac{1}{\lambda_k}\lambda_k v_k \rangle \\
&= \langle w, \frac{1}{\lambda_k} A v_k \rangle \\
&= \langle w, A \frac{1}{\lambda_k} v_k \rangle \\
&= \langle A^* w, \frac{1}{\lambda_k} v_k \rangle \\
&= 0
\end{align*}
Thus we have
\[
P u = \alpha v
\]
\end{proof}
\end{lemma}

Next, we rewrite our system in such a way as to exploit the geometry of the center subspace. By assumption, we can write the equilibrium equation as
\[
\partial_x( \partial_x^{2m}u + f(u, \partial_x u, \dots, \partial_x^{2m-1}u)) = 0
\]
which we can integrate once to get
\[
\partial_x^{2m}u + f(u, \partial_x u, \dots, \partial_x^{2m-1}u) = k
\]
for a constant $k$. Let
\[
U = (u_1, \dots, u_{2m}, k ) = 
(u, \partial_x u, \dots, \partial_x^{2m-1} u, k)^T
\]
Then we can write this as the following first order system in $\R^{2m+1}$.
\[
\partial_x
\begin{pmatrix}u_1 \\ \vdots \\ u_{2m-1} \\ u_{2m} \\ k \end{pmatrix} = 
\begin{pmatrix}
u_2 \\ \vdots \\ u_{2m} \\ -f(u_1, \dots, u_{2m}) \\ 0 
\end{pmatrix} 
\]
Linearize this about the primary pulse solution $Q(x)$, noting that $k = 0$ for this solution since it decays to a baseline of 0. The linearization will involve the partial derivatives of $f$ with respect to the $u_k$. Separating out the linear terms in $f$, 
\[
\frac{\partial}{\partial u_k} = c_{k-1} + g_{k-1}(q_1, \dots, q_n)(x)
\]
where the $g_k(x)$ decay exponentially to 0 since $f$ is smooth and the primary pulse is exponentially localized. Furthermore, by SYMMETRIES AND PRIOR DISCUSSION, $c_0 = c$ and $c_k = 0$ for $k$ odd. Thus for the linearization we have the first order linear system $V'(x) = \tilde{A}(Q(x))V(x)$, where
\[
\tilde{A}(Q(x)) = 
\begin{pmatrix}
0 & 1 & 0 & \dots & 0 & 0\\
0 & 0 & 1 & \dots & 0 & 0\\
&& \vdots && \vdots \\
0 & 0 & 0 & \dots & 1 & 0\\
-c - g_0(x) & -g_1(x) & -c_2 - g_2(x) & \dots 
& -g_{2m-1}(x) & 1 \\
0 & 0 & 0 & \dots & 0 & 0 \\
\end{pmatrix}
\]
$\tilde{A}(Q(x))$ is exponentially asymptotic to $\tilde{A}(0)$, which is given by
\[
\tilde{A}(0) = 
\begin{pmatrix}
0 & 1 & 0 & \dots & 0 & 0\\
0 & 0 & 1 & \dots & 0 & 0\\
&& \vdots && \vdots \\
0 & 0 & 0 & \dots & 1 & 0\\
-c & 0 & -c_2 & \dots 
& 0 & 1 \\
0 & 0 & 0 & \dots & 0 & 0 \\
\end{pmatrix}
\]
Since the top $2n-1$ rows are linearly independent and the bottom row is all 0s, $\tilde{A}(0)$ has a one dimensional kernel spanned by
\[
\tilde{V}_0 = (1/c, 0, \dots, 0, 1)^T
\]
and $\tilde{A}^*(0)$ has a one dimensional kernel spanned by
\[
\tilde{W}_0 = (0, 0, \dots, 0, 1)^T
\]

For the eigenvalue problem, we have the additional equation $k_x = \lambda v$. Writing things this way, we have
$V'(x) = \tilde{A}(Q(x); \lambda)V(x)$, where
\[
\tilde{A}(Q(x); \lambda) = 
\begin{pmatrix}
0 & 1 & 0 & \dots & 0 & 0\\
0 & 0 & 1 & \dots & 0 & 0\\
&& \vdots && \vdots \\
0 & 0 & 0 & \dots & 1 & 0\\
-c - g_0(x) & -g_1(x) & -c_2 - g_2(x) & \dots 
& -g_{2m-1}(x) & 1 \\
\lambda & 0 & 0 & \dots & 0 & 0 \\
\end{pmatrix}
\]

Now we can see what this gets us. Let $Q'(x) = (\partial_x q(x), \dots, \partial_x^{2m-2}, 0)^T$. Then by the previous lemma,
\[
P^c_0 Q'(x) = P_{\tilde{V}_0}(Q'(x)) = \langle \tilde{W}_0, Q'(x) \rangle \tilde{V}_0 = 0
\]


\section{The Search for the Krein Bubble}

For the 2-pulse, the block matrix $B$ takes the form

\[
B \approx \begin{pmatrix}
\begin{pmatrix}
e^{-\nu(\lambda)X_1} & -e^{\nu(\lambda)X_0} \\
-e^{\nu(\lambda)X_1} & e^{-\nu(\lambda)X_0} 
\end{pmatrix} &
\begin{pmatrix}
d_1 & d_2 \\ d_3 & d_4
\end{pmatrix} \\
M_\Psi \lambda
\begin{pmatrix}
e^{-\nu(\lambda)X_1} & e^{\nu(\lambda)X_0} \\
e^{\nu(\lambda)X_1} & e^{-\nu(\lambda)X_0} 
\end{pmatrix} &
\begin{pmatrix}
-a - \lambda^2 M & a \\
a & -a - \lambda^2 M
\end{pmatrix}
\end{pmatrix}
\]
where $M$, $M_\Psi$ are constants and
\[
a = \langle \Psi(X_0), Q'(-X_0) \rangle
+ \langle \Psi(X_1), Q'(-X_1) \rangle \approx \mathcal{O}(e^{-2\alpha X_0})
\]
since $X_1 >> X_0$. For now, take the upper right block to be to leading order
\[
\begin{pmatrix}
-d & d \\ d & -d
\end{pmatrix} 
\] 
which is plausible by symmetry. Using Mathematica to take the determinant, we have
\begin{align*}
\det B &\approx -2 \lambda^2 M \left[ (2a + \lambda^2 M) \sinh(\nu(\lambda)X) + 2 d M_\Psi \lambda 
(\cosh(\nu(\lambda)X) - \cosh(\nu(\lambda)(X_1 - X_0))
\right] \\
&\approx -2 \lambda^2 M \left[ (2a + \lambda^2 M) \sinh(\nu(\lambda)X) + 2 d M_\Psi \lambda 
(\cosh(\nu(\lambda)X) - \cosh(\nu(\lambda)(X - 2 X_0))
\right]
\end{align*}
where $X = X_0 + X_1$. Now we approximate the various terms. Krein bubbles occur (numerically) when interaction eigenvalues and essential spectrum eigenvalues collide. The essential spectrum eigenvalues occur at approximately 
\[
\lambda = c \frac{n \pi i}{X}
\]
for integer $n$. With this as motivation, we first take the scaling $\lambda = c \mu / X$. Then we have (from what we have already shown)
\begin{align*}
\nu(\lambda) &= \frac{1}{c} \lambda + \mathcal{O}(|\lambda|^3) 
\\
&= \frac{\mu}{X} + \mathcal{O}(\mu^3 / X^3)
\end{align*}
Next, we want $\mu$ to be a small perturbation of $n \pi i$, so take 
\[
\mu = n \pi i + h
\]
Substituting this into various things, we get
\begin{align*}
\sinh(\nu(\lambda)X) &= \sinh(n \pi i + h + \mathcal{O}(\mu^2 / X^3)) = (-1)^n h + \mathcal{O}(h^3 + \mu^2 / X^3) \\
\cosh(\nu(\lambda)X) &= \sinh(n \pi i + h + \mathcal{O}(\mu^2 / X^3)) = (-1)^n +  \mathcal{O}(h^2 + \mu^4 / X^6)
\end{align*}
and
\begin{align*}
\cosh( \nu(\lambda)(X - 2 X_0))
&= \cosh( \nu(\lambda)X) \cosh(2 \nu(\lambda)X_0) 
- \sinh( \nu(\lambda)X) \sinh(2 \nu(\lambda)X_0) 
\end{align*}
which we are going to neglect for now since $\nu(\lambda)$ is small and $X_0$ is also relatively small. Substituting these in and neglecting higher order terms, we have
\begin{align*}
\det B &\approx -2 \frac{\mu^2}{X^2} M \left[ \left(2a + \frac{\mu^2}{X^2} M\right)(-1)^n h + (-1)^n 2 d M_\Psi \frac{\mu}{X} \right] \\
&= -2 (-1)^n \frac{\mu^2}{X^2} M \left[ \left(2a + \frac{\mu^2}{X^2} M\right) h + 2 d M_\Psi \frac{\mu}{X} \right] 
\end{align*}
We want to solve
\[
\left(2a + \frac{\mu^2}{X^2} M\right) h + 2 d M_\Psi \frac{\mu}{X} = 0
\]
Multiplying by $X^2/M$, this becomes
\[
\left(\frac{2aX^2}{M} + \mu^2 \right) h + \frac{2 d M_\Psi X \mu}{M} = 0
\]
For convenience, take
\[
b = \sqrt{-\frac{2a}{M}}X 
\]
which must be pure imaginary for this matter. Then this simplifies to
\[
(\mu + b)(\mu - b) h + \frac{2 d M_\Psi X \mu}{M} = 0
\]
The Krein bubble only occurs when the interaction eigenvalues are close to the essential spectrum eigenvalues. For this to occur, we need either $b$ or $-b$ to be close to $n \pi i$ small or $\mu + b$ small. By symmetry, it does not matter which one we take, so let
\[
n \pi i - b = 2 k
\]
from which it follows that
\begin{align*}
\mu &= n \pi i + h = 2 k + h + b \\
\mu - b &= 2 k + h \\
\mu + b &= 2 k + h + 2b
\end{align*}
Substituting this in, we have
\[
(2k + h + 2b)(2k + h) h + \frac{2 d M_\Psi X}{M}(k + h + b) = 0
\]
Since $h$ and $k$ are small compared to $b$, this is approximately
\[
2b (2 k + h) h + \frac{2 d M_\Psi X}{M}b = 0
\]
which simplies to
\[
h^2 + 2kh + R = 0
\]
where
\[
R = \frac{d M_\Psi X}{M}
\]
which has solution
\[
h = -k \pm \sqrt{k^2 - R}
\]
Then for $|k| \leq \sqrt{R}$, we get
\[
h = -k \pm i \sqrt{R - k^2}
\]
So $h$ lies on a circle of radius $\sqrt{R}$ centered around the origin in the complex plane. Undoing the scaling, the radius of the Krein bubble has order
\[
\frac{\sqrt{R}}{X} = \frac{\sqrt{d}}{\sqrt{X}}
\] 
Numerics confirms the scaling in $X$. Ideally, we would like numerics to suggest a scaling in $X_0$, likely some power of $e^{-\alpha X_0}$, but this has produced equivocal results so far.

\section{Evaluating some Inner Products}

These inner products show up in the jump expressions and (likely) are key terms in the block matrix. Recall that the $\Phi^{s/u/c}(x, y; \lambda)$ are the evolution operators of the conjugated system, and $\Theta_i^{s/u/c}(x, y; \lambda)$ are the evolution operators of the original system.

First, we look at a term from the jump in the decaying adjoint direction.
\begin{align*}
\langle &\Psi(0), P_i^-(0; \lambda) \Phi^s(0, -X_{i-1}; \lambda) P_i^-(-X_{i-1}; \lambda)^{-1} P_0^s(\lambda) Q'(X_{i-1})\rangle \\
&= \langle \Psi(0), \Theta_-^s(0, -X_{i-1}; \lambda) P_0^s(\lambda) Q'(X_{i-1})\rangle \\
&= \langle \Psi(0), \Theta_-^s(0, -X_{i-1}; 0) P_0^s(0) Q'(X_{i-1})\rangle + \mathcal{O}(e^{-2\alpha X_{i-1}}(|\lambda| + e^{-\alpha X_m})) \\
&= \langle \Psi(0), \Theta_-^s(0, -X_{i-1}; 0) P^s_+(X_{i-1}) Q'(X_{i-1})\rangle + \mathcal{O}(e^{-2\alpha X_{i-1}}(|\lambda| + e^{-\alpha X_m})) \\
&= \langle \Psi(0), \Theta(0, -X_{i-1}; 0) P^s_-(-X_{i-1}) P^s_+(X_{i-1}) Q'(X_{i-1})\rangle + \mathcal{O}(e^{-2\alpha X_{i-1}}(|\lambda| + e^{-\alpha X_m})) \\
&= \langle \Psi(-X_{i-1}), P^s_+(X_{i-1}) Q'(X_{i-1})\rangle + \mathcal{O}(e^{-2\alpha X_{i-1}}(|\lambda| + e^{-\alpha X_m})) \\
&= \langle \Psi(-X_{i-1}), Q'(X_{i-1})\rangle + \mathcal{O}(e^{-2\alpha X_{i-1}}(|\lambda| + e^{-\alpha X_m})) 
\end{align*}

Similarly, for the jump in the center adjoint direction, we should have
First, we look at a term from the jump in the decaying adjoint direction.
\begin{align*}
\langle &\Psi^c(0), P_i^-(0; \lambda) \Phi^s(0, -X_{i-1}; \lambda) P_i^-(-X_{i-1}; \lambda)^{-1} P_0^s(\lambda) Q'(X_{i-1})\rangle \\
&= \langle \Psi^c(-X_{i-1}), Q'(X_{i-1})\rangle + \mathcal{O}(e^{-\alpha X_{i-1}}(|\lambda| + e^{-\alpha X_m}))
\end{align*}

\end{document}
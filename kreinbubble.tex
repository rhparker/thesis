\documentclass[thesis.tex]{subfiles}

\begin{document}

\section{The Search for the Krein Bubble}

For the 2-pulse, the block matrix $B$ takes the form

\[
B \approx \begin{pmatrix}
\begin{pmatrix}
e^{-\nu(\lambda)X_1} & -e^{\nu(\lambda)X_0} \\
-e^{\nu(\lambda)X_1} & e^{-\nu(\lambda)X_0} 
\end{pmatrix} &
\begin{pmatrix}
d_1 & d_2 \\ d_3 & d_4
\end{pmatrix} \\
M_\Psi \lambda
\begin{pmatrix}
e^{-\nu(\lambda)X_1} & e^{\nu(\lambda)X_0} \\
e^{\nu(\lambda)X_1} & e^{-\nu(\lambda)X_0} 
\end{pmatrix} &
\begin{pmatrix}
-a - \lambda^2 M & a \\
a & -a - \lambda^2 M
\end{pmatrix}
\end{pmatrix}
\]
where $M$, $M_\Psi$ are constants and
\[
a = \langle \Psi(X_0), Q'(-X_0) \rangle
+ \langle \Psi(X_1), Q'(-X_1) \rangle \approx \mathcal{O}(e^{-2\alpha X_0})
\]
since $X_1 >> X_0$. For now, take the upper right block to be to leading order
\[
\begin{pmatrix}
-d & d \\ d & -d
\end{pmatrix} 
\] 
which is plausible by symmetry. Using Mathematica to take the determinant, we have
\begin{align*}
\det B &\approx -2 \lambda^2 M \left[ (2a + \lambda^2 M) \sinh(\nu(\lambda)X) + 2 d M_\Psi \lambda 
(\cosh(\nu(\lambda)X) - \cosh(\nu(\lambda)(X_1 - X_0))
\right] \\
&\approx -2 \lambda^2 M \left[ (2a + \lambda^2 M) \sinh(\nu(\lambda)X) + 2 d M_\Psi \lambda 
(\cosh(\nu(\lambda)X) - \cosh(\nu(\lambda)(X - 2 X_0))
\right]
\end{align*}
where $X = X_0 + X_1$. Now we approximate the various terms. Krein bubbles occur (numerically) when interaction eigenvalues and essential spectrum eigenvalues collide. The essential spectrum eigenvalues occur at approximately 
\[
\lambda = c \frac{n \pi i}{X}
\]
for integer $n$. With this as motivation, we first take the scaling $\lambda = c \mu / X$. Then we have (from what we have already shown)
\begin{align*}
\nu(\lambda) &= \frac{1}{c} \lambda + \mathcal{O}(|\lambda|^3) 
\\
&= \frac{\mu}{X} + \mathcal{O}(\mu^3 / X^3)
\end{align*}
Next, we want $\mu$ to be a small perturbation of $n \pi i$, so take 
\[
\mu = n \pi i + h
\]
Substituting this into various things, we get
\begin{align*}
\sinh(\nu(\lambda)X) &= \sinh(n \pi i + h + \mathcal{O}(\mu^2 / X^3)) = (-1)^n h + \mathcal{O}(h^3 + \mu^2 / X^3) \\
\cosh(\nu(\lambda)X) &= 1 + \mathcal{O}(h^2 + \mu^4 / X^6)
\end{align*}
and
\begin{align*}
\cosh( \nu(\lambda)(X - 2 X_0))
&= \cosh( \nu(\lambda)X) \cosh(2 \nu(\lambda)X_0) 
- \sinh( \nu(\lambda)X) \sinh(2 \nu(\lambda)X_0) 
\end{align*}
which we are going to neglect for now since $\nu(\lambda)$ is small and $X_0$ is also relatively small. Substituting these in and neglecting higher order terms, we have
\begin{align*}
\det B &\approx -2 \frac{\mu^2}{X^2} M \left[ \left(2a + \frac{\mu^2}{X^2} M\right)(-1)^n h - 2 d M_\Psi \frac{\mu}{X} \right] 
\end{align*}
We want to solve
\[
\left(2a + \frac{\mu^2}{X^2} M\right)(-1)^n h - 2 d M_\Psi \frac{\mu}{X} = 0
\]
Multiplying by $X^2/M$, this becomes
\[
\left(\frac{2aX^2}{M} + \mu^2 \right)(-1)^n h - \frac{2 d M_\Psi X \mu}{M} = 0
\]
For convenience, take
\[
b = \sqrt{-\frac{2a}{M}}X 
\]
which must be pure imaginary for this matter. Then this simplifies to
\[
(\mu + b)(\mu - b)(-1)^n h - \frac{2 d M_\Psi X \mu}{M} = 0
\]
The Krein bubble only occurs when the interaction eigenvalues are close to the essential spectrum eigenvalues. For this to occur, we need either $b$ or $-b$ to be close to $n \pi i$ small or $\mu + b$ small. By symmetry, it does not matter which one we take, so let
\[
n \pi i - b = 2 k
\]
from which it follows that
\begin{align*}
\mu &= n \pi i + h = 2 k + h + b \\
\mu - b &= 2 k + h \\
\mu + b &= 2 k + h + 2b
\end{align*}
Substituting this in, we have
\[
(2k + h + 2b)(2k + h)(-1)^n h - \frac{2 d M_\Psi X}{M}(k + h + b) = 0
\]
Since $h$ and $k$ are small compared to $b$, this is approximately
\[
2b (2 k + h)(-1)^n h - \frac{2 d M_\Psi X}{M}b = 0
\]
which simplies to
\[
h^2 + 2kh - (-1)^n R = 0
\]
where
\[
R = \frac{d M_\Psi X}{M}
\]
which has solution
\[
h = -k \pm \sqrt{k^2 + (-1)^n R}
\]
For now, assume that this looks like
\[
h = -k \pm \sqrt{k^2 - R}
\]
Then for $|k| \leq \sqrt{R}$, we get
\[
h = -k \pm i \sqrt{R - k^2}
\]
So $h$ lies on a circle of radius $\sqrt{R}$ centered around the origin in the complex plane. Undoing the scaling, the radius of the Krein bubble has order
\[
\frac{\sqrt{R}}{X} = \frac{\sqrt{d}}{\sqrt{X}}
\] 
Numerics confirms the scaling in $X$. Ideally, we would like numerics to suggest a scaling in $X_0$, likely some power of $e^{-\alpha X_0}$, but this has produced equivocal results so far.

\section{Evaluating some Inner Products}

These inner products show up in the jump expressions and (likely) are key terms in the block matrix. Recall that the $\Phi^{s/u/c}(x, y; \lambda)$ are the evolution operators of the conjugated system, and $\Theta_i^{s/u/c}(x, y; \lambda)$ are the evolution operators of the original system.

First, we look at a term from the jump in the decaying adjoint direction.

\begin{align*}
\langle &\Psi(0), P_i^-(0; \lambda) \Phi^s(0, -X_{i-1}; \lambda) P_i^-(-X_{i-1}; \lambda)^{-1} P_0^s(\lambda) Q'(X_{i-1})\rangle \\
&= \langle \Psi(0), \Theta_-^s(0, -X_{i-1}; \lambda) P_0^s(\lambda) Q'(X_{i-1})\rangle \\
&= \langle \Psi(0), \Theta_-^s(0, -X_{i-1}; 0) P_0^s(0) Q'(X_{i-1})\rangle + \mathcal{O}(e^{-2\alpha X_{i-1}}(|\lambda| + e^{-\alpha X_m})) \\
&= \langle \Psi(0), \Theta_-^s(0, -X_{i-1}; 0) P^s_+(X_{i-1}) Q'(X_{i-1})\rangle + \mathcal{O}(e^{-2\alpha X_{i-1}}(|\lambda| + e^{-\alpha X_m})) \\
&= \langle \Psi(0), \Theta(0, -X_{i-1}; 0) P^s_-(-X_{i-1}) P^s_+(X_{i-1}) Q'(X_{i-1})\rangle + \mathcal{O}(e^{-2\alpha X_{i-1}}(|\lambda| + e^{-\alpha X_m}))
\end{align*}


\end{document}
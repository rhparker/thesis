\documentclass[thesis.tex]{subfiles}

\begin{document}

\chapter{Double Pulse}

For simplicity, we will start with the periodic 2-pulse. Let $\delta > 0$ be as in Theorem \ref{blockmatrixtheorem}, so that we will only consider eigenvalues $\lambda$ with $|\lambda| < \delta$. For the periodic 2-pulse, the matrix $A$ is
\[
A = \begin{pmatrix}
-a & a \\
a & -a
\end{pmatrix}
\]
where
\[
a = a_0 + a_1 = \langle \Psi(X_0), Q'(-X_0) \rangle + \langle \Psi(X_0), Q'(-X_0) \rangle
\]
and $a = \mathcal{O}(e^{-2\alpha X_m}) = \mathcal{O}(r)$. The eigenvalues of $A$ are $\{0, -2a\}$, thus we expect the interaction eigenvalues to be located at approximately $\pm \sqrt{-2a/M}$. These will be real if $2a/M < 0$ and imaginary if $2a/M > 0$. 

We will only consider the case where $2a/M > 0$ here. Let $b = \sqrt{2a/M} > 0$. We expect that there will be a pair of interaction eigenvalues near $\pm b i$ and essential spectrum eigenvalues near the set 
\[
S = \left\{ \lambda = \pm c \frac{k \pi i}{X} : k \in Z \text{ and }|\lambda| < \delta  \right\} 
\]

Numerical analysis suggests the presence of instability bubbles when essential spectrum eigenvalues collide with the interaction eigenvalues. From the numerics, these bubbles have radius of order $r^{1/2}/X^{1/2}$. Thus as long as the points $\pm b i$ are further than $r^{1/2}/X^{1/2}$ from every point in $S$, we do not expect to find any instability bubbles. For our analysis, we will place balls of radius $r^{1/2}/X^{1/2}$ around each point in $S$. Since these balls scale like $X^{-1/2}$ and the distance between the essential spectrum eigenvalues scales like $X^{-1}$, in order for the balls to not overlap, we require the condition
\begin{equation}\label{rXcondition}
r^{1/2}X^{1/2} \leq \frac{c \pi}{2}
\end{equation}
Since $X = C_1 ( |\log r| + |\log b| + C_2$, we can always choose $r$ sufficiently small so that this is satisfied. If we fix $r$, there is an upper bound for $X$. Essentially, this will prevent us from taking the domain size to infinity.

With this in mind, for the remainder of the section, we will only consider $\lambda$ with $|\Re \lambda| \leq r^{1/2}X^{1/2}$. In particular, this implies that we have the following bound.

\begin{lemma}\label{lemma:expnubound}
For $|\lambda|$ with $|\Re \lambda| \leq r^{1/2}X^{1/2}$,
\begin{equation}\label{expnubound}
\left|e^{\pm \nu(\lambda)X_j}\right|, \left|e^{\pm \nu(\lambda)X}\right| \leq C
\end{equation}
\begin{proof}
From Lemma \ref{nulambdalemma},
\[
\nu(\lambda) = \frac{1}{c}\lambda + \mathcal{O}(\lambda^3)
\]
Thus using \eqref{rXcondition}, we have the bound
\begin{align*}
|\nu(\lambda)X| &\leq \frac{1}{c}|\lambda| X + C |\lambda|^3 X \\
&\leq \frac{1}{c} r^{1/2} X^{1/2} + C r^{1/2} (r^{1/2} X^{1/2})^2 \\
&\leq \frac{\pi}{2} + C r^{1/2} \frac{c^2 \pi^2}{4} \\
&\leq C
\end{align*}
The bound \cref{expnubound} follows from this. Since $X_j \leq X$, the same bound holds.
\end{proof}
\end{lemma}

The first step is to compute the determinant of the block matrix in Theorem \ref{blockmatrixtheorem}.

\begin{lemma}\label{2blockmatrix}
For $|\lambda| \leq r^{1/2}X^{1/2}$, the determinant of the block matrix $B$ is 
\begin{equation}
\begin{aligned}
\det &B = -2 \lambda^2 M \Big[ (2a + \lambda^2 M 
+ 4 \lambda^2 p_1(k_0 \cosh(\nu(\lambda)X_0) + k_1 \cosh(\nu(\lambda)X_1)  ) ) \sinh(\nu(\lambda)X)  \\
&- 8 \lambda^2 p_1 \cosh(\nu(\lambda)X) \left( k_0 \sinh(\nu(\lambda)X_0) 
+ k_1 \sinh(\nu(\lambda)X_1) \right) \Big]
+ \mathcal{O}( (r^{1/2} + |\lambda|)^5 )
\end{aligned}
\end{equation}
\begin{proof}
First, we write the block matrix $B$ as $B = B_1 + B_2$, where
\[
B_1 = \begin{pmatrix}
\begin{pmatrix}
e^{-\nu(\lambda)X_1} & -e^{\nu(\lambda)X_0} \\
-e^{\nu(\lambda)X_1} & e^{-\nu(\lambda)X_0} 
\end{pmatrix} &
2 \lambda \begin{pmatrix}
-e^{-\nu(\lambda)X_1} k_1 + e^{\nu(\lambda)X_0} k_0 & e^{-\nu(\lambda)X_1} k_1 - e^{\nu(\lambda)X_0} k_0 \\ e^{-\nu(\lambda)X_0} k_0 - e^{\nu(\lambda)X_1} k_1 & -e^{-\nu(\lambda)X_0} k_0 + e^{\nu(\lambda)X_1} k_1
\end{pmatrix} \\
p_1 \lambda
\begin{pmatrix}
e^{-\nu(\lambda)X_1} & e^{\nu(\lambda)X_0} \\
e^{\nu(\lambda)X_1} & e^{-\nu(\lambda)X_0} 
\end{pmatrix} &
\begin{pmatrix}
-a - \lambda^2 M & a \\
a & -a - \lambda^2 M
\end{pmatrix}
\end{pmatrix}
\]
and $B_2$ is the block matrix 
\[
B_2 = \begin{pmatrix}
G_1 & G_2 \\ G_3 & G_4
\end{pmatrix}
\]
By our choice of $\lambda$, it follows from \cref{lemma:expnubound} that all the terms of the form $e^{\pm \nu(\lambda)X_j}$ are bounded by a constant. Thus the remainder matrices $G_i$ have bounds
\begin{align*}
|G_1| &\leq C |\lambda| (|\lambda| + r^{1/2} )\\
|G_2| &\leq C |\lambda| (|\lambda| + r^{1/2} )^2 \\
|G_3| &\leq C (|\lambda| + r^{1/2} )^2 \\
|G_4| &\leq C (|\lambda| + r^{1/2} )^3 \\
\end{align*}
where $r$ is the scaling parameter. We use Mathematica to evaluate the determinant. To do this, we write the matrices $G_j$ as
\[
G_j = \begin{pmatrix}t^j_1 & t^j_2 \\ t^j_3 & t^j_4 \end{pmatrix}
\]
where the orders of the $t^j_k$ are given in the bounds on the $G_i$ above. The remainder bound is the sum of the bounds on all the terms involving $t^j_k$. Doing all of this, we get
\begin{equation*}
\det B = \det B_1 + \mathcal{O}( (r^{1/2} + |\lambda|)^5)
\end{equation*}

From Mathematica, we also have
\begin{align*}
\det &B_1 = -2 \lambda^2 M \Big[ (2a + \lambda^2 M) \sinh(\nu(\lambda)X) \\
&- 2 \lambda^2 p_1 k_0 \left( \sinh(\nu(\lambda)(2 X_0 + X_1)) - 3 \sinh(\nu(\lambda)X_1)  \right) \\
&- 2 \lambda^2 p_1 k_1 \left( \sinh(\nu(\lambda)(2 X_1 + X_0)) - 3 \sinh(\nu(\lambda)X_0)  \right) \Big] 
\end{align*}
Expanding the second line, we get
\begin{align*}
\sinh&(\nu(\lambda)(2 X_0 + X_1)) - 3 \sinh(\nu(\lambda)X_1) \\
&= \sinh(\nu(\lambda)(X_0 + X_1))\cosh(\nu(\lambda)X_0) 
+ \cosh(\nu(\lambda)(X_0 + X_1))\sinh(\nu(\lambda)X_0) 
- 3 \sinh(\nu(\lambda)X_1) \\
&= \sinh(\nu(\lambda)X)\cosh(\nu(\lambda)X_0) 
+ \cosh(\nu(\lambda)X)\sinh(\nu(\lambda)X_0) 
- 3 \sinh(\nu(\lambda)X_1) \\
&= \sinh(\nu(\lambda)X)\cosh(\nu(\lambda)X_0) 
+ \cosh(\nu(\lambda)X)\sinh(\nu(\lambda)X_0) 
- 3 \sinh(\nu(\lambda)(X_1 + X_0 - X_0) \\
&= \sinh(\nu(\lambda)X)\cosh(\nu(\lambda)X_0) 
+ \cosh(\nu(\lambda)X)\sinh(\nu(\lambda)X_0) \\
&- 3 ( \sinh(\nu(\lambda)(X_1 + X_0))\cosh(\nu(\lambda)X_0) - \cosh(\nu(\lambda)(X_1 + X_0))\sinh(\nu(\lambda)X_0) ) \\
&= \sinh(\nu(\lambda)X)\cosh(\nu(\lambda)X_0) 
+ \cosh(\nu(\lambda)X)\sinh(\nu(\lambda)X_0) \\
&- 3 \sinh(\nu(\lambda)X)\cosh(\nu(\lambda)X_0) + 3 \cosh(\nu(\lambda)X)\sinh(\nu(\lambda)X_0)  \\
&= -2 \sinh(\nu(\lambda)X)\cosh(\nu(\lambda)X_0) 
+ 4 \cosh(\nu(\lambda)X)\sinh(\nu(\lambda)X_0) \\
\end{align*}
Similarly, expanding the third line,
\begin{align*}
\sinh&(\nu(\lambda)(2 X_1 + X_0)) - 3 \sinh(\nu(\lambda)X_0) \\
&= -2 \sinh(\nu(\lambda)X)\cosh(\nu(\lambda)X_1) 
+ 4 \cosh(\nu(\lambda)X)\sinh(\nu(\lambda)X_1) \\
\end{align*}
Substituting these into the expression for $\det B_1$, we have
\begin{align*}
\det &B_1 = -2 \lambda^2 M \Big[ (2a + \lambda^2 M 
+ 4 \lambda^2 p_1(k_0 \cosh(\nu(\lambda)X_0) + k_1 \cosh(\nu(\lambda)X_1)  ) ) \sinh(\nu(\lambda)X)  \\
&- 8 \lambda^2 p_1 \cosh(\nu(\lambda)X) \left( k_0 \sinh(\nu(\lambda)X_0) 
+ k_1 \sinh(\nu(\lambda)X_1) \right) \Big] 
\end{align*}
\end{proof}
\end{lemma}

To find the eigenvalues, we will solve $\det B = 0$. First, we will look at what happens when $\pm bi$ is separated by at least $r^{1/2}/X^{1/2}$ from the set $S$. This should be a sufficient condition for no Krein bubbles. First, we locate the interaction eigenvalues.

\begin{lemma}
\begin{proof}
By \cref{lemma:expnubound} and our choice of $\lambda$, $\cosh(\nu(\lambda)X_j)$, $\sinh(\nu(\lambda)X_j)$, and $\cosh(\nu(\lambda)X)$ are bounded by a constant. Since $k_j = \mathcal{O}(r^{1/2})$ and
\[
2a + \lambda^2 M = M( \lambda - b i) (\lambda + b i),
\]
the equation $\det B = 0$ simplifies to 
\begin{equation}\label{simpleDetB1}
-2 \lambda^2 M^2 ( \lambda - b i) (\lambda + b i)\sinh(\nu(\lambda)X) + \mathcal{O}( (r^{1/2} + |\lambda|)^5 ) = 0
\end{equation}
Since we expect the interaction eigenvalues to be near $bi = \mathcal{O}(r^{1/2})$, we scale out a factor of $r^{1/2}$ from everything. Let
\begin{align*}
b &= r^{1/2} \tilde{b} \\
\lambda &= r^{1/2} \tilde{\lambda}
\end{align*}
Making these substitutions and dividing by $r^2$ and the constants out front, we have
\begin{equation}\label{simpleDetB2}
\tilde{\lambda}^2 ( \tilde{\lambda} - \tilde{b} i) (\tilde{\lambda} + \tilde{b} i)\sinh(\nu(r^{1/2}\tilde{\lambda})X) + \mathcal{O}(r^{1/2}) = 0
\end{equation}
Next, let $\tilde{s} = \tilde{\lambda} - \tilde{b} i$. Then this becomes
\begin{equation}\label{simpleDetB3}
\tilde{s} (\tilde{s} + \tilde{b} i) (\tilde{s} + 2 \tilde{b} i)\sinh(\nu(r^{1/2}(\tilde{b}i + \tilde{s}))X) = \mathcal{O}(r^{1/2})
\end{equation}
As long as the $\sinh$ term is not too small, we should be able to solve this for $s$ near $s = 0$. This is of the form $F(\tilde{s}) = R$, where $F(0) = 0$. Taking the derivative at $\tilde{s} = 0$, we are left with only one term
\[
F'(0) = -2 \tilde{b}^2 \sinh(\nu(bi)X)
\]
We will obtain a lower bound on $\sinh(\nu(bi)X)$, from which it follows that $F'(0) \neq 0$. For any integer $k$, let
\[
d_k = bi - c\frac{k \pi i}{X}
\]
By our initial setup, $|d_k| \geq r^{1/2}/X^{1/2}$. Then we have
\begin{align*} 
\nu(bi) &= \frac{1}{c}bi + \mathcal{O}(bi)^3 \\
&= \frac{1}{c}\left( c\frac{k \pi i}{X} + \left( bi - c\frac{k \pi i}{X}\right) \right) + \mathcal{O}(r^{3/2}) \\
&= \frac{k \pi i}{X} + \frac{d_k}{c} + \mathcal{O}(r^{3/2})
\end{align*}
Since $r^{1/2}X^{1/2} \leq C$, $r^{3/2}X \leq C r^{1/2}$, thus we have
\begin{align*} 
\sinh( \nu(bi)X) &= \sinh\left( k \pi i + \frac{d_k X}{c} + \mathcal{O}(r^{1/2}) \right)
\end{align*}
from which it follows that
\[
|\sinh( \nu(bi)X)| \geq C |d_k X| \geq C r^{1/2}X^{1/2}
\]
Thus we have
\[
|F'(0)| \geq C \tilde{b}^2 r^{1/2}X^{1/2} > 0
\]
Using the inverse function theorem, $F$ is invertible near 0, thus for $R$ sufficiently small, $\tilde{s} = F^{-1}(R)$. For the derivative at 0, we have bound
\[
|F^{-1}'(0)| = \frac{1}{F'(F^{-1}(0))} \\
= \frac{1}{F'(0)} = \mathcal{O}(r^{-1/2}X^{-1/2})
\]
Expanding $F^{-1}$ in a Taylor series about $r = 0$, we have
\[
\tilde{s} = \mathcal{O}(r^{-1/2}X^{-1/2}) \mathcal{O}(R) = \mathcal{O}\left( \frac{1}{X^{1/2}}\right)
\]
\end{proof}
\end{lemma}

We do the same thing for the essential spectrum eigenvalues.

\begin{lemma}
\begin{proof}
As in the previous lemma, we wish to solve
\begin{equation*}
\lambda^2 ( \lambda - b i) (\lambda + b i)\sinh(\nu(\lambda)X) + \mathcal{O}( (r^{1/2} + |\lambda|)^5 ) = 0
\end{equation*}
Here, we are interested in what happens near where $\nu(\lambda) = \frac{k \pi i}{X}$. This time, let
\[
\nu(\lambda) = \frac{k \pi i}{X} + \frac{h}{X}
\]
for $k$ sufficiently small (DETAILS LATER) and, at minimum $h < \pi/2$ SINCE EXPLAIN. Expanding the $\sinh$ term in a Taylor series about $n \pi i$,
\begin{align*}
\sinh(\nu(\lambda)X) &= \sinh(k \pi i + h) \\
&= (-1)^k h + \mathcal{O}(h^3)
\end{align*}
Thus our equation to solve is
\begin{equation*}
\lambda^2 ( \lambda - b i) (\lambda + b i)((-1)^k h + \mathcal{O}(h^3)) + \mathcal{O}( (r^{1/2} + |\lambda|)^5 ) = 0
\end{equation*}
Since $\nu(0) = 0$, expanding $\nu^{-1}(t)$ in a Taylor series about $t = 0$, we have
\[
\nu^{-1}(t) = c t + \mathcal{O}(t^3)
\]
thus we have
\begin{align*}
\lambda &= \lambda_0 + c \frac{h}{X} + \mathcal{O}\left(\lambda_0 + \frac{h}{X} \right)^3 \\
&= \lambda_0 + c \frac{h}{X} + \mathcal{O}(\lambda_0)^3
\end{align*}
where 
\[
\lambda_0 = c \frac{k \pi i}{X}
\]
Substituting all of this in and dividing by $(-1)^k$, we have
\begin{align*}
(h &+ \mathcal{O}(h^3))\left(\lambda_0 + c \frac{h}{X} + \mathcal{O}(\lambda_0)^3 \right)^2\left(\lambda_0 - bi + c \frac{h}{X} + \mathcal{O}(\lambda_0)^3 \right)\left(\lambda_0 + bi + c \frac{h}{X} + \mathcal{O}(\lambda_0)^3 \right)\\
 &+ \mathcal{O}( (r^{1/2} + |\lambda_0|)^5 ) = 0
\end{align*}


\end{proof}
\end{lemma}

Fix $k$ such that $\frac{c \pi k}{X} < \delta$, where $\delta$ is given in Theorem \ref{locateeigtheorem}. We are interested in what happens near $\lambda = c \frac{k \pi i}{X}$ when $c \frac{k \pi i}{X}$ and $bi$ are close. Specifically, we will take $\lambda$ and $bi$ to both be within $r^{1/2}/X^{1/2}$ of $c \frac{k \pi i}{X}$. Let
\begin{align}\label{lambdah1}
\lambda = c \frac{k \pi i}{X} + \frac{c}{X}h
\end{align}
and
\begin{align}\label{lambdas1}
s = \frac{c \pi k}{X} - b
\end{align}
where
\begin{align}\label{hsbounds}
\left| \frac{c}{X}h \right|, |s| \leq \frac{r^{1/2}}{X^{1/2}}
\end{align}
Since $(r X)^{1/2} < c \pi/2$, we also have the bound on $h$
\begin{align}\label{hbound1}
|h| \leq \frac{\sqrt{r X}}{\pi c} \leq \frac{\pi}{2}.
\end{align}


In the next lemma, we evaluate $\det B$ in terms of $h$.

\begin{lemma}
In terms of $h$,
\begin{align*}
\det &B = -2 \lambda^2 M (-1)^k \Big[ (2a + \lambda^2 M) h - 8 i \lambda^2 p_1 \left( k_0 \sin\left(\frac{k \pi X_0}{X}\right) + k_1 \sin\left(\frac{k \pi X_1}{X}\right) \right) \Big] \\
&+ \mathcal{O}(r^2(h + r^{\tilde{\eta}/2}))
\end{align*}
\begin{proof}
From the previous lemma,
\begin{align*}
\nu(\lambda) X &= k \pi i + h + \mathcal{O}(r^{3/2}X) \\
&= k \pi i + h + \mathcal{O}(r^{1/2})
\end{align*}
Expanding $\sinh(\nu(\lambda)X)$ and $\cosh(\nu(\lambda)X)$ in a Taylor series about $k \pi i$,
\begin{align*}
\sinh(\nu(\lambda)X) &= \sinh(k \pi i + h + \mathcal{O}(r^{1/2})) = (-1)^k h  + \mathcal{O}(h^3 + r^{1/2}) \\
\cosh(\nu(\lambda)X) &= \cosh(k \pi i + h + \mathcal{O}(r^{1/2})) = (-1)^k +  \mathcal{O}(h^2 + r^{1/2})
\end{align*}
From the previous lemma,
\[
\nu(\lambda)X_j = \frac{k \pi X_j}{X}i + \frac{h X_j}{X} + \mathcal{O}(r^{3/2} X_j)
= \frac{k \pi X_j}{X}i + \frac{h X_j}{X} + \mathcal{O}(r^{1/2})
\]
Expanding $\cosh(\nu(\lambda)X)$ and $\sinh(\nu(\lambda)X)$in a Taylor series about $\frac{k \pi X_j}{X}i$
\begin{align*}
|\cosh&(\nu(\lambda)X_j)| = \left| 
\cosh\left( \frac{k \pi X_j}{X}i \right) + \mathcal{O}(h + r^{1/2}) \right| \leq C
\end{align*}
and
\begin{align*}
\sinh&(\nu(\lambda)X_j)
= \sinh\left( \frac{k \pi X_j}{X}i \right) + \mathcal{O}(h + r^{1/2}) \\
&= i \sin\left( \frac{k \pi X_j}{X}\right) + \mathcal{O}(h + r^{1/2})
\end{align*}
Substituting all of this in, we have
\begin{align*}
\det &B_1 = -2 \lambda^2 M \Big[ (2a + \lambda^2 M)((-1)^k h + \mathcal{O}(h^3 + r^{1/2})) + \mathcal{O}(r^{3/2}(h + r^{1/2})) \\
&- 8 \lambda^2 p_1 ((-1)^k + \mathcal{O}(h + r^{1/2})) \left( k_0 i \sin\left(\frac{k \pi X_0}{X}\right) 
+ k_1 i \sin\left(\frac{k \pi X_1}{X}\right) + \mathcal{O}(r^{1/2}(h + r^{1/2})) \right) \Big] 
\end{align*}
which simplifies to
\begin{align*}
\det &B_1 = -2 \lambda^2 M (-1)^k \Big[ (2a + \lambda^2 M) h \\
&- 8 i \lambda^2 p_1 \left( k_0 \sin\left(\frac{k \pi X_0}{X}\right) + k_1 \sin\left(\frac{k \pi X_1}{X}\right) \right) + \mathcal{O}(r(h + r^{1/2}))\Big] 
\end{align*}
Using \cref{lambdas1}, \cref{hsbounds}, and $k_j = \mathcal{O}r^{1/2})$, expanding the sine term in a Taylor series to get
\begin{align*}
k_j \sin\left( \frac{k \pi X_j}{X}\right) 
&= k_j \sin\left( b X_j + \mathcal{O}\left( \frac{r^{1/2}X_j }{X^{1/2}} \right) \right) \\
&= k_j \sin(b X_j) + \mathcal{O}\left(r X_j^{1/2} \right) \\
&= k_j \sin(b X_j) + \mathcal{O}\left(r X_j^{1/2} \right)
\end{align*}
\end{proof}
\end{lemma}

At this point, we neglect some higher order terms. Since $\nu(\lambda)$ is purely imaginary for the cases we care about, all the $\sinh$ and $\cosh$ terms are bounded by 1. Since 
\[
\lambda^2 p_1(k_0 \cosh(\nu(\lambda)X_0) + k_1 \cosh(\nu(\lambda)X_1)  ) = \mathcal{O}(e^{-\alpha X_0}|\lambda|^2)
\]
we can neglect that term. Since $k_1 << k_0$, we can also neglect the term $k_1 \sinh(\nu(\lambda)X_1) )$. This leaves us with 
\begin{align*}
\det &B \approx -2 \lambda^2 M \Big[ (2a + \lambda^2 M  ) \sinh(\nu(\lambda)X) - 8 \lambda^2 p_1 k_0 \cosh(\nu(\lambda)X) \sinh(\nu(\lambda)X_0) \Big]
\end{align*}

Now we approximate the various terms. Krein bubbles occur (numerically) when interaction eigenvalues and essential spectrum eigenvalues collide. The essential spectrum eigenvalues occur at approximately 
\[
\lambda = c \frac{n \pi i}{X}
\]
for integer $n$. With this as motivation, we first take the scaling $\lambda = c \mu / X$. Then we have from Lemma \ref{nulambdalemma}
\begin{align*}
\nu(\lambda) &= \frac{1}{c} \lambda + \mathcal{O}(|\lambda|^3) 
\\
&= \frac{\mu}{X} + \mathcal{O}(\mu^3 / X^3)
\end{align*}
Next, we want $\mu$ to be a small perturbation of $n \pi i$, so take 
\[
\mu = n \pi i + h i
\]
where the $i$ in front of $h i$ is for convenience. Substituting this into the $\sinh$ and $\cosh$ terms, we get
\begin{align*}
\sinh(\nu(\lambda)X) &= \sinh(n \pi i + h i + \mathcal{O}(\mu^2 / X^3)) = (-1)^n h i  + \mathcal{O}(h^3 + \mu^2 / X^3) \\
\cosh(\nu(\lambda)X) &= \sinh(n \pi i + h i + \mathcal{O}(\mu^2 / X^3)) = (-1)^n +  \mathcal{O}(h^2 + \mu^4 / X^6)
\end{align*}
Substituting these into the expression for $B$ and neglecting higher order terms, we have
\begin{align*}
\det &B \approx -2 \lambda^2 M \Big[ (2a + \lambda^2 M  ) (-1)^n h i - 8 \lambda^2 p_1 k_0 (-1)^n \sinh(\nu(\lambda)X_0) \Big] \\
&= -2 (-1)^n \lambda^2 M \Big[ (2a + \lambda^2 M  ) h i - 8 \lambda^2 p_1 k_0 \sinh(\nu(\lambda)X_0) \Big] \\
&= -2 (-1)^n \lambda^2 M \Big[ (2a + \lambda^2 M  ) h i - 8 \lambda^2 r_0 \Big]
\end{align*}
where $r_0 = p_1 k_0 \sinh(\nu(\lambda)X_0)$. Since we want to solve $\det B = 0$ and the stuff out front is nonzero, we can ignore the stuff out front. Plugging in $\lambda = c \mu / X$, this become
\[
\left(2a + \frac{c^2 \mu^2}{X^2} M\right) h i + r_0 \frac{c^2 \mu^2}{X^2} = 0
\]
Multiplying by $X^2/(M c^2)$, this becomes
\[
\left(\frac{2a}{M c^2} X^2 + \mu^2 \right) h i + \frac{r_0}{M}\mu^2 = 0
\]
Recall that Krein bubbles can only occur when the interaction eigenvalues are on the imaginary axis. This happens when $\frac{2a}{M} > 0$. For convenience, take
\begin{align*}
b &= \sqrt{ \frac{2a}{M} } \\
\end{align*}
where $b$ is real. Substituting this, we have
\begin{align*}
\left(\mu^2 - (bi)^2 \frac{X^2}{c^2} \right) h i + \frac{r_0}{M} \mu^2 &= 0 \\
\left(\mu + i b\frac{X}{c}\right)\left(\mu - i b\frac{X}{c}\right) h i + \frac{r_0}{M} \mu^2 &= 0
\end{align*}
The Krein bubble only occurs when the interaction eigenvalues are close to the essential spectrum eigenvalues (all of which are purely imaginary). For this to occur, we need either $i b X/c$ or $-i b X/c$ to be close to $n \pi i$ on the imaginary axis. By symmetry, it does not matter which one we do, so let
\[
n \pi - b X/c = 2 k
\]
where $k$ is real. It follows that
\begin{align*}
\mu &= n \pi i + h i = 2 k i + h i + i b X/c \\
\mu - i b X/c &= 2 k i + h i\\
\mu + i b X/c &= 2 k i + h i + 2 i b X/c
\end{align*}
Substituting these in, we have
\[
\left(2 k i + h i + 2i b\frac{X}{c}\right)\left(2k i + i h\right) h i + \frac{r_0}{M} \left(2 i k + h i + i b \frac{X}{c}\right)^2 = 0
\]
Dividing by $i^2$, this becomes
\[
\left(2 k + h + 2 b\frac{X}{c}\right)\left(2k + h\right) h i + \frac{r_0}{M} \left(2 k + h + b \frac{X}{c}\right)^2 = 0
\]
We note that everything except for $h$ is real. Since $h$ and $k$ are small compared to $b X/c \approx n \pi$, this is approximately
\[
2 b\frac{X}{c}\left(2k + h\right) h i + \frac{r_0}{M} \left( b \frac{X}{c}\right)^2 = 0
\]
which simplifies to
\[
\left(2k + h\right) h i + \frac{r_0 b X }{M c} = 0
\]
Expanding the $\sinh$ term in $r_0$ in a Taylor series about $0$, we have
\begin{align*}
r_0 &= p_1 k_0 \sinh(\nu(\lambda)X_0) \\
&= p_1 k_0 \sinh\left( \frac{n \pi i + h i}{X} X_0 + h.o.t. \right) \\
&= p_1 k_0 \sinh\left( \frac{n \pi i X_0}{X} h.o.t. \right) \\
&= i p_1 k_0 X_0 \frac{n \pi}{X} + h.o.t.
\end{align*}
Substituting $n \pi = b X/c + 2 k$, this becomes
\begin{align*}
r_0 &= i p_1 k_0 X_0 \frac{b X + 2 k c}{X c} + h.o.t. \\
&= i \frac{p_1 k_0 b X_0}{c} + h.o.t.
\end{align*}
Substituting this in, we have
\[
\left(2k + h\right) h i + i \frac{p_1 k_0 b^2 X X_0 }{M c^2} = 0
\]
Dividing by $i$, this simplies to
\[
h^2 + 2kh + R = 0
\]
where 
\[
R = \frac{p_1 k_0 b^2 X X_0 }{M c^2}
\]
which is real. This has solution
\[
h = -k \pm \sqrt{k^2 - R}
\]
Then for $|k| \leq \sqrt{R}$,
\[
h = -k \pm i \sqrt{R - k^2}
\]
and so $h$ lies on a circle of radius $\sqrt{R}$ centered around the origin in the complex plane. Substituting back for all the things
\[
\sqrt{R} = \frac{b}{c} \sqrt{\frac{p_1 k_0 X X_0 }{M} }
\]
This is the radius in the perturbation $h$. Since $\lambda = (n \pi i + h i)/X$, to get the radius in $\lambda$, we divide by $X$ to get the Krein bubble radius
\[
\frac{\sqrt{R}}{X} = \frac{b}{c} \sqrt{\frac{p_1 k_0 X_0 }{M X} } = \mathcal{O}\left(\frac{ X_0^{1/2} e^{-(5/2) \alpha X_0} }{X^{1/2}} \right)
\] 
Numerics confirms the scaling in $X$ to high accuracy. There is no easy way to confirm the scaling in $X_0$ numerically, but questionably crude numerics confirms the scaling in $X_0$ to reasonable accuracy.

\end{document}
\documentclass[thesis.tex]{subfiles}

\begin{document}

\iffulldocument\else
	\chapter{KdV5}
\fi

\section{Proof of \cref{lemma:1blockmatrix} }

We use the same piecewise ansatz as in \cref{Vpiecewise}, and we note that that there is only one parameter $c$ and one parameter $d$. The system of equations \cref{eigsystem} becomes
\begin{equation}\label{eigsystemper1p}
\begin{aligned}
(W^\pm)'(x) &= A(Q(x); \lambda) W^\pm(x) + G^\pm(x) W^\pm(x) + c e^{\mp\nu(\lambda)X} G^\pm(x)V^\pm(x; \lambda) + d \lambda^2 \tilde{H}^\pm(x) \\
W^+(X) &- W^-(-X) = C_i c \\
W^\pm(0) &\in \R \Psi(0) \oplus \R \Psi^c(0) \oplus Y^+ \oplus Y^- \\
W^+(0) &- W_i^-(0) \in \R \Psi(0) \oplus \R \Psi^c(0) 
\end{aligned}
\end{equation}
In particular, the second equation in \cref{eigsystemper1p} (the matching condition at the tail) no longer contains the term $D_i d$. Equation \cref{1pblockmatrix} follows directly from \cref{blockmatrixtheorem}.

We next turn to symmetry. Since the periodic single pulse is symmetric, $Q^-(x) = R Q^+(-x)$. Following the same procedure as in \cref{AQxsymmetrylemma}, it follows that $G^-(x) = -R G^+(-x)R$. Multiplying the first equation in \cref{eigsystemper1p} by $R$ on the right and using the symmetry relations,
\begin{align*}
(R W^\pm)'(x) &= R A(Q(x); \lambda) R R W^\pm(x) + R G^\pm(x) R R W^\pm(x) + c e^{\mp\nu(\lambda)X} R G^\pm(x)R R V^\pm(x; \lambda) + d \lambda^2 R \tilde{H}^\pm(x) \\
&= -A(Q(-x); -\lambda) R W^\pm(x) - G^\mp(-x) R W^\pm(x) - c e^{\pm\nu(-\lambda)X} G^\mp(-x) V^\mp(-x; -\lambda) + d \lambda^2 \tilde{H}^\mp(-x)
\end{align*}
Making the change of variables $x \mapsto -x$, this becomes
\begin{align*}
[-R W^\mp(-x)]' &= A(Q(x); -\lambda)[ R W^\mp(-x)] + G^\pm(x) [-R W^\mp(-x)] - c e^{\mp\nu(-\lambda)X} G^\pm(x) R V^\pm(x; -\lambda) + d \lambda^2 \tilde{H}^\pm(x)
\end{align*}
Multiplying the second equation in \cref{eigsystemper1p} by $R$,
\begin{align*}
RW^+(X) - RW^-(-X) &= c \left( e^{\nu(\lambda) X_i} R V^-(-X_i; \lambda) - e^{-\nu(\lambda) X_i} R V^+(X_i; \lambda) \right) \\
 [-RW^-(-X)] - [-RW^+(X)]&= -c \left( e^{\nu(-\lambda) X_i} V^-(-X_i; -\lambda) - e^{-\nu(-\lambda) X_i} V^+(X_i; -\lambda) \right)
\end{align*}
Since $Y^+ = R Y^-$, $R \Psi(0) = \Psi(0)$, and $R \Psi^c(0) = \Psi^c(0)$, the system of equations \cref{eigsystemper1p} is satisfied by $(-RW^-(-x), -RW^+(-x))$ when $(c, d, \lambda) \mapsto (-c, d, -\lambda)$. Since the solution $(W^+(x), W^-(x))$ is unique for a given $(c, d, \lambda)$, 
\[
\left(W^+(x; c, d, -\lambda), W^-(x; -c, d, -\lambda)\right)
= -\left(RW^-(-x; c, d, \lambda), RW^+(-x; -c, d, \lambda)\right).
\]
For the jump expressions at $x = 0$,
\[
W^+(0; c, d, -\lambda) - W^-(0; -c, d, -\lambda) = RW^+(0; -c, d, \lambda) - RW^-(0; c, d, \lambda).
\]
Since $R \Psi^c(0) = \Psi^c(0)$ and $R \Psi(0) = \Psi(0)$,
\begin{align*}
\langle \Psi^c(0), W^+(0; c, d, -\lambda) - W^-(0; c, d, -\lambda) \rangle &= \langle \Psi^c(0), W^+(0; -c, d, \lambda) - W^-(0; -c, d, \lambda) \rangle \\
\langle \Psi(0), W^+(0; c, d, -\lambda) - W^-(0; c, d, -\lambda) \rangle &= \langle \Psi(0), W^+(0; -c, d, \lambda) - W^-(0; -c, d, \lambda) \rangle 
\end{align*}
Since $W(x; c, d; \lambda)$ is linear in $(c, d)$ by \cref{Zinv2}, 
\begin{align}\label{1pdetSneg}
S(-\lambda) &= 
\begin{pmatrix}
2 \sinh(\nu(\lambda) X) + \tilde{M}\lambda \cosh(\nu(\lambda) X) & -\tilde{M}^c \lambda^2 \\
M^c \lambda \cosh(\nu(\lambda)X) & - M \lambda^2
\end{pmatrix} +
\begin{pmatrix}
-C_1 & D_1 \\ -C_2 & D_2
\end{pmatrix}
= S(\lambda) K,
\end{align}
where 
\[
K = \begin{pmatrix}-1 & 0 \\ 0 & 1 \end{pmatrix}
\]
Since $\det K = -1$, $\det S(-\lambda) = -\det S(\lambda)$. The bounds on $C_i$ are the same as in \cref{blockmatrixtheorem}. The bounds on $D_i$ follow from \cref{blockmatrixtheorem} and noting that $D_i d = 0$ for the periodic single pulse, thus the the parameter $d$ only appears as $\lambda^2 d$ in \cref{eigsystemper1p}. We then compute $\det S(\lambda)$ directly, using the bounds for the $C_i$ and $D_i$.

\section{Proof of \cref{theorem:1pess} }

First, we make a change of variables to simplify the problem. Since $\nu'(0) = 1/c$ and $\nu(0) = 0$, $\nu(\lambda)$ is invertible near 0. Let $\lambda = \nu^{-1}(\mu)$. Expanding in Taylor series about $\mu = 0$,
\begin{equation}\label{2plambdamu}
\lambda = \nu^{-1}(\mu) = c \mu + \mathcal{O}(\mu^3)
\end{equation}
Substituting this into \cref{1pblockmatrixdet} and simplifying, we obtain the equation
\begin{equation}\label{1pdetmu}
\det S(\mu, r) = c^2 \mu^2 \left( 2 M \sinh(\mu X)(1 + \mathcal{O}(|\mu| + r^{1/2})) + c \mu (M \tilde{M} - M_c \tilde{M}_c)\cosh(\mu X) + \mathcal{O}(|\mu|(|\mu| + r^{1/2}) \right)
\end{equation}
For convenience, let $K = M \tilde{M} - M_c \tilde{M}_c$. Since we are looking for the nonzero essential spectrum eigenvalues, and we know there is an eigenvalue with algebraic multiplicity 2 at 0, we divide by $c \mu^2$ to get the equation
\begin{equation}\label{1pdetmu2}
E(\mu, r) = 2 M \sinh(\mu X)(1 + \mathcal{O}(|\mu| + r^{1/2})) + c K \mu \cosh(\mu X) + \mathcal{O}(|\mu|(|\mu| + r^{1/2})
\end{equation}
Choose positive integer $m \leq N_1$. For our ansatz, we take
\begin{equation}\label{singlemu}
\mu = \frac{m \pi i}{X + c \frac{K}{2 M}} + \frac{h}{X}
\end{equation}
Factoring out an $X$ in the denominator, and expanding the denominator in a Taylor series, which we can do for sufficiently large $X$,
\begin{align*}
\mu &= \frac{m \pi i}{X\left(1  + c \frac{K}{2 M X} \right) } + \frac{h}{X} \\
&= \frac{m \pi i}{X}\left( 1 - c \frac{K}{2 M X} + c^2 \frac{K^2}{4 M^2 X^2} + \mathcal{O}\left(\frac{1}{X^3}\right) \right) + \frac{h}{X}
\end{align*}
Substituting this into \cref{1pdetmu2} and expanding the $\sinh$ and $\cosh$ terms in a Taylor series about $m \pi i$,
\begin{align*}
E(h, r) &= 2 M (-1)^m \left( -\frac{m \pi c K i}{2 M X} + \frac{m \pi c^2 K^2 i}{4 M^2 X^2} + h + \mathcal{O}\left(\frac{1}{X^3}\right) \right) \left(1 + \mathcal{O}\left( \frac{1}{X} + r^{1/2} \right) \right) \\
&\qquad+ c K (-1)^m \left( \frac{m \pi i}{X} - c \frac{m \pi K i}{2 M X^2} + \frac{h}{X} + \mathcal{O}\left(\frac{1}{X^3}\right) \right) \left( 1 + \mathcal{O}\left(\frac{1}{X^2}\right) \right) \\
&\qquad+ \mathcal{O}\left( \frac{1}{X} \left( \frac{1}{X^2} + r^{1/2} \right) \right) \\
&= (-1)^m\left( 2M + \frac{cK}{X} \right) h + \mathcal{O}\left( \frac{1}{X} \left( \frac{1}{X} + r^{1/2} \right) \right)
\end{align*}
Dividing by $(-1)^m$, we wish to solve 
\begin{align*}
G_m(h, r) = \left( 2M + \frac{cK}{X} \right)h + \mathcal{O}\left( \frac{1}{X} \left( \frac{1}{X} + r^{1/2} \right) \right)
\end{align*}
Since $X = X(r) = \mathcal{O}(|\log r|)$, this becomes
\begin{align*}
G_m(h, r) = \left( 2M + \mathcal{O}\left( \frac{1}{|\log r|} \right)\right) h + \mathcal{O}\left( \frac{1}{|\log r|^2} \right),
\end{align*}
where we subsumed the faster decaying $\mathcal{O}(r^{1/2})$ term into the slower decaying $1/|\log r|^2$ remainder term. Since
\begin{align*}
G_m(0,0) &= 0 \\
\partial_h G_m(0,0) &= 2M \neq 0,
\end{align*}
we can use the implicit function theorem to solve for $h$ in terms of $r$ near $(h,r) = (0,0)$. Specifically, there exists $r_1^m \leq r_0$ and a unique smooth function $h_m(r)$ with $h_m(0) = 0$ such that for all $r \leq r_1^m$, $G_m(h_m(r),r) = 0$. Expanding $h_m$ in a Taylor series about $r = 0$,
\[
h_m(r) = \mathcal{O}\left( \frac{1}{|\log r|^2} \right)
\]
Let $r_1 = \min\{ r_1^1, \dots, r_1^{N_1} \}$. Substituting $h_m(r)$ into \cref{singlemu} for $m = 1, \dots, N_1$, there are essential spectrum eigenvalues located at
\[
\mu_m(r) = \frac{m \pi i}{X + c \frac{K}{2 M}} + \mathcal{O}\left( \frac{1}{|\log r|^3} \right)
\]
Changing variables back to $\lambda$, these are located at
\[
\lambda_m(r) = c  \frac{m \pi i}{X + c \frac{K}{2 M}} + \mathcal{O}\left( \frac{1}{|\log r|^3} \right)
\]
By Hamiltonian symmetry, eigenvalues must come in quartets. Since there is nothing else above the real axis with similar magnitude, $\lambda_m(r)$ is pure imaginary and there is another essential spectrum eigenvalue $\lambda_{-m}(r) = -\lambda_m(r)$. We conclude that the nonzero essential spectrum eigenvalues are given by $\lambda = \pm \lambda_m^{\text{ess}}(r)$ for $m = 1, \dots, N_1$, where
\[
\lambda_m^{\text{ess}}(r) = c  \frac{m \pi i}{X + c \frac{K}{2 M}} + \mathcal{O}\left( \frac{1}{|\log r|^3} \right).
\]

\section{Proof of \cref{lemma:2blockmatrix} }

First, we note that the periodic 2-pulse is symmetric, i.e. $Q_i^-(x) = R Q_{i-1}^+(-x)$, where the subscript $i$ is taken $\Mod 2$. Using the same procedure as in \cref{AQxsymmetrylemma}, $G_i^-(x) = -R G_{i-1}^+(-x)R$. Following what we did in the proof of \cref{lemma:2blockmatrix}, we multiply the first two equations of \cref{eigsystem} by $R$ on the right, simplify, and make the change of variables $x \mapsto -x$ to get
\begin{align*}
[-R W_i^-(-x)]' &= A(Q(x); -\lambda)[-R W_i^-(-x)] + G_{i-1}^+(x) [-R W_i^-(-x)] \\
&\qquad -c_{i-1} e^{-\nu(\lambda)X_{i-1}} G_{i-1}^+(x) V^+(x; -\lambda) + d_i \lambda^2 \tilde{H}_{i-1}^+(x) \\
[-R W_i^+(-x)]' &= A(Q(x); -\lambda)[-R W_i^+(-x)] + G_{i-1}^-(x)[-R W_i^+(-x)] \\
&\qquad -c_i e^{\nu(\lambda)X_i } G_{i-1}^-(x)V^-(x; -\lambda) + d_i \lambda^2 \tilde{H}_{i-1}^-(x),
\end{align*}
which we reindex to get
\begin{align*}
[-R W_{i-1}^+(-x)]' &= A(Q(x); -\lambda)[-R W_{i-1}^+(-x)] + G_i^-(x)[-R W_{i-1}^+(-x)] \\
&\qquad - c_{i-1} e^{\nu(\lambda)X_{i-1}} G_i^-(x)V^-(x; -\lambda) + d_{i-1} \lambda^2 \tilde{H}_i^-(x) \\
[-R W_{i-1}^-(-x)]' &= A(Q(x); -\lambda)[-R W_{i-1}^-(-x)] + G_i^+(x) [-R W_{i-1}^-(-x)] \\
&\qquad -c_i e^{-\nu(\lambda)X_i} G_i^+(x) V^+(x; -\lambda) + d_{i-1} \lambda^2 \tilde{H}_i^+(x)
\end{align*}
Multiplying $C_i c$ and $D_i d$ by $R$, we get
\begin{align*}
R C_i c &= -c_i \left( e^{\nu(-\lambda) X_i} V^-(-X_i; -\lambda) - e^{-\nu(-\lambda) X_i} V^+(X_i; -\lambda) \right) \\
R D_i d &= d_i [ \partial_x Q_{i+1}^-(-X_i) + (-\lambda) \partial_c Q_{i+1}^-(-X_i) ] - d_{i+1}[\partial_x Q_i^+(X_i) + (-\lambda) \partial_c Q_i^+(X_i)]
\end{align*}
Since $Y^+ = R Y^-$, $R \Psi(0) = \Psi(0)$, and $R \Psi^c(0) = \Psi^c(0)$, the system of equations \cref{eigsystemper1p} is satisfied by $(-R W_1^+(-x), -R W_1^-(-x), -R W_0^+(-x), -R W_1^-(-x))$ when $(c_1, c_0, d_1, d_0, \lambda) \mapsto (-c_1, -c_0, d_0, d_1, -\lambda)$. Since the solution $(W^+(x), W^-(x))$ is unique for a given $(c, d, \lambda)$, 
% \begin{align*}
% \left( W_0^-&(x; c, d, -\lambda), W_0^+(x; c, d, -\lambda), W_1^-(x; c, d, -\lambda), W_1^+(x; c, d, -\lambda) \right) \\
% &= -\left( R W_1^+(-x; -c, T d, -\lambda), R W_1^-(-x; -c, T d, -\lambda), R W_0^+(-x; -c, T d, -\lambda), R W_0^-(-x; -c, T d, -\lambda) \right) .
% \end{align*}
where 
\[
T = \begin{pmatrix} 0 & 1 \\ 1 & 0 \end{pmatrix}
\]
Since $R \Psi^c(0) = \Psi^c(0)$ and $R \Psi(0) = \Psi(0)$,
\begin{align*}
\langle \Psi^c(0), W_i^+(0; c, d, -\lambda) - W_i^-(0; c, d, -\lambda) \rangle &= \langle \Psi^c(0), W_{i-1}^+(0; -c, T d, \lambda) - W_{i-1}^-(0; -c, T d, \lambda) \rangle \\
\langle \Psi(0), W_i^+(0; c, d, -\lambda) - W_i^-(0; c, d, -\lambda) \rangle &= \langle \Psi(0), W_{i-1}^+(0; -c, T d, \lambda) - W_{i-1}^-(0; -c, T d, \lambda) \rangle 
\end{align*}
Thus we have $S(-\lambda) = K_1 S(\lambda) K_2$, where
\begin{align*}
K_1 = \begin{pmatrix}T & 0 \\ 0 & T \end{pmatrix}, 
K_2 = \begin{pmatrix}-I & 0 \\ 0 & T \end{pmatrix}.
\end{align*}
It follows that $\det S(-\lambda) = -\det S(\lambda)$. Computing the determinant of $S(\lambda)$ with the aid of Mathematica,


\section{Proof of \cref{lemma:chara} }

For part (i), from \cref{lemma:ajparam}, Theorem \cref{2pulsebifurcation}, and the definition of $a(r)$ in \cref{2pa}, we have $a = r \tilde{a}(r; m_0, s_0)$, where
\begin{align*}
\tilde{a}(r; m_0, s_0) &= -2 p_0 e^{\alpha \phi/\beta_0} e^{-\frac{1}{\rho}(m_0 \pi + s_0)} \left( \beta \cos\left(m_0 \pi + s_0 \right) - \alpha \sin \left(m_0 \pi + s_0 \right) \right) + \mathcal{O}(r^{\gamma/2\alpha}) \\
&= -2 p_0 e^{\alpha \phi/\beta_0} e^{-\frac{1}{\rho}(m_0 \pi + s_0)} (-1)^{m_0} \left( \beta \cos s_0 - \alpha \sin s_0 \right) + \mathcal{O}(r^{\gamma/2\alpha})
\end{align*}
As in \cref{pitchforkH}, $\tilde{a}(0; m_0, s_0) = 0$ if and only if $s_0 = p^* = \arctan \rho$. For $m_0 = 0$, since $p_0 > 0$ by \cref{lemma:ajparam}, if $s_0 < p^*$, $\tilde{a}(0; 0, s_0) < 0$ and if $s_0 > p^*$, $\tilde{a}(0; 0, s_0) > 0$. These are reversed for $m_0 = 1$.

For part (ii), it again follows from \cref{lemma:ajparam}, Theorem \cref{2pulsebifurcation}, and the definition of $a(r)$ in \cref{2pa} that $a = r \tilde{a}(r; m_0, s_1)$, where
\begin{align*}
\tilde{a}(r; m_0, s_1) &= -p_0 e^{\alpha \phi/\beta_0} \Big[ b_0(m_0, s_1) \left( \beta \cos\left(-\rho \log b_0(m_0, s_1) \right) - \alpha \sin \left(-\rho \log b_0(m_0, s_1) \right) \right) \\
&\qquad+ b_1(s_1) \left( \beta \cos\left(-\rho \log b_1(s_1) \right) - \alpha \sin \left(-\rho \log b_1(s_1) \right) \right) \Big]  + \mathcal{O}(r^{\gamma/2\alpha}) \\
&= -p_0 e^{\alpha \phi/\beta_0} H_1( b_0(m_0, s_1), b_1(s_1) ) + \mathcal{O}(r^{\gamma/2\alpha}),
\end{align*}
where $H_1(b_0, b_1)$ is defined in \cref{perdefH1}.
$b_1(s_1) = e^{-\frac{1}{\rho}s_1}$, and for each $m_0 \in \{ 0, 1 \}$, $b_0(m_0, s_1)$ is a continuous function which of $s_1$. The exact form of $b_0(m_0, s_1)$ is given in the proof of Lemma \ref{armpersists}. All that matters here is that $(b_0(m_0, s_1), b_1(s_1))$ is contained entirely within the zero set of $H(b_0, b_1)$, where $H(b_0, b_1)$ is defined in \cref{perdefH}. By Hypothesis \ref{Hoverlaphyp}, the two zero sets overlap only at the pitchfork bifurcation points, thus $\tilde{a}(r; m_0, s_1) = 0$ if and only if $s_1 = p^*$. By continuity in $s_1$, $\tilde{a}(r; m_0, s_1)$ must have the same sign for all $s_1 > p^*$. It only remains to determine that sign.

From Lemma \ref{lemma:ajparam} and the proof of Lemma \ref{armpersists}, we can write $\tilde{a}(r; m_0, s_1)$ as 
\[
\tilde{a}(r; m_0, s_1) = \tilde{a}_1(r, s_1) + \tilde{a}_0(r, m_0, s_1), where
\]
For $\tilde{a}_1(r, s_1)$, we have 
\begin{align*}
\tilde{a}_1(r, s_1) &= -p_0 e^{\alpha_0 \phi/\beta_0} e^{-\frac{1}{\rho}s_1} \left( \beta \cos s_1 - \alpha \sin s_1 \right) + \mathcal{O}(r^{\gamma/2\alpha_0}) \\
&= \mathcal{O}\left(r^{\gamma/2\alpha_0} + e^{-\frac{1}{\rho}s_1} \right)
\end{align*}
From Lemma \ref{lemma:ajparam}, the proof of Lemma \ref{armpersists}, and the bound on $\theta^*(\theta, m)$ from Lemma \ref{thetaparamlemma}, we have for $\tilde{a}_0(r, m_0, s_1)$
\begin{align*}
\tilde{a}_0&(r, m_0, s_1) \\
&= -p_0 e^{\alpha_0 \phi/\beta_0} e^{-\frac{1}{\rho}m_0} \left( \beta \cos\left(m_0 \pi + \mathcal{O}\left(e^{-\frac{1}{\rho}s_1} \right) \right) - \alpha \sin \left(m_0 \pi + \mathcal{O}\left(e^{-\frac{1}{\rho}s_1} \right) \right) \right) + \mathcal{O}\left(r^{\gamma/2\alpha_0} + e^{-\frac{1}{\rho}s_1} \right) \\
&= -(-1)^{m_0} p_0 \beta_0 e^{\alpha_0 \phi/\beta_0} e^{-\frac{1}{\rho}m_0} + \mathcal{O}\left(r^{\gamma/2\alpha_0} + e^{-\frac{1}{\rho}s_1} \right) \\
&= \tilde{a}^*(m_0) + \mathcal{O}\left(r^{\gamma/2\alpha_0} + e^{-\frac{1}{\rho}s_1} \right) \\
\end{align*}
where $\tilde{a}^*(m_0)$ is defined by
\[
\tilde{a}^*(m_0) = (-1)^{m_0 + 1} p_0 \beta_0 e^{\alpha_0 \phi/\beta_0} e^{-\frac{1}{\rho}m_0}
\]
and is negative if $m_0 = 0$ and positive if $m_0 = 1$. Equation \cref{tildeas1limit} follows by taking $r = 0$. For all $s_1 > p^*$, the sign of $\tilde{a}(r; m_0, s_1)$ is the same as the sign of $\tilde{a}^*(m_0)$ by continuity.

\section{Change of variables}

To find the eigenvalues, we will solve $\det S(\lambda ) = 0$. From \cref{corr:2perDet1}, this is given by
\begin{equation}\label{2pdetSlambda}
\begin{aligned}
\det S(&\lambda,r) = -2 \lambda^2 M (2a + \lambda^2 M) \sinh(\nu(\lambda)X) + R(\lambda),
\end{aligned}
\end{equation}
This equation involves both $\lambda$ and $\nu(\lambda)$, which is annoying. We will simplify the problem by making a change of variables. Since $\nu'(0) = 1/c$ and $\nu(0) = 0$, $\nu(\lambda)$ is invertible near 0. Let $\lambda = \nu^{-1}(\mu)$. Expanding in Taylor series about $\mu = 0$, we have
\begin{equation}\label{2plambdamu}
\lambda = \nu^{-1}(\mu) = c \mu + \mathcal{O}(\mu^3)
\end{equation}
Substituting this into \cref{2pdetSlambda} and simplifying, we obtain the equation
\begin{equation}\label{2detBeqmu}
\det S(\mu, r) = -2 c^2 \mu^2 M \left( 2a + c^2 \mu^2 M\right)\sinh(\mu X) + \mathcal{O}(|\mu| + r^{1/2})^5.
\end{equation}

To conclude this section, we prove that Hamiltonian symmetry applies to $\mu$ as well.

\begin{lemma}\label{lemma:Hamsymmmu}
If $\mu$ is a zero of $S(\mu)$, so are $-\mu$, $\overline{\mu}$, and $-\overline{\mu}$.
\begin{proof}
If $\mu$ is a zero, then $\lambda = \nu^{-1}(\mu)$ is a zero of $S(\lambda)$. By Hamiltonian symmetry, $-\lambda$, $\overline{\lambda}$, and $-\overline{\lambda}$ are also zeros of $S(\lambda)$. By \cref{nulambdalemma},
\begin{align*}
\nu(-\lambda) &= -\nu(\lambda) = -\mu \\
\nu(\overline{\lambda}) &= \overline{ \nu(\lambda) } = \overline{\mu} \\
\nu(-\overline{\lambda}) &= -\overline{ \nu(\lambda) } = -\overline{\mu},
\end{align*}
it follows that $-\mu$, $\overline{\mu}$, and $-\overline{\mu}$ are all zeros of $S(\mu)$.
\end{proof}
\end{lemma}

\section{Proof of \cref{theorem:2peigsassym}}

Let $r_*$ be as in Theorem \ref{2pulsebifurcation}. Choose $m_0 \in \{ 0, 1\}$ and any $s_1 > p^*$. For $r \leq r_*$, let $Q_2(x; r) = Q_2(x; m_0, s_1, r)$ be the corresponding family of asymmetric periodic 2-pulses. Choose any positive integer $N$, and let $\delta(N,r)$ be as in Theorem \ref{blockmatrixtheorem}. Let $\tilde{a}(r) = \tilde{a}(r; m_0, s_1)$, be as in Lemma \cref{lemma:chara}. By \cref{lemma:chara}, $\tilde{a}(0) \neq 0$. Let
\begin{equation}\label{2ptildemustar}
\tilde{\mu}_*(r) = \sqrt{-\frac{2\tilde{a}(r)}{M c^2}},
\end{equation}
which is nonzero. We expect to find a pair of interaction eigenvalues near $\mu = \pm \tilde{\mu}_*(r) r^{1/2}$. Substituting \cref{2ptildemustar} into \cref{2detBeqmu}, using $a(r) = r \tilde{a}(r)$, and simplifying, we obtain the equation
\begin{equation}\label{2detBeqmu2}
\begin{aligned}
\det S(\mu, r) = -2 c^4 M^2 \mu^2 (\mu - \tilde{\mu}_*(r)r^{1/2})(\mu + \tilde{\mu}_*(r)r^{1/2}) \sinh(\mu X) + \mathcal{O}(|\mu| + r^{1/2})^5.
\end{aligned}
\end{equation}

To make sure that the essential spectrum eigenvalues are ``out of the way'' of the interaction eigenvalues, we will choose $r_0 \leq r_*$ sufficiently small so that for all $r \leq r_0$,
\begin{equation}\label{2pnobubble}
\left|\tilde{\mu}_*(r) \right| r^{1/2}  \leq \frac{\pi}{2 X(r)},
\end{equation}
where $X(r)$ is the domain size. Precisely, since
\[
X(r) \leq \frac{1}{2 \alpha_0} 2 |\log r| + \frac{1}{2\beta_0}\left( s_1 + 2 \pi \right) + 2 L_0, 
\]
choose $r_0 \leq r_*$ such that
\begin{align}\label{2pnobubblecond}
\left|\tilde{\mu}_*(r) \right| r^{1/2} \left( \frac{1}{\alpha_0} |\log r| + \frac{1}{2\beta_0}\left( s_1 + 2 \pi \right) + 2 L_0 \right) \leq \frac{\pi}{2}
\end{align}
for all $r \leq r_0$. Since $r^{1/2}|\log r| \rightarrow 0$ as $r \rightarrow 0$ and $\tilde{\mu}_*(r)$ is bounded, we can always find such a $r_0$. This guarantees that \cref{2pnobubble} is satisfied. For the rest of this section, we will take $r \leq r_0$.

\subsection{Essential spectrum eigenvalues}

First, we find the essential spectrum eigenvalues. Let $N_1 \leq N$ be the largest positive integer such that $N_1/|\log r| < \delta(N,r)$. Let $m$ be any nonzero integer with $|m| \leq N_1$, and let
\begin{equation}\label{defmum}
\mu_m = \frac{m \pi i}{X} + \frac{h}{X}
\end{equation}
Expanding the $\sinh(\mu_m X)$ terms in \cref{2detBeqmu2} in a Taylor series about $m \pi/X$, we have
\begin{align}\label{sinhTayloress}
\sinh(\mu_m X) &= (-1)^m h + \mathcal{O}(h^3).
\end{align}
By \cref{2pnobubble}, $|\mu_m| \geq C r^{1/2}$ for $r \leq r_0$. Substituting this and the Taylor expansion \cref{sinhTayloress} into \cref{2detBeqmu2}, we have 
\begin{equation}\label{Bess1}
\begin{aligned}
\det S(h, r) = -2 c^4 M^2 \mu^2 (\mu - \tilde{\mu}_*(r)r^{1/2})(\mu + \tilde{\mu}_*(r)r^{1/2}) \left( h + \mathcal{O}(h^3) \right) + \mathcal{O}\left( |\mu_m|^5 \right).
\end{aligned}
\end{equation}

We want to solve $\det S(\mu, r) = 0$. Dividing by $(-1)^m$ and the constants out front, we want to solve 
\begin{equation}\label{Bess2}
\begin{aligned}
\mu_m^2 (\mu_m - \tilde{\mu}_*(r)r^{1/2})(\mu_m + \tilde{\mu}_*(r)r^{1/2}) \left( h + \mathcal{O}(h^3) \right) + \mathcal{O}\left( |\mu_m|^5 \right) = 0
\end{aligned}
\end{equation}
Since $\mu_m \neq 0$, divide by $\mu_m^2$ to get
\begin{equation}\label{Bess3}
\begin{aligned}
(\mu_m - \mu_*(r))(\mu_m + \mu_*(r)) \left( h + \mathcal{O}(h^3) \right) + \mathcal{O}\left( |\mu_m|^3 \right) = 0
\end{aligned}
\end{equation}
Finally, multiply by $X^2$ and substitute \cref{defmum} for $\mu_m$. Since $|m| \leq N$, $|\mu_m| \leq C N/X$, so we can simplify the remainder terms to get the equation $G_m(h, r) = 0$, where
\begin{equation}\label{Bess4}
\begin{aligned}
G_m(h, r) &= \left(m \pi i + h - \tilde{\mu}_*(r) r^{1/2} X\right)\left(m \pi i + h + \tilde{\mu}_*(r) r^{1/2} X \right) \left( h + \mathcal{O}(h^3) \right) + \mathcal{O}\left( \frac{1}{X} \right)
\end{aligned}
\end{equation}
Since $X = X(r) = \mathcal{O}(1/|\log r|)$, this becomes
\begin{equation}\label{Bess5}
\begin{aligned}
G_m(h, r) &= \left( m \pi i + h + \mathcal{O}(r^{1/2}|\log r|) \right)\left(m \pi i + h + + \mathcal{O}(r^{1/2}|\log r|) \right) \left( h + \mathcal{O}(h^3) \right) + \mathcal{O}\left( \frac{1}{|\log r|} \right)
\end{aligned}
\end{equation}
Since $r^{1/2}|\log r| \rightarrow 0$ as $r \rightarrow 0$,
\begin{align*}
G_m(0,0) &= 0 \\
\partial_h G_m(0,0) &= -m^2 \pi^2
\end{align*}
For all $m$ between 1 and $N_1$, $|\partial_h G_m(0,0)| \geq \pi^2 > 0$. Thus we can use the implicit function theorem to solve for $h$ in terms of $r$ near $h = 0$. Specifically, there exists $r_2^m \leq r_0$ and a unique smooth function $h_m(r)$ with $h_m(0) = 0$ such that for all $r \leq r_2^m$, $G_m(h_m(r),r) = 0$. Expanding $h_m$ in a Taylor series about $r = 0$,
\[
h_m(r) = \mathcal{O}\left( \frac{1}{|\log r|} \right)
\]
Let $r_2 = \min\{ r_2^1, \dots, r_2^{N_1} \}$. Substituting $h_m(r)$ into \cref{defmum}, for all nonzero integers $m = 1, \dots, N_1$, there are essential spectrum eigenvalues located at
\[
\mu_m(r) = \frac{m \pi i}{X} + \mathcal{O}\left( \frac{1}{|\log r|^2} \right)
\]
Changing variables back to $\lambda$, these are located at
\[
\lambda_m(r) = c \frac{m \pi i}{X}\left[1 + \mathcal{O}\left( \frac{1}{|\log r|^2} \right) \right] +\mathcal{O}\left( \frac{1}{|\log r|^2} \right)
\]
Since $X = \mathcal{O}(|\log r|)$, this simplifies to
\[
\lambda_m(r) = c \frac{m \pi i}{X} +\mathcal{O}\left( \frac{1}{|\log r|^2} \right)
\]

By Hamiltonian symmetry, eigenvalues must come in quartets. Since there is nothing else above the real axis with similar magnitude, $\lambda_m(r)$ is pure imaginary and there is another essential spectrum eigenvalue $\lambda_{-m}(r) = -\lambda_m(r)$. We conclude that the nonzero essential spectrum eigenvalues are given by $\lambda = \pm \lambda_m^{\text{ess}}(r)$ for $m = 1, \dots, N_1$, where
\[
\lambda_m^{\text{ess}}(r) = c \frac{m \pi i}{X} + +\mathcal{O}\left( \frac{1}{|\log r|^2} \right),
\]

\subsection{Eigenvalues at 0}\label{sec:eigcount0}

In this section, we show that there are exactly three eigenvalues at 0, and that there are no other eigenvalues within a certain distance of 0. We know that there is an eigenvalue at 0 with at least algebraic multiplicity 2, since the corresponding eigenfunctions are $\partial_x Q_2(x)$ and $\partial_c Q_2(x)$. Since $\tilde{a}(0) \neq 0$, the eigenvalues of $\tilde{A}(0)$ are distinct, thus by \cref{lemma:centereigenfn}, there is a third eigenfunction $V_n^c(x)$ with eigenvalue 0. We will use Rouch\'{e}'s theorem to show that there are no more eigenfunctions in a small circle around 0. Let
\[
\tilde{\xi} = \frac{1}{2}|\tilde{\mu}_*(0)|
\]
and take $\mu$ on the circle $|\mu| = r^{1/2} \tilde{\xi}$. By \cref{2pnobubble}, $|\mu X| \leq 1/2$, thus we can expand the $\sinh$ term in \cref{2detBeqmu2} in a Taylor series about 0 to get
\begin{equation*}
\begin{aligned}
\sinh(\mu X) &= \mu X + \mathcal{O}((\mu X)^3) \\
\end{aligned}
\end{equation*}
Using this, equation \cref{2detBeqmu2} simplifies to
\begin{equation}\label{2detBeqmu3}
\begin{aligned}
\det S(\mu, r) &= -2 c^4 M^2 \mu^2 (\mu - \tilde{\mu}_*(r)r^{1/2})(\mu + \tilde{\mu}_*(r)r^{1/2}) ( \mu X + \mathcal{O}(\mu X)^3) + \mathcal{O}\left( r^{5/2} \right),
\end{aligned}
\end{equation}
Rescale the problem by taking $\mu = r^{1/2}\tilde{\mu}$. After simplifying, equation \cref{2detBeqmu3} becomes
\begin{equation}\label{2detBeqmu4}
\begin{aligned}
\det S(\tilde{\mu}, r) &= -2 c^4 M^2 \mu^2 (\mu - \tilde{\mu}_*(r)r^{1/2})(\mu + \tilde{\mu}_*(r)r^{1/2}) \tilde{\mu} X + \mathcal{O}\left( r^{5/2} + r^{7/2}X^3 \right)
\end{aligned}
\end{equation}
By \cref{lemma:ajparam}, $\tilde{\mu}_*(r) = \tilde{\mu}_*(0) + \mathcal{O}(r^{\gamma/4 \alpha})$, thus \cref{2detBeqmu4} becomes
\begin{equation}\label{2detBeqmu5}
\begin{aligned}
\det S(\tilde{\mu}, r) &= -2 c^4 M^2 r^{5/2} (\tilde{\mu} - \tilde{\mu}_*(0)) (\tilde{\mu} + \tilde{\mu}_*(0)) \tilde{\mu} X + \mathcal{O}\left( r^{5/2} \left( 1 + r^{\gamma/4 \alpha} X \right) + r^{7/2}X^3 \right)
\end{aligned}
\end{equation}
We are looking for zeros of \cref{2detBeqmu5}. Dividing by $r^{5/2}X$ and the constants out front and using the estimate $X = \mathcal{O}(|\log r|)$, we wish to solve $G(\tilde{\mu}, r) = 0$, where
\begin{equation}\label{2peigG}
G(\tilde{\mu}, r) = \tilde{\mu}^3 (\tilde{\mu} - \tilde{\mu}_*(0)) (\tilde{\mu} + \tilde{\mu}_*(0)) + \mathcal{O}\left( \frac{1}{|\log r|} + r |\log r|^2 + r^{\gamma/4 \alpha} \right)
\end{equation}
In particular, we wish to show that $G(\tilde{\mu}, r)$ has exactly three zeros inside the circle of radius $\tilde{\xi}$. Let $G(\tilde{\mu}, r) = G_1(\tilde{\mu}) + G_2(\tilde{\mu}, r)$, where 
\[
G_1(\tilde{\mu}) = \tilde{\mu}^3 (\tilde{\mu} - \tilde{\mu}_*(0)) (\tilde{\mu} + \tilde{\mu}_*(0)),
\]
which is independent of $r$, and
\[
G_2(\tilde{\mu}, r) = \mathcal{O}\left( \frac{1}{|\log r|} + r |\log r|^2 + r^{\gamma/4 \alpha} \right).
\]
On the circle $|\tilde{\mu}| = \tilde{\xi}$,
\[
|G_1(\tilde{\mu})| \geq \frac{|M|}{16}|\tilde{\mu}_*(0)|^5
\]
Since $|G_2(\tilde{\mu}, r)| \rightarrow 0$ as $r \rightarrow 0$ and $\tilde{\mu}_*(0) \neq 0$, there exists $r_3 \leq r_0$ such that for $r \leq r_3$, $|G_2(\tilde{\mu}, r)| < |G_1(\tilde{\mu})|$ on the circle $|\tilde{\mu}| = \tilde{\xi}$. By Rouch\'{e}'s theorem $G(\tilde{\mu}, r)$ and $G_1(\tilde{\mu})$ have the same number of zeros (counted with multiplicity) inside the circle of radius $\tilde{\xi}$. By the choice of $\tilde{\xi}$, $G_1(\tilde{\mu})$ has exactly 3 zeros inside the circle, thus $G(\tilde{\mu}, r)$ does as well. Undoing the scaling and changing variables back to $\lambda$, for $r \leq r_2$, there are exactly three eigenvalues inside a circle of radius $\sqrt{|2\tilde{a}(0)/M|}r^{1/2}$. These must correspond to the three kernel eigenvalues discussed above.

\subsection{Interaction eigenvalues}\label{sec:assyminteigs}

In this section, we will find the interaction eigenvalues. Using the same setup and scaling as in the previous section, by \cref{2detBeqmu4} we have
\begin{equation}\label{2detint2}
\begin{aligned}
\det S(\tilde{\mu}, r) &= -2 c^4 M^2 \mu^2 (\mu - \tilde{\mu}_*(r)r^{1/2})(\mu + \tilde{\mu}_*(r)r^{1/2}) \tilde{\mu} X + \mathcal{O}\left( r^{5/2} + r^{7/2}X^3 \right)
\end{aligned}
\end{equation}
Dividing by $r^{5/2}X$ and the constants out front and using the estimate $X = \mathcal{O}(|\log r|)$, we wish to solve $G(\tilde{\mu}, r) = 0$, where
\begin{equation}\label{2detintG1}
\begin{aligned}
G(\tilde{\mu}, r) &= (\tilde{\mu} - \tilde{\mu}_*(r)) (\tilde{\mu} + \tilde{\mu}_*(r))\tilde{\mu}^3 + \mathcal{O}\left( \frac{1}{|\log r|} + r |\log r|^2 \right)
\end{aligned}
\end{equation}
We expect to find the interaction eigenvalues near $\tilde{\mu} = \pm \tilde{\mu}_*(r)$. By symmetry, we only need to consider one of these. Let
\[
\tilde{\mu} = \tilde{\mu}_*(r) + \tilde{h}
\]
Then \cref{2detintG1} becomes
\begin{equation}\label{2detintG2}
\begin{aligned}
G(\tilde{h},r) = \tilde{h} ( \tilde{h} + \tilde{\mu}_*(r))^3 (\tilde{h} + 2 \tilde{\mu}_*(r)) + \mathcal{O}\left( \frac{1}{|\log r|} + r |\log r|^2 \right)
\end{aligned}
\end{equation}
Since $r |\log r|^2  \rightarrow 0$ as $r \rightarrow 0$, at $(\tilde{h},r) = (0, 0)$ we have
\begin{align*}
G(0, 0) &= 0 \\
\partial_{\tilde{h}} G(0, 0) &= 2 \tilde{\mu}_*(0)^4.
\end{align*}
Since $\tilde{\mu}_*(0) \neq 0$, we can use the implicit function theorem to solve $G(\tilde{h},r) = 0$ for $\tilde{h}$ in terms of $r$ near $\tilde{h} = 0$. Specifically, there exists $r_4 \leq r_0$ and a unique smooth function $\tilde{h}(r)$ with $\tilde{h}(0) = 0$ such that for all $r \leq r_4$, $G(\tilde{h}, r) = 0$. Expanding $\tilde{h}(r)$ in a Taylor series about $r = 0$, for all $r \leq r_4$,
\[
\tilde{h}(r) = \mathcal{O}\left( \frac{1}{|\log r|} + r |\log r|^2 \right) = \mathcal{O}\left( \frac{1}{|\log r|} \right)
\]
Undoing the scaling, there is an interaction eigenvalue located at
\begin{align*}
\mu(r) = \sqrt{-\frac{\tilde{a}(r)}{M c^2}}r^{1/2} + \mathcal{O}\left( \frac{r^{1/2}}{|\log r|} \right)
\end{align*}
Changing variables back to $\lambda$, this is located at
\begin{align*}
\lambda(r) = \sqrt{-\frac{2 \tilde{a}(r)}{M}}r^{1/2} + \mathcal{O}\left( \frac{r^{1/2}}{|\log r|} \right)
\end{align*}
By Hamiltonian symmetry there is also an eigenvalue at $-\lambda(r)$. Since Hamiltonian symmetry  dicatates that eigenvalues must come in quartets, and there only two eigenvalues of this magnitude, we conclude that for $r \leq r_4$, there is a pair of interaction eigenvalues given by $\lambda = \pm \lambda^{\text{int}}(r)$, where
\[
\lambda^{\text{int}}(r) = \sqrt{ -\frac{2 \tilde{a}(r)}{M} }r^{1/2} + \mathcal{O}\left( \frac{r^{1/2}}{|\log r|} \right).
\]
These are real if $\tilde{a}(0) < 0$ and purely imaginary if $\tilde{a}(0) > 0$. By \cref{lemma:chara}, the sign of $\tilde{a}(0)$ depends only on $m_0$ and the sign of $M$.

\subsection{Eigenvalue count}

Finally, we prove that we have accounted for all of the eigenvalues near 0. Let $N_1$ be as above, and let
\[
\xi = \left( N_1 + \frac{1}{2} \right)\pi
\]
Take $\mu$ on the circle $|\mu| = \xi/X$. We will use Rouch\'{e}'s theorem to show that there are no zeros of $\det S(\mu)$ inside the circle of radius $\xi$ besides the ones we have already found. This time, we rescale the problem by taking $\mu = h/X$. Making this substitution and noting that $|\mu| \geq C r^{1/2}$, equation \cref{2detBeqmu2} becomes
\begin{equation}\label{Gcount2}
\begin{aligned}
\det S(h, r) &= -2 c^4 M^2 \left(\frac{h}{X}\right)^2 \left( \frac{h}{X} - \tilde{\mu}_*(r)r^{1/2}\right)\left(\frac{h}{X} + \tilde{\mu}_*(r)r^{1/2}\right) \sinh(h) + \mathcal{O}\left(\frac{|h|^5}{X^5}\right)
\end{aligned}
\end{equation}
Multiplying by $X^4$ and dividing by the constants out front, we wish to find the zeros of $G(h, r)$, which is given by
\begin{equation}\label{Gcount3}
\begin{aligned}
G(h,r) &= h^2 \left( h - \tilde{\mu}_*(r)r^{1/2}X\right)\left(h + \tilde{\mu}_*(r)r^{1/2}X\right) \sinh(h) + \mathcal{O}\left(\frac{|h|^5}{X}\right)
\end{aligned}
\end{equation}
Write $G(h,r) = G_1(h) + G_2(h,r)$, where
\[
G_1(h, r) = h^2 \left( h - \tilde{\mu}_*(r)r^{1/2}X\right)\left(h + \tilde{\mu}_*(r)r^{1/2}X\right) \sinh(h)
\]
and
\[
G_2(h,r) = \mathcal{O}\left(\frac{|h|^5}{X}\right)
\]
On a circle of radius $|h| = \xi$, $|\sinh h| = 1$ by our choice of $\xi$. By \cref{2pnobubble}, $|h - \tilde{\mu}_*(r)r^{1/2}X|\geq \pi$. Thus on the circle $|h| = \xi$
\[
|G_1(h,r)| \geq \pi^2 \xi^2,
\]
which is independent of $r$. For $|h| = \xi$, $|G_2(h,r)| \rightarrow 0$ as $r \rightarrow 0$ since $X = \mathcal{O}(|\log r|)$. Thus there exists $r_5 \leq r_0$ such that for all $r \leq r_5$, $|G_2(h,r)| < |G_1(h,r)|$ on the circle $|h| = \xi$. By Rouch\'{e}'s theorem, $G(h,r)$ and $G_1(h,r)$ have the same number of zeros (counted with multiplicity) inside the circle $|h| = \xi$. By our choice of $\xi$, $G_1(h,r)$, and thus $G(h,r)$, has $2 N_1 + 5$ zeros inside this circle, which correspond exactly to the 3 eigenvalues at 0, the two interaction eigenvalues, and the first $2 N_1$ nonzero essential spectrum eigenvalues. Since we have accounted for all the zeros of $G(h,r)$, we conclude that there can be no more zeros of $G(h,r)$ inside the circle $|h| = \xi$. Undoing the scaling and changing variables back to $\lambda$, we have shown that there are no other eigenvalues inside a circle with radius slightly larger than $|\lambda_{N_1}^{\text{ess}}(r)|$.

\subsection{Proof of \cref{theorem:2peigsassym}}

The proof of \cref{theorem:2peigsassym} follows from combining the results of the previous four subsections and taking $r_1 = \min\{ r_2, r_3, r_4, r_5 \}$.

\section{Proof of \cref{theorem:2peigssym}}

To prove \cref{theorem:2peigssym}, we will first show that if we are sufficiently close to the bifurcation point, there are five eigenvalues in a small ball around the origin. We will then show that when were are at the bifurcation point, those eigenvalues are actually all at the origin.

\subsection{Count of eigenvalues near 0}

From the proof of \cref{lemma:chara}, for $m_1 \in \{0, 1\}$ and $s_0 \in [0, \pi)$,
\begin{align*}
\tilde{a}(r) = \tilde{a}(r, s_0) &= 2 (-1)^{m_0} s_0 \alpha_0 e^{\alpha_0 \phi/\beta_0} e^{-\frac{1}{\rho}(m_0 \pi + s_0) } \left( \rho \cos s_0 - \sin s_0 \right) + \mathcal{O}\left(r^{\gamma/2\alpha_0} \right)\\
\end{align*}
and $\tilde{a}(0) = 0$ if and only if $s_0 = p^*$. Since $\tilde{a}(r)$ is smooth in $s_0$ and $r$, there exist $\eta > 0$ and $r_2 \leq r_*$ such that for all $r \leq r_2$ and $s_0 \in (p^* - \eta, p^* + \eta)$, 
\begin{equation}
\sqrt{ \left| \frac{ 2 \tilde{a}(r, s_0) }{M} \right| }  r^{1/2} \leq \frac{\pi}{2 X(r)}
\end{equation}
Since $X(r) = \mathcal{O}(|\log r|)$, this is always possible. Let $\xi = c \pi / 2 X(r)$, and follow the same procedure as in \cref{sec:eigcount0}. Using Rouch\'{e}'s theorem, it follows that for $r \leq r_2$ and $s_0 \in (p^* - \eta, p^* + \eta)$, there are exactly 5 zeros of $S(\lambda)$ inside the circle of radius $\xi$. We know that two of the zeros must be at $\lambda = 0$ since they correspond to the kernel eigenfunctions $\partial_x Q_2(x)$ and $\partial_c Q_2(x)$. This leaves three eigenvalues unaccounted for.
By Hamiltonian symmetry, since eigenvalues must come in quartets, there must be a third eigenvalue at 0.

\subsection{Eigenvalues at bifurcation point}
Using the scaling $\mu = r^{1/2} \tilde{\mu}$ and following the same steps as in \cref{sec:assyminteigs}, we wish to solve
\begin{equation*}
\begin{aligned}
\begin{aligned}
0 &= (\tilde{\mu} - \tilde{\mu}_*(r)) (\tilde{\mu} + \tilde{\mu}_*(r))\tilde{\mu}^3 + \mathcal{O}\left( \frac{1}{|\log r|} + r |\log r|^2 \right)
\end{aligned}
\end{aligned}
\end{equation*}
Since we know that there will always be two eigenvalues at 0 corresponding to $\partial_x Q_n(x)$ and $\partial_c Q_n(x)$ and we proved the existence of a third eigenvalue at 0 in the previous section, we can divide by $\tilde{\mu}^3$ and the constants out front to get 
\begin{align*}
0 &= (\tilde{\mu} - \tilde{\mu}_*(r)) (\tilde{\mu} + \tilde{\mu}_*(r)) + \mathcal{O}\left( \frac{1}{|\log r|} + r |\log r|^2 \right) \\
&= \tilde{\mu}^2 - \tilde{\mu}_*(r)^2 + \mathcal{O}\left( \frac{1}{|\log r|} + r |\log r|^2 \right) \\
&= \tilde{\mu}^2 + \frac{2\tilde{a}(r)}{M c^2} + \mathcal{O}\left( \frac{1}{|\log r|} + r |\log r|^2 \right)\\
\end{align*}
Using the expression for $\tilde{a}(r)$ from \cref{lemma:chara}, this becomes
\begin{equation}\label{Gsymm1}
0 = \tilde{\mu}^2 + \frac{2\tilde{a}(0, s_0)}{M c^2} + \mathcal{O}\left( \frac{1}{|\log r|} + r |\log r|^2 + r^{\gamma/2\alpha_0} \right)
\end{equation}
Since we are interested in what happens near $s_0 = p^*$, let $s_0 = p^* + h$. Substituting this for $s_0$ into $\tilde{a}(0, s_0)$ and expanding in a Taylor series, we have
\begin{align*}
\tilde{a}(0, p^* + h) &= 2 (-1)^{m_0} s_0 \alpha_0 e^{\alpha_0 \phi/\beta_0} e^{-\frac{1}{\rho}(m_0 \pi + p^*) } e^{-\frac{1}{\rho}h }\left( \rho \cos(p^* + h) - \sin(p^* + h) \right) \\
&= 2 (-1)^{m_0} s_0 \alpha_0 e^{\alpha_0 \phi/\beta_0} e^{-\frac{1}{\rho}(m_0 \pi + p^*) }\left( 1 - \frac{1}{\rho}h + \mathcal{O}(|h|^2) \right) \left( -\frac{2}{\sqrt{1 + \rho^2} }h + \mathcal{O}(|h|^2) \right) \\
&= -\frac{4}{\sqrt{1 + \rho^2} }(-1)^{m_0} s_0 \alpha_0 e^{\alpha_0 \phi/\beta_0} e^{-\frac{1}{\rho}(m_0 \pi + p^*) }h + \mathcal{O}(|h|^2)
\end{align*}
Since the coefficient of $h$ is a constant, we can write this as
\[
\tilde{a}(p^* + h, 0) = C_1 h + \mathcal{O}(|h|^2) 
\]
where $C_1 \neq 0$. Substituting this into \cref{Gsymm1}, we wish to solve the equation the equation $G(\tilde{\mu}, h, r) = 0$, where
\begin{equation}\label{Gsymm2}
G(\tilde{\mu}, h, r) = \tilde{\mu}^2 + \frac{2C_1}{M c^2}h + \mathcal{O}\left( |h|^2 + \frac{1}{|\log r|} + r |\log r|^2 + r^{\gamma/2\alpha_0} \right).
\end{equation}
When $r = 0$ and $h = 0$, $G(\tilde{\mu}, h, r)$ has a double root at $\tilde{\mu} = 0$. We wish to show that this double root at $\tilde{\mu} = 0$ persists for small $r$.

For $\tilde{\mu}$ to be a double root, it must solve both $G(\tilde{\mu}, h, r) = 0$ and $\partial_{\tilde{\mu}} G(\tilde{\mu}, h, r) = 0$. Define $K(\tilde{\mu}, h, r)$ by
\begin{equation}
K(\tilde{\mu}, h, r) = 
\begin{pmatrix}G(\tilde{\mu}, h, r) \\ \partial_{\tilde{\mu}}G(\tilde{\mu}, h, r) \end{pmatrix} 
= \begin{pmatrix}
\tilde{\mu}^2 + \frac{2C_1}{M c^2}h \\
2 \tilde{\mu}
\end{pmatrix}
+ \mathcal{O}\left( |h|^2 + \frac{1}{|\log r|} + r |\log r|^2 + r^{\gamma/2\alpha_0} \right)
\end{equation}
When $r = 0$, $K(\tilde{\mu}, 0, 0) = 0$. The Jacobian with respect to $(\tilde{\mu}, h)$ is
\[
D_{(\tilde{\mu}, h)}K(\tilde{\mu}, h, r) = 
\begin{pmatrix}
2 \tilde{\mu} & \frac{2C_1}{M c^2} \\
2 & 0
\end{pmatrix}
+ \mathcal{O}\left( |h| + \frac{1}{|\log r|} + r |\log r|^2 + r^{\gamma/2\alpha_0} \right)
\]
At $(\tilde{\mu}, h) = (0,0)$, 
\[
D_{(\tilde{\mu}, h)}K(\tilde{\mu}, h, r)\Big|_{(0,0)} = 
\begin{pmatrix}
0 & \frac{2C_1}{M c^2} \\
2 & 0
\end{pmatrix},
\]
which is nonsingular since all the constants in the upper right block are nonzero. Using the implicit function theorem, we can solve for $(\tilde{\mu}, h)$ in terms of $r$ for sufficiently small $r$. pecifically, there exists $r_1 \leq r_2$ and a unique smooth function $(\tilde{\mu}, h)(r)$ with $(\tilde{\mu}, h)(0) = (0, 0)$ such that for all $r \leq r_1$, $K(\tilde{\mu}(r), h(r), r) = 0$. For $r \leq r_1$ and $h = h(r)$, $\tilde{\mu}(r)$ is a double root of $G(\tilde{\mu}, h, r)$. By Hamiltonian symmetry, we must have $\tilde{\mu}(r) = 0$. Expanding $h(r)$ in a Taylor series about $r$, 
\begin{equation*}
h(r) = \mathcal{O}\left( \frac{1}{|\log r|} + r |\log r|^2 + r^{\gamma/2\alpha_0} \right)
\end{equation*}

We conclude that when $s_0 = p^* + h(r)$, there are two more eigenvalues at 0. From the previous section, there are 5 zeros of $S(\lambda)$ inside a the circle of radius $\xi$. Thus when $s_0 = p^* + h(r)$, there is an eigenvalue at 0 with algebraic mutiplicty 5. Since the pitchfork bifurcation in the family of periodic 2-pulses occurs near $p^*$ at $p^*(r)$, it follows from standard PDE bifurcation theory that this quintuiple zero must occur at the pitchfork bifurcation point, i.e. $p^* + h(r) = p^*(r)$.

\iffulldocument\else
	\bibliographystyle{amsalpha}
	\bibliography{thesis.bib}
\fi

\end{document}
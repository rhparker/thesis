\documentclass[thesis.tex]{subfiles}

\begin{document}

\iffulldocument\else
	\chapter{KdV5}
\fi

\section{Proof of stability theorems for periodic 2-pulse}

\subsection{Proof of Corollary \ref{2blockmatrix}}

In this section, we will compute the determinant of the block matrix $S(\lambda)$ from \cref{blockmatrixtheorem} for the periodic 2-pulse. The computation is tedious, but it can be done with the aid of Mathematica. Choose a positive integer $N$, and let $\delta(r, N)$ be as in \cref{blockmatrixtheorem}. In particular, we note that $\delta(r,N) \leq N/|\log r|$. 

To compute the determinant, we write the block matrix $S(\lambda)$ as $S(\lambda) = S_1(\lambda) + S_2(\lambda)$, where the leading order terms are given by
\begin{align*}
S_1&(\lambda) = \\
&\begin{pmatrix}
\begin{pmatrix}
e^{-\nu(\lambda)X_1} & -e^{\nu(\lambda)X_0} \\
-e^{\nu(\lambda)X_1} & e^{-\nu(\lambda)X_0} 
\end{pmatrix} &
2 \lambda \begin{pmatrix}
-e^{-\nu(\lambda)X_1} q(X_1) + e^{\nu(\lambda)X_0} q(X_1) & e^{-\nu(\lambda)X_1} q(X_1) - e^{\nu(\lambda)X_0} q(X_1) \\ e^{-\nu(\lambda)X_0} q(X_1) - e^{\nu(\lambda)X_1} q(X_1) & -e^{-\nu(\lambda)X_0} q(X_1) + e^{\nu(\lambda)X_1} q(X_1)
\end{pmatrix} \\
-M^c \lambda
\begin{pmatrix}
e^{-\nu(\lambda)X_1} & e^{\nu(\lambda)X_0} \\
e^{\nu(\lambda)X_1} & e^{-\nu(\lambda)X_0} 
\end{pmatrix} &
\begin{pmatrix}
-a - \lambda^2 M & a \\
a & -a - \lambda^2 M
\end{pmatrix}
\end{pmatrix}
\end{align*}
where
\[
a = \langle \Psi(X_0), Q'(-X_0) \rangle + \langle \Psi(X_1), Q'(-X_1) \rangle
\]
and the remainder terms are in $S_2(\lambda)$, which is the block matrix 
\[
S_2(\lambda) = \begin{pmatrix}
C_1 & D_1 \\ C_2 & D_2
\end{pmatrix}
\]

We use Mathematica to evaluate the determinant. To do ths, we carefully track each remainder term in each remainder matrix in $S_2(\lambda)$ and use Lemma \ref{lemma:expnubound} to bound terms of the form $e^{\pm \nu(\lambda)X_j}$. Doing this and simplifying, we have
\begin{equation}\label{2detB1}
\begin{aligned}
\det S(&\lambda) = \sinh(\nu(\lambda)X)\left(-2 \lambda^2 M (2a + \lambda^2 M) +  \mathcal{O}( (r^{1/2} + |\lambda|)^5 \right) \\
&-\lambda^4 q(X_0) M M^c \left( \sinh(\nu(\lambda)(2 X_0 + X_1)) - 3 \sinh(\nu(\lambda)X_1)  \right) \\
&-\lambda^4 q(X_1) M M^c \left( \sinh(\nu(\lambda)(2 X_1 + X_0)) - 3 \sinh(\nu(\lambda)X_0)  \right) \\
&+ \mathcal{O}( (r^{1/2} + |\lambda|)^6) 
\end{aligned}
\end{equation}
Using standard hyperbolic trigonometric identities, we can show that 
\begin{align*}
\sinh(2 x + y) - 3 \sinh(y) &= 4 \cosh(x + y)\sinh(x) 
-2 \sinh(x+y)\cosh(x) 
\end{align*}
Using this with the second line of \cref{2detB1}, 
\begin{align*}
\lambda^4 &q(X_0) \left( \sinh(\nu(\lambda)(2 X_0 + X_1)) - 3 \sinh(\nu(\lambda)X_1)  \right) \\
&= 4 \lambda^4 q(X_0) \cosh(\nu(\lambda)X)\sinh(\nu(\lambda)X_0) - 3 \lambda^4 q(X_0) \sinh(\nu(\lambda)X)\cosh(\nu(\lambda)X_0) \\
&= 4 \lambda^4 q(X_0) \cosh(\nu(\lambda)X)\sinh(\nu(\lambda)X_0) + \sinh(\nu(\lambda)X)(\mathcal{O}(r^{1/2}|\lambda|^4))
\end{align*}
since $|\cosh(\nu(\lambda)X_0)|\leq C$ by Lemma \ref{lemma:expnubound}. Doing the same thing for the third line of \cref{2detB1}, substituting the results into \cref{2detB1}, and simplifying, we have
\begin{equation}\label{2pdetSlambda}
\begin{aligned}
\det S(&\lambda) = \left(-2 \lambda^2 M (2a + \lambda^2 M) + R_1  \right) \sinh(\nu(\lambda)X) \\
&-4 M M^c \lambda^4 ( q(X_0) \sinh(\nu(\lambda)X_0) + q(X_1) \sinh(\nu(\lambda)X_1) ) \cosh(\nu(\lambda)X)  + R_2
\end{aligned}
\end{equation}
where the remainder terms have bounds
\begin{align*}
|R_1| \leq C(r^{1/2} + |\lambda|)^5 \\
|R_2| \leq C(r^{1/2} + |\lambda|)^6
\end{align*}

\subsection{Change of variables}

We can now solve equation \cref{2pdetSlambda} to find the eigenvalues $\lambda$. However, this equation involves both $\lambda$ and $\nu(\lambda)$, which is annoying. We will simplify the problem by making a change of variables. Since $\nu'(0) = 1/c$ and $\nu(0) = 0$, $\nu(\lambda)$ is invertible near 0. Let $\lambda = \nu^{-1}(\mu)$. Expanding in Taylor series about $\mu = 0$, we have
\begin{equation}\label{2plambdamu}
\lambda = \nu^{-1}(\mu) = c \mu + \mathcal{O}(\mu^3)
\end{equation}
Substituting this into \cref{2pdetSlambda} and simplifying, we obtain the equation
\begin{equation}\label{2detBeqmu}
\begin{aligned}
\det S(&\mu) = \left(-2 c^2 \mu^2 M (2a + c^2 \mu^2 M) + R_1 )\right) \sinh(\mu X) \\
&-4 M M^c c^4 \mu^4 ( q(X_0) \sinh(\mu X_0) + q(X_1) \sinh(\mu X_1) ) \cosh(\mu X) + R_2
\end{aligned}
\end{equation}
Incorporating the higher order terms from \cref{2plambdamu} into the remainder terms, the bounds on the remainder terms are unchanged.
\begin{equation}\label{muRbounds}
\begin{aligned}
|R_1| \leq C(r^{1/2} + |\mu|)^5 \\
|R_2| \leq C(r^{1/2} + |\mu|)^6
\end{aligned}
\end{equation}

To conclude this section, we prove that Hamiltonian symmetry applies to $\mu$ as well.

\begin{lemma}\label{lemma:Hamsymmmu}
If $\mu$ is a zero of $S(\mu)$, so are $-\mu$, $\overline{\mu}$, and $-\overline{\mu}$.
\begin{proof}
If $\mu$ is a zero, then $\lambda = \nu^{-1}(\mu)$ is a zero of $S(\lambda)$. By Hamiltonian symmetry, $-\lambda$, $\overline{\lambda}$, and $-\overline{\lambda}$ are also zeros of $S(\lambda)$. By \cref{nulambdalemma},
\begin{align*}
\nu(-\lambda) &= -\nu(\lambda) = -\mu \\
\nu(\overline{\lambda}) &= \overline{ \nu(\lambda) } = \overline{\mu} \\
\nu(-\overline{\lambda}) &= -\overline{ \nu(\lambda) } = -\overline{\mu},
\end{align*}
it follows that $-\mu$, $\overline{\mu}$, and $-\overline{\mu}$ are all zeros of $S(\mu)$.
\end{proof}
\end{lemma}

\subsection{Proof of \cref{lemma:chara}}

\begin{enumerate}
	\item Proof of part (i) is straightforward and will be filled in.
	\item Proof of part (ii) is not complete, but numerics strongly suggests result.
	\item Proof of \cref{tildeas1limit} is as follows.

	From Lemma \ref{lemma:ajparam} and the proof of Lemma \ref{armpersists},
	\begin{align*}
	\tilde{a}_1(r, s_1) &= -s_0 e^{\alpha_0 \phi/\beta_0} e^{-\frac{1}{\rho}s_1} \left( \beta \cos s_1 - \alpha \sin s_1 \right) + \mathcal{O}(r^{\gamma/2\alpha_0}) \\
	&= \mathcal{O}\left(r^{\gamma/2\alpha_0} + e^{-\frac{1}{\rho}s_1} \right)
	\end{align*}
	From Lemma \ref{lemma:ajparam}, the proof of Lemma \ref{armpersists}, and the bound on $\theta^*(\theta, m)$ from Lemma \ref{thetaparamlemma}, 
	\begin{align*}
	\tilde{a}_0(r, m_0, s_1) &= -s_0 e^{\alpha_0 \phi/\beta_0} e^{-\frac{1}{\rho}m_0} \left( \beta \cos\left(m_0 \pi + \mathcal{O}\left(e^{-\frac{1}{\rho}s_1} \right) \right) - \alpha \sin \left(m_0 \pi + \mathcal{O}\left(e^{-\frac{1}{\rho}s_1} \right) \right) \right) + \mathcal{O}\left(r^{\gamma/2\alpha_0} + e^{-\frac{1}{\rho}s_1} \right) \\
	&= -(-1)^{m_0} s_0 \beta_0 e^{\alpha_0 \phi/\beta_0} e^{-\frac{1}{\rho}m_0} + \mathcal{O}\left(r^{\gamma/2\alpha_0} + e^{-\frac{1}{\rho}s_1} \right) \\
	&= \tilde{a}^*(m_0) + \mathcal{O}\left(r^{\gamma/2\alpha_0} + e^{-\frac{1}{\rho}s_1} \right) \\
	\end{align*}
	where $\tilde{a}^*(m_0)$ is defined by
	\[
	\tilde{a}^*(m_0) = (-1)^{m_0 + 1} s_0 \beta_0 e^{\alpha_0 \phi/\beta_0} e^{-\frac{1}{\rho}m_0}
	\]
	and is negative if $m_0 = 0$ and positive if $m_0 = 1$. Equation \cref{tildeas1limit} follows by taking $r = 0$.
\end{enumerate}

\subsection{Asymmetric periodic 2-pulse}\label{sec:assymkernel}

Let $r_*$ be as in Theorem \ref{2pulsebifurcation}. Choose $m_0 \in \{ 0, 1\}$ and any $s_1 > p^*$. For $r \leq r_*$, let $Q_2(x; r = Q_2(x; m_0, s_1, r)$ be the corresponding family of asymmetric periodic 2-pulses. Choose any positive integer $N$, and let $\delta(N,r)$ be as in Theorem \ref{blockmatrixtheorem}. Let $\tilde{a}(r) = \tilde{a}(r; m_0, s_1)$, be as in Lemma \cref{lemma:chara}. By Lemma \cref{lemma:chara}, $\tilde{a}(0) \neq 0$. Let
\begin{equation}\label{2ptildemustar}
\tilde{\mu}_*(r) = \sqrt{-\frac{2\tilde{a}(r)}{M c^2}}.
\end{equation}
Since $\tilde{a}(0) \neq 0$, $\tilde{\mu}_*(0) \neq 0$
We expect to find a pair of interaction eigenvalues near $\mu = \pm \tilde{\mu}_*(r) r^{1/2}$. 

To make sure that the essential spectrum eigenvalues are ``out of the way'' of the interaction eigenvalues, we will choose $r_0 \leq r_*$ sufficiently small so that for all $r \leq r_0$,
\begin{equation}\label{2pnobubble}
\left|\tilde{\mu}_*(r) \right| r^{1/2}  \leq \frac{\pi}{2 X(r)},
\end{equation}
where $X(r)$ is the domain size. Precisely, since
\[
X(r) \leq \frac{1}{2 \alpha_0} 2 |\log r| + \frac{1}{2\beta_0}\left( s_1 + 2 \pi \right) + 2 L_0, 
\]
choose $r_0 \leq r_*$ such that
\begin{align}\label{2pnobubblecond}
\left|\tilde{\mu}_*(r) \right| r^{1/2} \left( \frac{1}{\alpha_0} |\log r| + \frac{1}{2\beta_0}\left( s_1 + 2 \pi \right) + 2 L_0 \right) \leq \frac{\pi}{2}
\end{align}
for all $r \leq r_0$. Since $r^{1/2}|\log r| \rightarrow 0$ as $r \rightarrow 0$ and $\tilde{\mu}_*(r)$ is bounded, we can always find such a $r_0$. This guarantees that \cref{2pnobubble} is satisfied. For the rest of this section, we will take $r \leq r_0$.

\subsubsection{Essential spectrum eigenvalues}

In this section, we find the essential spectrum eigenvalues. Let $N_1 \leq N$ be the largest positive integer such that $N_1/|\log r| < \delta(N,r)$. Let $m$ be a nonzero integer with $|m| \leq N_1$, and let
\begin{equation}\label{defmum}
\mu_m = \frac{m \pi i}{X} + \frac{h}{X}
\end{equation}
Expanding the $\sinh(\mu_m X)$ and $\cosh(\mu_m X)$ terms in \cref{2detBeqmu} in a Taylor series about $m \pi/X$,
\begin{align*}
\sinh(\mu_m X) &= (-1)^m h + \mathcal{O}(h^3) \\
\cosh(\mu_m X) &= (-1)^m + \mathcal{O}(h^2)
\end{align*}
By \cref{lemma:expnubound}, the $\sinh(\mu_m X_j)$ terms in \cref{2detBeqmu} are bounded by a constant. The terms involving $q(X_j)$ are $\mathcal{O}(r^{1/2})$. By \cref{2pnobubble}, $|\mu_m| \geq C r^{1/2}$ for $r \leq r_0$. Substituting these into \cref{2detBeqmu}, we have 
\begin{equation}\label{Bess1}
\begin{aligned}
\det &S(h, r) = \left(-2 c^2 M  \mu_m^2 \left( 2a(r) + c^2 M \mu_m \right)^2 + \mathcal{O}( |\mu_m|)^5 \right) \left( (-1)^m h + \mathcal{O}(h^3) \right) \\
&+ \mu_m^4 \left( (-1)^m + \mathcal{O}(h^2)\right)\mathcal{O}(r^{1/2}) + \mathcal{O}\left( |\mu_m| \right)^6 \\
&= \left(-2 c^2 M  \mu_m^2 \left( 2a + c^2 M \mu_m \right)^2 + \mathcal{O}( |\mu_m|)^5 \right) \left( (-1)^m h + \mathcal{O}(h^3) \right) + \mathcal{O}\left( |\mu_m|^5 \right),
\end{aligned}
\end{equation}

We want to solve $\det S(\mu, r) = 0$. Dividing by $(-1)^m$, $M c^2$, and the constants out front and using \cref{2ptildemustar}, we want to solve 
\begin{equation}\label{Bess2}
\begin{aligned}
\left(\mu_m^2 (\mu_m - \tilde{\mu}_*(r)r^{1/2})(\mu_m + \tilde{\mu}_*(r)r^{1/2}) + \mathcal{O}( |\mu_m|)^5 \right) \left( h + \mathcal{O}(h^3) \right) + \mathcal{O}\left( |\mu_m|^5 \right) = 0
\end{aligned}
\end{equation}
Since $\mu_m \neq 0$, divide by $\mu_m^2$ to get
\begin{equation}\label{Bess3}
\begin{aligned}
\left((\mu_m - \mu_*(r))(\mu_m + \mu_*(r)) + \mathcal{O}( |\mu_m|)^3 \right) \left( h + \mathcal{O}(h^3) \right) + \mathcal{O}\left( |\mu_m|^3 \right) = 0
\end{aligned}
\end{equation}
Finally, multiply by $X^2$ and substitute \cref{defmum} for $\mu_m$. Since $|m| \leq N$, $|\mu_m| \leq C N/X$, so we can simplify the remainder terms to get the equation $G_m(h, r) = 0$, where
\begin{equation}\label{Bess4}
\begin{aligned}
G_m(h, r) &= (m \pi i + h - \tilde{\mu}_*(r) r^{1/2} X(m \pi i + h + \tilde{\mu}_*(r) r^{1/2} X \left( h + \mathcal{O}(h^3) \right) + \mathcal{O}\left( \frac{1}{X} \right)
\end{aligned}
\end{equation}
Since $X = X(r) = \mathcal{O}(1/|\log r|)$, this becomes
\begin{equation}\label{Bess4}
\begin{aligned}
G_m(h, r) &= \left( m \pi i + h + \mathcal{O}(r^{1/2}|\log r|) \right)\left(m \pi i + h + + \mathcal{O}(r^{1/2}|\log r|) \right) \left( h + \mathcal{O}(h^3) \right) + \mathcal{O}\left( \frac{1}{|\log r|} \right)
\end{aligned}
\end{equation}
Since $r^{1/2}|\log r| \rightarrow 0$ as $r \rightarrow 0$,
\begin{align*}
G_m(0,0) &= 0 \\
\partial_h G_m(0,0) &= -m^2 \pi^2
\end{align*}
For all $m$ between 1 and $N_1$, $|\partial_h G_m(0,0)| \geq \pi^2 > 0$. Thus we can use the implicit function theorem to solve for $h$ in terms of $r$ near $h = 0$. Specifically, there exists $r_2^m \leq r_0$ and a unique smooth function $h_m(r)$ with $h_m(0) = 0$ such that for all $r \leq r_2^m$, $G_m(h_m(r),r) = 0$. Expanding $h_m$ in a Taylor series about $r = 0$,
\[
h_m(r) = \mathcal{O}\left( \frac{1}{|\log r|} \right)
\]
Let $r_2 = \min\{ r_2^1, \dots, r_2^{N_1} \}$. Substituting $h_m(r)$ into \cref{defmum}, for all nonzero integers $m = 1, \dots, N_1$, there are essential spectrum eigenvalues located at
\[
\mu_m(r) = \frac{m \pi i}{X} + \mathcal{O}\left( \frac{1}{|\log r|^2} \right)
\]
Changing variables back to $\lambda$, these are located at
\[
\lambda_m(r) = c \frac{m \pi i}{X}\left[1 + \mathcal{O}\left( \frac{1}{|\log r|^2} \right) \right] +\mathcal{O}\left( \frac{1}{|\log r|^2} \right)
\]
Since $X = \mathcal{O}(|\log r|)$, this simplifies to
\[
\lambda_m(r) = c \frac{m \pi i}{X} +\mathcal{O}\left( \frac{1}{|\log r|^2} \right)
\]

By Hamiltonian symmetry, eigenvalues must come in quartets. Since there is nothing else above the real axis with similar magnitude, $\lambda_m(r)$ is pure imaginary and there is another essential spectrum eigenvalue $\lambda_{-m}(r) = -\lambda_m(r)$. We conclude that the nonzero essential spectrum eigenvalues are given by $\lambda = \pm \lambda_m^{\text{ess}}(r)$ for $m = 1, \dots, N_1$, where
\[
\lambda_m^{\text{ess}}(r) = c \frac{m \pi i}{X} + +\mathcal{O}\left( \frac{1}{|\log r|^2} \right),
\]

\subsubsection{Eigenvalues at 0}

In this section, we show that there are exactly three eigenvalues at 0, and that there are no other eigenvalues within a certain distance of 0. We know that there is an eigenvalue at 0 with at least algebraic multiplicity 2, since the corresponding eigenfunctions are $\partial_x Q_2(x)$ and $\partial_c Q_2(x)$. Since $\tilde{a}(0) \neq 0$, the eigenvalues of $\tilde{A}(0)$, thus by \cref{lemma:centereigenfn}, there is a third eigenfunction $V_n^c(x)$ with eigenvalue 0. We will use Rouch\'{e}'s theorem to show that there are no more eigenfunctions in a small circle around 0. Let
\[
\tilde{\xi} = \frac{1}{2}|\tilde{\mu}_*(0)|
\]
and take $\mu$ on the circle $|\mu| = r^{1/2} \tilde{\xi}$. By \cref{2pnobubble}, $|\mu X| \leq 1/2$, thus we can expand all of the $\sinh$ and $\cosh$ terms in \cref{2detBeqmu} in a Taylor series about 0 to get
\begin{equation*}
\begin{aligned}
\sinh(\mu X) &= \mu X + \mathcal{O}(\mu X^2) \\
\cosh(\mu X) &= \mathcal{O}(1) \\
\sinh(\mu X_j) &= \mathcal{O}(\mu X_j) = \mathcal{O}(\mu X).
\end{aligned}
\end{equation*}
Using these estimates, the estimate $q(X_j) = \mathcal{O}(r^{1/2})$, and \cref{2ptildemustar}, equation \cref{2detBeqmu} simplifies to
\begin{equation}\label{2detBeqmu3}
\begin{aligned}
\det S(\mu, r) &= -2 \mu^2 M (\mu - \tilde{\mu}_*(r)r^{1/2}) (\mu + \tilde{\mu}_*(r)r^{1/2}) ( \mu X + \mathcal{O}(\mu X)^3) + \mathcal{O}\left( r^{5/2}|\mu|X + |\mu|^5(r^{1/2} + |\mu|)X \right),
\end{aligned}
\end{equation}
Rescale the problem by taking $\mu = r^{1/2}\tilde{\mu}$. Then equation \cref{2detBeqmu3} becomes
\begin{equation}\label{2detBeqmu4}
\begin{aligned}
\det S(\tilde{\mu}, r) &= -2 \tilde{\mu}^2 M r^{5/2} (\tilde{\mu} - \tilde{\mu}_*(r)) (\tilde{\mu} + \tilde{\mu}_*(r))( \tilde{\mu} X + \mathcal{O}(\tilde{\mu}^3 r X^3) + \mathcal{O}\left( r^3 X \right)
\end{aligned}
\end{equation}
By \cref{lemma:ajparam}, $\tilde{\mu}_*(r) = \tilde{\mu}_*(0) + \mathcal{O}(r^{\gamma/4 \alpha)}$, thus \cref{2detBeqmu4} becomes
\begin{equation}\label{2detBeqmu5}
\begin{aligned}
\det S(\tilde{\mu}, r) &= -2 \tilde{\mu}^2 M r^{5/2} (\tilde{\mu} - \tilde{\mu}_*(0)) (\tilde{\mu} + \tilde{\mu}_*(0))( \tilde{\mu} X + \mathcal{O}(\tilde{\mu}^3 r X^3) + \mathcal{O}\left( r^{5/2} \left( r^{1/2} + r^{\gamma/4 \alpha)} \right) X \right)
\end{aligned}
\end{equation}
We are looking for zeros of \cref{2detBeqmu5}. Dividing by $r^{5/2}X$ and the constants out front and using the estimate $X = \mathcal{O}(|\log r|)$, we wish to solve $G(\tilde{\mu}, r) = 0$, where
\begin{equation}\label{2peigG}
G(\tilde{\mu}, r) = -2 \tilde{\mu}^3 M (\tilde{\mu} - \tilde{\mu}_*(0)) (\tilde{\mu} + \tilde{\mu}_*(0)) + \mathcal{O}\left( r |\log r|^2  + r^{1/2} + r^{\gamma/4 \alpha} \right)
\end{equation}
In particular, we wish to show that $G(\tilde{\mu}, r)$ has exactly three zeros inside the circle of radius $\tilde{\xi}$. Let $G(\tilde{\mu}, r) = G_1(\tilde{\mu}) + G_2(\tilde{\mu}, r)$, where 
\[
G_1(\tilde{\mu}) = -2 \tilde{\mu}^3 M (\tilde{\mu} - \tilde{\mu}_*(0)) (\tilde{\mu} + \tilde{\mu}_*(0)),
\]
which is independent of $r$, and
\[
G_2(\tilde{\mu}, r) = \mathcal{O}\left( r |\log r|^2 + \frac{1}{|\log r|} \right).
\]
On the circle $|\tilde{\mu}| = \tilde{\xi}$,
\[
|G_1(\tilde{\mu})| \geq \frac{|M|}{16}|\tilde{\mu}_*(0)|^5
\]
Since $|G_2(\tilde{\mu}, r)| \rightarrow 0$ as $r \rightarrow 0$ and $\tilde{\mu}_*(0) \neq 0$, there exists $r_3 \leq r_0$ such that for $r \leq r_3$, $|G_2(\tilde{\mu}, r)| < |G_1(\tilde{\mu})|$ on the circle $|\tilde{\mu}| = \tilde{\xi}$. By Rouch\'{e}'s theorem $G(\tilde{\mu}, r)$ and $G_1(\tilde{\mu})$ have the same number of zeros (counted with multiplicity) inside the circle of radius $\tilde{\xi}$. By the choice of $\tilde{\xi}$, $G_1(\tilde{\mu})$ has exactly 3 zeros inside the circle, thus $G(\tilde{\mu}, r)$ does as well. Undoing the scaling and changing variables back to $\lambda$, for $r \leq r_2$, there are exactly three eigenvalues inside a circle of radius $\sqrt{|2\tilde{a}(0)/M|}r^{1/2}$. These must correspond to the three kernel eigenvalues discussed above.

\subsubsection{Interaction eigenvalues}\label{sec:assyminteigs}

In this section, we will find the interaction eigenvalues. Using the same setup and scaling as in the previous section, simplify the remainder terms in \cref{2detBeqmu4} to get
\begin{equation}\label{2detint2}
\begin{aligned}
\det S(\tilde{\mu}, r) &= -2 M r^{5/2} \tilde{\mu}^3 (\tilde{\mu} - \tilde{\mu}_*(r)) (\tilde{\mu} + \tilde{\mu}_*(r)) X + \mathcal{O}\left( r^{7/2} X^3 + r^3 X \right)
\end{aligned}
\end{equation}
Dividing by $r^{5/2}X$ and the constants out front and using the estimate $X = \mathcal{O}(|\log r|)$, we wish to solve $G(\tilde{\mu}, r) = 0$, where
\begin{equation}\label{2detintG1}
\begin{aligned}
G(\tilde{\mu}, r) &= -2 M (\tilde{\mu} - \tilde{\mu}_*(r)) (\tilde{\mu} + \tilde{\mu}_*(r))\tilde{\mu}^3 + \mathcal{O}\left( r |\log r|^2 + r^{1/2} \right)
\end{aligned}
\end{equation}
We expect to find the interaction eigenvalues near $\tilde{\mu} = \pm \tilde{\mu}_*(r)$. By symmetry, we only need to consider one of these. Let
\[
\tilde{\mu} = \tilde{\mu}_*(r) + \tilde{h}
\]
Then \cref{2detintG1} becomes
\begin{equation}\label{2detintG2}
\begin{aligned}
G(\tilde{h},r) = \tilde{h} ( \tilde{h} + \tilde{\mu}_*(r))^3 (\tilde{h} + 2 \tilde{\mu}_*(r)) + \mathcal{O}\left( r |\log r|^2 + r^{1/2} \right)
\end{aligned}
\end{equation}
Since $r |\log r|^2  \rightarrow 0$ as $r \rightarrow 0$,
\begin{align*}
G(0, 0) &= 0 \\
\partial_{\tilde{h}} G(0, 0) &= 2 \tilde{\mu}_*(0)^4
\end{align*}
Since $\tilde{\mu}_*(0) \neq 0$, we can use the implicit function theorem to solve $G(\tilde{h},r) = 0$ for $\tilde{h}$ in terms of $r$ near $\tilde{h} = 0$. Specifically, there exists $r_4 \leq r_0$ and a unique smooth function $\tilde{h}(r)$ with $\tilde{h}(0) = 0$ such that for all $r \leq r_4$, $G(\tilde{h}, r) = 0$. Expanding $\tilde{h}(r)$ in a Taylor series about $r = 0$, for all $r \leq r_4$,
\[
\tilde{h}(r) = \mathcal{O}\left( r |\log r|^2 + r^{1/2} \right) = \mathcal{O}\left( r^{1/2} \right)
\]
Undoing the scaling, there is an interaction eigenvalue located at
\begin{align*}
\mu(r) = \sqrt{-\frac{\tilde{a}(r)}{M c^2}}r^{1/2} + \mathcal{O}\left( r \right)
\end{align*}
Changing variables back to $\lambda$, this is located at
\begin{align*}
\lambda(r) = \sqrt{-\frac{2 \tilde{a}(r)}{M}}r^{1/2} + \mathcal{O}\left( r \right)
\end{align*}
By Hamiltonian symmetry there is also an eigenvalue at $-\lambda(r)$. Since Hamiltonian symmetry  dicatates that eigenvalues must come in quartets, and there only two eigenvalues of this magnitude, we conclude that for $r \leq r_4$, there is a pair of interaction eigenvalues given by
\[
\lambda^{\text{int}}(r) = \pm \left( \sqrt{ -\frac{2 \tilde{a}(r)}{M} }r^{1/2} + \mathcal{O}\left( r \right) \right)
\]
which are real if $\tilde{a}(0) < 0$ and purely imaginary if $\tilde{a}(0) > 0$. By \cref{lemma:chara}, the sign of $\tilde{a}(0)$ depends only on $m_0$.

\subsubsection{Eigenvalue count}

Finally, we prove that we have accounted for all of the eigenvalues near 0. Let $N_1$ be as above, and let
\[
\xi = \left( N_1 + \frac{1}{2} \right)\pi
\]
Take $\mu$ on the circle $|\mu| = \xi/X$. We will use Rouch\'{e}'s theorem to show that there are no zeros of $\det S(\mu)$ inside the circle of radius $\xi$ besides the ones we have already found.

Substituting \cref{2ptildemustar} into \cref{2detBeqmu} and dividing by the constants out front, we are looking to solve
\begin{equation}\label{Gcount1}
\begin{aligned}
0 &= \mu^2 (\mu - \tilde{\mu}_*(r)r^{1/2}) (\mu + \tilde{\mu}_*(r)r^{1/2}) \sinh(\mu X) \\
&\qquad +2 M^c c^2 \mu^4 ( q(X_0) \sinh(\mu X_0) + q(X_1) \sinh(\mu X_1) ) \cosh(\mu X) + R_1\sinh(\mu X) + R_2
\end{aligned}
\end{equation}
This time, we rescale the problem by taking $\mu = h/X$. Making this substitution, equation \cref{Gcount1} becomes
\begin{equation}\label{Gcount2}
\begin{aligned}
0 &= \left(\frac{h}{X}\right)^2 \left( \frac{h}{X} - \tilde{\mu}_*(r)r^{1/2}\right)\left(\frac{h}{X} + \tilde{\mu}_*(r)r^{1/2}\right) \sinh(h) \\
&\qquad +2 M^c c^2 \left(\frac{h}{X}\right)^4 \left( q(X_0) \sinh\left(h \frac{X_0}{X} \right) + q(X_1) \sinh\left(h \frac{X_1}{X} \right) \right) \cosh(h) + R_1\sinh(h) + R_2
\end{aligned}
\end{equation}
Multiplying by $X^4$, we wish to find the zeros of $G(h, r)$, which is given by
\begin{equation}\label{Gcount3}
\begin{aligned}
G(h,r) &= h^2 \left( h - \tilde{\mu}_*(r)r^{1/2}X\right)\left(h + \tilde{\mu}_*(r)r^{1/2}X\right) \sinh(h) \\
&\qquad +2 M^c c^2 h^4 \left( q(X_0) \sinh\left(h \frac{X_0}{X} \right) + q(X_1) \sinh\left(h \frac{X_1}{X} \right) \right) \cosh(h) + (R_1\sinh(h) + R_2)X^4
\end{aligned}
\end{equation}
Write $G(h,r) = G_1(h) + G_2(h,r)$, where
\[
G_1(h, r) = h^2 \left( h - \tilde{\mu}_*(r)r^{1/2}X\right)\left(h + \tilde{\mu}_*(r)r^{1/2}X\right) \sinh(h)
\]
and
\[
G_2(h,r) = 2 M^c c^2 h^4 \left( q(X_0) \sinh\left(h \frac{X_0}{X} \right) + q(X_1) \sinh\left(h \frac{X_1}{X} \right) \right) \cosh(h) + (R_1\sinh(h) + R_2)X^4
\]
On a circle of radius $|h| = \xi$, $|\sinh h| \geq 1$. By \cref{2pnobubble}, $|h - \tilde{\mu}_*(r)r^{1/2}X|\geq \pi$. Thus on the circle $|h| = \xi$
\[
|G_1(h,r)| \geq \pi^2 \xi^2,
\]
which is independent of $r$. For $G_2(h,r)$, using $q(X_j) = \mathcal{O}(r^{1/2})$, $X = \mathcal{O}(|\log r|)$, the remainder bounds for $R_1$ and $R_2$ from \cref{muRbounds}, and the scaling $\mu = h/X$, on the circle of radius $|h| = \xi$, we have
\[
|G_2(h,r)| = \mathcal{O}\left( r^{1/2} + \frac{1}{|\log r|} \right)
\]
Since $|G_2(h,r)| \rightarrow 0$ as $r \rightarrow 0$, there exists $r_5 \leq r_0$ such that for all $r \leq r_5$, $|G_2(h,r)| < |G_1(h,r)|$ on the circle $|h| = \xi$. Thus by Rouch\'{e}'s theorem, $G(h,r)$ and $G_1(h,r)$ have the same number of zeros (counted with multiplicity) inside the circle $|h| = \xi$. By our choice of $\xi$, $G_1(h,r)$, thus $G(h,r)$, has $2 N_1 + 5$ zeros inside this circle, which correspond exactly to the 3 eigenvalues at 0, the two interaction eigenvalues, and the first $2 N_1$ nonzero essential spectrum eigenvalues. Since we have accounted for all the zeros of $G(h,r)$, we conclude that there can be no more zeros of $G(h,r)$ inside the circle $|h| = \xi$. Undoing the scaling and changing variables back to $\lambda$, we have shown that there are no other eigenvalues inside a circle with radius slightly larger than $|\lambda_{N_1}^{\text{ess}}(r)|$.

\subsubsection{Proof of \cref{theorem:2peigsassym}}

The proof of \cref{theorem:2peigsassym} follows from combining the results of the previous four subsections and taking $r_1 = \min\{ r_2, r_3, r_4, r_5 \}$.

\subsection{Proof of \cref{2peigsym}}

To prove \cref{2peigsym}, we will first show that if we are sufficiently close to the bifurcation point, there are five eigenvalues in a small ball around the origin. We will then show that when were are at the bifurcation point, those eigenvalues are actually all at the origin.

\subsubsection{Count of eigenvalues near 0}

From the proof of \cref{lemma:chara}, for $m_1 \in \{0, 1\}$ and $s_0 \in [0, \pi)$,
\begin{align*}
\tilde{a}(r) = \tilde{a}(r, s_0) &= 2 (-1)^{m_0} s_0 \alpha_0 e^{\alpha_0 \phi/\beta_0} e^{-\frac{1}{\rho}(m_0 \pi + s_0) } \left( \rho \cos s_0 - \sin s_0 \right) + \mathcal{O}\left(r^{\gamma/2\alpha_0} \right)\\
\end{align*}
and $\tilde{a}(0) = 0$ if and only if $s_0 = p^*$. Since $\tilde{a}(r)$ is smooth in $s_0$ and $r$, there exist $\eta > 0$ and $r_2 \leq r_*$ such that for all $r \leq r_r$ and $s_0 \in (p^* - \eta, p^* + \eta)$, 
\begin{equation}
\sqrt{ \left| \frac{ 2 \tilde{a}(r, s_0) }{M} \right| }  r^{1/2} \leq \frac{\pi}{2 X(r)}
\end{equation}
Since $X(r) = \mathcal{O}(|\log r|)$, this is always possible. Let $\xi = c \pi / 2 X(r)$, and follow the same procedure as in \cref{sec:assymkernel}. Using Rouch\'{e}'s theorem, it follows that for $r \leq r_1$ and $s_0 \in (p^* - \eta, p^* + \eta)$, there are exactly 5 zeros of $S(\lambda)$ inside the circle of radius $\xi$. We know that two of the zeros must be at $\lambda = 0$ since they correspond to the kernel eigenfunctions $\partial_x Q_2(x)$ and $\partial_c Q_2(x)$. This leaves three eigenvalues unaccounted for.
By Hamiltonian symmetry, since eigenvalues must come in quartets, there must be a third eigenvalue at 0.

\subsubsection{Eigenvalues at bifurcation point}
Using the scaling $\mu = r^{1/2} \tilde{\mu}$ and following the same steps as in \cref{sec:assyminteigs}, we wish to solve
\begin{equation*}
\begin{aligned}
0 &= -2 M (\tilde{\mu} - \tilde{\mu}_*(r)) (\tilde{\mu} + \tilde{\mu}_*(r))\tilde{\mu}^3 + \mathcal{O}\left( r |\log r|^2 + r^{1/2} \right)
\end{aligned}
\end{equation*}
Since we know that there will always be two eigenvalues at 0 corresponding to $\partial_x Q_n(x)$ and $\partial_c Q_n(x)$ and we proved the existence of a third eigenvalue at 0 in the previous section, we can divide by $\tilde{\mu}^3$ and the constants out front to get 
\begin{align*}
0 &= (\tilde{\mu} - \tilde{\mu}_*(r)) (\tilde{\mu} + \tilde{\mu}_*(r)) + \mathcal{O}\left( r |\log r|^2 + r^{1/2} \right) \\
&= \tilde{\mu}^2 - \tilde{\mu}_*(r)^2 + \mathcal{O}\left( r |\log r|^2 + r^{1/2} \right) \\
&= \tilde{\mu}^2 + \frac{2\tilde{a}(r)}{M c^2} + \mathcal{O}\left( r |\log r|^2 + r^{1/2} \right) \\
\end{align*}
Using the expression for $\tilde{a}(r)$ from \cref{lemma:chara}, this become
\begin{equation}\label{Gsymm1}
0 = \tilde{\mu}^2 + \frac{2\tilde{a}(0, s_0)}{M c^2} + \mathcal{O}\left( r |\log r|^2 + r^{1/2} + r^{\gamma/2\alpha_0} \right)
\end{equation}
Since we are interested in what happens near $s_0 = p^*$, let $s_0 = p^* + h$. Substituting this for $s_0$ into $\tilde{a}(0, s_0)$ and expanding in a Taylor series, we have
\begin{align*}
\tilde{a}(0, p^* + h) &= 2 (-1)^{m_0} s_0 \alpha_0 e^{\alpha_0 \phi/\beta_0} e^{-\frac{1}{\rho}(m_0 \pi + p^*) } e^{-\frac{1}{\rho}h }\left( \rho \cos(p^* + h) - \sin(p^* + h) \right) \\
&= 2 (-1)^{m_0} s_0 \alpha_0 e^{\alpha_0 \phi/\beta_0} e^{-\frac{1}{\rho}(m_0 \pi + p^*) }\left( 1 - \frac{1}{\rho}h + \mathcal{O}(|h|^2) \right) \left( -\frac{2}{\sqrt{1 + \rho^2} }h + \mathcal{O}(|h|^2) \right) \\
&= -\frac{4}{\sqrt{1 + \rho^2} }(-1)^{m_0} s_0 \alpha_0 e^{\alpha_0 \phi/\beta_0} e^{-\frac{1}{\rho}(m_0 \pi + p^*) }h + \mathcal{O}(|h|^2)
\end{align*}
Since the coefficient of $h$ is a constant, we can write this as
\[
\tilde{a}(p^* + h, 0) = C_1 h + \mathcal{O}(|h|^2) 
\]
where $C_1 \neq 0$. Substituting this into \cref{Gsymm1}, we wish to solve the equation the equation $G(\tilde{\mu}, h, r) = 0$, where
\begin{equation}\label{Gsymm1}
G(\tilde{\mu}, h, r) = \tilde{\mu}^2 + \frac{2C_1}{M c^2}h + \mathcal{O}\left( |h|^2 + r |\log r|^2 + r^{1/2} + r^{\gamma/2\alpha_0} \right).
\end{equation}
When $r = 0$ and $h = 0$, $G(\tilde{\mu}, h, r)$ has a double root at $\tilde{\mu} = 0$. We wish to show that this double root at $\tilde{\mu} = 0$ persists for small $r$.

For $\tilde{\mu}$ to be a double root, it must solve both $G(\tilde{\mu}, h, r) = 0$ and $\partial_{\tilde{\mu}} G(\tilde{\mu}, h, r) = 0$. Define $K(\tilde{\mu}, h, r)$ by
\begin{equation}
K(\tilde{\mu}, h, r) = 
\begin{pmatrix}G(\tilde{\mu}, h, r) \\ \partial_{\tilde{\mu}}G(\tilde{\mu}, h, r) \end{pmatrix} 
= \begin{pmatrix}
\tilde{\mu}^2 + \frac{2C_1}{M c^2}h \\
2 \tilde{\mu}
\end{pmatrix}
+ \mathcal{O}\left( |h|^2 + r |\log r|^2 + r^{1/2} + r^{\gamma/2\alpha_0} \right)
\end{equation}
When $r = 0$, $K(\tilde{\mu}, 0, 0) = 0$. The Jacobian with respect to $(\tilde{\mu}, h)$ is
\[
D_{(\tilde{\mu}, h)}K(\tilde{\mu}, h, r) = 
\begin{pmatrix}
2 \tilde{\mu} & \frac{2C_1}{M c^2} \\
2 & 0
\end{pmatrix}
+ \mathcal{O}\left( |h| + r |\log r|^2 + r^{1/2} + r^{\gamma/2\alpha_0} \right)
\]
At $(\tilde{\mu}, h) = (0,0)$, 
\[
D_{(\tilde{\mu}, h)}K(\tilde{\mu}, h, r)\Big|_{(0,0)} = 
\begin{pmatrix}
0 & \frac{2C_1}{M c^2} \\
2 & 0
\end{pmatrix},
\]
which is nonsingular since all the constants in the upper right block are nonzero. Using the implicit function theorem, we can solve for $(\tilde{\mu}, h)$ in terms of $r$ for sufficiently small $r$. pecifically, there exists $r_1 \leq r_2$ and a unique smooth function $(\tilde{\mu}, h)(r)$ with $(\tilde{\mu}, h)(0) = (0, 0)$ such that for all $r \leq r_1$, $K(\tilde{\mu}(r), h(r), r) = 0$. For $r \leq r_1$ and $h = h(r)$, $\tilde{\mu}(r)$ is a double root of $G(\tilde{\mu}, h, r)$. By Hamiltonian symmetry, we must have $\tilde{\mu}(r) = 0$. Expanding $h(r)$ in a Taylor series about $r$, 
\begin{equation*}
h(r) = \mathcal{O}\left( r |\log r|^2 + r^{1/2} + r^{\gamma/2\alpha_0} \right)
\end{equation*}

We conclude that when $s_0 = p^* + h(r)$, there are two more eigenvalues at 0. From the previous section, there are 5 zeros of $S(\lambda)$ inside a the circle of radius $\xi$. Thus when $s_0 = p^* + h(r)$, there is an eigenvalue at 0 with algebraic mutiplicty 5. Since the pitchfork bifurcation in the family of periodic 2-pulses occurs near $p^*$ at $p^*(r)$, it follows from standard PDE bifurcation theory that this quintuiple zero must occur at the pitchfork bifurcation point, i.e. $p^* + h(r) = p^*(r)$.

\section{Proof of Theorem \ref{theorem:kreinbubbles}}

\subsection{Proof of \cref{lemma:s1choice}}

We wish to solve
\begin{equation}\label{solvefors1}
\frac{c \pi}{X(r, m_0, s_1)} = \lambda^* r^{1/2}
\end{equation}
for $s_1$ in terms of $r$. We note that $\lambda_*$ is independent of both $r$ and $s_1$. Substituting \cref{X2pdomain} from \cref{2pulsebifurcation} and rearranging, equation \cref{solvefors1} becomes
\begin{align*}
\lambda_* r^{1/2} \left( \frac{1}{\alpha_0} |\log r| + \frac{1}{2\beta_0} (m_0 \pi + s_1 ) + 2 L_0 + \mathcal{O}\left(e^{-\frac{1}{\rho}s_1}\right) \right) = c \pi
\end{align*}
Rearranging this, we obtain the equation $g(s_1) = h(r)$, where
\begin{align*}
g(s_1) &= s_1 + \mathcal{O}\left(e^{-\frac{1}{\rho}s_1}\right) \\
h(r) &= \frac{2 \beta_0 c \pi}{\lambda_*} r^{-1/2} - \frac{2 \beta_0}{\alpha_0} |\log r| - m_0 \pi - 4 \beta_0 L_0
\end{align*}
Differentiating $g(s_1)$ with respect to $s_1$, 
\[
g'(s_1) = 1 + \mathcal{O}\left(e^{-\frac{1}{\rho}s_1}\right)
\]
Thus there exists $R > 0$ such that for all $s_1 \geq R$, $g'(s_1) > 1/2$. For $s_1 \geq R$, $g(s_1)$ is strictly increasing, thus invertible. Since $h(r) \rightarrow \infty$ as $r \rightarrow 0$, there exists $r_1 \leq r_*$ such that for all $r \leq r_1$, we can solve for $s_1$ to get
\[
s_1^*(r) = g^{-1}(h(r))
\]
This proves \cref{lemma:s1choice}.

We are interested in what happens as $\frac{c \pi}{X(r, m_0, s_1)}$ passes through $\lambda_* r^{1/2}$ as $s_1$ is varied. To do this, following the same argument as above, there exists $r_2 \leq r_1$ such that for all $r \leq r_2$, there exists a function $s_1: \mathcal{R} \times [1/2, 3/2] \rightarrow \R$ such that for all $t \in [1/2, 3/2]$,
\begin{equation}\label{solvefors1range}
\frac{c \pi}{X(r, m_0, s_1(r, t))} = t \left( \lambda^* r^{1/2} \right),
\end{equation}
$s_1(r, 1) = s_1^*(r)$, and $s_1(r, t)$ is strictly decreasing. To obtain $s_1(r, t)$, solve $g(s_1) = h(r, t)$, where $g(s_1)$ is the same as above, and
\begin{align*}
h(r, t) &= \frac{2 \beta_0 c \pi}{t \lambda_*} r^{-1/2} - \frac{2 \beta_0}{\alpha_0} |\log r| - m_0 \pi - 4 \beta_0 L_0
\end{align*}
Let $s_1^-(r) = s_1(r, 3/2)$ and $s_1^+(r) = s_1(r, 1/2)$, and let 
\[
I_s(r) = [s_1^-(r), s_1^+(r)]
\]

Finally, since $|e^{-\frac{1}{\rho}s_1(r, t)} \leq 1|$, substituting $s_1(r, t)$ into $g(s_1) = h(r, t)$ (and decreasing $r_2$ if necessary), there exists a constant $R_s$ such that for all $r \leq r_2$ and $t \in [1/2, 3/2]$, we have the lower bound
\begin{equation}\label{s1starLB}
s_1(r, t) \geq R_s r^{-1/2}
\end{equation}
From Lemma \ref{lemma:chara},
\[
\tilde{a}(0; m_0 , s_1) = \tilde{a}^*(m_0) + \mathcal{O}\left( e^{-\frac{1}{\rho}s_1} \right)
\]
Combining this with \cref{s1starLB}, we have
\begin{align}\label{tildeasub}
\tilde{a}(0; m_0 , s_1) = \tilde{a}^*(m_0) + \mathcal{O}\left( r^{1/2} \right)
\end{align}
for all $s_1 \in I_s(r)$. From this point forward we will take $s_1 \in I_s(r)$.

\subsection{Setup}

Since we will be changing variables by taking $\lambda = \nu^{-1}(\mu)$, let 
\[
\tilde{\mu}_* = \frac{1}{c} \tilde{\lambda_*},
\]
which depends on neither $r$ nor $s_1$. Let $\mu_* = \tilde{\mu}_* r^{1/2}$. We will solve $\det S(\mu) = 0$ for $\mu$ close to $\tilde{\mu}_* r^{1/2} i$, where $\det S(\mu)$ is given by \cref{2detBeqmu}.

Since we are looking for nonzero eigenvalues which are order $\mu = \mathcal{O}(r^{1/2})$, dividing \cref{2detBeqmu} by $\mu^2$ and the constants out front and simplifying, we obtain the equation
\begin{equation}\label{KreinB1}
\begin{aligned}
0 &= \left( (2a + c^2 \mu^2 M) + \mathcal{O}( r^{3/2} )\right) \sinh(\mu X) \\
&+ 2 c^2 \mu^2 M^c ( q(X_0) \sinh(\mu X_0) + q(X_1) \sinh(\mu X_1) ) \cosh(\mu X) + \mathcal{O}( r^2 ) 
\end{aligned}
\end{equation}

Since we are taking $s_1 \in I_s(r)$, it follows from \cref{tildeasub} and \cref{lemma:chara} that
\[
a = r \tilde{a}(r) = 
\]
\tilde{a}^*(m_0) + \mathcal{O}\left( r^{1/2} \right)







Dividing by $M c^2$ and using \cref{kreinmustar} and $a(r) = r \tilde{a}(r)$ from Lemma \ref{lemma:chara}, equation \cref{KreinB1} becomes
\begin{equation}\label{KreinB2}
\begin{aligned}
0 &= \left( (\mu - \mu_*(r) i)( \mu + \mu_*(r) i) +  \mathcal{O}( r^{3/2} )\right) \sinh(\mu X) \\
&+\frac{2 M^c}{M} \mu^2 ( q(X_0)\sinh(\mu X_0) + q(X_1) \sinh(\mu X_1) ) \cosh(\mu X) + \mathcal{O}( r^2 ) 
\end{aligned}
\end{equation}
The lengths $X$, $X_0$, and $X_1$ depend on $s_1$ and $r$, but we have suppressed that dependence for now for notational convenience. Let
\begin{equation}\label{defmu1}
\mu_1 = \frac{1}{X}
\end{equation}

\subsection{Parameterization}

We will characterize the system using two parameters: the difference $h$ between $\mu$ and $\mu_* i$ and the distance $s$ along the imaginary axis between $\mu_* i$ and $\mu_1 i$. Note that $s$ is real. We will only consider the upper half of the complex plane; what happens in the lower half is the same by Hamiltonian symmetry. Let 
\begin{align*}
h &= \mu - \mu_* i \\
s &= \mu_1 - \mu_*
\end{align*}
where $h \in \C$ and $s \in \R$. At the end, we will relate $s$ to the parameter $s_1$ and provide an appropriate dimensionless scaling. Our goal is derive a relationship between $s$ and $h$. In order to expand the $\sinh(\mu X)$ and $\cosh(\mu X)$ terms around $\mu_1$, we write $\mu$ as
\begin{align*}
\mu &= \mu - \mu_1 i + \mu_1 i \\
&= (\mu_* - \mu_1)i + h + \mu_1 i \\
&= \mu_1 i + s i + h
\end{align*}
Using these, we have the Taylor expansions
\begin{align*}
\sinh(\mu X) &= \sinh((\mu_1 + s i + h)X)
= -(s i + h)X + \mathcal{O}\left( (s i +h)^3 X^3 \right) \\
\cosh(\mu X) &= \cosh((\mu_1 + s i + h)X)
= -1 + \mathcal{O}\left( (s i + h)^2 X^2 \right) \\
\end{align*}
Substituting all of these into \cref{KreinB2} and dividing by $-1$, we obtain the equation
\begin{equation}\label{KreinB3}
\begin{aligned}
0 &= \left( h ( h + 2 \mu_* i) +  \mathcal{O}( r^{3/2} )\right) \left( (s i + h)X + \mathcal{O}\left( (si+h)^3 X^3 \right)  \right) \\
&+\frac{2 M^c}{M} ( h + \mu_* i)^2 ( q(X_0)\sinh(\mu X_0) + q(X_1) \sinh(\mu X_1) ) \left( 1 + \mathcal{O}\left( (s i +h)^2 X^2 \right) \right) + \mathcal{O}( r^2 ) 
\end{aligned}
\end{equation}

\subsection{First rescaling}

Next, we will rescale our system. In fact, we will rescale it twice. While we could take the final scaling from the outset, doing it this way is more intuitive and we will be able to see where the second scaling comes from. First, we scale out a factor of $r^{1/2}$. Let
\begin{align*}
h &= r^{1/2} \tilde{h} \\
s &= r^{1/2} \tilde{s} \\
\mu_*(r) &= r^{1/2} \tilde{\mu}_*(r) \\
q(X_0) &= r^{1/2} \tilde{q}
\end{align*}
We also want to scale $r^{1/2}$ out from $\sinh(\mu X_j)$ terms. Let 
\[
\frac{X_1}{X_0} = 1 + 2 \eta
\]
where $\eta > 1$. Then
\begin{align*}
e^{-\alpha X_1} &= e^{-(\alpha X_0)\frac{X_1}{X_0}}
= \left( e^{-(\alpha X_0)} \right)^{1 + 2 \eta} = r^{1/2}r^{\eta}
\end{align*}
from which it follows that $q(X_1) = \mathcal{O}(r^{1/2}r^{\eta}$. Expanding the $c\sinh(\mu X_j)$ terms in a Taylor series about 0, which we can do since $X_j < X$, we get
\begin{align*}
\sinh(\mu X_j) &= \mu X_j + \mathcal{O}(\mu X_j)^3 \\
&= r^{1/2}(\mu_*(r)i + \tilde{h})X_j + \mathcal{O}(r^{3/2} (\mu_*(r)i + \tilde{h})^3 X_j^3)
\end{align*}
from which it follows that
\[
q(X_0) \sinh(\mu X_0) + q(X_1) \sinh(\mu X_1)
= \tilde{q}(\tilde{h} + \tilde{\mu}_*(r)i )r X_0 +  \mathcal{O}(r^{1 + \eta}) + \mathcal{O}(r^2 (\tilde{h} + \tilde{\mu}_*(r))^3 X^3))
\]
Substituting all of these into \cref{KreinB3} and dividing by $r^{3/2}$, we obtain the rescaled equation
\begin{equation}\label{KreinB4}
\begin{aligned}
0 &= \left( \tilde{h} ( \tilde{h} + 2 \tilde{\mu}_*(r) i) +  \mathcal{O}( r^{1/2} )\right) \left( (\tilde{s}i + \tilde{h})X + \mathcal{O}\left( (\tilde{s}i+\tilde{h})^3 r X^3 \right)  \right) \\
&+\frac{2 M^c}{M} ( \tilde{h} + \tilde{\mu}_*(r) i)^2 \left( \tilde{q}(\tilde{h} + \tilde{\mu}_*(r)i )r^{1/2} X_0 + \mathcal{O}(r^{1/2 + \eta}) + \mathcal{O}(r^{3/2} (\tilde{h} + \tilde{\mu}_*(r))^3 X^3)) \right) \left( 1 + \mathcal{O}\left( (\tilde{s}i+\tilde{h})^2 r X^2 \right) \right) \\
&+ \mathcal{O}( r^{1/2} ) 
\end{aligned}
\end{equation}

\subsection{Second rescaling}

At this point, we are most of the way there, except for the $r^{1/2}X_0$ term on the second line. To figure out an appropriate scaling ansatz to eliminate this term, suppose that $\tilde{s}$ and $\tilde{h}$ are smaller than $\tilde{\mu}_*(r)$. Eliminating all terms involving $r$ other than the $r^{1/2}X_0$ term, we obtain an equation of the form
\begin{equation*}
\begin{aligned}
\tilde{h} (\tilde{s}i + \tilde{h}) \tilde{\mu}_*(r) i X
- C \tilde{h}^3 r^{1/2} X_0 (\tilde{\mu}_*(r))^3 = 0.
\end{aligned}
\end{equation*} 
Since this is quadratic in $\tilde{h}$, it suggests that $\tilde{h} = \mathcal{O}(r^{1/4}(X_0/X)^{1/2})$. Adopting this scaling for $\tilde{h}$ and $\tilde{s}$, let
\begin{align*}
\tilde{h} &= r^{1/4}\frac{X_0^{1/2}}{X^{1/2}} h_0 \\
\tilde{s} &= 2 r^{1/4}\frac{X_0^{1/2}}{X^{1/2}} s_0
\end{align*}
where the factor of $2$ is $\tilde{s}$ is for convenience later. Substituting these into \cref{KreinB4} and dividing by $r^{1/2}$, we obtain the equation
\begin{equation}\label{KreinB5}
\begin{aligned}
&\left( \frac{X_0^{1/2}}{X^{1/2}} h_0 \left( r^{1/4}\frac{X_0^{1/2}}{X^{1/2}} h_0 + 2 \tilde{\mu}_*(r) i\right) + \mathcal{O}( r^{1/4} ) \right) 
\left( \frac{X_0^{1/2}}{X^{1/2}}(2 s_0 i + h_0) X + \mathcal{O}\left( \left( 2 s_0 i + h_0 \right)^3 r^{3/2} X_0^{3/2} X^{3/2} \right) \right) \\
&+\frac{2 M^c}{M} \left( r^{1/4}\frac{X_0^{1/2}}{X^{1/2}} h_0 + \tilde{\mu}_*(r) i\right)^2 \left( \tilde{q} \left(r^{1/4}\frac{X_0^{1/2}}{X^{1/2}} h_0 + \tilde{\mu}_*(r)i \right) X_0 + \mathcal{O}(r^{\eta}) \right.\\
&+ \left. \mathcal{O} \left (r X^3 \left(r^{1/4} \frac{X_0^{1/2}}{X^{1/2}} h_0 + \tilde{\mu}_*(r)\right)^3 \right) \right) \left( 1 + \mathcal{O}\left( \left(2 s_0 i + h_0\right)^2 r^{3/2} X_0 X \right) \right) + \mathcal{O}(1) = 0
\end{aligned}
\end{equation}
Rearranging the top line, we get
\begin{equation*}
\begin{aligned}
&\left( X_0 h_0 \left( r^{1/4}\frac{X_0^{1/2}}{X^{1/2}} h_0 + 2 \tilde{\mu}_*(r) i\right) + \mathcal{O}\left( r^{1/4} X_0^{1/2} X^{1/2} \right) \right) 
\left( 2 s_0 i + h_0 + \mathcal{O}\left( \left( 2 s_0 i + h_0 \right)^3 r^{3/2} X_0 X \right) \right) \\
&+\frac{2 M^c}{M} \left( r^{1/4}\frac{X_0^{1/2}}{X^{1/2}} h_0 + \tilde{\mu}_*(r) i\right)^2 \left( \tilde{q}\left(r^{1/4}\frac{X_0^{1/2}}{X^{1/2}} h_0 + \tilde{\mu}_*(r)i \right) X_0 + \mathcal{O}(r^{\eta}) \right.\\
&+ \left. \mathcal{O} \left (r X^3 \left(r^{1/4} \frac{X_0^{1/2}}{X^{1/2}} h_0 + \tilde{\mu}_*(r)\right)^3 \right) \right) \left( 1 + \mathcal{O}\left( \left(2 s_0 i + h_0\right)^2 r^{3/2} X_0 X \right) \right) + \mathcal{O}(1) = 0
\end{aligned}
\end{equation*}
and divide by $X_0$ to get
\begin{equation*}
\begin{aligned}
&\left( h_0 \left( r^{1/4}\frac{X_0^{1/2}}{X^{1/2}} h_0 + 2 \tilde{\mu}_*(r) i\right) + \mathcal{O}\left( r^{1/4} \frac{X^{1/2}}{X_0^{1/2}} \right) \right) 
\left( 2 s_0 i + h_0 + \mathcal{O}\left( \left( 2 s_0 i + h_0 \right)^3 r^{3/2} X_0 X \right) \right) \\
&+\frac{2 p_1}{M^c} \left( r^{1/4}\frac{X_0^{1/2}}{X^{1/2}} h_0 + \tilde{\mu}_*(r) i\right)^2 \left( \tilde{q}\left(r^{1/4}\frac{X_0^{1/2}}{X^{1/2}} h_0 + \tilde{\mu}_*(r)i \right) + \mathcal{O}\left(\frac{r^{\eta}}{X_0} \right) \right.\\
&+ \left. \mathcal{O} \left (r X^3 \left(r^{1/4} \frac{X_0^{1/2}}{X^{1/2}} h_0 + \tilde{\mu}_*(r)\right)^3 \right) \right) \left( 1 + \mathcal{O}\left( \left(2 s_0 i + h_0\right)^2 r^{3/2} X_0 X \right) \right) + \mathcal{O}\left(\frac{1}{X_0}\right) = 0
\end{aligned}
\end{equation*}
Since $X = \mathcal{O}(|\log r|)$, $X_1 = \mathcal{O}(|\log r|)$, and $r^{a}(|\log r|)^b \rightarrow 0$ as $r \rightarrow 0$ for any positive powers $a$ and $b$, we can simplify this equation to get
\begin{equation*}
\begin{aligned}
0 &= h_0 (2 \mu_*(r) i)(2 s_0 i + h_0) + \frac{2 M^c \tilde{q}}{M}(\mu_*(r) i)^3 + \mathcal{O}\left( \frac{1}{|\log r|}\right)
\end{aligned}
\end{equation*}
Finally, divide by $2 \mu_*(r) i$ to get the equation $G(h_0, s_0, r) = 0$, where $G: \C \times \R \times \mathcal{R} \rightarrow \C$ is defined by
\begin{equation}\label{BsimpleG0}
\begin{aligned}
G(h_0, s_0, r) &= h_0 (2 s_0 i + h_0) + \frac{M^c \tilde{q}}{M}(\mu_*(r) i)^2 + \mathcal{O}\left( \frac{1}{|\log r|}\right),
\end{aligned}
\end{equation}
which simplifies to
\begin{equation}\label{BsimpleG1}
\begin{aligned}
G(h_0, s_0, r) &= h_0^2 + 2 i s_0 h_0 - \tilde{R}(r) + \mathcal{O}\left( \frac{1}{|\log r|}\right),
\end{aligned}
\end{equation}
where 
\[
\tilde{R}(r) = \frac{M^c \tilde{q} }{M} \mu^*(r)^2.
\]

\subsection{Krein bubble}

To leading order, equation \cref{BsimpleG1} is a quadratic in $h_0$. When $r = 0$, equation \cref{BsimpleG1} has solution
\begin{equation}\label{Bquadsol}
h_0(s_0) = -s_0 i \pm \sqrt{ \tilde{R}(0) - s_0^2 }
\end{equation}
Assume $R_0(0) > 0$. Then $\tilde{R}(0) > 0$ as well, and for $|s_0| \leq \sqrt{\tilde{R}}$, $h_0(s_0)$ describes a circle of radius $\sqrt{\tilde{R}(0)}$ in the complex plane centered at the origin. For $|s_0| \geq \sqrt{\tilde{R}(0)}$, $h_0(s_0)$ is on the imaginary axis. Thus for $r = 0$, this is the Krein bubble!

We will now show that this persists for small $r$. To do that, we will use the implicit function theorem to solve for $h_0$ in terms of $s_0$ for small $r$. Computing the derivative with respect to $h_0$ and noting that all the remainder terms vanish at 0,
\begin{align*}
\partial_{h_0} G(h_0, s_0, 0) 
&= 2 ( h_0 + s_0 i)
\end{align*}
Plugging in $h_0(s_0)$ from \cref{Bquadsol},
\begin{align*}
\partial_{h_0} G(h_0(s_0), s_0, 0) 
&= \pm \sqrt{ \tilde{R}(0) - s_0^2 }
\end{align*}
This is nonzero as long as $s_0 \neq \pm \sqrt{\tilde{R}}$, at which point there is a bifurcation. We will first handle what happens away from the bifurcation, and then we will deal with the bifurcation point.

Choose any $\epsilon > 0$. Then on the set
\[
S_\epsilon = [-2 \sqrt{\tilde{R}(0)} -\sqrt{\tilde{R}(0)} - \epsilon]
\cup [-\sqrt{\tilde{R}(0)} + \epsilon, \sqrt{\tilde{R}(0)} - \epsilon]
\cup [\sqrt{\tilde{R}(0)} + \epsilon, 2\sqrt{\tilde{R}(0)}]
\]
$\partial_{h_0} G(s_0, h_0(s_0), 0)$ is bounded away from 0, with bound dependent on $\epsilon$. Thus using the uniform contraction mapping principle, we can find $r_1 \leq r_*$ such that for all $s \in S_\epsilon$, we can solve for $h_0$ in terms of $s_0$ and $r$. Specifically, there exists $r_2 \leq r_1$ and smooth functions $h_0^\pm: S_\epsilon \times \mathcal{R} \rightarrow \C$ with
\[
h_0^\pm(s_0, 0) = -s_0 i \pm \sqrt{ \tilde{R}(0) - s_0^2 }
\]
such that for all $s_0 \in S_\epsilon$ and $r \leq r_2$, $G(h_0^\pm(s_0,r),s_0,r) = 0$. Expanding in a Taylor series, for $s_0 \in S_\epsilon$,
\[
h_0^\pm(s_0, r) = -s_0 i \pm \sqrt{ \tilde{R}(0) - s_0^2 } + \mathcal{O}\left( \frac{1}{|\log r|} \right)
\]
Undoing both of our scalings, let
\begin{align*}
h^\pm(r) &= \frac{X_0^{1/2}}{X^{1/2}}r^{3/4} h_0^\pm(r) \\
s &= \frac{X_0^{1/2}}{X^{1/2}}r^{3/4} s_0 \\
\end{align*}

\subsection{Bifurcation points}

All that remains is to show what happens at the bifurcation points $s_0 = \pm \sqrt{\tilde{R}}$ for small $r$. We will consider the bifurcation point at $s_0 = \sqrt{\tilde{R}}$. The other one is similar. When $r = 0$ and $s_0 = \sqrt{\tilde{R}}$, $h_0 = -\sqrt{\tilde{R}}i$ is a double root of \cref{BsimpleG1}, i.e. both $G(-\sqrt{\tilde{R}}i, \sqrt{\tilde{R}}, 0) = 0$ and $\partial_{h_0} G(-\sqrt{\tilde{R}}i, \sqrt{\tilde{R}}, 0) = 0$. Define $K: \C \times \R \times \mathcal{R} \rightarrow \C^2$ by
\begin{equation}
K(h_0, s_0, r) = 
\begin{pmatrix}G(h_0, s_0, r) \\ \partial_{h_0}G(h_0, s_0, r) \end{pmatrix} 
= \begin{pmatrix}
h_0^2 + 2 i s_0 h_0 - \tilde{R} \\
2(h_0 + s_0 i) 
\end{pmatrix}
+ \mathcal{O}\left( \frac{1}{|\log r|} \right)
\end{equation}
When $r = 0$, $K\left(-\sqrt{\tilde{R}}i, \sqrt{\tilde{R}}, 0\right) = 0$. The Jacobian with respect to $(h_0, s_0)$ is
\[
D_{(z, \epsilon)}K(z, \epsilon, r) = 
\begin{pmatrix}
2 h_0 + 2 i s_0 & 2 i s_0  \\
2 & 2 i
\end{pmatrix}
+ \mathcal{O}\left( \frac{1}{|\log r|} \right)
\]
At $(h_0, s_0, r) = \left( -\sqrt{\tilde{R}}i, \sqrt{\tilde{R}},0\right)$, 
\[
D_{(h_0, s_0)}K\left( -\sqrt{\tilde{R}}i, \sqrt{\tilde{R}},0\right) = 
2 \begin{pmatrix}
0 & \sqrt{\tilde{R}}i \\
1 & i
\end{pmatrix}
\]
which is nonsingular since the columns are linearly independent. Using the implicit function theorem, we can solve for $(h_0, s_0)$ near $(-\sqrt{\tilde{R}}i, \sqrt{\tilde{R}})$ in terms of $r$ for sufficiently small $r$. Specifically, there exists $r_2 \leq r_1$ and a unique smooth function $(h_0^-, s_0^-)(r)$ with $(h_0^-, s_0^-)(0) = (-\sqrt{\tilde{R}}i, \sqrt{\tilde{R}})$ such that for all $r \leq r_2$, $K(h_0^-(r), s_0^-(r), r) = 0$. For $r \leq r_2$ and $s_0 = s_0^-(r)$, $h_0^-(r)$ is a double root of $G(h_0, s_0, r)$. Expanding in a Taylor series, we have
\begin{align*}
h_0^-(r) &= -\sqrt{\tilde{R}}i + \mathcal{O}\left( \frac{1}{|\log r|} \right) \\
s_0^-(r) &= \sqrt{\tilde{R}} + \mathcal{O}\left( \frac{1}{|\log r|} \right) \\
\end{align*}

Similarly, we can find $h_0^+(r)$ and $s_0^+(r)$ with
\begin{align*}
h_0^+(r) &= \sqrt{\tilde{R}}i + \mathcal{O}\left( \frac{1}{|\log r|} \right) \\
s_0^+(r) &= -\sqrt{\tilde{R}} + \mathcal{O}\left( \frac{1}{|\log r|} \right) \\
\end{align*}
such that for $r \leq r_2$ and $s_0 = s_0^+(r)$, $h_0^+(r)$ is a double root of $G(h_0, s_0, r)$.

\subsection{Undoing the scalings}

Undoing both scalings, let 
\begin{align*}
h^\pm(r) &= \frac{X_0^{1/2}}{X^{1/2}}r^{3/4} h_0^\pm(r) \\
s^\pm(r) &= \frac{X_0^{1/2}}{X^{1/2}}r^{3/4} h_0^\pm(r) \\
\end{align*}
Then for $r \leq r_2$, when 
\[
s = s^\pm(r) = \mp \frac{X_0^{1/2}}{X^{1/2}}r^{3/4} \sqrt{\tilde{R}} + \mathcal{O}\left( \frac{1}{|\log r|} \right),
\]
$S(\mu)$ has double zero at
\[
\mu^\pm(r) = \left( \mu_*(r) \pm \frac{X_0^{1/2}}{X^{1/2}}r^{3/4} \sqrt{\tilde{R}} \right)i + \mathcal{O}\left( \frac{1}{|\log r|} \right)
\]
By Lemma \ref{lemma:Hamsymmmu}, Hamiltonian symmetry applies to $\mu$ as well, thus the only way this is possible is if $\mu^\pm(r)$ is on the imaginary axis. 


\iffulldocument\else
	\bibliographystyle{amsalpha}
	\bibliography{thesis.bib}
\fi

\end{document}
\documentclass[thesis.tex]{subfiles}

\begin{document}

\thispagestyle{empty}

% Thesis abstract

Solitary waves are localized disturbances that maintain their shape as they propagate at a constant velocity. They have applications in diverse fields such as fluid mechanics, nonlinear optics, and neuroscience. The fifth-order Korteweg-de Vries equation (KdV5) is a nonlinear, Hamiltonian partial differential equation used to model dispersive phenomena such as plasma waves and capillary-gravity water waves. In addition to having solitary wave solutions, for certain parameter regimes, KdV5 exhibits multi-pulse traveling wave solutions. Multi-pulses are disturbances which resemble multiple, well separated copies of a single solitary wave. Although a multi-pulse structure maintains its shape as it is transmitted, there are underlying nonlinear interactions between neighboring pulses which are revealed when the structure is perturbed. These interactions are captured by eigenvalues in the spectrum of the linearization of KdV5 about a multi-pulse. We refer to these as interaction eigenvalues.

\noindent We use a spatial dynamics approach and mathematical tools such as Lin's method to prove the existence of these multi-pulse structures and to determine the interaction eigenvalues. For multi-pulses on the real line, we are able to obtain leading order estimates for these eigenvalues which are in good agreement with numerical analysis. This allows us to prove an instability criterion for multi-pulses but not a rigorous stability criterion. We also look at multi-pulses with periodic boundary conditions. As long as the periodic domain is not too large, we can locate both the interaction eigenvalues and eigenvalues corresponding to the essential spectrum. In addition, we present strong numerical evidence that as the length of the periodic domain is increased, a brief instability bubble forms when an essential spectrum eigenvalue collides with an interaction eigenvalue on the imaginary axis. Finally, we summarize results we have obtained regarding the spectral stability of multi-pulse solutions in two other systems: the Chen-McKenna suspension bridge equation and the discrete nonlinear Schr{\"o}dinger equation.

\end{document}
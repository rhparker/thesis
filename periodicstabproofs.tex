\documentclass[thesis.tex]{subfiles}

\begin{document}

\iffulldocument\else
	\chapter{KdV5}
\fi

\section{Proof of stability theorems} 

% proof of nu lambda lemma










\subsection{Inversion}

Define the spaces
\begin{align*}
V_Z &= \bigoplus_{i=0}^{n-1} C_b([-X_{i-1},0]:\C^{2m+1}) \oplus C_b([0,X_i]:\C^{2m+1})  \\
V_a &= \bigoplus_{i=0}^{n-1} E^u(\lambda) \oplus E^s(\lambda) \\
V_b &= \bigoplus_{i=0}^{n-1} E^u(0) \oplus E^s(0) \\
V_c, V_{\tilde{c}} &= \bigoplus_{i=0}^{n-1} E^c(\lambda) = \bigoplus_{i=0}^{n-1} \C V_0(\lambda) \\
V_d &= \bigoplus_{i=0}^{n-1} \C \\
V_\lambda &= B_\delta(0) \subset \C
\end{align*}
where the subscripts are all taken $\Mod n$, since we are on a periodic domain. All the product spaces are endowed with the maximum norm, e.g. for $V_b$, 
\[
|b| = \max(|b_0^-|, \dots, |b_{n-1}^-|, |b_0^+|, \dots, |b_{n-1}^+|)
\]
In addition, we take the following notational convention. If we eliminate a subscript or superscript, we are taking the maximum over the eliminated subscript or superscript. For example,
\begin{enumerate}
	\item $|b_i| = \max(|b_i^+|, |b_i^-|)$ 
	\item $|b^+| = \max(|b_0^+|, \dots, |b_{n-1}^+|)$
\end{enumerate}

As in \cite{Sandstede1998}, we will write \eqref{systemZ} in integrated form as a set of fixed point equations. This is similar to \cite[(3.14)]{Sandstede1998}, except we are on a periodic domain and there is an additional center subspace. Using the variation of constants formula and the eigenprojections for $A(\lambda)$, we write \eqref{systemZ} as the set of fixed point equations
\begin{equation}\label{Zfpeq}
\begin{aligned}
Z_i^-(x) &= \Phi^s(x, -X_{i-1}; \lambda) a_{i-1}^- + \Phi^u(x, 0; \lambda) b_i^- + \Phi^c(x, -X_{i-1}; \lambda) c_{i-1} \\
&+ \int_0^x \Phi^u(x, y; \lambda)[P_-(y; \lambda)^{-1} G_i^-(y) P_-(y; \lambda)Z_i^-(y) + \lambda^2 d_i P_-(y; \lambda)^{-1} \tilde{H}_i^-(y)] dy \\
&+ \int_{-X_{i-1}}^x \Phi^s(x, y; \lambda)[P_-(y; \lambda)^{-1} G_i^-(y) P_-(y; \lambda)Z_i^-(y) + \lambda^2 d_i P_-(y; \lambda)^{-1} \tilde{H}_i^-(y)] dy \\
&+ \int_{-X_{i-1}}^x \Phi^c(x, y; \lambda) [P_-(y; \lambda)^{-1} G_i^-(y) P_-(y; \lambda)Z_i^-(y) + \lambda^2 d_i P_-(y; \lambda)^{-1} \tilde{H}_i^-(y)] dy  \\ 
Z_i^+(x) &= \Phi^u(x, X_i; \lambda) a_i^+ + \Phi^s(x, 0; \lambda) b_i^+ + \Phi^c(x, X_i; \lambda)(c_i + \tilde{c}_i) \\
&+ \int_0^x \Phi^s(x, y; \lambda) [P_+(y; \lambda)^{-1} G_i^+(y) P_+(y; \lambda) Z_i^+(y) + \lambda^2 d_i P_+(y; \lambda)^{-1} \tilde{H}_i^+(y)] dy \\
&+ \int_{X_i}^x \Phi^u(x, y; \lambda) [P_+(y; \lambda)^{-1} G_i^+(y) P_+(y; \lambda) Z_i^+(y) + \lambda^2 d_i P_+(y; \lambda)^{-1} \tilde{H}_i^+(y)] dy \\
&+ \int_{X_i}^x \Phi^c(x, y; \lambda) [P_+(y; \lambda)^{-1} G_i^+(y) P_+(y; \lambda) Z_i^+(y) + \lambda^2 d_i P_+(y; \lambda)^{-1} \tilde{H}_i^+(y)] dy \\
\end{aligned}
\end{equation}
where $i = 0, \dots, n-1$. 

As in \cite{Sandstede1998}, we will solve the eigenvalue problem in a series of inversion steps. First, we will solve equation \eqref{systemZ} for $Z_i^\pm$. To do this, we rewrite \eqref{Zfpeq} as
\begin{equation}\label{L1L2eq}
[I - L_1(\lambda)]Z = L_2(\lambda)(a,b,c,\tilde{c},d)Z
\end{equation}
where $L_1(\lambda)$ is the linear operator consisting of the terms from the RHS of \eqref{Zfpeq} involving $Z_i^\pm$ and $L_2(\lambda)(a,b,c,\tilde{c},d)$ is the linear operator consisting of the terms from the RHS of \eqref{Zfpeq} which do not involve $Z_i^\pm$. Specifically,
\begin{align*}
[L_1&(\lambda)_i^- Z](x) 
= \int_0^x \Phi^u(x, y; \lambda)P_-(y; \lambda)^{-1} G_i^-(y) P_-(y; \lambda) Z_i^-(y) dy \\
&+ \int_{-X_{i-1}}^x \Phi^s(x, y; \lambda) P_-(y; \lambda)^{-1} G_i^-(y) P_-(y; \lambda) Z_i^-(y) dy \\
&+ \int_{-X_{i-1}}^x \Phi^c(x, y; \lambda) P_-(y; \lambda)^{-1} G_i^-(y) P_-(y; \lambda) Z_i^-(y) dy  \\ 
[L_1&(\lambda)_i^+ Z](x) = \int_0^x \Phi^s(x, y; \lambda) P_+(y; \lambda)^{-1} G_i^+(y) P_+(y; \lambda) Z_i^+(y) dy \\
&+ \int_{X_i}^x \Phi^u(x, y; \lambda) P_+(y; \lambda)^{-1} G_i^+(y) P_+(y; \lambda) Z_i^+(y) dy \\
&+ \int_{X_i}^x \Phi^c(x, y; \lambda) P_+(y; \lambda)^{-1} G_i^+(y) P_+(y; \lambda) Z_i^+(y) dy \\
\end{align*}
and
\begin{align*}
[L_2&(\lambda)_i^-(a,b,c,\tilde{c},d)](x) = \Phi^s(x, -X_{i-1}; \lambda) a_{i-1}^- + \Phi^u(x, 0; \lambda) b_i^- + \Phi^c(x, -X_{i-1}; \lambda) c_{i-1} \\
&+ \lambda^2 d_i \left( 
\int_0^x \Phi^u(x, y; \lambda) P_-(y; \lambda)^{-1} \tilde{H}_i^-(y) dy 
+ \int_{-X_{i-1}}^x \Phi^s(x, y; \lambda) P_-(y; \lambda)^{-1} \tilde{H}_i^-(y) dy \right. \\
&+ \left. \int_{-X_{i-1}}^x \Phi^c(x, y; \lambda) P_-(y; \lambda)^{-1} \tilde{H}_i^-(y)] dy \right) \\
[L_2&(\lambda)_i^+(a,b,c,\tilde{c},d)](x) = \Phi^u(x, X_i; \lambda) a_i^+ + \Phi^s(x, 0; \lambda) b_i^+ + \Phi^c(x, X_i; \lambda)(c_i + \tilde{c}_i) \\
&+ \lambda^2 d_i \left( \int_0^x \Phi^s(x, y; \lambda) P_+(y; \lambda)^{-1} \tilde{H}_i^+(y)] dy 
+ \int_{X_i}^x \Phi^u(x, y; \lambda) P_+(y; \lambda)^{-1} \tilde{H}_i^+(y)] dy \right. \\
&+ \left. \int_{X_i}^x \Phi^c(x, y; \lambda) P_+(y; \lambda)^{-1} \tilde{H}_i^+(y) dy \right)
\end{align*}

We obtain bounds for $L_1$ and $L_2$ in the next two lemmas.

\begin{lemma}\label{L1boundlemma}
For the linear operator $L_1$, we have uniform bound
\begin{equation}\label{L1uniformbound}
\|L_1(\lambda)Z\| \leq C e^{-\alpha_1 X_m}\|Z\|
\end{equation}
\begin{proof}
The bound on $L_1$ will depend on the integral involving the center subspace, since there is potential growth in that subspace. For the ``minus'' piece, using the bound for $G_i^\pm(x)$ from Lemma \ref{stabestimateslemma}
\begin{align*}
\left| \int_{-X_{i-1}}^x \right.&\left.\Phi^c(x, y; \lambda) P_-(y; \lambda)^{-1} G_i^-(y) P_-(y; \lambda) Z_i^-(y) dy \right| \\ 
&\leq C \| Z_i^-\|\int_{-X_{i-1}}^x e^{\eta (x - y)} \left( e^{-\alpha_0 X_{i-1}} e^{-\alpha_0(X_{i-1} + y) } + e^{-2 \alpha_0 X_i} e^{\alpha_0 y} \right) dy \\
&\leq C \| Z_i^-\|\int_{-X_{i-1}}^x e^{\eta X_{i-1}} e^{-\alpha_0 X_{i-1}} \left( e^{-\alpha_0(X_{i-1} + y) } + e^{-\alpha_0 X_i} e^{\alpha_0 y} \right) dy \\
&\leq C e^{-\alpha_1 X_{i-1}} \| Z_i^-\| \int_{-X_{i-1}}^0 \left( e^{-\alpha_0(X_{i-1} + y) } + e^{-\alpha_0 X_i} e^{\alpha_0 y} \right) dy \\
&\leq C e^{-\alpha_1 X_{i-1}} \| Z_i^-\| 
\end{align*}
For the ``plus'' piece, 
\begin{align*}
\left| \int_{X_i}^x \Phi^c(x, y; \lambda) P_+(y; \lambda)^{-1} \tilde{H}_i^+(y) dy \right| \leq C e^{-\alpha_1 X_i} \| Z_i^+\| 
\end{align*}
Since the integrals involving the stable and unstable subspaces have stronger bounds, we have the uniform bound \eqref{L1uniformbound}.
\end{proof}
\end{lemma}

\begin{lemma}\label{L2boundlemma}
For the linear operator $L_1$, we have piecewise bound
\begin{equation}\label{L2bound}
\begin{aligned}
\| L_2(\lambda)_i^-(a,b,c,\tilde{c},d) \| &\leq C(|a_{i-1}^-| + |b_i^-| + |e^{\nu(\lambda)X_{i-1}}c_{i-1}| + |c_{i-1}| + e^{-\alpha X_{i-1}}|\lambda|^2|d|)\\
\| L_2(\lambda)_i^+(a,b,c,\tilde{c},d) \| &\leq C(|a_i^+| + |b_i^+| + |e^{-\nu(\lambda)X_i} c_i| + |c_i| + e^{\eta X_i}|\tilde{c}_i| + e^{-\alpha X_i}|\lambda|^2|d|)
\end{aligned}
\end{equation}
\begin{proof}
We will bound the ``minus'' piece first. For the term involving $c_{i-1}^-$, 
\begin{align*}
\Phi^c(x, -X_{i-1}; \lambda) c_{i-1}
&= e^{\nu(\lambda)(x + X_{i-1})}
\end{align*}
Since we do not know whether this term will grow or decay, we will bound it by
\begin{align*}
|\Phi^c(x, -X_{i-1}; \lambda) c_{i-1}|
&\leq |e^{\nu(\lambda)X_{i-1}}c_{i-1}| + |c_{i-1}|
\end{align*}
For the other terms not involving integrals,
\[
\left| \Phi^s(x, -X_{i-1}; \lambda) a_{i-1}^- + \Phi^u(x, 0; \lambda) b_i^- \right| \leq C(|a_{i-1}^-| + |b_i^-|)
\]
For the integral terms, the bound is again determined by the integral involving the center subspace. This is similar to the corresponding integral in Lemma \ref{L1boundlemma}, except the bound for $\tilde{H}$ involves the decay constant $\alpha_1$; this gives us a similar bound for the center integral term, except it involves the decay constant $\alpha$. Putting all of this together,
\[
\| L_2(\lambda)_i^-(a,b,c,\tilde{c},d) \| \leq C(|a_{i-1}^-| + |b_i^-| + |e^{\nu(\lambda)X_{i-1}}c_{i-1}| + |c_{i-1}| + e^{-\alpha X_{i-1}}|\lambda|^2|d|)
\]

For the ``plus'' piece, 
\begin{align*}
\Phi^c(x, X_i; \lambda) c_i
&= e^{\nu(\lambda)(x - X_i)}
\end{align*}
which we will bound by 
\begin{align*}
|\Phi^c(x, X_i; \lambda) c_i|
&\leq |e^{-\nu(\lambda)X_i}c_i| + |c_i|
\end{align*}
We bound the piece involving $\tilde{c}_i$ by
\[
|\Phi^c(x, X_i; \lambda) \tilde{c}_i|
\leq e^{\eta X_i}|\tilde{c}_i|
\]
Combining all of these, we have the bound
\[
\| L_2(\lambda)_i^+(a,b,c,\tilde{c},d) \| \leq C(|a_i^+| + |b_i^+| + |e^{-\nu(\lambda)X_i} c_i| + |c_i| + e^{\eta X_i}|\tilde{c}_i| + e^{-\alpha X_i}|\lambda|^2|d|)
\]
\end{proof}
\end{lemma}

We can now solve equation \eqref{systemZ} for $Z$, which we do in the following lemma.

\begin{lemma}\label{Zinv0}
There exists an operator $Z_1: V_\lambda \times V_a \times V_b \times V_c \times V_{\tilde{c}} \times V_d \rightarrow V_z$ such that $Z = Z_1(\lambda)(a,b,c,\tilde{c},d)$ solves \eqref{systemZ}. The operator $Z_1$ is analytic in $\lambda$ and linear in $(a,b,c,\tilde{c},d)$, and we have the piecewise estimates
\begin{equation}\label{Z1bound}
\begin{aligned}
\| Z_1(\lambda)_i^-(a,b,c,\tilde{c},d) \| &\leq C(|a_{i-1}^-| + |b_i^-| + |e^{\nu(\lambda)X_{i-1}}c_{i-1}| + |c_{i-1}| + e^{-\alpha X_{i-1}}|\lambda|^2|d|) \\
\| Z_1(\lambda)_i^+(a,b,c,\tilde{c},d) \| &\leq C(|a_i^+| + |b_i^+| + |e^{-\nu(\lambda)X_i} c_i| + |c_i| + e^{\eta X_i}|\tilde{c}_i| + e^{-\alpha X_i}|\lambda|^2|d|)
\end{aligned}
\end{equation}
\begin{proof}

Using the bound on $L_1$ from Lemma \ref{L1boundlemma}, 
\[
\|L_1(\lambda)Z\| \leq C e^{-\alpha_1 X_m}\|Z\|
\]
Since we have chosen $X_m$ sufficiently large so that
$e^{-\alpha_1 X_m} \leq e^{-\tilde{\alpha} X_m} < \delta$, we have
\[
\|L_1(\lambda)Z\| \leq C \delta \|Z\|
\]
Thus, for sufficiently small $\delta$, $I - L_1(\lambda)$ is invertible on $V_Z$. The inverse $(I - L_1(\lambda))^{-1}$ is analytic in $\lambda$, thus we have the solution
\begin{equation}\label{L1L2eq}
Z = (I - L_1(\lambda)^{-1}L_2(\lambda)(a,b,c,\tilde{c},d)
\end{equation}
The bounds \eqref{Z1bound} follow by using the bounds \eqref{L2bound} from Lemma \ref{L2boundlemma} on the appropriate pieces of $Z$.
\end{proof}
\end{lemma}

In the next lemma, we solve equation \eqref{systemmiddle}, which are the matching conditions at $\pm X_i$.
\begin{align*}
P_i^+(X_i; \lambda) Z_i^+(X_i) - P_{i+1}^-(-X_i; \lambda) Z_{i+1}^-(-X_i) &= D_i d && i = 0, \dots, n-1
\end{align*}

% first inversion lemma : match at \pm X_i
\begin{lemma}\label{Zinv1}
There exists operators
\begin{align*}
A_1: &V_\lambda \times V_b \times V_c \times V_d \rightarrow V_a \times V_{\tilde{c}} \\
Z_2: &V_\lambda \times V_b \times V_c \times V_d \rightarrow V_Z
\end{align*}
such that 
\[
((a, \tilde{c}), Z) = (A_1(\lambda)(b, c, d), Z_2(\lambda)(b,c,d))
\]
solves \eqref{systemZ} and \eqref{systemmiddle} for any $(b, c, d)$ and $\lambda$. These operator are analytic in $\lambda$ and linear in $(b, c, d)$. Piecewise bounds for $A_1$ and $Z_2$ are given by
\begin{align}\label{A1bound}
|A_1&(\lambda)_i(b, c, d)|
\leq C \Big( e^{-\alpha X_i} (|b_i^+| + |b_{i+1}^-| + |c_i| + |\lambda^2||d|) + |D_i||d| \Big)
\end{align} 
and
\begin{equation}\label{Z2bound}
\begin{aligned}
\| Z_2(\lambda)_i^-(b,c,d) \| &\leq C(|b_i^-| + e^{-\alpha X_{i-1}}|b_{i-1}^+| + |e^{\nu(\lambda)X_{i-1}}c_{i-1}| + |c_{i-1}| + e^{-(\alpha - \eta) X_m}|\lambda|^2|d| + |D_{i-1})|d| \\
\| Z_2(\lambda)_i^+(b,c,d) \| &\leq C(|b_i^+| + e^{-\alpha X_i}|b_{i+1}^-| + |e^{-\nu(\lambda)X_i} c_i| + |c_i| + e^{-(\alpha - \eta) X_m}|\lambda|^2|d| + |D_i||d|)
\end{aligned}
\end{equation}

\begin{proof}
At $\pm X_i$, the fixed point equations \eqref{Zfpeq} are
\begin{align*}
Z_{i+1}^-&(-X_i) = a_i^- + \Phi^u(-X_i, 0; \lambda) b_{i+1}^- + c_i \\ 
&+ \int_0^{-X_i} \Phi^u(-X_i, y; \lambda) [P_-(y; \lambda)^{-1} G_{i+1}^-(y) P_-(y; \lambda)Z_{i+1}^-(y) + \lambda^2 d_{i+1} P_-(y; \lambda)^{-1} \tilde{H}_{i+1}^-(y)] dy \\
Z_i^+&(X_i) = a_i^+ + \Phi^s(X_i, 0; \lambda) b_i^+ + c_i + \tilde{c}_i \\
&+ \int_0^{X_i} [P_+(y; \lambda)^{-1} G_i^+(y) P_+(y; \lambda) Z_i^+(y) + \lambda^2 d_i P_+(y; \lambda)^{-1} \tilde{H}_i^+(y)] \Phi^s(X_i, y; \lambda) dy
\end{align*}
To obtain these, we used the fact that, for example, $a_i^- \in E^s(\lambda)$ and $\Phi^s(-X_{i-1}, -X_{i-1}; \lambda)$ is the identity on $E^s(\lambda)$. Applying the appropriate projections, subtracting, and using equation \eqref{projTheta}, we obtain the equation 
\begin{align*}
D_i &d = (I + \Theta_+(X_i; \lambda))a_i^+ + (I + \Theta_+(X_i; \lambda))(c_i + \tilde{c}_i) + P_+(X_i; \lambda)\Phi^s(X_i, 0; \lambda) b_i^+ \\
&+ P_+(X_i; \lambda) \int_0^{X_i} \Phi^s(X_i, y; \lambda)[P_+(y; \lambda)^{-1} G_i^+(y) P_+(y; \lambda) Z_i^+(y) + \lambda^2 d_i P_+(y; \lambda)^{-1} \tilde{H}_i^+(y)] dy \\
&- (I + \Theta_-(-X_i; \lambda))a_i^- - (I + \Theta_-(-X_i; \lambda))c_i - P_-(-X_i; \lambda)\Phi^u(-X_i, 0; \lambda) b_{i+1}^- \\ 
&- P_-(-X_i; \lambda) \int_0^{-X_i} \Phi^u(-X_i, y; \lambda) [P_-(y; \lambda)^{-1} G_{i+1}^-(y) P_-(y; \lambda)Z_{i+1}^-(y) + \lambda^2 d_{i+1} P_-(y; \lambda)^{-1} \tilde{H}_{i+1}^-(y)] dy
\end{align*}
which simplifies to
\begin{equation}\label{Didexpansion}
\begin{aligned}
D_i &d = a_i^+ - a_i^- + \tilde{c}_i^- \\
&+ \Theta_+(X_i; \lambda)a_i^+ - \Theta_-(-X_i; \lambda))a_i^- + \Theta_+(X_i; \lambda)\tilde{c}_i \\
&+ P_+(X_i; \lambda)\Phi^s(X_i, 0; \lambda) b_i^+ - P_-(-X_i; \lambda)\Phi^u(-X_i, 0; \lambda) b_{i+1}^- + \Theta_+(X_i; \lambda) c_i  - \Theta_-(-X_i; \lambda))c_i \\
&+ P_+(X_i; \lambda) \int_0^{X_i} \Phi^s(X_i, y; \lambda)[P_+(y; \lambda)^{-1} G_i^+(y) P_+(y; \lambda) Z_i^+(y) + \lambda^2 d_i P_+(y; \lambda)^{-1} \tilde{H}_i^+(y)] dy \\ 
&- P_-(-X_i; \lambda) \int_0^{-X_i} \Phi^u(-X_i, y; \lambda) [P_-(y; \lambda)^{-1} G_{i+1}^-(y) P_-(y; \lambda)Z_{i+1}^-(y) + \lambda^2 d_{i+1} P_-(y; \lambda)^{-1} \tilde{H}_{i+1}^-(y)] dy
\end{aligned}
\end{equation}
This is of the form
\begin{align}\label{Dideq1}
D_i d &= a_i^+ - a_i^- + \tilde{c}_i + L_3(\lambda)_i(a, b, c, \tilde{c}, d)
\end{align}
where the linear operator $L_3(\lambda)_i(a, b, c, \tilde{c}, d)$ is linear in $a,b,c,\tilde{c}$ and $d$, analytic in $\lambda$, and $L_3(\lambda)_i(a, b, c, \tilde{c}, d)$ is defined by the RHS of \cref{Didexpansion}. To get a bound on $L_3$, we will bound the individual terms. 
\begin{enumerate}
\item For the terms involving $a_i^\pm$, we use \eqref{conjthetadecay} to get
\[
|\Theta_+(X_i; \lambda)a_i^+ - \Theta_-(-X_i; \lambda)a_i^-| \leq C e^{-\alpha_1 X_i}(|a_i^+| + |a_i^-|)
\]
\item For the terms involving $c_i$, and $\tilde{c}_i$, we use \eqref{conjthetadecay} to get
\[
|\Theta_+(X_i; \lambda)c_i + \Theta_+(X_i; \lambda)\tilde{c}_i - \Theta_-(-X_i; \lambda)c_i| \leq 
C e^{-\alpha_1 X_i} (|c_i| + |\tilde{c}_i|)
\]

\item For the terms involving $b$, we use \eqref{Zevolbounds} to get
\[
| P_+(X_i; \lambda)\Phi^s(X_i, 0; \lambda) b_i^+ - P_-(-X_i; \lambda) \Phi^u(-X_i, 0; \lambda) b_{i+1}^-| \leq C e^{-\alpha X_i} (|b_i^+| + |b_{i+1}^-|)
\]

\item For the integral terms involving $Z$, we use the bound $Z_1$ from Lemma \ref{Zinv0} to get
\begin{align*}
&\left|
P^+(X_i; \beta_i^+, \lambda) \int_0^{X_i} \Phi^s(X_i, y; \lambda) P_+(y; \lambda)^{-1} G_i^+(y) P_+(y; \lambda) Z_i^+(y) dy \right| \\
&\leq \int_0^{X_i} e^{-\alpha(X_i - y)} |G_i^+(y)| |Z_i^+(y)| dy \\
&\leq C \left(|a_i^+| + |b_i^+| + e^{\eta X_i}(|c_i| + |\tilde{c}_i|) + e^{-\alpha X_i}|\lambda|^2|d| \right) \int_0^{X_i} e^{-\alpha(X_i - y)} |G_i^+(y)| dy \\
&\leq C \left(|a_i^+| + |b_i^+| + |c_i| + |\tilde{c}_i| + e^{-\alpha_1 X_i}|\lambda|^2|d| \right) \int_0^{X_i}  e^{\eta X_i} e^{-\alpha(X_i - y)} |G_i^+(y)| dy
\end{align*}
To evaluate the integral, we use the estimates on $G$ from Lemma \ref{stabestimateslemma} to get
\begin{align*}
\int_0^{X_i} &e^{\eta X_i} e^{-\alpha(X_i - y)} |G_i^+(y)| dy \leq \int_0^{X_i} e^{\eta X_i} e^{-\alpha(X_i - y)} \left(e^{-\alpha_0(X_i - y)}e^{-\alpha_0 X_i} + e^{-2 \alpha_0 X_{i-1}} e^{-\alpha_0 y} \right) dy \\
&\leq e^{-\alpha X_i} \int_0^{X_i} e^{-(\alpha + \alpha_0)(X_i - y)} dy + e^{-2 \alpha X_{i-1}} \int_0^{X_i} e^{-(\alpha - 2 \eta)(X_i - y)}e^{-2 \eta(X_i - y)}e^{\eta X_i}e^{-\alpha_0 y} dy \\
&\leq e^{-\alpha X_i} \int_0^{X_i} e^{-(\alpha + \alpha_0)(X_i - y)} dy + e^{-2 \alpha X_{i-1}} \int_0^{X_i} e^{-(\alpha - 2\eta)(X_i - y)}e^{-(\alpha_0 - 2 \eta)y} dy \\
&\leq C (e^{-\alpha X_i} + e^{-\eta X_i} e^{-2 \alpha X_{i-1}})
\end{align*}
where we extracted an extra $e^{-\eta X_i}$ in the second integral to cancel a term later on. Putting this all together,
\begin{align*}
&\left|
P^+(X_i; \beta_i^+, \lambda) \int_0^{X_i} \Phi^s(X_i, y; \lambda) P_+(y; \lambda)^{-1} G_i^+(y) P_+(y; \lambda) Z_i^+(y) dy \right| \\
&\leq C (e^{-\alpha X_i} + e^{-\eta X_i} e^{-2 \alpha X_{i-1}}) \left(|a_i^+| + |b_i^+| + |c_i| + |\tilde{c}_i| + e^{-\alpha X_i}|\lambda|^2|d| \right)
\end{align*}
The other integral is similar, except we have a term in $e^{-2 \alpha X_{i+1}}$.
\begin{align*}
&\left|
P^+(X_i; \beta_i^+, \lambda) \int_0^{-X_i} \Phi^u(-X_i, y; \lambda) P_-(y; \lambda)^{-1} G_{i+1}^-(y) P_-(y; \lambda)Z_{i+1}^-(y) \right| \\
&\leq C (e^{-\alpha X_i} + e^{-\eta X_i} e^{-2 \alpha X_{i+1}}) \left(|a_i^-| + |b_{i+1}^-| + |c_i| + e^{-\alpha X_i}|\lambda|^2|d| \right)
\end{align*}

\item For the integral terms not involving $Z$, we use \eqref{Zevolbounds} and the estimates from Lemma \ref{stabestimates} to get
\begin{align*}
&\left|
P_+(X_i; \lambda) \int_0^{X_i} \Phi^s(X_i, y; \lambda) P_+(X_i; \lambda)^{-1} \tilde{H}_i^+(y) dy \right| \\
&\leq C \int_0^{X_i} e^{-\alpha(X_i - y)}e^{-\alpha_1 y} dy \\
&\leq C e^{-\alpha X_i} \int_0^{X_i} e^{-(\alpha_1 - \alpha)y} dy \\
&= C e^{-\alpha X_i} \int_0^{X_i} e^{-\eta y} dy \\ 
&= C e^{-\alpha X_i}
\end{align*}
The other integral is similar.
\end{enumerate}

Putting all of these together, we have the following bound for $L_3$.
\begin{equation}\label{L3bound}
|L_3(\lambda)_i(a, b, c, \tilde{c}, d)| \leq C (e^{-\alpha X_i} + e^{-\eta X_i}e^{-2\alpha X_m}) \left( |a_i| + |b_i^+| + |b_{i+1}^-| + |c_i| + |\tilde{c}_i| + |\lambda^2| |d| \right)
\end{equation}
Since $e^{-\alpha X_m} < \delta$, this becomes
\begin{align*}
|L_3(\lambda)_i(a, b, c, \tilde{c}, d)| \leq C \delta ( |a_i| + |\tilde{c}| ) + C e^{-\alpha X_i} \left( |b_i^+| + |b_{i+1}^-| + |c_i| + |\lambda^2| |d| \right)
\end{align*}
Let 
\[
J_1: \bigoplus_{j=1}^n (E^s(\lambda) \times E^u(
\lambda) \times E^c(\lambda) ) \rightarrow \bigoplus_{j=1}^n \rightarrow \C^{2m+1}
\]
be defined by $(J_1)_i(a_i^+, a_i^-, \tilde{c}_i) = (a_i^+ - a_i^-, \tilde{c}_i)$. The map $J_i$ is a linear isomorphism since $E^s(\lambda) \oplus E^u(\lambda) \oplus E^c(\lambda) = \C^{2m+1}$. Consider the map
\[
S_1(a, \tilde{c}) = J_1 (a, \tilde{c}) + L_3(\lambda)(a, 0, c, 0, 0) = J_1( I + J_1^{-1} L_3(\lambda)(a, 0, c, 0, 0))
\]
For sufficiently small $\delta$, $||J_1^{-1} L_3(\lambda)(a, 0, \tilde{c}, 0, 0)|| < 1$, thus the operator $S_1(a, \tilde{c})$ is invertible. We can then solve for $(a, \tilde{c})$ to get
\[
(a, \tilde{c}) = A_1(\lambda)(b, c, d) = S_i^{-1}(-D d + L_3(\lambda)(0, b, 0, c, d)
\]
Using the bound on $L_3$ and noting which pieces are involved, $A_1$ has piecewise bounds
\begin{align*}
|A_1&(\lambda)_i(b, c, d)|
\leq C \Big( (e^{-\alpha X_i} + e^{-\eta X_i}e^{-2\alpha X_m}) (|b_i^+| + |b_{i+1}^-| + |c_i| + |\lambda^2||d|) + |D_i||d| \Big)
\end{align*} 
We can plug this into the bound $Z_1$ from Lemma \ref{Zinv0} to get $Z_2(\lambda)(b,c,d)$, which has bound
\begin{align*}
\| Z_2(\lambda)_i^-(b,c,d) \| &\leq C(|b_i^-| + e^{-\alpha X_{i-1}}|b_{i-1}^+| + |e^{\nu(\lambda)X_{i-1}}c_{i-1}| + |c_{i-1}| + e^{-(\alpha - \eta) X_m}|\lambda|^2|d| + |D_{i-1})|d| \\
\| Z_2(\lambda)_i^+(b,c,d) \| &\leq C(|b_i^+| + e^{-\alpha X_i}|b_{i+1}^-| + |e^{-\nu(\lambda)X_i} c_i| + |c_i| + e^{-(\alpha - \eta) X_m}|\lambda|^2|d| + |D_i||d|)
\end{align*}
\end{proof}
\end{lemma}

In the next two lemmas, we derive expressions for $a_i^\pm$ and $\tilde{c}_i$ which we will use in evaluating the jump conditions. First, we have the following expressions for $a_i^\pm$.

\begin{lemma}\label{lemma:aipm}
For the initial conditions $a_i^\pm$, we have the expressions
\begin{equation}\label{aipmexp1}
\begin{aligned}
a_i^+ &= P_i^+(X_i; \lambda)^{-1} P_0^u(\lambda) D_i d + A_2(\lambda)_i^+(b, c, d) \\
a_i^- &= -P_i^-(-X_i; \lambda)^{-1} P_0^s(\lambda) D_i d + A_2(\lambda)_i^-(b, c, d)
\end{aligned}
\end{equation}
$A_2$ is linear in $(b, c, d)$, and has piecewise bounds
\begin{align}
|A_2(\lambda)_i(b, c, d)|
&\leq C e^{-\alpha X_i} \left( |b_i^+| + |b_{i+1}^-| + |c_i| + |\lambda|^2|d| + |D_i||d| \right) \label{A2bound}
\end{align}

\begin{proof}
We apply the projections $P_0^{s/u}(\lambda)$ on the eigenspaces $E^{s/u}(\lambda)$ to \eqref{Dideq1} to obtain the expressions
\begin{align*}
a_i^+ &= P_0^u(\lambda) D_i d + A_2(\lambda)_i^+(b, c^-, d) \\
a_i^- &= -P_0^s(\lambda) D_i d + A_2(\lambda)_i^-(b, c^-, d) \\
\end{align*}
The bound on the remainder term $A_2(\lambda)_i(b, c^-, d)$ is found by substituting the bound for $A_1$ into the bound for $L_3$ and simplifying. 
\begin{align*}
|A_2&(\lambda)_i(b, c^-, d)|
\leq C (e^{-\alpha X_i} + e^{-\eta X_i}e^{-2\alpha X_m}) \left( |b_i^+| + |b_{i+1}^-| + |c_i| + |\lambda|^2|d| + |D_i||d| \right)
\end{align*} 

Anticipating what we will need at the end, we will modify the expressions for $a_i^+$ and $a_i^-$ to involve the conjugation projections. Using the conjugation operator $P_+(X_i; \lambda)$, we write $a_i^+$ as
\begin{align*}
a_i^+ &= P_+(X_i; \lambda)a_i^+ + (I - P_+(X_i; \lambda))a_i^+ \\
&= P_+(X_i; \lambda)a_i^+ - \Theta_+(X_i; \lambda))a_i^+
\end{align*}
Rearranging this and substituting the expression above for $a_i^+$, we get
\begin{align*}
P_+(&X_i; \lambda) a_i^+ = P_0^u(\lambda) D_i d + A_2(\lambda)_i^+(b, c^-, d) + \Theta_+(X_i; \lambda))a_i^+ \\
&= P_0^u(\lambda) D_i d + A_2(\lambda)_i^+(b, c^-, d) + \mathcal{O}\Big( e^{-\alpha X_i} ( e^{-\alpha X_i} (|b_i^+| + |b_{i+1}^-|) + |c_i| + e^{-\alpha X_i} |\lambda^2||d| + |D_i||d| )\Big)
\end{align*}
where we used the bound $A_1$ and the estimate \eqref{Thetadecay}. The last term on the RHS is the same (or higher) order as $A_2$, so we incorporate that into the bound on $A_2(\lambda)_i^+(b, c^-, d)$ to get
\begin{align*}
P_+(X_i; \lambda)a_i^+ &= P_0^u(\lambda) D_i d + A_2(\lambda)_i^+(b, c^-, d)
\end{align*}
Finally, apply the operator $P_+(X_i; \lambda)^{-1}$ on the left on both sides to solve for $a_i^+$. Since $P_+(X_i; \lambda)^{-1}$ a bounded operator, we will also incorporate this into $A_2(\lambda)_i^+(b, c^-, d)$, which will leave the bound unchanged. Thus we have
\begin{align*}
a_i^+ &= P_+(X_i; \lambda)^{-1} P_0^u(\lambda) D_i d + A_2(\lambda)_i^+(b, c^-, d)
\end{align*}
We do the same thing for $a_i^-$, which gives us
\begin{align*}
a_i^- &= -P_-(-X_i; \lambda)^{-1} P_0^s(\lambda) D_i d + A_2(\lambda)_i^-(b, c^-, d)
\end{align*}
\end{proof}
\end{lemma}

Next, we derive an expression for $\tilde{c}_i$.

\begin{lemma}\label{lemma:tildec1}
For the $\tilde{c}$, we have the expression
\begin{align}\label{tildeciexp1}
\tilde{c}_i &= 2 \lambda \langle \tilde{W}_0, Q'(X_i) \rangle (d_{i+1} - d_i ) + C_2(\lambda)_i(b, c, d)
\end{align}
$C_2$ is linear in $(b, c, d)$ and has piecewise bounds
\begin{align}
|C_2(\lambda)_i(b, c, d)| &\leq C |\lambda| e^{-\alpha X_i} \left( |b_i^+| + |b_{i+1}^-| + |c_i| + (|D_i| + |\lambda|)|d| \right) \label{C2bound}
\end{align}
\begin{proof}
We apply the projection $P^c(\lambda)$ on $E^c(
\lambda)$ to \eqref{Dideq1}, and write the resulting equation as
\begin{equation}\label{PcDid}
P^c(\lambda)D_i d = P^c(\lambda) P_+(X_i; \lambda)\tilde{c}_i + P^c(\lambda) \tilde{L}_3(
\lambda)_i(a,b,\tilde{c},c,d)
\end{equation}
where $\tilde{L}_3(\lambda)(a,b,\tilde{c},c,d)$ is the linear operator defined by
\begin{align*}
\tilde{L}_3&(\lambda)_i(a,b,\tilde{c},c,d) = (I + \Theta_+(X_i; \lambda))P_0^u(\lambda) a_i^+ - (I + \Theta_-(-X_i; \lambda))P_0^s(\lambda)a_i^- \\
&+ (I + \Theta_+(X_i; \lambda))\Phi^s(X_i, 0; \lambda) b_i^+ - (I + \Theta_-(-X_i; \lambda))\Phi^u(-X_i, 0; \lambda) b_{i+1}^- \\
&+ (\Theta_+(X_i; \lambda) - \Theta_-(-X_i; \lambda))c_i  \\
&+ (I + \Theta_+(X_i; \lambda)) \int_0^{X_i} \Phi^s(X_i, y; \lambda) P_+(y; \lambda)^{-1} G_i^+(y) P_+(y; \lambda) Z_i^+(y) dy \\
&+ \lambda^2 d_i (I + \Theta_+(X_i; \lambda))  \int_0^{X_i} \Phi^s(X_i, y; \lambda) P_+(y; \lambda)^{-1} \tilde{H}_i^+(y) dy \\ 
&- (I + \Theta_-(-X_i; \lambda)) \int_0^{-X_i} \Phi^u(-X_i, y; \lambda) P_-(y; \lambda)^{-1} G_{i+1}^-(y) P_-(y; \lambda)Z_{i+1}^-(y) dy \\
&- \lambda^2 d_{i+1} (I + \Theta_-(-X_i; \lambda)) \int_0^{-X_i} \Phi^u(-X_i, y; \lambda) P_-(y; \lambda)^{-1} \tilde{H}_{i+1}^-(y) dy
\end{align*}

First, we use the expansion for $W_0(\lambda)$ from Lemma \ref{nulambdalemma} to evaluate $P^c(\lambda)D_i d$.
\begin{align*}
P^c(&\lambda)D_i d = \langle W_0(\lambda), D_i d \rangle \\
&= \langle W_0 + \overline{\lambda} \tilde{W}_0 + \mathcal{O}(\overline{\lambda}^2), D_i d \rangle \\
&= \langle W_0, D_i d \rangle + \langle \overline{\lambda} \tilde{W}_0, (Q'(X_i) + Q'(-X_i))(d_{i+1} - d_i ) \rangle + \mathcal{O}(e^{-\alpha_1 X_i} |\lambda|(e^{\alpha_0 X_i} + |\lambda|)|d|) \\
&= \lambda \langle \tilde{W}_0, (Q'(X_i) + Q'(-X_i))(d_{i+1} - d_i ) \rangle + \mathcal{O}(e^{-\alpha_1 X_i} |\lambda|(e^{\alpha_0 X_i} + |\lambda|)|d|) \\
&= \lambda ( \langle \tilde{W}_0, Q'(X_i) \rangle + \langle \tilde{W}_0, -R Q'(X_i)\rangle )(d_{i+1} - d_i ) \rangle + \mathcal{O}(e^{-\alpha_1 X_i} |\lambda|(e^{\alpha_0 X_i} + |\lambda|)|d|)\\
&= \lambda ( \langle \tilde{W}_0, Q'(X_i) \rangle + \langle -R \tilde{W}_0, Q'(X_i)\rangle )(d_{i+1} - d_i ) \rangle + \mathcal{O}(e^{-\alpha_1 X_i} |\lambda|(e^{\alpha_0 X_i} + |\lambda|)|d|)
\end{align*}
Using the symmetry relation $\tilde{W}_0 = -R \tilde{W}_0$ from Lemma \ref{nulambdalemma}, this becomes
\begin{equation}\label{PcDid2}
P^c(\lambda)D_i d = 2 \lambda \langle \tilde{W}_0, Q'(X_i) \rangle (d_{i+1} - d_i ) + \mathcal{O}(e^{-\alpha_1 X_i} |\lambda|(e^{\alpha_0 X_i} + |\lambda|)|d|)
\end{equation}

Next, we obtain a bound for $P^c(\lambda)\tilde{L}_3(\lambda)$. We note that applying the projection $P^c(\lambda)$ is equivalent to taking the inner product with $W_0(\lambda)$. When we do this, all of the terms in $\tilde{L}_3(\lambda)$ which are in $E^s(\lambda)$ or $E^u(\lambda)$ are eliminated outright. For the term involving $c_i$, we use a symmetry argument to get
\begin{align*}
\langle &W_0(\lambda), (\Theta_+(X_i; \lambda) - \Theta_-(-X_i; \lambda))c_i\rangle = \langle W_0 + \mathcal{O}(\lambda), (\Theta_+(X_i; \lambda) - \Theta_-(-X_i; \lambda) c_i \rangle \\
&= \langle W_0 + \mathcal{O}(\lambda), (\Theta_+(X_i; 0) - R \Theta_+(X_i; 0)R + \mathcal{O}(|\lambda|e^{-\alpha X_i} ))c_i \rangle \\
&= \langle W_0, \Theta_+(X_i; 0) - R \Theta_+(X_i; 0)R \rangle c_i + \mathcal{O}(|\lambda|e^{-\alpha X_i} |c_i|) \\
&= (\langle W_0, \Theta_+(X_i; 0) \rangle - \langle R W_0 R, \Theta_+(X_i; 0) \rangle) c_i + \mathcal{O}(|\lambda|e^{-\alpha X_i} |c_i|) \\
&= (\langle W_0, \Theta_+(X_i; 0) \rangle - \langle W_0, \Theta_+(X_i; 0) \rangle) c_i + \mathcal{O}(|\lambda|e^{-\alpha X_i} |c_i|) \\
&= \mathcal{O}(|\lambda|e^{-\alpha X_i} |c_i|) 
\end{align*}
since $W_0 R = R W_0 = W_0$. For the remaining terms, we use Lemma \ref{trichotomyprojunconj}(iii) and the bounds on the individual terms from Lemma \ref{Zinv1} to get
\begin{align*}
|P^c(\lambda)\tilde{L}_3&(\lambda)_i(a,b,\tilde{c},c,d)| \leq C |\lambda| e^{-\alpha X_i} \left( |a_i^+| + |a_i^-| + |b_i^+| + |b_{i+1}^-| + |c_i| + e^{-\alpha X_i} |\lambda^2| |d| \right)
\end{align*}
Substituting the bound $A_1$ for $|a_i^+|$ and $|a_i^-|$, this becomes
\begin{align*}
|P^c(\lambda)\tilde{L}_3&(\lambda)_i(a,b,\tilde{c},c,d)| \leq C |\lambda| e^{-\alpha X_i} \left( |b_i^+| + |b_{i+1}^-| + |c_i| + |D_i||d| + e^{-\alpha X_i} |\lambda^2| |d| \right)
\end{align*}

Rearranging equation \eqref{PcDid} and substituting \eqref{PcDid2}, we get
\begin{equation}\label{PcDid3}
\begin{aligned}
P^c(\lambda)&(I + \Theta_+(X_i; \lambda)) \tilde{c}_i = 2 \lambda \langle \tilde{W}_0, Q'(X_i) \rangle (d_{i+1} - d_i ) + |\lambda|)|d|) + P^c(\lambda) \tilde{L}_3(\lambda)_i(a,b,\tilde{c},c,d) \\
&+\mathcal{O}(e^{-\alpha_1 X_i} |\lambda|(e^{\alpha_0 X_i} + |\lambda|)|d|)
\end{aligned}
\end{equation}
Looking at the bound for $\tilde{L}_3$, we see that if $\lambda = 0$, $\tilde{c}_i = 0$ as well. We want our expression for $\tilde{c}_i$ to capture that. We also want to ensure that the lowest order term is $\langle \tilde{W}_0, Q'(X_i) \rangle$. In order to do that, we first need to obtain an improved bound for $\tilde{c}_i$. Expanding the conjugation operator $P_+(X_i; \lambda)$ and recalling that $P^c(\lambda)$ is the identity on $E^c(\lambda)$, we write \eqref{PcDid2} as
\begin{align*}
P^c(\lambda)(I + \Theta_+(X_i; \lambda)) \tilde{c}_i &= P^c(\lambda)D_i d + P^c(\lambda) \tilde{L}_3(
\lambda)_i(a,b,\tilde{c},c,d) \\
\tilde{c}_i &= P^c(\lambda)D_i d - \Theta_+(X_i; \lambda))\tilde{c}_i + P^c(\lambda) \tilde{L}_3(
\lambda)_i(a,b,\tilde{c},c,d)
\end{align*}
Using the bound $A_1$ for $\tilde{c}$ on the RHS, the bound for $\tilde{L}_3$, and equation \eqref{PcDid2} for $P^c(\lambda)D_i d$, we obtain the estimate
\begin{equation}\label{tildecest}
|\tilde{c}_i| \leq C \left( e^{-\alpha X_i}(|\lambda| + e^{-\alpha X_m} )(|b_i^+| + |b_{i+1}^-| + |c_i| + |d|) \right)
\end{equation}

Finally, we solve equation \eqref{PcDid} for $\tilde{c}_i$. Expanding the LHS of \eqref{PcDid} in $\lambda$ and writing $\tilde{c}_i \in E^c(\lambda)$ as $\tilde{c}_i V_0(\lambda)$, we get
\begin{align*}
P^c&(\lambda)P_+(X_i; \lambda)) \tilde{c}_i
= (P^c(0) + \mathcal{O}(\lambda))(P_+(X_i; 0) + \mathcal{O}(\lambda))(V_0 + \mathcal{O}(\lambda))\tilde{c}_i \\
&= P^c(0)P_+(X_i; 0)V_0 \tilde{c}_i + \mathcal{O}(|\lambda||\tilde{c}_i|) \\
&= \langle W_0, V^c(X_i) \rangle \tilde{c}_0 + \mathcal{O}(|\lambda||\tilde{c}_i|) \\
&= \tilde{c}_0 + \mathcal{O}(|\lambda||\tilde{c}_i|)
\end{align*}
since $\langle W_0, V^c(X_i) \rangle = \langle \Psi^c(X_i), V^c(X_i) \rangle = 1$. Using this, \eqref{PcDid2} becomes
\begin{align*}
\tilde{c}_i &= 2 \lambda \langle \tilde{W}_0, Q'(X_i) \rangle (d_{i+1} - d_i ) + P^c(\lambda) \tilde{L}_3(
\lambda)_i(a,b,\tilde{c},c,d) + \mathcal{O}(e^{-\alpha_1 X_i} |\lambda|(e^{-\alpha_0 X_i} + |\lambda|)|d|) + \mathcal{O}(|\lambda||\tilde{c}_i|) \\
&= 2 \lambda \langle \tilde{W}_0, Q'(X_i) \rangle (d_{i+1} - d_i ) + C_2(\lambda)_i(b, c, d)
\end{align*}
Using the estimate for $\tilde{L}_3$ and the estimate \eqref{tildecest} for $\tilde{c}_i$, we have the bound
\begin{align*}
|C_2(\lambda)(b, c, d)_i| \leq C |\lambda| e^{-\alpha X_i} \left( |b_i^+| + |b_{i+1}^-| + |c_i| + (|D_i| + |\lambda|)|d| \right)
\end{align*}
\end{proof}
\end{lemma}

In the next lemma, we solve for the conditions at $x = 0$, which are
\begin{equation}\label{centercond}
\begin{aligned}
P_\pm(0; \lambda) Z_i^\pm(0) &\in \oplus Y^+ \oplus Y^- \oplus \C \Psi(0) \oplus \C \Psi^c(0) \\
P_+(0; \lambda) Z_i^+(0) - P_-(0; \lambda) Z_i^-(0) &\in \C \Psi(0) \oplus \C \Psi^c(0)
\end{aligned}
\end{equation}
Recall that we have the decomposition
\begin{equation}\label{DSdecomp}
\C^{2m+1} = \C Q'(0) \oplus Y^+ \oplus Y^- \oplus \C \Psi(0) \oplus \C \Psi^c(0)
\end{equation}
Thus \eqref{centercond} is equivalent to the three projections
\begin{equation}\label{centercond2}
\begin{aligned}
P(\C Q'(0) ) P_-(0; \lambda) Z_i^-(0) &= 0 \\
P(\C Q'(0) ) P_+(0; \lambda) Z_i^+(0) &= 0 \\
P(Y_i^+ \oplus Y_i^-) ( P_+(0; \lambda) Z_i^+(0) - P_-(0; \lambda) Z_i^-(0) ) &= 0
\end{aligned}
\end{equation}
where the kernel of each projection is the remaining spaces in the direct sum decomposition \eqref{DSdecomp}. We do not need to include $\C Q'(0)$ in the third equation of \eqref{centercond2} since we eliminated any component in $\C Q'(0)$ in the first two equations.

% second inversion lemma
\begin{lemma}\label{Zinv2}
There exist operators
\begin{align*}
B_1: &V_\lambda \times V_c \times V_d \rightarrow V_b \\
A_3: &V_\lambda \times V_c \times V_d \rightarrow V_a \times V_{\tilde{c}} \\
Z_3: &V_\lambda \times V_c \times V_d \rightarrow V_Z
\end{align*}
such that $( (a, \tilde{c}) , b, Z ) = ( A_3(\lambda)(c, d), B_1(\lambda)(c, d), Z_3(\lambda)(c, d) )$ solves \eqref{systemZ}, \eqref{systemmiddle}, \eqref{systemcenter1}, and \eqref{systemcenter2} for any $(c, d)$. These operators are analytic in $\lambda$ and linear in $(c, d)$. Bounds for $B_1$ and $A_3$ are given by
\begin{align}
|B_1&(\lambda)_i(c, d)| \leq C \Big( (|\lambda| + e^{-\alpha X_m})(|e^{\nu(\lambda)X_{i-1}} c_{i-1}| + |e^{-\nu(\lambda)X_i} c_i|) \nonumber \\
&+ e^{-\alpha X_m}(|c_{i-1}| + |c_i|) + (|\lambda| + e^{-\alpha X_m})^2 |d|  \Big)\label{B1bound} \\
|A_3&(\lambda)_i(c, d)|
\leq C \Big(  
e^{-\alpha X_i} (|\lambda| + e^{-\alpha X_m})(|e^{\nu(\lambda)X_{i-1}} c_{i-1}| + |e^{-\nu(\lambda)X_{i+1}}c_{i+1}|) \nonumber \\
&+ e^{-(\alpha - \eta)X_i}|c_i| + e^{-\alpha X_i} e^{-\alpha X_m}(|c_{i-1}| + |c_{i+1}|) + e^{-\alpha X_i} |\lambda^2||d| + |D_i||d| \Big) \label{A3bound}
\end{align} 
Piecewise bounds for $Z_3$ are given by
\begin{equation}\label{Z3bound}
\begin{aligned}
\| Z_3&(\lambda)_i^-(b,c,d) \| \leq C\Big(|c_{i-1}| + e^{-\alpha X_m}(|c_i| + e^{-\alpha X_{i-1}} |c_{i-2}|) + e^{-(\alpha - \eta) X_m}|\lambda|^2|d| + |D_{i-1})|d| \\
&+ |e^{\nu(\lambda)X_{i-1}}c_{i-1}| + (|\lambda| + e^{-\alpha X_m})(|e^{-\nu(\lambda)X_i} c_i| + e^{-\alpha X_{i-1}} |e^{\nu(\lambda)X_{i-2}} c_{i-2}|)\Big) \\
\| Z_3&(\lambda)_i^+(b,c,d) \| \leq C\Big(|c_i| + e^{-\alpha X_m}(|c_{i-1}| + e^{-\alpha X_i} |c_{i+1}|) + e^{-(\alpha - \eta) X_m}|\lambda|^2|d| + |D_i||d|) \\
&+ |e^{-\nu(\lambda)X_i} c_i| + (|\lambda| + e^{-\alpha X_m})(|e^{\nu(\lambda)X_{i-1}} c_{i-1}| + e^{-\alpha X_i} |e^{-\nu(\lambda)X_{i+1}} c_{i+1}|)\Big)
\end{aligned}
\end{equation}
In addition, we can write
\begin{align*}
a_i^+ &= P_i^+(X_i; \lambda) P_0^u(\lambda) D_i d + A_4(\lambda)_i^+(c, d) \\
a_i^- &= -P_i^-(-X_i; \lambda) P_0^s(\lambda) D_i d + A_4(\lambda)_i^-(c, d) \\
\tilde{c}_i &= 2 \lambda \langle \tilde{W}_0, Q'(X_i) \rangle (d_{i+1} - d_i ) + C_4(\lambda)_i(c, d) )
\end{align*}
where $A_4$ and $C_4$ are linear in $(c, d)$, and have piecewise bounds
\begin{align}
|A_4&(\lambda)_i(c, d)|
\leq C \Big(  
e^{-\alpha X_i} (|\lambda| + e^{-\alpha X_m})(|e^{\nu(\lambda)X_{i-1}} c_{i-1}| + |e^{-\nu(\lambda)X_{i+1}}c_{i+1}|) \nonumber \\
&+ e^{-(\alpha - \eta)X_i}|c_i| + e^{-\alpha X_i} e^{-\alpha X_m}(|c_{i-1}| + |c_{i+1}|) + e^{-\alpha X_i} |\lambda^2||d| + e^{-\alpha X_i}|D||d| \Big) \label{A4bound} \\
|C_4&(\lambda)_i(c, d)| \leq C |\lambda| e^{-\alpha X_i} \Big( (|\lambda| + e^{-\alpha X_m})(|e^{\nu(\lambda)X_{i-1}} c_{i-1}| + |e^{-\nu(\lambda)X_{i+1}}c_{i+1}|) \nonumber \\
&+ |c_i| + (|D_i| + |\lambda|)|d| \Big) \label{C4bound}
\end{align}

\begin{proof}
At $x = 0$, the fixed point equations \eqref{Zfpeq} become
\begin{align*}
Z_i^-(0) &= \Phi^s(0, -X_{i-1}; \lambda) a_{i-1}^- + \Phi^u(0, 0; \lambda) b_i^- + \Phi^c(0, -X_{i-1}; \lambda) c_{i-1} \\
&+ \int_{-X_{i-1}}^0 \Phi^s(0, y; \lambda)[P_-(y; \lambda)^{-1} G_i^-(y) P_-(y; \lambda)Z_i^-(y) + \lambda^2 d_i P_-(y; \lambda)^{-1} \tilde{H}_i^-(y)] dy \\
&+ \int_{-X_{i-1}}^0 \Phi^c(0, y; \lambda) [P_-(y; \lambda)^{-1} G_i^-(y) P_-(y; \lambda)Z_i^-(y) + \lambda^2 d_i P_-(y; \lambda)^{-1} \tilde{H}_i^-(y)] dy \\ 
Z_i^+(0) &= \Phi^u(0, X_i; \lambda) a_i^+ + \Phi^s(0, 0; \lambda) b_i^+ + \Phi^c(0, X_i; \lambda) c_i + \Phi^c(0, X_i; \lambda) \tilde{c}_i \\
&+ \int_{X_i}^0 \Phi^u(0, y; \lambda) [P_+(y; \lambda)^{-1} G_i^+(y) P_+(y; \lambda) Z_i^+(y) + \lambda^2 d_i P_+(y; \lambda)^{-1} \tilde{H}_i^+(y)] dy \\
&+ \int_{X_i}^0 \Phi^c(0, y; \lambda) [P_+(y; \lambda)^{-1} G_i^+(y) P_+(y; \lambda) Z_i^+(y) + \lambda^2 d_i P_+(y; \lambda)^{-1} \tilde{H}_i^+(y)] dy \\
\end{align*}

First, recall that at $Q(0)$, the tangent spaces to the stable and unstable manifold are given by
\begin{align*}
T_{Q(0)} W^u(0) &= \R Q'(0) \oplus Y^- \\
T_{Q(0)} W^s(0) &= \R Q'(0) \oplus Y^+
\end{align*}
Thus we have
\begin{align*}
P^-(0)^{-1} Q'(0) &= V^- \in E^u(0) \\
P^+(0)^{-1} Q'(0) &= V^+ \in E^s(0)
\end{align*}
Let
\begin{align*}
E^u(0) &= \C V^- \oplus E^- \\
E^s(0) &= \C V^+ \oplus E^+ \\
\end{align*}
Then we have
\begin{align*}
P^-(0)^{-1} Y^- = E^- \\
P^+(0)^{-1} Y^+ = E^+ \\
\end{align*}
We will use this to decompose $b_i^\pm$ uniquely as $b_i^\pm = x_i^\pm + y_i^\pm$, where $x_i^\pm \in \C V^\pm$ and $y_i^\pm \in E^\pm$. Since the evolution operators $\Phi(0, 0; \lambda)$ involve $\lambda$ but the projections we will be taking are for $\lambda = 0$, we make the following substitutions.
\begin{align*}
\Phi^u(0, 0; \lambda) = P_0^u(\lambda) &= P_0^u(0) + (P_0^u(\lambda) - P_0^u(0)) \\
P_0^c(\lambda) &= P_0^c(0) + (P_0^c(\lambda) - P_0^c(0))
\end{align*}
Using these together with equation \eqref{centerevol} for the evolution $\Phi^c$ on $E^c(\lambda)$, we have
\begin{align*}
Z_i^-&(0) = \Phi^s(0, -X_{i-1}; \lambda) a_{i-1}^- + x_i^- + y_i^- + (P_0^u(\lambda) - P_0^u(0))b_i^- \\
&+ P_0^c(0) e^{\nu(\lambda) X_{i-1}} c_{i-1} + (P_0^c(\lambda) - P_0^c(0)) e^{\nu(\lambda) X_{i-1}} c_{i-1} \\
&+ \int_{-X_{i-1}}^0 \Phi^s(0, y; \lambda) [P_-(y; \lambda)^{-1} G_i^-(y) P_-(y; \lambda)Z_i^-(y) + \lambda^2 d_i P_-(y; \lambda)^{-1} \tilde{H}_i^-(y)] dy \\
&+ \int_{-X_{i-1}}^0 \Phi^c(0, y; \lambda) [P_-(y; \lambda)^{-1} G_i^-(y) P_-(y; \lambda)Z_i^-(y) + \lambda^2 d_i P_-(y; \lambda)^{-1} \tilde{H}_i^-(y)] dy  \\ 
Z_i^+&(0) = \Phi^u(0, X_i; \lambda) a_i^+ + x_i^+ + y_i^+ + (P_0^s(\lambda) - P_0^s(0)) b_i^+ \\
&+ P_0^c(0) e^{-\nu(\lambda)X_i} c_i + (P_0^c(\lambda) - P_0^c(0)) e^{-\nu(\lambda)X_i} c_i \\
&+ P_0^c(0) e^{-\nu(\lambda)X_i} \tilde{c}_i + (P_0^c(\lambda) - P_0^c(0)) e^{-\nu(\lambda)X_i} \tilde{c}_i \\
&+ \int_{X_i}^0 \Phi^u(0, y; \lambda) [P_+(y; \lambda)^{-1} G_i^+(y) P_+(y; \lambda) Z_i^+(y) + \lambda^2 d_i P_+(y; \lambda)^{-1} \tilde{H}_i^+(y)] dy \\
&+ \int_{X_i}^0 \Phi^c(0, y; \lambda) [P_+(y; \lambda)^{-1} G_i^+(y) P_+(y; \lambda) Z_i^+(y) + \lambda^2 d_i P_+(y; \lambda)^{-1} \tilde{H}_i^+(y)] dy 
\end{align*}
Finally, we apply the conjugation operators $P_\pm(0; \lambda)$ in \eqref{centercond2}. For the $c$, $\tilde{c}$, and $b$ terms, we expand the conjugations operators in $\lambda$ as
\[
P_\pm(0; \lambda) = P_\pm(0; 0) + (P_\pm(0; \lambda) - P_\pm(0; 0))
\]
to obtain the expressions
\begin{align*}
P_-&(0; \lambda) Z_i^-(0) = P_-(0; 0)( x_i^- + y_i^- + P_0^c(0) e^{\nu(\lambda) X_{i-1}} c_{i-1} ) \\
&+ P_-(0; \lambda) \Phi^s(0, -X_{i-1}; \lambda) a_{i-1}^- + (P_-(0; \lambda) - P_-(0; 0))b_i^- + P_-(0; \lambda)(P_0^u(\lambda) - P_0^u(0))b_i^- \\
&+ (P_-(0; \lambda) - P_-(0; 0)) P_0^c(0) e^{\nu(\lambda) X_{i-1}} c_{i-1} + P_-(0; \lambda) (P_0^c(\lambda) - P_0^c(0)) e^{\nu(\lambda) X_{i-1}} c_{i-1} \\
&+ P_-(0; \lambda) \int_{-X_{i-1}}^0 \Phi^s(0, y; \lambda) [P_-(y; \lambda)^{-1} G_i^-(y) P_-(y; \lambda)Z_i^-(y) + \lambda^2 d_i P_-(y; \lambda)^{-1} \tilde{H}_i^-(y)] dy \\
&+ P_-(0; \lambda) \int_{-X_{i-1}}^0 \Phi^c(0, y; \lambda) [P_-(y; \lambda)^{-1} G_i^-(y) P_-(y; \lambda)Z_i^-(y) + \lambda^2 d_i P_-(y; \lambda)^{-1} \tilde{H}_i^-(y)] dy 
\end{align*}
and
\begin{align*}
P_+&(0; \lambda) Z_i^+(0) = P_+(0; 0)( x_i^+ + y_i^+ + P_0^c(0) e^{-\nu(\lambda)X_i} c_i + P_0^c(0) e^{-\nu(\lambda)X_i} \tilde{c}_i )\\
&+ P_+(0; \lambda) \Phi^u(0, X_i; \lambda) a_i^+ + (P_+(0; \lambda) - P_+(0; 0)) b_i^+ + P_+(0; \lambda) (P_0^s(\lambda) - P_0^s(0)) b_i^+ \\
&+ (P_+(0; \lambda) - P_+(0; 0))P_0^c(0) e^{-\nu(\lambda)X_i} c_i + P_+(0; \lambda) (P_0^c(\lambda) - P_0^c(0)) e^{-\nu(\lambda)X_i} c_i \\
&+ (P_+(0; \lambda) - P_+(0; 0))P_0^c(0) e^{-\nu(\lambda)X_i} \tilde{c}_i + P_+(0; \lambda) (P_0^c(\lambda) - P_0^c(0)) e^{-\nu(\lambda)X_i} \tilde{c}_i \\
&+ P_+(0; \lambda) \int_{X_i}^0 \Phi^u(0, y; \lambda) [P_+(y; \lambda)^{-1} G_i^+(y) P_+(y; \lambda) Z_i^+(y) + \lambda^2 d_i P_+(y; \lambda)^{-1} \tilde{H}_i^+(y)] dy \\
&+ P_+(0; \lambda) \int_{X_i}^0 \Phi^c(0, y; \lambda) [P_+(y; \lambda)^{-1} G_i^+(y) P_+(y; \lambda) Z_i^+(y) + \lambda^2 d_i P_+(y; \lambda)^{-1} \tilde{H}_i^+(y)] dy \\
\end{align*}

With this setup, the projections on $Q'(0)$ and $Y^+ \oplus Y^-$ will either eliminate or act as the identity on the terms in the first lines of $P_i^-(0; \lambda) Z_i^-(0)$ and $P_i^+(0; \lambda) Z_i^+(0)$. Thus, applying the projections in \eqref{centercond2}, we obtain an expression of the form
\begin{equation}\label{projxy}
\begin{pmatrix}x_i^- \\ x_i^+ \\ 
y_i^+ - y_i^- \end{pmatrix} + L_4(\lambda)_i(b, c, d) = 0
\end{equation}
To get a bound on $L_4$, we will bound the individual terms involved. 

\begin{enumerate}
\item For the $a_i$ terms, we substitute $A_1$ from Lemma \ref{Zinv1} and use the bound \eqref{A1bound} to get
\begin{align*}
|P_-(0; \lambda) \Phi^s(0, -X_{i-1}; \lambda) a_{i-1}^-| 
&\leq C \left( e^{-2 \alpha X_{i-1}} (|b_{i-1}^+| + |b_i^-| + |c_{i-1}| + |\lambda^2||d|) + e^{- \alpha X_{i-1}} |D_{i-1}||d| \right)\\
|P_+(0; \lambda) \Phi^u(0, X_i; \lambda) a_i^+| 
&\leq C \left( e^{-2 \alpha X_i} (|b_i^+| + |b_{i+1}^-| + |c_i| + |\lambda|^2|d|) + e^{-\alpha X_i}|D_i||d| \right)
\end{align*}

\item For the $b_i$ and $c_i$ terms, since the conjugation operators $P_\pm(x; \lambda)$ and eigenprojections $P_0^{s/u}(\lambda)$ are smooth in $\lambda$, we have
\begin{align*}
|(P_-(0; \lambda) &- P_-(0; 0))b_i^- + P_-(0; \lambda)(P_0^u(\lambda) - P_0^u(0))b_i^-| \\
&\leq C |\lambda| |b_i^-|
\end{align*}
The $b_i^+$ is similar. For the $c_i$ terms, we similarly have
\begin{align*}
|(P_-(0; \lambda) &- P_-(0; 0)) P_0^c(0) e^{\nu(\lambda) X_{i-1}} c_{i-1} + P_-(0; \lambda) (P_0^c(\lambda) - P_0^c(0)) e^{\nu(\lambda) X_{i-1}} c_{i-1} | \leq C|\lambda||e^{\nu(\lambda) X_{i-1}} c_{i-1}|
\end{align*}
and
\begin{align*}
|(P_+(0; \lambda) &- P_+(0; 0))P_0^c(0) e^{-\nu(\lambda)X_i} c_i + P_+(0; \lambda) (P_0^c(\lambda) - P_0^c(0)) e^{-\nu(\lambda)X_i} c_i| \leq C |\lambda||e^{-\nu(\lambda)X_i} c_i|
\end{align*}

\item For the $\tilde{c}_i$ terms, we use the expression for $\tilde{c}_i$ from Lemma \ref{lemma:tildec1} and the bounds on $C_2$ from that lemma, to get
\begin{align*}
|(P_+&(0; \lambda) - P_+(0; 0))P_0^c(0) e^{-\nu(\lambda)X_i} \tilde{c}_i + P_+(0; \lambda) (P_0^c(\lambda) - P_0^c(0)) e^{-\nu(\lambda)X_i} \tilde{c}_i| \\
&\leq C |\lambda| e^{-|\nu(\lambda)|X_i} |\lambda| e^{-\alpha X_i} \left( |b_i^+| + |b_{i+1}^-| + |c_i| + |d| \right) \\
&\leq C |\lambda|^2 e^{-(\alpha - \eta)X_i} \left( |b_i^+| + |b_{i+1}^-| + |c_i| + |d| \right) 
\end{align*}

\item The bound on the integral terms is determined by the integrals involving the center subspace, since the other integrals will have stronger bounds. For the integrals involving $Z$, we use $Z_2$ from Lemma \ref{Zinv2} and the bound on $G_i^\pm$ from Lemma \ref{stabestimateslemma} to get
\begin{align*}
&\left| P_i^+(0; \lambda) \int_{X_i}^0 \Phi^c(0, y; \lambda) P_+(y; \lambda)^{-1} G_i^+(y) P_+(y; \lambda) Z_i^+(y) dy \right| \\
&\leq C \| Z_i^+(y) \| \int_0^{X_i} e^{\eta y} \left( e^{-\alpha_0 X_i} e^{-\alpha_0(X_i - y) } + e^{-2 \alpha_0 X_{i-1}} e^{-\alpha_0 y} \right) dy \\
&\leq C \| Z_i^+(y) \| \left( e^{-\alpha X_i} \int_0^{X_i} e^{-\alpha_0(X_i - y) } dy + e^{-2 \alpha_0 X_{i-1}} \int_0^{X_i} e^{-(\alpha_0 - \eta) y}dy \right) \\
&\leq C e^{-\alpha X_m} (|b| + |e^{-\nu(\lambda)X_i} c_i| + |c_i| + e^{-(\alpha - \eta) X_m}|\lambda|^2|d| + |D_i||d|)
\end{align*}
Similarly,
\begin{align*}
&\left| \int_{-X_{i-1}}^0 \Phi^c(0, y; \lambda) P_-(y; \lambda)^{-1} G_i^-(y) P_-(y; \lambda)Z_i^-(y) dy \right| \\
&\leq C e^{-\alpha X_m} (|b| + |e^{-\nu(\lambda)X_i} c_i| + |c_i| + e^{-(\alpha - \eta) X_m}|\lambda|^2|d| + |D_i||d|)
\end{align*}

\item For the integrals not involving $Z$, we have
\begin{align*}
\left| \lambda^2 d_i P_-(0; \lambda) \int_{-X_{i-1}}^0 \Phi^c(0, y; \lambda) P_-(y; \lambda)^{-1} \tilde{H}_i^-(y) dy \right| &\leq C |\lambda|^2 |d| \int_{-X_{i-1}}^0 e^{-\eta y} e^{\alpha_1 y} dy \\
&\leq C |\lambda|^2 |d|
\end{align*}
The other term is similar.
\end{enumerate}

Combining all of these individual bounds together, using the fact that $D_i = \mathcal{O}(e^{-\alpha X_m})$, and simplifying, we obtain the bound for $L_4(\lambda)_i(b, c, d)$.
\begin{align*}
L_4(\lambda)_i&(b, c, d) \leq 
C\Big( (|\lambda| + e^{-\alpha X_m})|b|  
+ e^{-\alpha X_m}(|c_{i-1}| + |c_i|) \\
&+ (|\lambda| + e^{-\alpha X_m})(|e^{\nu(\lambda)X_{i-1}} c_{i-1}| + |e^{-\nu(\lambda)X_i} c_i|) + (|\lambda| + e^{-\alpha X_m})^2 |d|  \Big) 
\end{align*}
Since $|\lambda|, e^{-\alpha X_m} < \delta$, this becomes
\begin{align*}
L_4(\lambda)_i&(b, c, d) \leq 
C \delta |b| + C \Big( (|\lambda| + e^{-\alpha X_m})|b|  
+ e^{-\alpha X_m}(|c_{i-1}| + |c_i|) \\
&+ (|\lambda| + e^{-\alpha X_m})(|e^{\nu(\lambda)X_{i-1}} c_{i-1}| + |e^{-\nu(\lambda)X_i} c_i|) + (|\lambda| + e^{-\alpha X_m})^2 |d|  \Big) 
\end{align*}
which is uniform in $|b|$. Define the map
\[
J_2: \left( \bigoplus_{j=1}^n \C V^+ \oplus \C V^- \right) \oplus
\left( \bigoplus_{j=1}^n E^+ \oplus E^- \right) 
\rightarrow \bigoplus_{j=1}^n \C V^+ \oplus \C V^- \oplus (E^+ \oplus E^-)
\]
by 
\[
J_2( (x_i^+, x_i^-),(y_i^+, y_i^-))_i = ( x_i^+, x_i^-, y_i^+ - y_i^- )
\]
Since $\C^{2m} = E^s(0) \oplus E^u(0) = \C V^+ \oplus \C V^- \oplus (E^+ \oplus E^-)$, $J_2$ is an isomorphism. Using this as the fact that $b_i = (x_i^- + y_i^-, x_i^+ + y_i^+)$, we can write \eqref{projxy} as
\begin{equation}\label{projxy2}
J_2( (x_i^+, x_i^-),(y_i^+, y_i^-))_i 
+ L_4(\lambda)_i(b_i, 0, 0) + L_4(\lambda)_i(0, c, d) = 0
\end{equation}
Consider the map
\begin{align*}
S_2(b)_i &= J_2( (x_i^+, x_i^-),(y_i^+, y_i^-))_i 
+ L_4(\lambda)_i(b_i, 0, 0) 
\end{align*}
Substituting this in \eqref{projxy2}, we have
\begin{align*}
S_2(b) &= -L_4(\lambda)(0, c, d)
\end{align*}

For sufficiently small $\delta$, the operator $S_2(b)$ is invertible. Thus we can solve for $b$ to get
\begin{align}
b = B_1(\lambda)(c,d) 
= -S_2^{-1} L_4(\lambda)(0, c, d)
\end{align}
The bound on $B_1$ is given by the bound on $L_4$, where we note which piece is involved.
\begin{align*}
|B_1(\lambda)_i&(c, d)| \leq C \Big( (|\lambda| + e^{-\alpha X_m})(|e^{\nu(\lambda)X_{i-1}} c_{i-1}| + |e^{-\nu(\lambda)X_i} c_i|) \\
&+ e^{-\alpha X_m}(|c_{i-1}| + |c_i|) + (|\lambda| + e^{-\alpha X_m})^2 |d|  \Big)
\end{align*}
We now plug $B_1$ into all of the previous bounds. Plugging $B_1$ into the bound \eqref{A1bound} for $A_1$, we get $A_3$ with bound
\begin{align*}
|A_3&(\lambda)_i(c, d)|
\leq C \Big(  
e^{-\alpha X_i} (|\lambda| + e^{-\alpha X_m})(|e^{\nu(\lambda)X_{i-1}} c_{i-1}| + |e^{-\nu(\lambda)X_{i+1}}c_{i+1}|) \\
&+ e^{-(\alpha - \eta)X_i}|c_i| + e^{-\alpha X_i} e^{-\alpha X_m}(|c_{i-1}| + |c_{i+1}|) + e^{-\alpha X_i} |\lambda^2||d| + |D_i||d| \Big)
\end{align*} 
Next, we plug $B_1$ into the bound $Z_2$ to get $Z_3$ with bounds
\begin{align*}
\| Z_3&(\lambda)_i^-(c,d) \| \leq C\Big(|c_{i-1}| + e^{-\alpha X_m}(|c_i| + e^{-\alpha X_{i-1}} |c_{i-2}|) + e^{-(\alpha - \eta) X_m}|\lambda|^2|d| + |D_{i-1})|d| \\
&+ |e^{\nu(\lambda)X_{i-1}}c_{i-1}| + (|\lambda| + e^{-\alpha X_m})(|e^{-\nu(\lambda)X_i} c_i| + e^{-\alpha X_{i-1}} |e^{\nu(\lambda)X_{i-2}} c_{i-2}|)\Big) \\
\| Z_3&(\lambda)_i^+(c,d) \| \leq C\Big(|c_i| + e^{-\alpha X_m}(|c_{i-1}| + e^{-\alpha X_i} |c_{i+1}|) + e^{-(\alpha - \eta) X_m}|\lambda|^2|d| + |D_i||d|) \\
&+ |e^{-\nu(\lambda)X_i} c_i| + (|\lambda| + e^{-\alpha X_m})(|e^{\nu(\lambda)X_{i-1}} c_{i-1}| + e^{-\alpha X_i} |e^{-\nu(\lambda)X_{i+1}} c_{i+1}|)\Big)
\end{align*}
We can also plug $B_1$ into the bound \eqref{A2bound} for $A_2$ to get $A_4$ with bound
\begin{align*}
|A_4&(\lambda)_i(c, d)|
\leq C \Big(  
e^{-\alpha X_i} (|\lambda| + e^{-\alpha X_m})(|e^{\nu(\lambda)X_{i-1}} c_{i-1}| + |e^{-\nu(\lambda)X_{i+1}}c_{i+1}|) \\
&+ e^{-(\alpha - \eta)X_i}|c_i| + e^{-\alpha X_i} e^{-\alpha X_m}(|c_{i-1}| + |c_{i+1}|) + e^{-\alpha X_i} |\lambda^2||d| + e^{-\alpha X_i}|D||d| \Big)
\end{align*} 
Finally, we plug in $B_1$ into the bound for $C_2$ to get $C_4$ with bound
\begin{align*}
|C_4(\lambda)&(c, d)_i| \leq C |\lambda| e^{-\alpha X_i} \Big( (|\lambda| + e^{-\alpha X_m})(|e^{\nu(\lambda)X_{i-1}} c_{i-1}| + |e^{-\nu(\lambda)X_{i+1}}c_{i+1}|)\\
&+ |c_i| + (|D_i| + |\lambda|)|d| \Big)
\end{align*} 

\end{proof}
\end{lemma}

\subsection{Jump Conditions}

Up to this point, we have solved uniquely for $a$, $b$, and $\tilde{c}$. We will not be able to obtain a unique solution for $c$ and $d$. We will instead compute jump conditions in the direction of $\Psi(0)$ and $\Psi^c(0)$. Obtaining a nontrivial solution for $c_i$ and $d$ will be equivalent to these jump conditions being 0.

First, we compute the jump in the direction of $\Psi^c(0)$. Before we do that, we prove the following lemma regarding inner products with $\Psi^c(0)$ and $\Psi(0)$.

% lemma : inner products with Psi and Psi^c
\begin{lemma}\label{PsiIP}
We have the following expressions involving the inner product with $\Psi^c(0)$.
\begin{enumerate}[(i)]
	\item $\langle \Phi^c(0), P^\pm(0) V \rangle = V$ for all $V \in E^c(0)$.
	\item $\langle \Phi(0), P^\pm(0) V \rangle = 0$ for all $V \in E^c(0)$.
	\item $\langle \Phi^c(0), P^-(0) V \rangle = 0$ and $\langle \Phi(0), P^-(0) V \rangle = 0$ for all $V \in E^u(0)$.
	\item $\langle \Phi^c(0), P^+(0) V \rangle = 0$ and $\langle \Phi(0), P^+(0) V \rangle = 0$ for all $V \in E^s(0)$.
\end{enumerate}
\begin{proof}
For (i), recall that $E^c(0) = \Span\{ V_0 \}$, where $V_0$ is the eigenvector of $A(0)$ corresponding to the eigenvalue 0. Furthermore, the constant function $Z(x) = V_0$ solves $Z' = A(0) Z$ with initial condition $V_0$. Let $W^-(x) = P^-(x) Z(x) = P^-(x) V_0$. By the conjugation lemma, we can write
\[
W^-(x) = V_0 + \mathcal{O}({e^{-\tilde{\alpha}|x|}})
\]
Since the inner product $\langle \Phi^c(x), W^-(x) \rangle$ is constant in $x$, sending $x \rightarrow -\infty$, we conclude by the continuity of the inner product that
\[
\langle \Phi^c(0), W^-(0) \rangle = \langle W_0, V_0 \rangle = 1 
\]
Thus $\langle \Phi^c(0), P^-(0) V \rangle = V$ for all $V \in E^c(0)$. The same holds for $\langle \Phi^c(0), P^+(0) V \rangle$.

For (ii), we use the same argument as in (i), except we look at the inner product $\langle \Phi(x), W^-(x) \rangle$. Since this is constant in $x$, we send $x \rightarrow \infty$. This time, $W^-(x)$ remains bounded, but $\Phi(x)$ decays to 0, thus by the continuity of the inner product, we conclude that $\langle \Phi(0), W^-(0) \rangle = 0$, from which (ii) follows.

For (iii) and (iv), we note that $P^-(0)E^u = \C Q'(0) \oplus Y^-$ and $P^+(0)E^u = \C Q'(0) \oplus Y^+$. The result follows since $\Psi^c(0), \Psi(0) \perp \C Q'(0) \oplus Y^+ \oplus Y^-$.
\end{proof}
\end{lemma}

We can now compute the jump in the direction of $\Psi^c(0)$.

% jump lemma : center adjoint
\begin{lemma}\label{jumpcenteradj}
The jumps in the direction of $\Psi^c(0)$ are given by
\begin{equation}\label{xic}
\begin{aligned}
\xi^c_i = e^{-\nu(\lambda) X_i} c_i - e^{\nu(\lambda) X_{i-1}} c_{i-1} + 2 \lambda e^{-\nu(\lambda)X_i} \langle \tilde{W}_0, Q'(X_i) \rangle (d_{i+1} - d_i )  + R^c(\lambda)_i(c, d)
\end{aligned}
\end{equation}
The remainder term $R^c_i(c, d)$ is analytic in $\lambda$, linear in $(c, d)$, and has bound
\begin{align*}
||R^c&(c, d)_i|| \leq C |\lambda| \Big(
(|\lambda| + e^{-\alpha X_m})(|e^{\nu(\lambda)X_{i-1}}c_{i-1}| + |e^{-\nu(\lambda)X_i}c_i|) \\
&+ (|\lambda| + e^{-\alpha X_m})( e^{-\alpha X_{i-1}} |e^{\nu(\lambda)X_{i-2}}c_{i-2}| + e^{-\alpha X_i} |e^{-\nu(\lambda)X_{i+1}}c_{i+1}|)  \\
&+ e^{-\alpha X_m}(|c_{i+1}|+|c_{i-2}|) + e^{-2 \alpha X_m}(|c_i|+|c_{i-1}|) + (|\lambda| + e^{-\alpha X_m}|\lambda| + e^{-(2 \alpha - \eta) X_m })|d|
\Big)
\end{align*}

The jump conditions can be written as the matrix equation
\begin{equation}\label{matrixjumpc}
(K(\lambda) + C_1 K(\lambda) + K_1(\lambda) + C_2) c + (\lambda \tilde{A} + D_1) d = 0
\end{equation}
where
\begin{align*}
K(\lambda) =  
\begin{pmatrix}
e^{-\nu(\lambda)X_1} & & & & & -e^{\nu(\lambda)X_0} \\
-e^{\nu(\lambda)X_1} & e^{-\nu(\lambda)X_2} \\
& -e^{\nu(\lambda)X_2} & e^{-\nu(\lambda)X_3} \\
\vdots & & \vdots & &&  \vdots \\
& & & & -e^{\nu(\lambda)X_{n-1}} & e^{-\nu(\lambda)X_0}
\end{pmatrix}
\end{align*}
and
\begin{align*}
\tilde{A} &= \begin{pmatrix}
-e^{-\nu(\lambda)X_1} k_1 & e^{-\nu(\lambda)X_1} k_1 \\
& -e^{-\nu(\lambda)X_2} k_2 & e^{-\nu(\lambda)X_2} k_2 \\
& \ddots \\
e^{-\nu(\lambda)X_0} k_0 & &  & -e^{-\nu(\lambda)X_0} k_0 & 
\end{pmatrix}
\end{align*}
where 
\begin{equation}\label{defki}
k_i = 2 \langle \tilde{W}_0, Q'(X_i) \rangle
\end{equation}
The matrices have uniform bounds
\begin{align*}
\tilde{A} &= \mathcal{O}(e^{-\alpha X_m}) \\
C_1 &= \mathcal{O}(|\lambda|e^{-\alpha X_m}(|\lambda| + e^{-\alpha X_m})) \\
C_2 &= \mathcal{O}(|\lambda|e^{-\alpha X_m}) \\
D_1 &= \mathcal{O}(|\lambda|(|\lambda| + e^{-\alpha X_m})^2|d|)
\end{align*}
$K_1(\lambda)$ has the same form as $K(\lambda)$, but each term in $K(\lambda)$ has been multiplied by a factor of order $\mathcal{O}(|\lambda|(|\lambda| + e^{-\alpha X_m}))$. The specific forms of $K_1(\lambda)$ and $C_1$ are given in the proof.

\begin{proof}
Recall that $\Psi^c(0) = W_0$. From Lemma \ref{Zinv2}, $P_i^\pm(0; \lambda) Z_i^\pm(0)$ are given by
\begin{align*}
P_-&(0; \lambda) Z_i^-(0) = P_-(0; 0)( b_i^- + P_0^c(0) e^{\nu(\lambda) X_{i-1}} c_{i-1} ) \\
&+ P_-(0; \lambda) \Phi^s(0, -X_{i-1}; \lambda) a_{i-1}^- + (P_-(0; \lambda) - P_-(0; 0))b_i^- + P_-(0; \lambda)(P_0^u(\lambda) - P_0^u(0))b_i^- \\
&+ (P_-(0; \lambda) - P_-(0; 0)) P_0^c(0) e^{\nu(\lambda) X_{i-1}} c_{i-1} + P_-(0; \lambda) (P_0^c(\lambda) - P_0^c(0)) e^{\nu(\lambda) X_{i-1}} c_{i-1} \\
&+ P_-(0; \lambda) \int_{-X_{i-1}}^0 \Phi^s(0, y; \lambda) [P_-(y; \lambda)^{-1} G_i^-(y) P_-(y; \lambda)Z_i^-(y) + \lambda^2 d_i P_-(y; \lambda)^{-1} \tilde{H}_i^-(y)] dy \\
&+ P_-(0; \lambda) \int_{-X_{i-1}}^0 \Phi^c(0, y; \lambda) [P_-(y; \lambda)^{-1} G_i^-(y) P_-(y; \lambda)Z_i^-(y) + \lambda^2 d_i P_-(y; \lambda)^{-1} \tilde{H}_i^-(y)] dy
\end{align*}
and
\begin{align*}
P_+&(0; \lambda) Z_i^+(0) = P_+(0; 0)( b_i^+ + P_0^c(0) e^{-\nu(\lambda)X_i} c_i + P_0^c(0) e^{-\nu(\lambda)X_i} \tilde{c}_i )\\
&+ P_+(0; \lambda) \Phi^u(0, X_i; \lambda) a_i^+ + (P_+(0; \lambda) - P_+(0; 0)) b_i^+ + P_+(0; \lambda) (P_0^s(\lambda) - P_0^s(0)) b_i^+ \\
&+ (P_+(0; \lambda) - P_+(0; 0))P_0^c(0) e^{-\nu(\lambda)X_i} c_i + P_+(0; \lambda) (P_0^c(\lambda) - P_0^c(0)) e^{-\nu(\lambda)X_i} c_i \\
&+ (P_+(0; \lambda) - P_+(0; 0))P_0^c(0) e^{-\nu(\lambda)X_i} \tilde{c}_i + P_+(0; \lambda) (P_0^c(\lambda) - P_0^c(0)) e^{-\nu(\lambda)X_i} \tilde{c}_i \\
&+ P_+(0; \lambda) \int_{X_i}^0 \Phi^u(0, y; \lambda) [P_+(y; \lambda)^{-1} G_i^+(y) P_+(y; \lambda) Z_i^+(y) + \lambda^2 d_i P_+(y; \lambda)^{-1} \tilde{H}_i^+(y)] dy \\
&+ P_+(0; \lambda) \int_{X_i}^0 \Phi^c(0, y; \lambda) [P_+(y; \lambda)^{-1} G_i^+(y) P_+(y; \lambda) Z_i^+(y) + \lambda^2 d_i P_+(y; \lambda)^{-1} \tilde{H}_i^+(y)] dy \\
\end{align*}

The leading order terms are those involving $c$ and $\tilde{c}$. For the leading order terms involving $c$, 
\begin{align*}
\langle W_0, P_-(0; 0) P_0^c e^{\nu(\lambda) X_{i-1}} c_{i-1} \rangle &= \langle W_0, V^c(0) e^{\nu(\lambda) X_{i-1}} c_{i-1} \rangle \\
&= e^{\nu(\lambda) X_{i-1}} c_{i-1}
\end{align*}
Similarly,
\begin{align*}
\langle W_0, P_+(0; 0) P^c(0) e^{-\nu(\lambda) X_i} c_i \rangle &= e^{-\nu(\lambda) X_i} c_i 
\end{align*}
For the term involving $\tilde{c}$, using the expression from Lemma \ref{Zinv2}, we have
\begin{align*}
\langle &\Psi^c(0), P_i^+(0; 0) P_0^c(0) e^{-\nu(\lambda)X_i} \tilde{c}_i \rangle
= 2 \lambda e^{-\nu(\lambda)X_i} \langle \tilde{W}_0, Q'(X_i) \rangle (d_{i+1} - d_i ) \\
&+ \mathcal{O}\Big( |\lambda| e^{-(\alpha - \eta) X_i} \Big( (|\lambda| + e^{-\alpha X_m})(|e^{\nu(\lambda)X_{i-1}} c_{i-1}| + |e^{-\nu(\lambda)X_{i+1}}c_{i+1}|) + |c_i| + (|D_i| + |\lambda|)|d| \Big)
\end{align*}

The rest of the terms are higher order.
\begin{enumerate}

\item For the terms involving $a$, we use the bound $A_3$ together with Lemma \ref{centerprojlemma}(iv) and the exponential dichotomy evolution bounds to get
\begin{align*}
|P_-(0; \lambda) &\Phi^s(0, -X_{i-1}; \lambda) a_{i-1}^-| \\
&\leq C |\lambda| e^{-\alpha X_{i-1}} \Big(  
e^{-\alpha X_{i-1}} (|\lambda| + e^{-\alpha X_m})(|e^{\nu(\lambda)X_{i-2}} c_{i-2}| + |e^{-\nu(\lambda)X_i} c_i |) \\
&+ e^{-(\alpha - \eta)X_i}|c_{i-1}| + e^{-\alpha X_{i-1}} e^{-\alpha X_m}(|c_{i-2}| + |c_i|) + e^{-\alpha X_{i-1}} |\lambda^2||d| + |D_{i-1}||d| \Big)
\end{align*} 
and
\begin{align*}
|P_+(0; \lambda) &\Phi^u(0, X_i; \lambda) a_i^+| \\
&\leq C |\lambda| e^{-\alpha X_i} \Big( e^{-\alpha X_i} (|\lambda| + e^{-\alpha X_m})(|e^{\nu(\lambda)X_{i-1}} c_{i-1}| + |e^{-\nu(\lambda)X_{i+1}}c_{i+1}|) \\
&+ e^{-(\alpha - \eta)X_i}|c_i| + e^{-\alpha X_i} e^{-\alpha X_m}(|c_{i-1}| + |c_{i+1}|) + e^{-\alpha X_i} |\lambda^2||d| + |D_i||d| \Big)
\end{align*}

\item For the terms involving $b$, the terms $P_-(0; 0) b_i^-$ and $P_+(0, 0)b_i^+$ are eliminated outright when we take the inner product with $W_0$. For the other terms, we use the estimate for $B_1$ from Lemma \ref{Zinv2}.
\begin{align*}
|(P_-&(0; \lambda) - P_-(0; 0))b_i^- + P_-(0; \lambda)(P_0^u(\lambda) - P_0^u(0))b_i^-| \\
&\leq C |\lambda| \Big( (|\lambda| + e^{-\alpha X_m})(|e^{\nu(\lambda)X_{i-1}} c_{i-1}| + |e^{-\nu(\lambda)X_i} c_i|) \\
&+ e^{-\alpha X_m}(|c_{i-1}| + |c_i|) + (|\lambda| + e^{-\alpha X_m})^2 |d|  \Big)
\end{align*}

\item We will do the center integrals first. In both cases, when $\lambda = 0$, the integrands are eliminated by the center projection. For the integrals involving $Z$, we first evaluate
\begin{align*}
P_+(0; \lambda) &\Phi^c(0, y; \lambda) P_+(y; \lambda)^{-1} G_i^+(y) P_+(y; \lambda) Z_i^+(y) \\
&= P_+(0; 0) \Phi^c(0, y; 0) P_+(y; 0)^{-1} G_i^+(y) P_+(y; \lambda) Z_i^+(y) + \mathcal{O}(e^{\eta y}|\lambda||G_i^+(y) Z_i^+(y)|) \\
&= e^{-\nu(\lambda) y} P_+(0; 0) P_+(y; 0)^{-1} G_i^+(y) P_+(y; \lambda) Z_i^+(y) + \mathcal{O}(e^{\eta y}|\lambda||G_i^+(y) Z_i^+(y)|) \\
&= e^{-\nu(\lambda) y} G_i^+(y) P_+(y; \lambda) Z_i^+(y) + \mathcal{O}(e^{\eta y}|\lambda||G_i^+(y) Z_i^+(y)|)
\end{align*}
Since the bottom row of $G_i^\pm(y)$ is all zeros for all $y$, 
\[
\langle W_0, e^{-\nu(\lambda) y} G_i^+(y) P_+(y; \lambda) Z_i^+(y) \rangle = 0
\]
Using this together with the estimate $Z_3$ and the estimate of the integral from Lemma \ref{Zinv2},
\begin{align*}
&\left| \langle W_0, P_+(0; \lambda) \int_{X_i}^0 \Phi^c(0, y; \lambda) P_+(y; \lambda)^{-1} G_i^+(y) P_+(y; \lambda) Z_i^+(y) dy \rangle \right| \\
&\leq C |\lambda| e^{-\alpha X_m}\Big(|c_i| + e^{-\alpha X_m}(|c_{i-1}| + e^{-\alpha X_i} |c_{i+1}|) + e^{-(\alpha - \eta) X_m}|\lambda|^2|d| + |D_i||d|) \\
&+ |e^{-\nu(\lambda)X_i} c_i| + (|\lambda| + e^{-\alpha X_m})(|e^{\nu(\lambda)X_{i-1}} c_{i-1}| + e^{-\alpha X_i} |e^{-\nu(\lambda)X_{i+1}} c_{i+1}|)\Big)
\end{align*}
Similarly,
\begin{align*}
&\left| \langle W_0, P_-(0; \lambda) \int_{-X_{i-1}}^0 \Phi^c(0, y; \lambda) P_-(y; \lambda)^{-1} G_i^-(y) P_-(y; \lambda)Z_i^-(y) dy \rangle \right| \\
&\leq C |\lambda| e^{-\alpha X_m} \Big(|c_{i-1}| + e^{-\alpha X_m}(|c_i| + e^{-\alpha X_{i-2}} |c_{i-2}|) + e^{-(\alpha - \eta) X_m}|\lambda|^2|d| + |D_{i-1})|d| \\
&+ |e^{\nu(\lambda)X_{i-1}}c_{i-1}| + (|\lambda| + e^{-\alpha X_m})(|e^{-\nu(\lambda)X_i} c_i| + e^{-\alpha X_{i-1}} |e^{\nu(\lambda)X_{i-2}} c_{i-2}|)\Big)
\end{align*}

For the integrals not involving $Z$, we first evaluate
\begin{align*}
P_-&(0; \lambda) \Phi^c(0, y; \lambda) P_-(y; \lambda) \tilde{H}_i^-(y) \\
&= P_-(0; 0) \Phi^c(0, y; 0) P_-(y; 0) \tilde{H}_i^-(y) + \mathcal{O}(e^{-\eta y}|\lambda||\tilde{H}_i^-(y)|) \\
&= e^{-\nu(\lambda) y} P_+(0; 0) P_+(y; 0)^{-1}\tilde{H}_i^-(y) + \mathcal{O}(e^{-\eta y}|\lambda||\tilde{H}_i^-(y)|)  \\
&= e^{-\nu(\lambda) y} \tilde{H}_i^-(y) + \mathcal{O}(e^{-\eta y}|\lambda||\tilde{H}_i^-(y)|) 
\end{align*}
Since the last component of $\tilde{H}_i^\pm(y)$ is 0 for all $y$,
\[
\langle W_0, e^{-\nu(\lambda) y} \tilde{H}_i^-(y) \rangle = 0
\]
Thus we have
\begin{align*}
&\left| \langle W_0, P_-(0; \lambda) \lambda^2 d_i \int_{-X_{i-1}}^0 \Phi^c(0, y; \lambda) P_-(y; \lambda) \tilde{H}_i^-(y) dy \rangle \right| \\
&\leq C |\lambda|^3 |d| \int_{-X_{i-1}}^0 e^{\alpha_1 y} e^{-\eta y} dy \\
&\leq C |\lambda|^3 |d| 
\end{align*}
The other integral is similar.

\item Finally, we look at the noncenter integrals. For the integral involving $Z$, we first evaluate 
\begin{align*}
P_-(0; &\lambda) \Phi^s(0, y; \lambda) P_-(y; \lambda)^{-1} G_i^-(y) P_-(y; \lambda)Z_i^-(y) \\
&= P_-(0; 0) \Phi^s(0, y; 0) P_-(y; 0)^{-1}G_i^-(y) P_-(y; \lambda)Z_i^-(y) + \mathcal{O}(|\lambda| G_i^-(y) Z_i^-(y)) \\
&= \tilde{P}_-^s(0)G_i^-(y) P_-(y; \lambda)Z_i^-(y) + \mathcal{O}(|\lambda| G_i^-(y) Z_i^-(y))
\end{align*}
From Lemma \ref{centerprojlemma},
\[
\langle W_0, \tilde{P}_-^s(0)G_i^-(y) P_-(y; \lambda)Z_i^-(y) \rangle = 0
\]
The remaining integral has the same bound as for the center integral involving $Z$. The integral not involing $Z$ is handled in the a similar way. Thus the estimates for the noncenter integrals are the same as for the center integrals.

\end{enumerate}

Putting all of this together, we obtain the center jump expressions
\begin{align*}
\xi^c_i = e^{-\nu(\lambda) X_i} c_i^- - e^{\nu(\lambda) X_{i-1}} c_{i-1}^- + 2 \lambda e^{-\nu(\lambda)X_i} \langle \tilde{W}_0, Q'(X_i) \rangle (d_{i+1} - d_i )  + R^c(\lambda)_i(c, d)
\end{align*}
The remainder term $R^c_i(c, d)$ has bound
\begin{align*}
||R^c&(c, d)_i|| \leq C |\lambda| \Big(
(|\lambda| + e^{-\alpha X_m})(|e^{\nu(\lambda)X_{i-1}}c_{i-1}| + |e^{-\nu(\lambda)X_i}c_i|) \\
&+ (|\lambda| + e^{-\alpha X_m})( e^{-\alpha X_{i-1}} |e^{\nu(\lambda)X_{i-2}}c_{i-2}| + e^{-\alpha X_i} |e^{-\nu(\lambda)X_{i+1}}c_{i+1}|)  \\
&+ e^{-\alpha X_m}(|c_{i+1}|+|c_{i-2}|) + e^{-2 \alpha X_m}(|c_i|+|c_{i-1}|) + (|\lambda| + e^{-\alpha X_m})^2 |d|
\Big)
\end{align*}

To write this in matrix form, let
\begin{align*}
K(\lambda) =  
\begin{pmatrix}
e^{-\nu(\lambda)X_1} & & & & & -e^{\nu(\lambda)X_0} \\
-e^{\nu(\lambda)X_1} & e^{-\nu(\lambda)X_2} \\
& -e^{\nu(\lambda)X_2} & e^{-\nu(\lambda)X_3} \\
\vdots & & \vdots & &&  \vdots \\
& & & & -e^{\nu(\lambda)X_{n-1}} & e^{-\nu(\lambda)X_0}
\end{pmatrix}
\end{align*}
The leading order terms in the center jump expression involving $c$ are given by $K(\lambda)c$, where $c = (c_1, \dots, c_{n-1}, c_0)^T$. We will organize the remainder terms into three groups.

\begin{enumerate}
\item Let $G_i$ be the sum of the remainder terms of the form $e^{\pm \nu(\lambda) X_i} c_i$. Then 
\[
G_i = \gamma_{i,i-1} e^{\nu(\lambda)X_{i-1}}c_{i-1} + \gamma_{i,i} e^{-\nu(\lambda)X_i}c_i + \gamma_{i,i-2} e^{\nu(\lambda)X_{i-2}}c_{i-2} + \gamma_{i,i+1} e^{-\nu(\lambda)X_{i+1}}c_{i+1}
\] 
where from the remainder bound we have for the coefficients $\gamma_{i, j}$
\begin{align*}
\gamma_{i,i-1}, \gamma_{i,i} &= \mathcal{O}(|\lambda|(|\lambda| + e^{-\alpha X_m})) \\
\gamma_{i,i-2} &= \mathcal{O}(e^{-\alpha X_{i-1}}|\lambda|(|\lambda| + e^{-\alpha X_m})) \\
\gamma_{i,i+1} &= \mathcal{O}(e^{-\alpha X_i}|\lambda|(|\lambda| + e^{-\alpha X_m}))
\end{align*}
Adding and subtracting $e^{\nu(\lambda)X_i}c_i$ and $e^{-\nu(\lambda)X_{i-1}}c_{i-1}$, this becomes
\begin{align*}
G_i &= \gamma_{i,i-1} e^{\nu(\lambda)X_{i-1}}c_{i-1} + \gamma_{i,i} e^{-\nu(\lambda)X_i}c_i + \gamma_{i,i-2} ( e^{\nu(\lambda)X_{i-2}}c_{i-2} - e^{-\nu(\lambda)X_{i-1}}c_{i-1}) \\
&+ \gamma_{i,i-2} e^{-\nu(\lambda)X_{i-1}}c_{i-1} + \gamma_{i,i+1} (e^{-\nu(\lambda)X_{i+1}}c_{i+1} - e^{\nu(\lambda)X_i}c_i) + \gamma_{i,i+1} e^{\nu(\lambda)X_i}c_i
\end{align*}
Next, we note that
\begin{align*}
\gamma_{i,i-2} e^{-\nu(\lambda)X_{i-1}}c_{i-1} &= \mathcal{O}(e^{-(\alpha - 3 \eta) X_{i-1}}|\lambda|(|\lambda| + e^{-\alpha X_m})e^{\nu(\lambda)X_{i-1}}c_{i-1} \\
\gamma_{i,i+1} e^{\nu(\lambda)X_i}c_i &= \mathcal{O}(e^{-(\alpha - 3 \eta) X_i}|\lambda|(|\lambda| + e^{-\alpha X_m})e^{-\nu(\lambda)X_i}c_i
\end{align*}
Both of these coefficients are higher order than $\gamma_{i,i-1}$ and $\gamma_{i,i}$. Substituting these into $G_i$, collecting terms, and keeping the notation $\gamma_{i,j}$ for the resulting coefficients, we have
\begin{align*}
G_i(&\lambda) = \gamma_{i,i-1} e^{\nu(\lambda)X_{i-1}}c_{i-1} + \gamma_{i,i} e^{-\nu(\lambda)X_i}c_i \\
&+ \gamma_{i,i-2} ( e^{\nu(\lambda)X_{i-2}}c_{i-2} - e^{-\nu(\lambda)X_{i-1}}c_{i-1}) + \gamma_{i,i+1} (e^{-\nu(\lambda)X_{i+1}}c_{i+1} - e^{\nu(\lambda)X_i}c_i)
\end{align*}
where we have the same bounds on the coefficients $\gamma_{i,j}$. Using these, we can write the terms of the form $e^{\pm \nu(\lambda) X_i} c_i$ in matrix form as
\[
(K(\lambda) + C_1 K(\lambda) + K_1(\lambda)) c
\]
where $K_1(\lambda)$ is ``$\gamma-$perturbation'' of $K(\lambda)$ given by
\begin{align*}
K_1(\lambda) =  
\begin{pmatrix}
e^{-\nu(\lambda)X_1} \gamma_{1,1} & & & & & e^{\nu(\lambda)X_0}\gamma_{1,0} \\
e^{\nu(\lambda)X_1}\gamma_{2,1} & e^{-\nu(\lambda)X_2}\gamma_{2,2} \\
& e^{\nu(\lambda)X_2}\gamma_{3,2} & e^{-\nu(\lambda)X_3}\gamma_{3,3} \\
\vdots & & \vdots & &&  \vdots \\
& & & & e^{\nu(\lambda)X_{n-1}}\gamma_{0,n-1} & e^{-\nu(\lambda)X_0}\gamma_{0,0} 
\end{pmatrix}
\end{align*}
with 
\[
\gamma_{i,i-1}, \gamma_{i,i} = \mathcal{O}(|\lambda|(|\lambda| + e^{-\alpha X_m}))
\] 
and $C_1$ is the periodic, banded matrix
\begin{align*}
C_1 &= \begin{pmatrix}
0 & \gamma_{1,2} & 0 & 0 & \dots & 0 & -\gamma_{n-1,0} & 0 \\
0 & 0 & \gamma_{2,3} & 0 & \dots & 0 & 0 & -\gamma_{2,1} \\
-\gamma_{3,1} & 0 & 0 & \gamma_{3,4} & \dots & 0 & 0 & 0 \\
&  & & \ddots  \\
0 & 0 & 0 & 0 & \dots & 0 & 0 & \gamma_{n-1,0} \\
\gamma_{0,1} & 0 & 0 & 0 & \dots & -\gamma_{0, n-2} & 0 & 0 
\end{pmatrix}
\end{align*}
which has uniform bound
\begin{align*}
||C_1|| &= 
\mathcal{O}(e^{-\alpha X_m}|\lambda|(|\lambda| + e^{-\alpha X_m}))
\end{align*}

\item For the terms involving $c$ without the exponential, we write them in matrix form as $C_2 c$, where
\[
C_2 = \mathcal{O}(|\lambda|e^{-\alpha X_m})
\]

\item For the terms involving $D$, we first separate out leading order terms $2 \lambda e^{-\nu(\lambda)X_i} \langle \tilde{W}_0, Q'(X_i) \rangle (d_{i+1} - d_i )$, and place them in a matrix $\lambda \tilde{A}$, where
\begin{align*}
\tilde{A} &= \begin{pmatrix}
-e^{-\nu(\lambda)X_1} k_1 & e^{-\nu(\lambda)X_1} k_1 \\
& -e^{-\nu(\lambda)X_2} k_2 & e^{-\nu(\lambda)X_2} k_2 \\
& \ddots \\
e^{-\nu(\lambda)X_0} k_0 & &  & -e^{-\nu(\lambda)X_0} k_0 & 
\end{pmatrix}
\end{align*}
and
\[
k_i = 2 \langle \tilde{W}_0, Q'(X_i) \rangle.
\]
Since $Q'(X_i) = \mathcal{O}(e^{-\alpha_0 X_i})$, $e^{-\nu(\lambda)X_i} k_i = \mathcal{O}(e^{-\alpha X_i})$, thus we have the uniform bound
\[
\tilde{A} = \mathcal{O}(e^{-\alpha X_m})
\]

Let $D_1$ be the matrix we get from the terms involving $d$ in $R^c(\lambda)_i(c, d)$. Then $D_1$ has uniform bound
\begin{align*}
D_1 &= \mathcal{O}(|\lambda|(|\lambda| + e^{-\alpha X_m})^2|d|)
\end{align*}
\end{enumerate}

Combining everything, the center jump condition can be written in matrix form as
\[
(K(\lambda) + C_1 K(\lambda) + K_1(\lambda) + C_2) c + (\lambda \tilde{A} + D_1) d = 0
\]
\end{proof}
\end{lemma}

Finally, we compute the jump in the direction of $\Psi(0)$.
% lemma : jump in decaying adjoint direction
\begin{lemma}\label{jumpadj}
The jumps in the direction of $\Psi(0)$ are given by
\begin{equation}\label{jumpPsi0}
\begin{aligned}
\xi_i &= \langle \Psi(X_i), Q'(-X_i) \rangle (d_{i+1} - d_i ) + \langle \Psi(-X_{i-1}), Q'(X_{i-1}) \rangle (d_i - d_{i-1} ) \\
&+ p_1 \lambda( e^{-\nu(\lambda)X_i}c_i + e^{\nu(\lambda)X_{i-1}}c_{i-1})
- \lambda_2 d_i M + R_i(\lambda)(c, d)
\end{aligned}
\end{equation}
$M$ is the higher order Melnikov integral
\begin{equation}\label{M}
M = \int_{-\infty}^\infty \langle \Psi(y), B H(y) \rangle dy
\end{equation}
and $p_1 = \langle \Psi(0), \partial_\lambda V^+(0; 0) \rangle$. The remainder term $R(\lambda)(c, d)$ is analytic in $\lambda$, linear in $(c, d)$, and has piecewise bound
\begin{align*}
|R(\lambda)_i&(c_i, d)| \leq C \Big( (|\lambda| + e^{-\alpha X_m})(|\lambda| + e^{-(\alpha - \eta) X_m})(|e^{\nu(\lambda)X_{i-1}}c_{i-1}| + |e^{-\nu(\lambda)X_i}c_i|) \\
&+ e^{-(\alpha - \eta)X_m} (|\lambda| + e^{-\alpha X_m})(e^{-\alpha X_{i-1}} |e^{\nu(\lambda)X_{i-2}}c_{i-2}| + e^{-\alpha X_i} |e^{-\nu(\lambda)X_{i+1}}c_{i+1}|) \\
&+ (|\lambda| + e^{-\alpha X_m})(|\lambda| + e^{-(\alpha - \eta) X_m})(|c_i| + |c_{i-1}|) \\
&+ e^{-(2 \alpha - \eta)X_m} (|\lambda| + e^{-\alpha X_m})(|c_{i-2}| + |c_{i+1}|) 
+ (|\lambda| + e^{-(\alpha - \eta) X_m})(|\lambda| + e^{-\alpha X_m})^2 |d| \Big)
\end{align*}
We can write these conditions in matrix form as
\begin{equation}
(\lambda p_1 \tilde{K}(\lambda) + C_3 K(\lambda) + K_2(\lambda) + C_4 )c + (A - \lambda^2 M I + D_2)d = 0
\end{equation}
The matrix $A$ is given by
\begin{align*}
A &= \begin{pmatrix}
-a_0 -a_1 & a_0 + a_1 \\
-a_0 + a_1 & -a_0 - a_1
\end{pmatrix} && n = 2 \\
A &= \begin{pmatrix}
-a_{n-1} - a_0 & a_0 & & & \dots & a_{n-1}\\
a_0 & -a_0 - a_1 &  a_1 \\
& a_1 & -a_1 - a_2 &  a_2 \\
& & \vdots & & \vdots \\
a_{n-1} & & & & a_{n-2} & -a_{n-2} - a_{n-1} \\
\end{pmatrix} && n > 2
\end{align*}
with
\begin{align*}
a_i &= \langle \Psi(X_i), Q'(-X_i) \rangle \\
\end{align*}
The matrix $K(\lambda)$ is defined in Lemma \ref{jumpcenteradj}. The matrix $\tilde{K}(\lambda)$ is given by
\begin{align*}
\tilde{K}(\lambda) =  
\begin{pmatrix}
e^{-\nu(\lambda)X_1} & & & & & e^{\nu(\lambda)X_0} \\
e^{\nu(\lambda)X_1} & e^{-\nu(\lambda)X_2} \\
& e^{\nu(\lambda)X_2} & e^{-\nu(\lambda)X_3} \\
\vdots & & \vdots & &&  \vdots \\
& & & & e^{\nu(\lambda)X_{n-1}} & e^{-\nu(\lambda)X_0} 
\end{pmatrix}
\end{align*}
The remainder matrices have uniform bounds
\begin{align*}
C_3 &= \mathcal{O}(e^{-(2 \alpha - \eta)X_m} (|\lambda| + e^{-\alpha X_m})) \\
C_4 &= \mathcal{O}((|\lambda| + e^{-\alpha X_m})(|\lambda| + e^{-(\alpha - \eta) X_m})) \\
D_2 &= \mathcal{O}((|\lambda| + e^{-(\alpha - \eta) X_m})(|\lambda| + e^{-\alpha X_m})^2)
\end{align*}
$K_2(\lambda)$ has the same form as $K(\lambda)$, but each term in $K(\lambda)$ has been multiplied by a factor of order $\mathcal{O}((|\lambda| + e^{-\alpha X_m})(|\lambda| + e^{-(\alpha - \eta) X_m}))$. The specific forms of $K_2(\lambda)$ and $C_2$ are given in the proof.

\begin{proof}
The expressions for $P_\pm(0; \lambda) Z_i^\pm(0)$ we will use this time are
\begin{align*}
P_-&(0; \lambda) Z_i^-(0) = P_-(0; 0) b_i^- + P_-(0; \lambda)P_0^c(\lambda) e^{\nu(\lambda) X_{i-1}} c_{i-1}  \\
&+ P_-(0; \lambda) \Phi^s(0, -X_{i-1}; \lambda) a_{i-1}^- + (P_-(0; \lambda) - P_-(0; 0))b_i^- + P_-(0; \lambda)(P_0^u(\lambda) - P_0^u(0))b_i^- \\
&+ P_-(0; \lambda) \int_{-X_{i-1}}^0 \Phi^s(0, y; \lambda) [P_-(y; \lambda)^{-1} G_i^-(y) P_-(y; \lambda)Z_i^-(y) + \lambda^2 d_i P_-(y; \lambda)^{-1} \tilde{H}_i^-(y)] dy \\
&+ P_-(0; \lambda) \int_{-X_{i-1}}^0 \Phi^c(0, y; \lambda) [P_-(y; \lambda)^{-1} G_i^-(y) P_-(y; \lambda)Z_i^-(y) + \lambda^2 d_i P_-(y; \lambda)^{-1} \tilde{H}_i^-(y)] dy
\end{align*}
and
\begin{align*}
P_+&(0; \lambda) Z_i^+(0) = P_+(0; 0) b_i^+ + P_+(0; \lambda) P_0^c(\lambda) e^{-\nu(\lambda)X_i} c_i + P_+(0; \lambda) P_0^c(\lambda) e^{-\nu(\lambda)X_i} \tilde{c}_i \\
&+ P_+(0; \lambda) \Phi^u(0, X_i; \lambda) a_i^+ + (P_+(0; \lambda) - P_+(0; 0)) b_i^+ + P_+(0; \lambda) (P_0^s(\lambda) - P_0^s(0)) b_i^+ \\
&+ P_+(0; \lambda) \int_{X_i}^0 \Phi^u(0, y; \lambda) [P_+(y; \lambda)^{-1} G_i^+(y) P_+(y; \lambda) Z_i^+(y) + \lambda^2 d_i P_+(y; \lambda)^{-1} \tilde{H}_i^+(y)] dy \\
&+ P_+(0; \lambda) \int_{X_i}^0 \Phi^c(0, y; \lambda) [P_+(y; \lambda)^{-1} G_i^+(y) P_+(y; \lambda) Z_i^+(y) + \lambda^2 d_i P_+(y; \lambda)^{-1} \tilde{H}_i^+(y)] dy \\
\end{align*}
As in Lemma \ref{jumpcenteradj}, we begin by computing the leading order terms.

\begin{enumerate}
\item The non-center integral will give us the higher order Melnikov integral. For the piece on $\R^-$ we use the uniform estimate $\tilde{H}_i^-(y) = H(y) + \mathcal{O}(e^{-\alpha_0 X_m})$ from Lemma \ref{stabestimates} to get
\begin{align*}
&\langle \Psi(0), P_-(0; \lambda) \int_{-X_{i-1}}^0 \Phi^s(0, y; \lambda) P_-(y; \lambda)^{-1} \tilde{H}_i^-(y) dy \rangle \\
&= \int_{-X_{i-1}}^0 \langle \Psi(0), P_-(0; \lambda), \Phi^s(0, y; \lambda) P_-(y; \lambda)^{-1} \tilde{H}_i^-(y) \rangle dy \\
&= \int_{-X_{i-1}}^0 \langle \Psi(0), P_-(0; \lambda), \Phi^s(0, y; \lambda) P_-(y; \lambda)^{-1} H(y) \rangle dy + \mathcal{O}({e^{-\alpha_0 X_m}})\\
&= \int_{-X_{i-1}}^0 \langle \Psi(0), \Theta(0, y) H(y) \rangle dy + \mathcal{O}(|\lambda| + {e^{-\alpha X_m}})\\
&= \int_{-X_{i-1}}^0 \langle \Theta(y, 0)^* \Psi_i(0), H(y) \rangle dy + \mathcal{O}(|\lambda| + {e^{-\alpha X_m}})\\
&= \int_{-X_{i-1}}^0 \langle \Psi(y), H(y) \rangle dy + \mathcal{O}(|\lambda| + {e^{-\alpha X_m}})\\
&= \int_{-\infty}^0 \langle \Psi(y), H(y) \rangle dy + \mathcal{O}(|\lambda| + {e^{-\alpha X_m}})
\end{align*}
The piece on $\R^+$ is similar, and gives us the other half of the Melnikov integral.

\item For the terms involving $a$, we use the expressions from Lemma \ref{Zinv2}. For the term involving $a_{i-1}^-$, we have 
\begin{align*}
\langle &\Psi(0), P_-(0; \lambda) \Phi^s(0, -X_{i-1}; \lambda) a_{i-1}^- \rangle \\
&= -\langle \Psi(0), P_-(0; \lambda) \Phi^s(0, -X_{i-1}; \lambda) P_-(-X_i; \lambda) P_0^s(\lambda) D_i d \rangle + \mathcal{O}(e^{-\alpha X_{i-1}}|A_4(\lambda)_i^-(c, d)|) \\
&= -\langle \Psi(0), P_-(0; 0) \Phi^s(0, -X_{i-1}; 0) P_-(-X_i; 0) P_0^s(0) D_i d \rangle + \mathcal{O}(e^{-\alpha X_{i-1}}|A_4(\lambda)_i^-(c, d)| + |\lambda|e^{-2\alpha X_m}|d|) \\
&= -\langle \Psi(0), \Phi^s(0, -X_{i-1}) \tilde{P}_-^s(-X_{i-1}) P_0^s(0) D_i d \rangle + \mathcal{O}(e^{-\alpha X_{i-1}}|A_4(\lambda)_i^-(c, d)| + |\lambda|e^{-2\alpha X_m}|d|) \\
&= -\langle \Phi^s(-X_{i-1}, 0)^* \Psi(0), \tilde{P}_-^s(-X_{i-1}) P_0^s(0) D_i d \rangle + \mathcal{O}(e^{-\alpha X_{i-1}}|A_4(\lambda)_i^-(c, d)| + |\lambda|e^{-2\alpha X_m}|d|) \\
&= -\langle \Psi(-X_{i-1}), P_0^s(0) D_i d \rangle + \mathcal{O}(e^{-\alpha X_{i-1}}|A_4(\lambda)_i^-(c, d)| + e^{-2\alpha X_m}(|\lambda| + e^{-\alpha X_m})|d|)
\end{align*}
For the leading order term, we substitute the expression for $D_{i-1}d$ from Lemma \ref{stabestimates} to get 
\begin{align*}
\langle &\Psi(-X_{i-1}), \tilde{P}_-^s(-X_{i-1}) P_0^s(0) D_i d \rangle = \langle \Psi(-X_{i-1}), P_0^s(0) D_i d \rangle + \mathcal{O}(e^{-3 \alpha X_m}|d|) \\
&= \langle \Psi(-X_{i-1}), P_0^s(0)( Q'(-X_{i-1}) + Q'(X_{i-1})) \rangle (d_i - d_{i-1} ) + \mathcal{O}(e^{-2 \alpha X_m}(|\lambda| + e^{-\alpha X_m})|d|) \\
\end{align*}
Since
\begin{align*}
\langle \Psi(-X_{i-1}), P_0^s(0) Q'(X_{i-1})\rangle
&= \langle \Psi(-X_{i-1}), \tilde{P}^s(X_{i-1}) Q'(X_{i-1})\rangle + \mathcal{O}(e^{-3 \alpha X_m}|d|) \\
&= \langle \Psi(-X_{i-1}), Q'(X_{i-1})\rangle + \mathcal{O}(e^{-3 \alpha X_m})
\end{align*}
and
\begin{align*}
\langle \Psi(-X_{i-1}), P_0^s(0) Q'(-X_{i-1})\rangle
&= \langle \Psi(-X_{i-1}), \tilde{P}^s(-X_{i-1}) Q'(-X_{i-1})\rangle + \mathcal{O}(e^{-3 \alpha X_m}|d|) \\
&= \langle \Psi(-X_{i-1}), Q'(-X_{i-1})\rangle + \mathcal{O}(e^{-3 \alpha X_m}|d|) \\
&= \mathcal{O}(e^{-3 \alpha X_m})
\end{align*}
the leading order term gives us
\begin{align*}
\langle &\Psi(-X_{i-1}), P_0^s(0) D_i d \rangle = \langle \Psi(-X_{i-1}), Q'(X_{i-1}) \rangle (d_i - d_{i-1} ) + \mathcal{O}(e^{-2 \alpha X_m}(|\lambda| + e^{-\alpha X_m})|d|) \\
\end{align*}
Combining all of these, collecting remainder terms, and using the bound for $A_4$ from Lemma \ref{Zinv2}, we have for the $a_{i-1}^-$ term,
\begin{align*}
\langle \Psi(0), &P_-(0; \lambda) \Phi^s(0, -X_{i-1}; \lambda) a_{i-1}^- \rangle = -\langle \Psi(-X_{i-1}), Q'(X_{i-1}) \rangle (d_i - d_{i-1} ) \\
&+ \mathcal{O}\Big(  
e^{-2\alpha X_{i-1}} (|\lambda| + e^{-\alpha X_m})(|e^{\nu(\lambda)X_{i-2}} c_{i-2}| + |e^{-\nu(\lambda)X_i}c_i|) \\
&+ e^{-(2 \alpha - \eta)X_{i-1}}|c_{i-1}| + e^{-2\alpha X_{i-1}} e^{-\alpha X_m}(|c_{i-2}| + |c_i|) + (e^{-\alpha X_m} + |\lambda|)^3 |d| \Big) 
\end{align*}
Similarly, for the $a_i^+$ term, we have
\begin{align*}
\langle \Psi(0), &P_+(0; \lambda) \Phi^u(0, X_i; \lambda) a_i^+ \rangle = \langle \Psi(X_i), Q'(-X_i) \rangle (d_{i+1} - d_i ) \\
&+ \mathcal{O}\Big( e^{-2\alpha X_i} (|\lambda| + e^{-\alpha X_m})(|e^{\nu(\lambda)X_{i-1}} c_{i-1}| + |e^{-\nu(\lambda)X_{i+1}}c_{i+1}|) \nonumber \\
&+ e^{-(2\alpha - \eta)X_i}|c_i| + e^{-2\alpha X_i} e^{-\alpha X_m}(|c_{i-1}| + |c_{i+1}|) + (e^{-\alpha X_m} + |\lambda|)^3 |d| \Big) 
\end{align*}
\end{enumerate}

The remaining terms will be higher order. Doing these in turn, we have

\begin{enumerate}

\item For the terms involving $b$, we first note that by Lemma \ref{PsiIP}, the terms $P_-(0) b_i^-$ and $P_+(0)b_i^+$ will vanish when we take the inner product with $\Psi(0)$. For the remaining terms, we substitute the estimate for $B_1$ from Lemma \ref{Zinv2}.
\begin{align*}
&|\langle \Psi(0), (P_-(0; \lambda) - P_-(0; 0))b_i^- + P_-(0; \lambda)(P_0^u(\lambda) - P_0^u(0))b_i^- \rangle | \\
&\leq C |\lambda| \Big( (|\lambda| + e^{-\alpha X_m})(|e^{\nu(\lambda)X_{i-1}} c_{i-1}| + |e^{-\nu(\lambda)X_i} c_i|) \\
&+ e^{-\alpha X_m}(|c_{i-1}| + |c_i|) + (|\lambda| + e^{-\alpha X_m})^2 |d|  \Big)
\end{align*}

\item For the terms involving $c$, we use Lemmas \ref{centerprojlemma} and \ref{lemma:VpmPsiIP} to get
\begin{align*}
\langle \Psi(0), &P_-(0; \lambda) P_0^c(\lambda) e^{\nu(\lambda) X_{i-1}} c_{i-1} \rangle = \langle \Psi(0), V^-(0; \lambda) e^{\nu(\lambda) X_{i-1}} c_{i-1} \rangle  \\
&= -p_1 \lambda e^{\nu(\lambda) X_{i-1}} c_{i-1} + \mathcal{O}(|\lambda|^2 |e^{\nu(\lambda) X_{i-1}} c_{i-1}|)
\end{align*}
where $p_1 = \langle \Psi(0), \partial_\lambda V^+(0; 0) \rangle$. Similarly, 
\begin{align*}
\langle \Psi(0), P_-(0; \lambda) P_0^c(\lambda) e^{-\nu(\lambda) X_i} c_i \rangle 
&= p_1 \lambda e^{-\nu(\lambda) X_i} c_i + \mathcal{O}(|\lambda|^2|e^{-\nu(\lambda) X_i} c_i|)
\end{align*}

\item For the term involving $\tilde{c}$, as with the terms involving $c$, we first use Lemmas \ref{centerprojlemma} and \ref{lemma:VpmPsiIP} to get
\begin{align*}
\langle \Psi(0), P_+(0; \lambda) e^{-\nu(\lambda) X_i}\tilde{c}_i \rangle &= p_1 \lambda |e^{-\nu(\lambda) X_i}\tilde{c}_i|
\end{align*}
Using the expression for $\tilde{c}$ from Lemma \ref{Zinv2},
\begin{align*}
|\langle \Psi(0), &P_+(0; \lambda) e^{-\nu(\lambda) X_i}\tilde{c}_i \rangle| \leq C |\lambda|^2 e^{-\alpha X_i} e^{\eta X_i} \Big( (|\lambda| + e^{-\alpha X_m})(|e^{\nu(\lambda)X_{i-1}} c_{i-1}| + |e^{-\nu(\lambda)X_{i+1}}c_{i+1}|) + |c_i| + |d| \Big)
\end{align*}

\item Next, we evaluate the center integral terms. As in Lemma \ref{jumpcenteradj}, the integrands are both zero when $\lambda = 0$. Thus these integral terms have the same bounds as in the previous lemma. For the center integral terms involving $Z$,
\begin{align*}
&\left| \langle \Psi(0), P_+(0; \lambda) \int_{X_i}^0 \Phi^c(0, y; \lambda) P_+(y; \lambda)^{-1} G_i^+(y) P_+(y; \lambda) Z_i^+(y) dy \rangle \right| \\
&\leq C |\lambda| e^{-\alpha X_m}\Big(|c_i| + e^{-\alpha X_m}(|c_{i-1}| + e^{-\alpha X_i} |c_{i+1}|) + e^{-(\alpha - \eta) X_m}|\lambda|^2|d| + |D_i||d|) \\
&+ |e^{-\nu(\lambda)X_i} c_i| + (|\lambda| + e^{-\alpha X_m})(|e^{\nu(\lambda)X_{i-1}} c_{i-1}| + e^{-\alpha X_i} |e^{-\nu(\lambda)X_{i+1}} c_{i+1}|)\Big)
\end{align*}
and
\begin{align*}
&\left| \langle \Psi(0), P_-(0; \lambda) \int_{-X_{i-1}}^0 \Phi^c(0, y; \lambda) P_-(y; \lambda)^{-1} G_i^-(y) P_-(y; \lambda)Z_i^-(y) dy \rangle \right| \\
&\leq C |\lambda| e^{-\alpha X_m} \Big(|c_{i-1}| + e^{-\alpha X_m}(|c_i| + e^{-\alpha X_{i-1}} |c_{i-2}|) + e^{-(\alpha - \eta) X_m}|\lambda|^2|d| + |D_{i-1}|)|d| \\
&+ |e^{\nu(\lambda)X_{i-1}}c_{i-1}| + (|\lambda| + e^{-\alpha X_m})(|e^{-\nu(\lambda)X_i} c_i| + e^{-\alpha X_{i-1}} |e^{\nu(\lambda)X_{i-2}} c_{i-2}|)\Big)
\end{align*}
For the center integrals not involving $Z$, 
\begin{align*}
&\left| \langle \Psi(0), P_-(0; \lambda) \lambda^2 d_i \int_{-X_{i-1}}^0 \Phi^c(0, y; \lambda) P_-(y; \lambda) \tilde{H}_i^-(y) dy \rangle \right| \\
&\leq C |\lambda|^3 |d| \int_{-X_{i-1}}^0 e^{\alpha_1 y} e^{-\eta y} dy \\
&\leq C |\lambda|^3 |d| 
\end{align*}
The other integral is similar.

\item Finally, we evaluate the noncenter integral involving $Z$.

\begin{align*}
&\left| \langle \Psi(0), P_+(0; \lambda) \int_{X_i}^0 \Phi^u(0, y; \lambda) P_+(y; \lambda)^{-1} G_i^+(y) P_+(y; \lambda) Z_i^+(y) dy \rangle \right| \\
&\leq C \int_0^{X_i} e^{-\alpha y} |G_i^+(y)| |Z_i^+(y)| dy \\
&\leq C \|Z_i^+(y)\| \int_0^{X_i} e^{-\alpha y} \left( e^{-\alpha_0 X_i} e^{-\alpha_0(X_i - y) } + e^{-2 \alpha_0 X_{i-1}} e^{-\alpha_0 y} \right) dy \\
&\leq C \|Z_i^+(y)\| \left( e^{-\alpha_0 X_i} \int_0^{X_i} e^{-(\alpha_0 - \eta) y}  e^{-(\alpha_0 - 2 \eta)(X_i - y) } dy + e^{-2\alpha_0 X_{i-1}} \int_0^{X_i} e^{-2 \alpha y}  dy \right) \\
&= C \|Z_i^+(y)\| \left( e^{-2 \alpha X_i } \int_0^{X_i} e^{-\eta y} dy + e^{-2\alpha_0 X_{i-1}} \int_0^{X_i} e^{-2 \alpha_0 y} dy \right) \\
&\leq C e^{-2 \alpha X_m } \|Z_i^+(y)\|
\end{align*}
Substituting the bound for $Z_3(\lambda)_i^+$, this becomes
\begin{align*}
&\left| \langle \Psi(0), P_+(0; \lambda) \int_{X_i}^0 \Phi^u(0, y; \lambda) P_+(y; \lambda)^{-1} G_i^+(y) P_+(y; \lambda) Z_i^+(y) dy \rangle \right| \\
&= C e^{-2 \alpha X_m } \Big(|c_i| + e^{-\alpha X_m}(|c_{i-1}| + e^{-\alpha X_i} |c_{i+1}|) + e^{-(\alpha - \eta) X_m}|\lambda|^2|d| + |D_i||d|) \\
&+ |e^{-\nu(\lambda)X_i} c_i| + (|\lambda| + e^{-\alpha X_m})(|e^{\nu(\lambda)X_{i-1}} c_{i-1}| + e^{-\alpha X_i} |e^{-\nu(\lambda)X_{i+1}} c_{i+1}|)\Big)
\end{align*}
Similarly,
\begin{align*}
&\left| \langle \Psi(0), P_-(0; \lambda) \int_{-X_{i-1}}^0 \Phi^s(0, y; \lambda) P_-(y; \lambda)^{-1} G_i^-(y) P_-(y; \lambda)Z_i^-(y) \rangle \right| \\
&= C e^{-2 \alpha X_m } \Big(|c_{i-1}| + e^{-\alpha X_m}(|c_i| + e^{-\alpha X_{i-1}} |c_{i-2}|) + e^{-(\alpha - \eta) X_m}|\lambda|^2|d| + |D_{i-1}|)|d| \\
&+ |e^{\nu(\lambda)X_{i-1}}c_{i-1}| + (|\lambda| + e^{-\alpha X_m})(|e^{-\nu(\lambda)X_i} c_i| + e^{-\alpha X_{i-1}} |e^{\nu(\lambda)X_{i-2}} c_{i-2}|)\Big) 
\end{align*}

\end{enumerate}

Putting this all together, we have the jump expressions
\begin{align*}
\xi_i = \langle \Psi(X_i), Q'(-X_i) \rangle (d_{i+1} - d_i ) + \langle \Psi(-X_{i-1}), Q'(X_{i-1}) \rangle (d_i - d_{i-1} ) \\
+ p_1 \lambda( e^{-\nu(\lambda)X_i}c_i + e^{\nu(\lambda)X_{i-1}}c_{i-1})
- \lambda_2 d_i M + R_i(\lambda)(c, d)
\end{align*}
where $M$ is the higher order Melnikov integral
\[
M = \int_{-\infty}^\infty \langle \Psi(y), H(y) \rangle dy
\]
and the remainder term has piecewise bound
\begin{align*}
|R(\lambda)_i&(c_i^-, d)| \leq C \Big( (|\lambda| + e^{-\alpha X_m})(|\lambda| + e^{-(\alpha - \eta) X_m})(|e^{\nu(\lambda)X_{i-1}}c_{i-1}| + |e^{-\nu(\lambda)X_i}c_i|) \\
&+ e^{-(\alpha - \eta)X_m} (|\lambda| + e^{-\alpha X_m})(e^{-\alpha X_{i-1}} |e^{\nu(\lambda)X_{i-2}}c_{i-2}| + e^{-\alpha X_i} |e^{-\nu(\lambda)X_{i+1}}c_{i+1}|) \\
&+ (|\lambda| + e^{-\alpha X_m})(|\lambda| + e^{-(\alpha - \eta) X_m})(|c_i| + |c_{i-1}|) \\
&+ e^{-(2 \alpha - \eta)X_m} (|\lambda| + e^{-\alpha X_m})(|c_{i-2}| + |c_{i+1}|) 
+ (|\lambda| + e^{-(\alpha - \eta) X_m})(|\lambda| + e^{-\alpha X_m})^2 |d| \Big)
\end{align*}

As in Lemma \ref{jumpcenteradj}, we will write these jump expressions in matrix form. The terms involving the $c$ terms work out similarly to those in Lemma \ref{jumpcenteradj}. They are given by
\[
(\lambda p_1 \tilde{K}(\lambda) + C_3 K(\lambda) + K_2(\lambda) + C_4 )c
\]
$K(\lambda)$ is the same matrix as in Lemma \ref{jumpcenteradj}. $\tilde{K}(\lambda)$ is the same matrix as $K(\lambda)$, except all terms are positive.
\begin{align*}
\tilde{K}(\lambda) =  
\begin{pmatrix}
e^{-\nu(\lambda)X_1} & & & & & e^{\nu(\lambda)X_0} \\
e^{\nu(\lambda)X_1} & e^{-\nu(\lambda)X_2} \\
& e^{\nu(\lambda)X_2} & e^{-\nu(\lambda)X_3} \\
\vdots & & \vdots & &&  \vdots \\
& & & & e^{\nu(\lambda)X_{n-1}} & e^{-\nu(\lambda)X_0} 
\end{pmatrix}
\end{align*}
$K_2(\lambda)$ is the ``$\tilde{\gamma}-$perturbation'' of $K(\lambda)$ given by
\begin{align*}
K_2(\lambda) =  
\begin{pmatrix}
e^{-\nu(\lambda)X_1} \tilde{\gamma}_{1,1} & & & & & e^{\nu(\lambda)X_0}\tilde{\gamma}_{1,0} \\
e^{\nu(\lambda)X_1}\tilde{\gamma}_{2,1} & e^{-\nu(\lambda)X_2}\tilde{\gamma}_{2,2} \\
& e^{\nu(\lambda)X_2}\tilde{\gamma}_{3,2} & e^{-\nu(\lambda)X_3}\tilde{\gamma}_{3,3} \\
\vdots & & \vdots & &&  \vdots \\
& & & & e^{\nu(\lambda)X_{n-1}}\tilde{\gamma}_{0,n-1} & e^{-\nu(\lambda)X_0}\tilde{\gamma}_{0,0} 
\end{pmatrix}
\end{align*}
where 
\begin{align*}
\tilde{\gamma}_{i,i-1}, \tilde{\gamma}_{i,i} &= \mathcal{O}((|\lambda| + e^{-\alpha X_m})(|\lambda| + e^{-(\alpha - \eta) X_m}))
\end{align*}
$C_3$ is the periodic, banded matrix
\begin{align*}
C_3 &= \begin{pmatrix}
0 & \tilde{\gamma}_{1,2} & 0 & 0 & \dots & 0 & -\tilde{\gamma}_{n-1,0} & 0 \\
0 & 0 & \gamma_{2,3} & 0 & \dots & 0 & 0 & -\tilde{\gamma}_{2,1} \\
-\tilde{\gamma}_{3,1} & 0 & 0 & \tilde{\gamma}_{3,4} & \dots & 0 & 0 & 0 \\
&  & & \ddots  \\
0 & 0 & 0 & 0 & \dots & 0 & 0 & \tilde{\gamma}_{n-1,0} \\
\tilde{\gamma}_{0,1} & 0 & 0 & 0 & \dots & -\tilde{\gamma}_{0, n-2} & 0 & 0 
\end{pmatrix}
\end{align*}
where
\begin{align*}
\tilde{\gamma}_{i,i-2}, \tilde{\gamma}_{i,i+1} &= \mathcal{O}(e^{-(2 \alpha - \eta)X_m} (|\lambda| + e^{-\alpha X_m})) 
\end{align*}
The matrices $C_3$ and $C_4$ have uniform bound
\begin{align*}
C_3 &= \mathcal{O}(e^{-(2 \alpha - \eta)X_m} (|\lambda| + e^{-\alpha X_m})) \\
C_4 &= \mathcal{O}((|\lambda| + e^{-\alpha X_m})(|\lambda| + e^{-(\alpha - \eta) X_m}))
\end{align*}

For the terms involving $d$, let
\[
a_i = \langle \Psi(X_i), Q'(-X_i) \rangle 
\]
By reversibility, $Q(-x) = R Q(x)$ implies $Q'(-x) = -R Q'(x)$. From Lemma \ref{varadjsolutions}, $\Psi(x) = R \Psi(-x)$. Thus, since $R^2 = I$ and $R$ is self-adjoint, we have as in Lemma \ref{jumpcenteradj},
\begin{align*}
\langle \Psi(-X_i), Q'(X_i) \rangle &= \langle \Psi(-X_i), R^2 Q'(X_i) \rangle \\
&= -\langle R \Psi(-X_i), Q'(-X_i) \rangle \\
&= -\langle \Psi(X_i), Q'(-X_i) \rangle \\
&= -a_i
\end{align*}
Thus the terms involving $d$ are given in matrix form by
\[
(A - \lambda^2 M I + D_2)d
\]
The matrix $A$ is given by
\begin{align*}
A &= \begin{pmatrix}
-a_0 -a_1 & a_0 + a_1 \\
-a_0 + a_1 & -a_0 - a_1
\end{pmatrix} && n = 2 \\
A &= \begin{pmatrix}
-a_{n-1} - a_0 & a_0 & & & \dots & a_{n-1}\\
a_0 & -a_0 - a_1 &  a_1 \\
& a_1 & -a_1 - a_2 &  a_2 \\
& & \vdots & & \vdots \\
a_{n-1} & & & & a_{n-2} & -a_{n-2} - a_{n-1} \\
\end{pmatrix} && n > 2
\end{align*}
$M$ is the higher order Melnikov integral
\[
M = \int_{-\infty}^\infty \langle \Psi(y), H(y) \rangle dy
\]
The remainder terms involving $D$ are collected into the matrix $D_2$, which has uniform bound
\begin{align*}
D_2 &= \mathcal{O}((|\lambda| + e^{-(\alpha - \eta) X_m})(|\lambda| + e^{-\alpha X_m})^2)
\end{align*}
\end{proof}
\end{lemma}

\subsection{Proof of Theorem \ref{blockmatrixtheorem}}

Theorem \ref{blockmatrixtheorem} combines the jump matrix formulas from Lemma \ref{jumpcenteradj} and Lemma \ref{jumpadj} into a single block matrix.

\iffulldocument\else
	\bibliographystyle{amsalpha}
	\bibliography{thesis.bib}
\fi

\end{document}
\documentclass[thesis.tex]{subfiles}

\begin{document}

\iffulldocument\else
	\chapter{KdV5}
\fi

\section{Proof of eigenvalue location theorems}

Let $Q_n(x)$ be a periodic n-pulse with length parameters $\{ b_0(r), \dots, b_{n-1}(r)\}$. In this section, we prove Theorem \ref{locateeigtheorem}, in which we solve the block matrix equation from Theorem \ref{blockmatrixtheorem} to locate the PDE eigenvalues $\lambda$ for $|\lambda| < \delta$. Throughout this section, we assume that the essential spectrum eigenvalues are ``out of the way''. 

We proceed as follows. Let $\{ 0, \lambda_*^1(r), \dots, \lambda_*^{n-1}(r) \}$ be the eigenvalues of $A$, where we have made the dependence on the scaling parameter $r$ explicit. Using the expressions for $a_j(r)$ from \cref{lemma:ajparam}, rescale the matrix $A$ by taking $a_j(r) = r \tilde{a}(r)$. Thus we have $A = r \tilde{A}$. $\tilde{A}$ is the same matrix as $A$ with the entries $a_j$ replaced by $\tilde{a}_j$. For $j = 1,\dots, n-1$, let 
\[
\lambda_*^j(r) = r \tilde{\lambda}_*^j(r)
\] 
Then the eigenvalues of $\tilde{A}$ are $\{ 0, \tilde{\lambda}_*^1(r), \dots, \tilde{\lambda}_*^{n-1}(r) \}$. We expect to find interaction eigenvalues near $\pm \sqrt{\tilde{\lambda}_*^j(r)/M}$.

We expect that the first nonzero essential spectrum eigenvalue will be located at $c \pi i/X$. To ensure that all of the essential spectrum eigenvalues are ``out of the way'', we will choose $r_0$ sufficiently small so that
\[
\max_{j = 1, \dots, n-1}
\left| \sqrt{\frac{\lambda_*^j(r)}{M}} \right| \leq \frac{c}{4 X}
\]
In terms of the scaling parameter alone, we have chosen $r_0$ sufficiently small so that
\begin{equation}\label{nobubblecondgen}
\max_{j = 1, \dots, n-1} \left| \sqrt{\frac{\tilde{\lambda}_*^j(r)}{M}}\right|r_0^{1/2} \leq \frac{1}{4} \frac{c}{\left( n |\log r_0| + |\log( b_0(0) \cdots b_{n-1}(0) |\right)}
\end{equation}
As $r_0 \rightarrow 0$, the LHS of \cref{nobubblecondgen} goes to 0 faster than the RHS, thus we can always find such an $r_0$.

We now prove the theorem below in a series of lemmas. The basic strategy is as follows. To find the interaction eigenvalues, we solve the top row of the block matrix equation for $c$, then plug it into the bottom row. We do the reverse for the essential spectrum eigenvalues. The condition \cref{nobubblecondgen} provides sufficient bounds on the inverses of the appropriate matrices so that this is possible.

\subsection{Rescaling}

Before we can solve the block matrix equation, we will rescale the problem by adopting the same scaling and parameterization as in the existence problem. In the next lemma, we rewrite the terms in the block matrix equation \eqref{blockeq} from Theorem \ref{blockmatrixtheorem} using this parameterization. In addition, to simplify the analysis, we make the substitution $\lambda = \nu^{-1}(\mu)$.

% block matrix reparameterization
\begin{lemma}\label{reparam}
The block matrix equation from Theorem \ref{blockmatrixtheorem} takes the form
\begin{equation}\label{blockeq2}
\begin{pmatrix}
K(\mu) + C_1 K(\mu) + K_1(\mu) + C_2 & c \mu S(\mu) + D_1 \\
-c \mu M^c \tilde{K}(\mu) + C_3 K(\mu) + K_2(\lambda) + C_4 & r \tilde{A} - c^2 \mu^2 MI + D_2
\end{pmatrix}
\begin{pmatrix} c \\ d \end{pmatrix} = 0
\end{equation}
where $\mu = \nu(\lambda)$ and $\tilde{A} = r^{-1} A$. The remainder terms have bounds
\begin{align*}
S(\mu) = \mathcal{O}(|\mu|r^{\tau/2})
C_1 &= \mathcal{O}(|\mu|r^{\tau/2}(|\mu| + r^{\tau/2})) \\
C_2 &= \mathcal{O}(|\mu|r^{\tau/2}) \\
C_3 &= \mathcal{O}(r^{\tau/2} (|\mu| + r^{\tau/2})(|\mu| + r^{1/2}))  \\
C_4 &= \mathcal{O}((|\mu| + r^{\tau/2})(|\mu| + r^{1/2})) \\
D_1 &= \mathcal{O}(|\mu|(|\mu| + r^{\tau/2})(|\mu| + r^{1/2})) \\
D_2 &= \mathcal{O}((|\mu| + r^{\tau/2})(|\mu| + r^{1/2})^2)
\end{align*}
where $0 < \tau < 1$ (FOR NOW). $K_1(\mu)$ and $K_2(\mu)$ are obtained from $K(\mu)$ by multiplying each nonzero entry by $\mathcal{O}(|\mu|(|\mu| + r^{\tau/2}))$ and $\mathcal{O}((|\mu| + r^{\tau/2})(|\mu| + r^{1/2}))$ (respectively). 

\begin{proof}
Let $\lambda = \nu^{-1}(\mu)$. As in \cref{sec:blockdet},
\[
\nu^{-1}(\mu) = c \mu + \mathcal{O}(\mu^3)
\]
Making this substutition, the block matrix equation \cref{blockeq} becomes
\begin{equation*}
\begin{pmatrix}
K(\mu) + C_1 K(\mu) + K_1(\mu) + C_2 & c \mu S(\mu) + D_1 \\
-c \mu M^c \tilde{K}(\mu) + C_3 K(\mu) + K_2(\lambda) + C_4 & A - c^2 \mu^2 MI + D_2
\end{pmatrix}
\begin{pmatrix} c \\ d \end{pmatrix} = 0
\end{equation*}
Absorbing the $\mathcal{O}(\mu^3)$ terms from the substitution into the remainder matrices, the remainder matrices have the same bounds as those from the orginal block matrix equation, with $\lambda$ replaced with $\mu$. 

Now we rescale the system. From the scaling used in the existence problem,
\[
e^{-\alpha_1 X_m} = C r^{1/2} 
\]
Recall that $\alpha = \alpha_1 - \eta$, and let 
\[
\tau = \frac{\alpha}{\alpha_1} = \frac{\alpha_1 - \eta}{\alpha_1}
\]
Then we have
\begin{align*}
e^{-\alpha X_m} &= \left( e^{-\alpha_1 X_m} \right)^{\alpha/\alpha_1} \\
&= r^{\tau / 2}
\end{align*}
Making these substitutions, we obtain the bounds in the statement of the lemma.
\end{proof}
\end{lemma}

Let
\[
\mu_*^j(r) = \sqrt{\frac{\lambda_*^j(r)}{M c^2}} =  r^{1/2}
\sqrt{\frac{\tilde{\lambda}_*^j(r)}{M c^2}} 
\]
and let $\mu_*^j = r^{1/2} \tilde{\mu}_*^j$. After the change of variable to $\mu$, we expect to find the interaction eigenvalues near $\pm \mu_*^j(r)$.

\subsection{Characterization of \texorpdfstring{$K(\mu)$}{K} }

In order to find the interaction eigenvalues, we will solve the first row of the block matrix equation for $c$. The key step in doing this is inverting the matrix $K(\mu)$, which we can do as long as it is nonsingular. 

The first step is to prove the following general result about the determinant of a periodic, bi-diagonal matrix.

% bidiagonal determinant
\begin{lemma}\label{bidiag}
Let $A$ be the periodic bi-diagonal matrix
\begin{equation}
A = \begin{pmatrix}
a_1 & & & & & & b_n \\
b_1 & a_2 \\
& b_2 & a_3 \\
\vdots & & & \vdots & &&  \vdots \\
& & & & b_{n-2} & a_{n-1} \\
& & & & & b_{n-1} & a_n
\end{pmatrix}
\end{equation}
Then 
\begin{equation}
\det{A} = \prod_{k = 1}^n a_k + (-1)^n \prod_{k = 1}^{n-1} b_k
\end{equation}
\begin{proof}
Expanding by minors using the last column, we have
\begin{align*}
\det A &= a_n \det
\begin{pmatrix}
a_1 \\
b_1 & a_2 \\
& b_2 & a_3 \\
\vdots & & & & \vdots \\
& & & & b_{n-2} & a_{n-1}
\end{pmatrix}
+ (-1)^{n-1} \det
\begin{pmatrix}
b_1 & a_2 \\
& b_2 & a_3 \\
\vdots & & & & \vdots \\
& & & & & b_{n-2} & a_{n-1} \\
& & & & & & b_{n-1}
\end{pmatrix} \\
&= \prod_{k = 1}^n a_k + (-1)^{n-1} \prod_{k = 1}^n b_k
\end{align*}
since both of the matrices on the RHS are triangular.
\end{proof}
\end{lemma}

As a corollary, we can compute the determinant of $K(\mu)$.
% determinant of K(\lambda)
\begin{corollary}\label{detKcorr}
For the matrix $K(\mu)$, 
\begin{enumerate}[(i)]
\item 
\begin{equation}\label{detK}
\det K(\mu) = e^{-\mu X} - e^{\mu X} = -2 \sinh (\mu X)
\end{equation}
where $X = X_0 + X_1 + \dots + X_{n-1}$ is half the length of the periodic domain. 
\item $\det K(\mu = 0$ if and only if 
\begin{align*}
\mu &= \frac{n \pi i}{X} && n \in Z.
\end{align*} 
\end{enumerate}
\begin{proof}
Since $K(\lambda)$ is a periodic, bi-diagonal matrix, by Lemma \ref{bidiag} we have
\begin{align*}
\det K(\mu) &= \prod_{k = 0}^{n-1} e^{-\mu X_k} + (-1)^{n-1} \prod_{k = 1}^n (-e^{\mu X_k}) \\
&= e^{-\mu(X_0 + X_1 + \dots X_{n-1})} + (-1)^{n-1} (-1)^n e^{\mu(X_0 + X_1 + \dots X_{n-1})} \\
&= e^{-\mu X} - e^{\mu X} \\
&= -2 \sinh (\mu X)
\end{align*}
Part (ii) follows immediately.
\end{proof}
\end{corollary}

In the next lemma, we derive an expression for the inverse of $K(\mu)$ (when it is nonsingular).

% lemma : inverse of K(lambda)
\begin{lemma}\label{Kinvlemma}
If $\det K(\mu) \neq 0$,
\begin{equation}\label{Klambdainv}
K(\mu)^{-1} = \frac{1}{\det K(\mu} \tilde{K}(\mu)
\end{equation}
where
\begin{align}\label{tildeK}
\tilde{K}&(\mu) = \\
&\begin{pmatrix}
e^{-\mu(X_2+\dots+X_{n-1}+X_0)} & e^{-\mu(-X_2-\dots-X_{n-1}-X_0)} &
e^{-\mu(X_2-\dots-X_{n-1}-X_0)} & \dots & e^{-\mu(X_2+\dots+X_{n-1}-X_0)}  \\ 
e^{-\mu(X_3+\dots+X_0-X_1)} & e^{-\mu(X_3+\dots+X_0+X_1)} &
e^{-\mu(-X_3-\dots-X_0-X_1)} & \dots & e^{-\mu(X_3+\dots-X_0-X_1)} \\ 
& \ddots & \ddots \\
e^{-\mu(-X_1-X_2 -\dots-X_{n-1})} & e^{-\mu(X_1-X_2 -\dots-X_{n-1})} &
e^{-\mu(X_1+X_2 -\dots-X_{n-1})} & \dots & e^{-\mu(X_1+X_2+\dots+X_{n-1})}  \nonumber 
\end{pmatrix}
\end{align}
and we have the bound
\begin{equation}\label{Klambdainvnorm}
||K(\mu)^{-1}|| \leq C \frac{e^{|\text{Re }\mu|X }}{| \det K(\mu) |}
\end{equation}

\begin{proof}
This can be verified directly. Note that each row is essentially a cyclic permutation of the previous row. Everything is shifted one place to the right, but a different index omitted in each row; for example, row $k$ omits index $k$ (this is taken $\text{mod } n$, which means that row $n$ omits the index 0). For row $j$, we can verify by matrix multiplication that
\begin{align*}
[K(\lambda)\tilde{K}(\lambda)]_{jj} &= e^{-\mu(X_0 + \dots + X_{n-1})} - e^{\mu(X_0 + \dots + X_{n-1})} = \det K(\lambda) \\
[K(\lambda)\tilde{K}(\lambda)]_{jk} &= 0 && j \neq k
\end{align*}
where $\det(K(\mu))$ is given in Corollary \ref{detKcorr}. The result immediately follows. Since every entry in $\tilde{K}(\lambda)$ is bounded by $e^{|\text{Re }\mu|X }$, we have the bound \cref{Klambdainvnorm}.
\end{proof}
\end{lemma}

Next, we obtain a lower bound for $\det K(\lambda)$. Recall that we are looking for interaction eigenvalues near $\pm \mu_*^j(r)$. Let 
\begin{align*}
R_{\min} &= \min\{ |\tilde{\mu}_*^1(0)|, \dots, |\tilde{\mu}_*^{n-1}(0)| \} \\
R_{\max} &= \max \{ |\tilde{\mu}_*^1(0)|, \dots, |\tilde{\mu}_*^{n-1}(0)| \}
\end{align*}
Then we can only consider $\mu$ with $\frac{1}{2} r^{1/2}R_{\min} \leq |\mu| \leq 2 r^{1/2} R_{\max}$. By the condition \cref{nobubblecondgen}, for $r \leq r_0$, $2 r^{1/2} R_{\max} \leq 1/2X$. Thus we are only considering $\mu$ which satisfy the condition
\begin{equation}\label{mucondition}
\frac{1}{2} r^{1/2}R_{\min} \leq |\mu| \leq 2 r^{1/2} R_{\max} \leq \frac{1}{2X}
\end{equation}
We note that $K(\mu)$ is invertible for all $\mu$ satisfying \cref{mucondition}. In the following lemma, we derive a lower bound for $\det K(\mu)$.  

\begin{lemma}\label{detKlemma}
For $\mu$ satisfying \cref{mucondition} and sufficiently small $r$,
\[
|\det K(\mu)| \geq C r^{1/2}X
\]
\begin{proof}
Since $|\mu| \leq X/2$, $|\mu X| \leq 1/2$, thus we can expand $\sinh(\mu X)$ in a Taylor series about $0$ to get
\begin{align*}
\det K(\mu) &= \sinh(\mu X) = \mu X + \mathcal{O}(\mu X)^3
\end{align*}
Since we also have the lower bound $|\mu| \geq r^{1/2}R_{\min}$, we can find a constant $C$ (depending only on $R_{\min}$) so that for sufficiently small $r$ we have
\[
|\det K(\mu)| \geq C r^{1/2}X
\]
\end{proof}
\end{lemma}

In the final lemma of this section, we obtain a bounds on $K(\mu)^{-1}$ as well as certain products involving $K(\mu)^{-1}$.

% lemma : bounds on K
\begin{lemma}\label{Kinvboundslemma}
For $\mu$ satisfying \cref{mucondition},
\begin{equation}\label{Kinvbound}
||K(\mu)^{-1}|| \leq C \frac{r^{-1/2}}{X}
\end{equation}
\begin{enumerate}
\item 
\begin{align}
||K_1(\mu)K(\mu)^{-1}|| &\leq C \frac{r^{-1/2}}{X} |\mu|(|\mu| + r^{\tau/2}) \label{K1Kinvbound} \\
||K_2(\mu)K(\mu)^{-1}|| &\leq C \frac{r^{-1/2}}{X} |\mu|(|\mu| + r^{\tau/2})(|\mu| + r^{1/2}) \label{K2Kinvbound} \\
||\mu \tilde{K}(\mu)K(\mu)^{-1}|| &\leq C \frac{r^{-1/2}}{X} |\mu| \label{K2Kinvbound} \\
\end{align}
\end{enumerate}
\begin{proof}
From Lemma \ref{Kinvlemma} and Lemma \ref{detKlemma},
\begin{align*}
||K(\mu)^{-1}|| \leq C \frac{e^{|\text{Re }\mu|X }}{| \det K(\mu) |} \leq C \frac{r^{1/2}}{X} e^{|\text{Re }\mu|X }
\end{align*}
By \cref{mucondition},
\[
e^{|\text{Re }\mu|X } \leq e^{1/2} = C
\]
thus we have the bound \cref{Kinvbound}.

For the bound on $K_1(\mu)K(\mu)^{-1}$, we use equation for $K(\mu)^{-1}$ from Lemma \ref{Kinvlemma} and the form of $K_1(\mu)$ from the proof of Lemma \ref{jumpcenteradj}. Combining these with the bound \cref{Kinvbound} and the estimate $\gamma_{ij} = \mathcal{O}(|\mu|(|\mu| + r^{\tau/2}))$, we have the estimate
\begin{align*}
||K_1(\mu)K(\mu)^{-1}|| &\leq 
C ||K(\mu)^{-1}|| \max \{|\gamma|_{ij}\} \\
&\leq C \frac{r^{-1/2}}{X} |\mu|(|\mu| + r^{\tau/2})
\end{align*}
The estimate for $||K_2(\mu)K(\mu)^{-1}||$ is similarly obtained using the bounds on the entries of $K_2(\mu)$. The estimate for $||\mu \tilde{K}(\mu)K(\mu)^{-1}||$ is similar.
\end{proof}
\end{lemma}

\subsection{Interaction Eigenvalues}

We can now use the results from the previous section to find the interaction eigenvalues. To do this, we take the scaling
\[
\mu = r^{1/2}\tilde{\mu}
\]
In the following lemma, we derive an equation we can solve to find the interaction eigenvalues.

% equation for d
\begin{lemma}\label{deqlemma}
For sufficiently small $r$, the interaction eigenvalues are the values of $\tilde{\mu}$ for which the equation
\begin{equation}\label{eqford}
(\tilde{A} - \tilde{\mu}^2 c^2 M I + \tilde{D}_3)d = 0
\end{equation}
has a nontrivial solution. The remainder term $\tilde{D}_3$ has bound
\begin{equation}\label{tildeD3bound}
||\tilde{D}_3|| \leq C r^{1/2}
\end{equation}

\begin{proof}
First, we solve the top line of the block matrix equation \eqref{blockeq} for $c$. By the condition \cref{mucondition}, $K(\mu)$ is invertible, and we can write the top line of \eqref{blockeq} as
\begin{align*}
(I + C_1 + K_1(\mu)K(\mu)^{-1} + C_2 K(\mu)^{-1} ) K(\mu) c = -(c \mu S(\mu) + D_1) d
\end{align*}
Taking $\mu = \mathcal{O}(r^{1/2})$ and using the bounds from \cref{reparam} and \cref{Kinvboundslemma}, we have the bound
\begin{align*}
||C_1 + K_1(\mu)K(\mu)^{-1} C_2 K(\mu)^{-1}|| &\leq C\left( r^{\tau/2} r^{1/2} + \frac{r^{\tau/2}}{X} \right)
\end{align*}
Since $X = \mathcal{O}(|\log r|)$, $r^{\tau/2}/X \rightarrow 0$ as $r \rightarrow 0$. Thus for sufficiently small $r$, $I + C_1 + K_1(\mu)K(\mu)^{-1} + C_2 K(\mu)^{-1}$ is invertible. Let $S_1 = (I + C_1 + K_1(\mu)K(\mu)^{-1} + C_2 K(\mu)^{-1})^{-1}$. Then we can solve for $c$ to get
\[
c = -K(\mu)^{-1} S_1(c \mu S(\mu) + D_1) d
\]
Plugging this into the second line of \eqref{blockeq}, we get the folowing equation for $d$.
\begin{align}\label{eqford1}
(r \tilde{A} - r \tilde{\mu}^2 c^2 MI + D_2)d - 
(-c \mu M^c \tilde{K}(\mu) + C_3 K(\mu) + K_2(\lambda) + C_4)K(\mu)^{-1} S_1(c \mu S(\mu) + D_1) d  &= 0 \\
\end{align}
Let 
\[
\tilde{D} = D_2 -
(-c \mu M^c \tilde{K}(\mu) + C_3 K(\mu) + K_2(\mu) + C_4)K(\mu)^{-1} S_1(c \mu S(\mu) + D_1)
\]
be the remainder term, so that \cref{eqford1} becomes
\begin{equation}\label{eqford2}
(r\tilde{A} - r \tilde{\mu}^2 c^2  MI + \tilde{D})d = 0
\end{equation}
Using the scaling $\mu = \mathcal{O}(r^{1/2})$ together with all the bounds we have obtained, we have
\begin{align*}
||\tilde{D}|| &\leq C \left( \right)
\end{align*}
Dividing by $r$, we get the equation 
\[
(\tilde{A} - \tilde{\lambda}^2 MI + \tilde{D}_3)d = 0
\]
where 
\[
||\tilde{D}_3|| \leq C r^{1/2}
\]
\end{proof}
\end{lemma}

Because of our scaling, the parameter $r$ only occurs in the remainder term. If we assume Hypothesis \ref{Adistincteigs}, we can solve \eqref{eqford} for the interaction eigenvalues.

% solve for int eigs
\begin{lemma}\label{inteigslemma}
Assume Hypothesis \ref{Adistincteigs}, and let $0, \tilde{\mu}_1, \dots, \tilde{\mu}_{n-1}$ be the eigenvalues of $\tilde{A}$. Then there exists $r_1 \leq r_0$ such that for $r \in \mathcal{R}$ with $r \leq r_1$, we have $2n - 2$ pairs of interaction eigenvalues
\begin{align*}
\lambda &= \pm \lambda^{\text{int}}_j(r) && j = 0, \dots, n-2
\end{align*}
where
\begin{align}\label{inteigsformula}
\lambda^{\text{int}}_j(r) = r^{1/2} \sqrt{\tilde{\mu}_j / M} + \mathcal{O}(r^{3/4})
\end{align}
These interaction eigenvalue pairs are either real or purely imaginary, and the remainder term cannot move them off of the real or imaginary axis.

\begin{proof}
Let $\{0, \mu_1, \dots, \mu_{n-1}\}$ be the eigenvalues of $A$, which we are distinct by Hypothesis \ref{Adistincteigs}. Since $A = r \tilde{A}$, the eigenvalues of $\tilde{A}$ are $\{0, \tilde{\mu}_1, \dots, \tilde{\mu}_{n-1}\}$, which are also distinct, and $\mu_j = r \tilde{\mu}_j$.\\

Equation \eqref{eqford} has a nontrivial solution if and only if
\[
\tilde{E}(\tilde{\lambda}, r) = \det
\left( \tilde{A} - \tilde{\lambda}^2 MI + \mathcal{O}(r^{1/2}) \right) = 0
\]
When $r = 0$, $\tilde{E}(\tilde{\lambda}, 0) = \det(\tilde{A} - \tilde{\lambda}^2 MI)$, which we can evaluate by taking $\mu = \tilde{\lambda}^2 M$ in the characteristic polynomial of $\tilde{A}$.
\begin{equation}\label{tildeE1}
\tilde{E}(\tilde{\lambda}, 0) = \tilde{\lambda}^2
\left( \tilde{\lambda} - \sqrt{\tilde{\mu}_1 / M} \right)
\left( \tilde{\lambda} + \sqrt{\tilde{\mu}_1 / M} \right) \dots
\left( \tilde{\lambda} - \sqrt{\tilde{\mu}_{n-1} / M} \right)
\left( \tilde{\lambda} + \sqrt{\tilde{\mu}_{n-1} / M} \right)
\end{equation}

For $j = 1, \dots, n-1$, $\tilde{E}(\pm \sqrt{\tilde{\mu}_j / M}, 0) = 0$. Since the eigenvalues of $A_0$ are distinct, $\partial_{\tilde{\lambda}} \tilde{E}(\pm \sqrt{\tilde{\mu}_j / M}, 0) \neq 0$. Thus there exists $r_1 \leq r_0$ so that for $r \leq r_1$, we can use the IFT to solve for $\tilde{\lambda}$ in terms of $r$ near the $2n-2$ roots $\pm \sqrt{\tilde{\mu}_j / M}$ of \eqref{tildeE1}. In other words, for $r \leq r_1$, there are unique smooth functions $\tilde{\lambda}_j^\pm(r)$ such that $\tilde{\lambda}_j^\pm(0) = \pm \sqrt{\tilde{\mu}_j / M}$ and $\tilde{E}(\tilde{\lambda}_j^\pm(r); r) = 0$. We also have the estimate
\[
\tilde{\lambda}_j^\pm(r) = \pm \sqrt{\tilde{\mu}_j/ M} + \mathcal{O}(r^{1/4})
\]

Undoing the scaling, let
\[
\lambda_j^\pm(r) = r^{1/2} \tilde{\lambda}_j^\pm(r)
\]
These are the interaction eigenvalues we seek. By Hamiltonian symmetry, eigenvalues must come in quartets $\pm a \pm b i$. Since  the eigenvalues $\tilde{\mu}$ of $\tilde{A}$ are distinct, the only way we can satisfy Hamiltonian symmetry is if $\lambda_j^+(r) = \lambda_j^-(r)$, in which case the eigenvalue pairs must be real or purely imaginary. Thus the interaction eigenvalues are given by $\lambda = \pm \lambda^{\text{int}}_j(r)$, where
\begin{align*}
\lambda^{\text{int}}_j(r) = r^{1/2} \sqrt{\tilde{\mu}_j / M} + \mathcal{O}(r^{3/4})
\end{align*}
By Hamiltonian symmetry, the remainder term cannot move these off of the real or imaginary axis.
\end{proof}
\end{lemma}

\subsection{Characterization of \texorpdfstring{$A - \lambda^2 MI$}{Matrix A} }

We can now find the ``essential spectrum'' eigenvalues, which we expect to occur near the points where $K(\lambda)$ is singular. In order to do this, we will need to invert $A - \lambda^2 MI$ away from the points where it is singular.

As in the previous section, we will assume that the $\epsilon-$ball condition holds. In addition, since we expect that the smallest nonzero ``essential spectrum'' eigenvalues will occur at approximately $\lambda = \pm c \pi i / X$, we will restrict ourselves to $\lambda$ with $|\lambda| \geq C/X$.

We first prove a lower bound the determinant of $(A - \lambda^2 MI)$.

% lemma : bound on det (A - \lambda^2 MI)
\begin{lemma}\label{detAboundlemma}
We have the following lower bounds for $\det(A - \lambda^2 M I)$.
\begin{enumerate}[(i)]
\item If $|\lambda| \geq C/X$ and the $\epsilon-$ball condition holds,
\begin{equation}\label{detAbound1}
|\det(A - \lambda^2 M I)|
\geq C \frac{1}{X^2} \left( \frac{r^{1/4}}{X} \right)^{n-1} \left( |\lambda|^2 + r \right)^{(n-1)/2}
\end{equation}

\item If $|\lambda| \geq 2 r^{1/2} \sqrt{\tilde{\mu}_M/M}$, where $\tilde{\mu}_M = \max\{\tilde{\mu}_1, \dots, \tilde{\mu}_{n-1} \}$, then
\begin{equation}\label{detAbound2}
|\det(A - \lambda^2 M I)|
\geq C |\lambda|^{n+1} \left( |\lambda|^2 + r \right)^{(n-1)/2}
\end{equation}

\end{enumerate}
\begin{proof}
$\det(A - \lambda^2 MI)$ is the characteristic polynomial $p(t)$ of $A$ with $t = \lambda^2 / M$. Since the eigenvalues of $A$ are $\{0, r \tilde{\mu}_1, \dots, r\tilde{\mu}_{n-1}\}$, the roots of $\det(A - \lambda^2 MI)$ are 
\[
\{0, \pm r^{1/2} \sqrt{\tilde{\mu}_1/M}, \dots, \pm r^{1/2} \sqrt{\tilde{\mu}_{n-1}/M}\}
\]
where the root at 0 has algebraic multiplicty 2. Thus we have
\begin{align}\label{detAlambda}
\det(A &- \lambda^2 M I) \\ 
&= C \lambda^2 (\lambda - r^{1/2} \sqrt{\tilde{\mu_1}/M} )(\lambda + r^{1/2} \sqrt{\tilde{\mu}_1/M} )
\dots(\lambda - r^{1/2} \sqrt{\tilde{\mu}_{n-1}/M})(\lambda + r^{1/2} \sqrt{\tilde{\mu}_{n-1}/M} ) \nonumber
\end{align}

For the bound (i), suppose $|\lambda| \geq C/X$ and that the $\epsilon-$ball condition holds. For the $\lambda^2$ term in \eqref{detAlambda}, we use the lower bound $|\lambda| \geq C/X$. Since the pairs $\pm \mu_j$ are symmetric across the origin, we have the lower bound for each pair
\begin{align*}
(\lambda - r^{1/2} \sqrt{\tilde{\mu}/M} )(\lambda + r^{1/2} \sqrt{\tilde{\mu}_1/M} )
&\geq \epsilon \sqrt{ |\lambda|^2 + |r^{1/2} \sqrt{\tilde{\mu}_1/M}|^2 } \\
&\geq C \frac{r^{1/4}}{X} \sqrt{ |\lambda|^2 + r } \\
\end{align*}
Combining these, we have the bound
\begin{align*}
|\det(A - \lambda^2 M I)|
&\geq C \frac{1}{X^2} \left( \frac{r^{1/4}}{X} \right)^{n-1} \left( |\lambda|^2 + r \right)^{(n-1)/2} \\
\end{align*}

For bound (ii), suppose $|\lambda| \geq 2 r^{1/2} \sqrt{\tilde{\mu}_M/M}$. For the pairs $\pm \mu_j$ in \eqref{detAlambda}, we have the lower bound
\begin{align*}
(\lambda - r^{1/2} \sqrt{\tilde{\mu}/M} )(\lambda + r^{1/2} \sqrt{\tilde{\mu}_1/M} )
&\geq \frac{|\lambda|}{2} \sqrt{ |\lambda|^2 + |r^{1/2} \sqrt{\tilde{\mu}_1/M}|^2 } \\
&\geq C |\lambda| \sqrt{ |\lambda|^2 + r } \\
\end{align*}
Combining these, we have the bound
\[
|\det(A - \lambda^2 M I)|
\geq C |\lambda|^{n+1} \left( |\lambda|^2 + r \right)^{(n-1)/2}
\]
\end{proof}
\end{lemma}

In the next lemma, we derive a bound for $(A - \lambda^2 M I)^{-1}$.

% bound on (A - \lambda^2 M I)^{-1}
\begin{lemma}\label{Ainvboundlemma}
Choose $\lambda \in \C$ such that $|\lambda| \geq C/X$ and the $\epsilon-$ball condition holds. Then we have the following bounds for $(A - \lambda^2 M I)^{-1}$.
\begin{enumerate}[(i)]
\item If $|\lambda| \leq 2 r^{1/2} \sqrt{\tilde{\mu}_M/M}$, where $\tilde{\mu}_M = \max\{\tilde{\mu}_1, \dots, \tilde{\mu}_{n-1}\}$, then
\begin{align}\label{Ainvbound1}
||(A - \lambda^2 M I)^{-1}|| &\leq C X ^{n+1} r^{(n-1)/4}
\end{align}

\item If $|\lambda| \geq 2 r^{1/2} \sqrt{\tilde{\mu}_M/M}$, then
\begin{equation}\label{Ainvbound2}
||(A - \lambda^2 M I)^{-1}|| \leq \frac{C}{|\lambda|^2}
\end{equation}
\end{enumerate}
\begin{proof}
When it is nonsingular, the inverse $(A - \lambda^2 M I)^{-1}$ is given by the formula
\[
(A - \lambda^2 M I)^{-1} = \frac{1}{\det(A - \lambda^2 M I)}\text{Adj}(A - \lambda^2 M I)
\]
where $\text{Adj}(A - \lambda^2 M I)$ is the adjugate matrix (the transpose of the cofactor matrix) corresponding to $A - \lambda^2 M I$. Since $A$ is $n \times n$, each entry in $\text{Adj}(A - \lambda^2 M I)$ involves sums of products of $n-1$ of the entries of $A - \lambda^2 M I$, each of which is $\mathcal{O}(r + |\lambda|^2)$, thus $||\text{Adj}(A - \lambda^2 M I)|| = \mathcal{O}(r + |\lambda|^2)^{n-1}$.

For the bound (i), we use lower bound \eqref{detAbound1} on $\det(A - \lambda^2 M I)$ from Lemma \ref{detAboundlemma} to get
\begin{align*}
||(A - \lambda^2 M I)^{-1}|| &\leq C X^2 \left(\frac{X}{r^{1/4}}\right)^{n-1} 
\frac{\left( |\lambda|^2 + r \right)^{n-1}}{\left( |\lambda|^2 + r \right)^{(n-1)/2}} \\
&= C X^2 X^{n - 1} r^{-(n-1)/4}\left( |\lambda|^2 + r \right)^{(n-1)/2} \\
&\leq C X ^{n+1} r^{(n-1)/4}
\end{align*}
where the last inequality holds since $|\lambda| \leq C r^{1/2}$.

For the bound (ii), we have $|\lambda| \geq C r^{1/2}$, and we use the lower bound \eqref{detAbound2} on $\det(A - \lambda^2 M I)$ from Lemma \ref{detAboundlemma} to get
\begin{align*}
||(A - \lambda^2 M I)^{-1}|| &\leq C \frac{1}{|\lambda|^{n+1}} 
\frac{\left( |\lambda|^2 + r \right)^{n-1}}{\left( |\lambda|^2 + r \right)^{(n-1)/2}} \\
&= C \frac{1}{|\lambda|^{n+1}} \left( |\lambda|^2 + r \right)^{(n-1)/2} \\
&\leq C \frac{1}{|\lambda|^{n+1}} |\lambda|^{n-1} \\
&=\frac{C}{|\lambda|^2}
\end{align*}
\end{proof}
\end{lemma}

\subsection{``Essential spectrum'' eigenvalues}

To find the ``essential spectrum'' eigenvalues, we solve the second line of of the block matrix equation \eqref{blockeq} for $d$ and plug it into the first line. In the next lemma, we derive an equation we can solve to obtain the ``essential spectrum'' eigenvalues.

% lemma : equation for c
\begin{lemma}\label{ceqlemma}
Let $\lambda \in \C$ such that $|\lambda| \geq C/X$ and the $\epsilon-$ball condition holds. Then sufficiently small $r$, the ``essential spectrum'' eigenvalues are the values of $\lambda$ for which 
\begin{align}\label{eqforc}
(K(\lambda) + C_5 K_3(\lambda)c &= 0
\end{align}
has a nontrivial solution, where $K_3(\lambda)$ has the same form as $K_1(\lambda)$, and
\begin{align*}
||C_5|| &\leq C
\end{align*}

\begin{proof}
As long as $\lambda$ is not one of the $2n - 1$ points $\{0, \pm \sqrt{\mu_1/M}, \dots, \pm \sqrt{\mu_{n-1}/M}$ where $A - \lambda^2 MI$ is singular, we can invert $A - \lambda^2 MI$ and write the bottom line of \eqref{blockeq} as 
\begin{align}\label{blockeqbottom}
(C_2 K(\lambda) + K_2(\lambda))c 
+ (A - \lambda^2 MI)(I + (A - \lambda^2 MI)^{-1} D_2))d = 0
\end{align}

To continue, we need to bound $(A - \lambda^2 MI)^{-1} D_2$, which we will do for the two cases in Lemma \ref{Ainvboundlemma}. For $|\lambda| \geq 2 r^{1/2} \sqrt{\tilde{\mu}_M/M}$, we have
\begin{align*}
|| (A - \lambda^2 MI)^{-1} D_2 || &\leq \frac{C}{|\lambda|^2} (|\lambda| + r^{1/2})^3 \\ 
&\leq C |\lambda|
\end{align*}
Since $|\lambda| < \delta$, for sufficiently small $\delta$ this will be less than 1. 

For $|\lambda| \leq 2 r^{1/2} \sqrt{\tilde{\mu}_M/M}$,
\begin{align*}
|| (A - \lambda^2 MI)^{-1} D_2 || &\leq C X^{n+1}\left( |\lambda|^2 + r \right)^{(n-1)/4} (|\lambda| + r^{1/2})^3 \\
&\leq C X^{n+1} r^{(n-1)/4} r^{3/2} \\ 
&= C X^{n+1} r^{(n+1)/4} r \\
&= C r (r^{1/4} X)^{n+1}
\end{align*}
Using the expression for $X$ from Lemma \ref{reparam}, this becomes
\begin{align*}
|| (A - \lambda^2 MI)^{-1} D_2 ||
&\leq C r \left( r^{1/4} (n |\log r| + C_b )\right)^{n+1}
\end{align*}
Since $r^{1/4} |\log r| \rightarrow 0$ as $r \rightarrow 0$, we can find $r_2 \leq r_1$ such that for $r \in \mathcal{R}$ with $\leq r_2$, $|| (A - \lambda^2 MI)^{-1} D_2 || < 1$. 

Thus for both cases, $(I + (A - \lambda^2 MI)^{-1} D_2)$ is invertible. Let $D_3 = (I + (A - \lambda^2 MI)^{-1} D_2)^{-1}$. Then we can solve \eqref{blockeqbottom} for $d$ to get
\begin{align*}
d &= -(A - \lambda^2 MI)^{-1} D_3 (C_2 K(\lambda) + K_2(\lambda))c
\end{align*}

Plugging this in for $c$ in the first line of the block matrix equation, we get
\begin{align*}
(K(\lambda) + C_1 K(\lambda) + K_1(\lambda))c + D_1 d &= 0 \\
(K(\lambda) + C_1 K(\lambda) + K_1(\lambda))c - D_1 (A - \lambda^2 MI)^{-1} D_3 (C_2 K(\lambda) + K_2(\lambda))c &= 0 \\
(K(\lambda) + C_1 K(\lambda) + K_1(\lambda) - D_1 (A - \lambda^2 MI)^{-1} D_3 C_2 K(\lambda) - D_1 (A - \lambda^2 MI)^{-1} D_3 K_2(\lambda))c &= 0 \\
(I + C_1 - D_1 (A - \lambda^2 MI)^{-1} D_3 C_2) K(\lambda)c + (K_1(\lambda) - D_1 (A - \lambda^2 MI)^{-1} D_3 K_2(\lambda))c &= 0
\end{align*}

Next, we will obtain bounds on the terms in the above equation. For the term $D_1 (A - \lambda^2 MI)^{-1}$, we consider the same two cases we did above. For $|\lambda| \geq 2 r^{1/2} \sqrt{\tilde{\mu}_M/M}$, we have
\begin{align*}
|| D_1 (A - \lambda^2 MI)^{-1} || &\leq \frac{C}{|\lambda|^2} (|\lambda| + r^{1/2})^2 \\ 
&\leq C
\end{align*}
For $|\lambda| \leq 2 r^{1/2} \sqrt{\tilde{\mu}_M/M}$, we have
\begin{align*}
|| D_1 (A - \lambda^2 MI)^{-1} || &\leq C r^{1/2} \left( r^{1/4} (n |\log r| + |\log(b_0\cdots b_{n-1})| \right)^{n+1}
\end{align*}
If necessary, decrease $r_2$ so that $X r^{(2 \tilde{\gamma} - 1)/4} \leq 1$. Thus for $r \leq r_2$ we will always have $|| D_1 (A - \lambda^2 MI)^{-1} || \leq C$.

We can now bound $C_1 - D_1 (A - \lambda^2 MI)^{-1} D_3 C_2$ to get
\begin{align*}
|| C_1 - D_1 (A - \lambda^2 MI)^{-1} D_3 C_2 || &\leq C \left( r^{\tilde{\gamma}/2} (|\lambda| + r^{1/2}) + r^{\tilde{\gamma}/2} (|\lambda| + r^{1/2})^2 \right) \\
&\leq C r^{\tilde{\gamma}/2} (|\lambda| + r^{1/2}) 
\end{align*}

Since $|\lambda| < \delta$, for sufficiently small $\delta$ and $r \leq r_2$ (again, decreasing $r_2$ if necessary), $|| C_1 - D_1 (A - \lambda^2 MI)^{-1} D_3 C_2 || < 1$, thus $I + C_1 - D_1 (A - \lambda^2 MI)^{-1} D_3 C_2$ is invertible. Let $C_4 = (I + C_1 - D_1 (A - \lambda^2 MI)^{-1} D_3 C_2)^{-1}$, where we have the bound $||C_4|| \leq C$. Then our equation becomes
\begin{align*}
K(\lambda)c + C_4(K_1(\lambda) - D_1 (A - \lambda^2 MI)^{-1} D_3 K_2(\lambda))c &= 0
\end{align*}

Let $D_4 = -C_4 D_1 (A - \lambda^2 MI)^{-1} D_3$. This has bound
\begin{align*}
||D_4|| &\leq C (|\lambda| + r^{1/2})^3
\end{align*}
Substituting this in, our equation becomes
\begin{align*}
(K(\lambda) + C_4 K_1(\lambda) + D_4 K_2(\lambda))c &= 0
\end{align*}

Finally, we note that since $K_1(\lambda)$ and $K_2(\lambda)$ have the same form, and $C_4$ is a weaker bound than $D_4$, we can write this as
\begin{align*}
(K(\lambda) + C_5 K_3(\lambda))c &= 0
\end{align*}
where $K_3(\lambda)$ has the same form as $K_1(\lambda)$ and $||C_5|| \leq C$.
\end{proof}
\end{lemma}

In the next lemma, we find the ``essential spectrum'' eigenvalues.
% find essential spectrum eigs
\begin{lemma}\label{essspeclemma}
There exists $r_2 \leq r_1$ such that for $r \in \mathcal{R}$ with $r \leq r_2$, the following is true. For all positive integers $k$ with $|\lambda^K(X,k)| < \delta$, there is a pair of purely imaginary ``essential spectrum'' eigenvalues which are given by $\lambda = \pm \lambda^{ess}(X,k; r)$, where
\begin{equation}\label{lambdaess}
\lambda^{ess}(X, k; r) = c \frac{k \pi i }{X} \left( 1 + \mathcal{O}\left( \frac{1}{X} \right)\right) + \mathcal{O}\left( \frac{r^{1/2}}{X} \right)
\end{equation}
The remainder terms cannot move this off of the imaginary axis.

\begin{proof}
From Lemma \ref{ceqlemma}, we have a nontrivial solution to \eqref{eqford} if and only if 
\begin{align*}
E(\lambda) = \det (K(\lambda) + C_5 K_3(\lambda)) = 0
\end{align*}
Since we know the form of $K_3(\lambda)$, we have the following expression for $C_5 K_3(\lambda)$
\[
C_5 K_3(\lambda) = 
\begin{pmatrix}
c_{1,1}^- e^{-\nu(\lambda)X_1} - c_{1,1}^+ e^{\nu(\lambda)X_1} 
& \dots & 
c_{1, n-1}^- e^{-\nu(\lambda)X_{n-1}} - c_{1,n-1}^+ e^{\nu(\lambda)X_{n-1}} &
c_{1,0}^- e^{-\nu(\lambda)X_0} - c_{1,0}^+ e^{\nu(\lambda)X_0}  \\
\vdots & & \vdots & \\
c_{n,1}^- e^{-\nu(\lambda)X_1} - c_{n,1}^+ e^{\nu(\lambda)X_1}
& \dots & 
c_{1, n-1}^- e^{-\nu(\lambda)X_{n-1}} - c_{1,n-1}^+ e^{\nu(\lambda)X_{n-1}} &
c_{n,0}^- e^{-\nu(\lambda)X_0} - c_{n,0}^+ e^{\nu(\lambda)X_0} 
\end{pmatrix}
\]
where $c_{i,j}$ are constants with $|c_{i,j}| \leq C(|\lambda| + r^{1/2})$. 

To solve $E(\lambda) = 0$, we use the definition of the determinant of an $n \times n$ matrix $A$.
\begin{align*}
\det A = \sum_{\sigma \in S_n} \left( \text{sgn}(\sigma) \prod_{i=1}^n a_{i, \sigma(i)} \right)
\end{align*}
where $S_n$ is the symmetric group on $n$ elements. Applying this to $K(\lambda) + C_5 K_3(\lambda)$ and simplifying, we get
\begin{align}\label{Elambdaess}
E(\lambda)
&= -2 \sinh(\nu(\lambda)X) + \sum_{\tau \in T_n}
c_\tau \prod_{j = 0}^{n-1} e^{\tau(j) \nu(\lambda)X_j}
\end{align}

where $T_n = \{ (\pm 1, \pm 1, \dots, \pm 1 \}$ and $c_\tau = \mathcal{O}(|\lambda| + r^{1/2})$. From Lemma \ref{detKlemma}, $\det K(\lambda^K(X,k)) = 0$. Since we are looking for a small perturbation of $\lambda^K(X,k)$, let
\begin{equation}\label{tildelambdadef}
\lambda = \lambda^K(X,k) + \frac{\tilde{\lambda}}{X}
\end{equation}
where $k \in \Z$ with $|\lambda^K(X,k)| < \delta$. From the proof of Lemma \ref{detKlemma}, 
\begin{align*}
\nu\left( \lambda^K(X, k) + \frac{\tilde{\lambda}}{X} \right) 
&= \frac{k \pi i}{X} + \frac{1}{c}\frac{\tilde{\lambda}}{X} \left( 1 + \mathcal{O} \left(\frac{k}{X}\right)^2 \right) + \mathcal{O}\left( \frac{\tilde{\lambda}}{X}\right)^2 \\
&= \frac{k \pi i}{X} + C_k \frac{\tilde{\lambda}}{X} + \mathcal{O}\left( \frac{\tilde{\lambda}}{X}\right)^2 
\end{align*}
where $C_k = 1/c = \mathcal{O}(1)$. 

Substituting this into the term $\sinh(\nu(\lambda)X)$ in \eqref{Elambdaess}, we have
\begin{align*}
\sinh\left(\nu\left(\lambda^K(X, k) + \frac{\tilde{\lambda}}{X}\right)X\right)
&= \sinh\left(\left(\frac{k \pi i}{X} + C_k \frac{\tilde{\lambda}}{X} + \mathcal{O}\left( \frac{\tilde{\lambda}}{X}\right)^2 \right) X\right) \\
&= \sinh\left( k \pi i + C_k \tilde{\lambda} + \mathcal{O}\left( \frac{\tilde{\lambda}^2}{X}\right) \right) \\
&= (-1)^k \left( C_k \tilde{\lambda} + \mathcal{O}\left( \frac{\tilde{\lambda}^2}{X}\right) \right) + \mathcal{O}\left( \tilde{\lambda} + \frac{\tilde{\lambda}^2}{X} \right)^3 \\
&= (-1)^k C_k \tilde{\lambda} + \mathcal{O}\left( \frac{\tilde{\lambda}^2}{X} + \tilde{\lambda}^3 \right)
\end{align*}

For the remainder term of $E(\lambda)$, we have
\begin{align*}
c_\tau \prod_{j = 0}^{n-1} &\exp\left( {\tau(j) \nu(\lambda)X_j} \right)
= c_\tau \prod_{j = 0}^{n-1} 
\exp\left( \tau_j \left( \frac{k \pi i}{X} + C_k \frac{\tilde{\lambda}}{X} + \mathcal{O}\left( \frac{\tilde{\lambda}}{X}\right)^2 \right) X_j\right) \\
&= c_\tau \exp\left( \left( \sum_{j=0}^{n-1} \frac{\tau_j X_j}{X} \right)
\left( k \pi i + C_k \tilde{\lambda} + \mathcal{O}\left( \frac{\tilde{\lambda}^2}{X} \right) \right) \right) \\
&= c_\tau \exp\left( r_\tau
\left( k \pi i + C_k \tilde{\lambda} + \mathcal{O}\left( \frac{\tilde{\lambda}^2}{X} \right) \right) \right) \\ 
&= c_\tau e^{i k \pi r_\tau} \exp \left( r_\tau C_k \tilde{\lambda} + \mathcal{O}\left( \frac{\tilde{\lambda}^2}{X} \right) \right)
\end{align*}
where $r_\tau = \left( \sum_{j=0}^{n-1} \frac{\tau_j X_j}{X} \right)$ and $|r_\tau| \leq 1$ for all $\tau \in T_n$. Expanding the exponential in a Taylor series, we have
\begin{align*}
c_\tau \prod_{j = 0}^{n-1} \exp\left( {\tau(j) \nu(\lambda)X_j} \right)
&= \tilde{c}_\tau \left( 1 + r_\tau C_k \tilde{\lambda} + \mathcal{O}\left(\tilde{\lambda}^2 \right) \right) 
\end{align*}
where $\tilde{c}_\tau = c_\tau e^{i k \pi r_\tau} = \mathcal{O}(|\lambda| + r^{1/2})$. 
Since the remainder term of \eqref{Elambdaess} consists of a finite sum of terms of this form, we can write it as
\begin{align*}
\sum_{\tau \in T_n} c_\tau \prod_{j = 0}^{n-1} \exp\left( {\tau(j) \nu(\lambda)X_j} \right)
&= \mathcal{O}\left( (|\lambda| + r^{1/2}) \left( 1 + r_\tau C_k \tilde{\lambda} + \mathcal{O}\left(\tilde{\lambda}^2 \right) \right)\right) \\
&= \mathcal{O} \left( \frac{k \pi}{X} + \frac{\tilde{\lambda}}{X} + r^{1/2} \right)
\end{align*}

Combining all of these, we obtain an expression for $E(\lambda)$ which is entirely terms of $\tilde{\lambda}$.
\begin{align*}
E(\tilde{\lambda})
&= (-1)^k C_k \tilde{\lambda} + \mathcal{O}\left( \frac{\tilde{\lambda}^2}{X} + \tilde{\lambda}^3 \right) + \mathcal{O} \left( \frac{k \pi}{X} + \frac{\tilde{\lambda}}{X} + r^{1/2} \right) \\
&= (-1)^k C_k \tilde{\lambda} + \mathcal{O}\left( \frac{\tilde{\lambda}}{X} + \tilde{\lambda}^3 \right) + \mathcal{O} \left( \frac{k \pi}{X} + r^{1/2} \right) \\
&= (-1)^k C_k \tilde{\lambda} + \mathcal{O}\left( \frac{\tilde{\lambda}}{n|\log r| + \log|b_0\cdots b_{n-1}|} + \tilde{\lambda}^3 \right) + \mathcal{O} \left( \frac{k \pi}{X} + r^{1/2} \right)
\end{align*}

where we used our expression for $X$ from Lemma \ref{reparam}. Note that the last term on the RHS does not involve $\tilde{\lambda}$. We wish to solve $E(\tilde{\lambda}) = 0$. To do this, define
\[
F(\tilde{\lambda}) = (-1)^k C_k \tilde{\lambda} + \mathcal{O}\left( \frac{\tilde{\lambda}}{n|\log r| + \log|b_0\cdots b_{n-1}|} + \tilde{\lambda}^3 \right)
\]
so that we wish to solve
\[
F(\tilde{\lambda}) = \mathcal{O} \left( \frac{k \pi}{X} + r^{1/2} \right)
\]
Note that $F(0) = 0$ and for sufficiently small $r$, 
\[
\frac{\partial}{\partial\tilde{\lambda}}F(\tilde{\lambda})\big|_{\tilde{\lambda} = 0}
= (-1)^k C_k + \mathcal{O}\left( \frac{1}{n|\log r| + \log|b_0\cdots b_{n-1}|} \right) \neq 0
\]
since $C_k = \mathcal{O}(1)$. Thus by the inverse function theorem, $F$ is invertible in a neighborhood of 0. Recall that $|\frac{k \pi}{X}| < \delta$. Thus, decreasing $r_2$ and $\delta$ if needed, we can solve uniquely for $\tilde{\lambda}$, i.e. we have
\[
\tilde{\lambda} = F^{-1}\left( \mathcal{O} \left( \frac{k \pi}{X} + r^{1/2} \right)\right)
\]
In particular, since $F^{-1}$ is smooth with $F^{-1}(0) = 0$,
\[
\tilde{\lambda} = \mathcal{O}\left( \frac{k \pi}{X} + r^{1/2} \right)
\]
Substituting this into \eqref{tildelambdadef}, we have eigenvalues $\lambda$ at
\begin{align*}
\lambda &= \lambda^K(X,k) + \frac{\tilde{\lambda}}{X} \\
&= \lambda^K(X,k) + \mathcal{O}\left( \frac{1}{X} \left( \frac{k \pi}{X} + r^{1/2} \right) \right)\\
&= c \frac{k \pi i }{X} \left( 1 + \mathcal{O}\left( \frac{1}{X} \right)\right) + \mathcal{O}\left( \frac{r^{1/2}}{X} \right)
\end{align*}

By Hamiltonian symmetry, these must be purely imaginary since they cannot come in quartets. Thus, the ``essential spectrum'' eigenvalues are given by $\lambda = \pm \lambda^{ess}(X, k; r)$
\[
\lambda^{ess}(X, k; r) = c \frac{k \pi i }{X} \left( 1 + \mathcal{O}\left( \frac{1}{X} \right)\right) + \mathcal{O}\left( \frac{r^{1/2}}{X} \right)
\]
where $k$ is a positive integer with $k \pi/X < \delta$. The remainder terms cannot move these off of the imaginary axis.
\end{proof}
\end{lemma}

\subsection{Eigenvalue counts}

Finally, we will perform two counts of the small eigenvalues so that we can conclude that we have accounted for everything. First, we will count the eigenvalues with $|\lambda| < \delta$.

\begin{lemma}\label{eigcount}
There are $2n + 2 k_M + 1$ eigenvalues inside the circle $|\lambda| = \delta$, where $k_M$ is the largest positive integer $k$ such that $|\lambda^K(k,X) < \delta$. There are no other eigenvalues inside $|\lambda| = \delta$ besides these.

\begin{proof}
Use the radius $\delta$ from Theorem \ref{blockmatrixtheorem}. If necessary, decrease $\delta$ a little so that the circle of radius $\delta$ about the origin in the complex plane cuts exactly halfway between consecutive points $\lambda^K(X, k)$.

Take $\lambda$ with $|\lambda| = \delta$. Then $K(\lambda)$ is invertible. For all $k$ such that $|\lambda^K(X, k)| \leq \delta$, this implies
\[
| \lambda - \lambda^K(X, k)| \geq C \frac{1}{X}
\]
Following the proofs of Lemma \ref{detKlemma} and \ref{Kinvboundslemma}, we have
\[
|\det K(\lambda)| \geq C
\]
which implies
\begin{equation}\label{Kinvbounddelta}
||K(\lambda)^{-1}|| \leq C
\end{equation}

Since the $2n-2$ nonzero roots of $\det(A - \lambda^2 M I)$ are $\mathcal{O}(r^{1/2})$, we can find $r_3 \leq r_2$ such that for all $r \leq r_3$, $2 r^{1/2} \sqrt{\tilde{\mu}_M/M} \leq \delta$, where $\tilde{\mu}_M = \max\{|\tilde{\mu}_1|, \dots, |\tilde{\mu}_{n-1}| \}$. Thus for $|\lambda| = \delta$, using the bound \eqref{Ainvbound2} from Lemma \ref{Ainvboundlemma}, 
\begin{equation}\label{Ainvbounddelta}
||(A - \lambda^2 M I)^{-1}|| \leq \frac{C}{\delta^2}
\end{equation}

Factoring $K(\lambda)$ out of the top left, write \eqref{blockeq} as
\begin{equation}\label{blockeq2}
\begin{pmatrix}
(I + C_1 + K_1(\lambda)K(\lambda)^{-1})K(\lambda) & D_1 \\
C_2 K(\lambda) + K_2(\lambda) & A - \lambda^2 MI + D_2
\end{pmatrix}
\begin{pmatrix} c \\ d \end{pmatrix} = 0
\end{equation}
From the proof of Lemma \ref{deqlemma}, $I + C_1 + K_1(\lambda)K(\lambda)^{-1}$ is invertible. As in that lemma, let $C_3 = (I + C_1 + K_1(\lambda)K(\lambda)^{-1})$. Multiplying the top row of \eqref{blockeq2} by $C_3$, we obtain the equivalent formulation
\begin{equation}\label{blockeq3}
\begin{pmatrix}
K(\lambda) & C_3 D_1 \\
C_2 K(\lambda) + K_2(\lambda) & A - \lambda^2 MI + D_2
\end{pmatrix}
\begin{pmatrix} c \\ d \end{pmatrix} = 0
\end{equation}
Thus finding $\lambda$ for which \eqref{blockeq} has a nontrivial solution is equivalent to finding the zeros of $E(\lambda)$, where
\begin{equation}
E(\lambda) = \det 
\begin{pmatrix}
K(\lambda) & C_3 D_1 \\
C_2 K(\lambda) + K_2(\lambda) & A - \lambda^2 MI + D_2
\end{pmatrix}
\end{equation}

Using a standard determinant identity, since $K(\lambda)$ and $A - \lambda^2 M I$ are both invertible, we can write $E(\lambda)$ as
\begin{align*}
E(\lambda) &= \det(K(\lambda))
\det ( A - \lambda^2 MI + D_2 - (C_2 K(\lambda) + K_2(\lambda))K(\lambda)^{-1}C_3 D_1 ) \\
&= \det(K(\lambda))\det(A - \lambda^2 MI)
\det ( I + (A - \lambda^2 MI)^{-1}(D_2 - (C_2 + K_2(\lambda)K(\lambda)^{-1})C_3 D_1 ) \\
&= \det(K(\lambda))\det(A - \lambda^2 MI)\det(I + R(\lambda))
\end{align*}
where
\[
R(\lambda) = 
(A - \lambda^2 MI)^{-1}(D_2 - (C_2 + K_2(\lambda)K(\lambda)^{-1})C_3 D_1)
\]
Using the bounds \eqref{Ainvbounddelta} and \eqref{Kinvbounddelta} together with the bounds from Lemma \ref{reparam}, and recalling that $r^{1/2} < |\lambda| = \delta$, we have the bound on $R(\lambda)$
\begin{align*}
||R(\lambda)|| \leq C \frac{1}{\delta^2}
( |\delta|^3 + (r^{\tilde{\gamma}/2}\delta + \delta)\delta^2) = C \delta
\end{align*}

Thus we can write $R(\lambda) = \delta \tilde{R}(\lambda)$, where $\tilde{R}(\lambda) = \mathcal{O}(1)$. From a standard expansion of the determinant, 
\begin{align*}
\det(I + R(\lambda)) &= 1 + \delta \text{Tr}(\tilde{R}(\lambda)) + \mathcal{O}(\delta^2) \\
&= 1 + \mathcal{O}(\delta)
\end{align*}
For sufficiently small $\delta$, $\det(I + R(\lambda)) = 1 + \tilde{\delta}$, where $|\tilde{\delta}| < 1$. This gives us 
\begin{equation}
E(\lambda) = \det(K(\lambda))\det(A - \lambda^2 MI) + \tilde{\delta} \det(K(\lambda))\det(A - \lambda^2 MI)
\end{equation}

Since $\tilde{\delta} < 1$ and we are taking $\lambda = \delta$, by Rouch\'e's Theorem, $E(\lambda)$ and $\det(K(\lambda))\det(A - \lambda^2 MI)$ have the same number of zeros (counting multiplicty) inside the circle $|\lambda| = \delta$. By our choice of $\delta$, 
\begin{enumerate}[(i)]
\item $\det(A - \lambda^2 MI)$ has exactly $2n$ zeros inside the circle $|\lambda| = \delta$, which are given by $\{ 0, \pm r^{1/2} \sqrt{ \tilde{\mu}_1 /M}, \dots, \pm r^{1/2} \sqrt{\tilde{\mu}_{n-1}/M} \}$, where 0 has algebraic multiplicty 2.

\item Let $k_M$ be the largest positive integer $k$ such that $\lambda^K(k,X) < \delta$. Then $\det(K(\lambda))$ has exactly $2 K_M + 1$ zeros inside the circle $|\lambda| = \delta$, which are given by $\{0, \pm \lambda^K(1,X), \dots, \lambda^K(k_M,X)\}$, where 0 has algebraic multiplicity 1.
\end{enumerate}

Thus there are exactly $2n + 2 k_M + 1$ eigenvalues inside the circle $|\lambda| = \delta$. 
\end{proof}
\end{lemma}

We will also count the eigenvalues in a small ball around the 0 

\begin{lemma}\label{eigcount2}
Let 
\begin{equation}\label{xiradius}
\xi = \min\left\{ \frac{\pi}{2X}, \frac{r^{1/2}\sqrt{\tilde{\mu}_m/M} }{2} \right\} = 
\min\left\{ \frac{\pi}{2 C( n |\log X| + |\log(b_1\cdots b_{n-1}|)}, \frac{r^{1/2} \sqrt{\tilde{\mu}_m/M}}{2} \right\} 
\end{equation}
where $\tilde{\mu}_m = \min \{ |\tilde{\mu_1}|, \dots, |\tilde{\mu}_{n-1}| \}$. Then for sufficiently small $r$, there are exactly 3 eigenvalues inside the circle of radius $\xi$ in the complex plane.

\begin{proof}
For $|\lambda| = \xi$, $K(\lambda)$ is invertible. We look for nontrivial solutions to \eqref{blockeq}, which, as in the previous lemma, we rewrite as \eqref{blockeq2}.

To bound $K_1(\lambda)K(\lambda)^{-1}$, we follow the proof of Lemma \ref{Kinvboundslemma}. By our choice of $\xi$, the $\epsilon$ criterion in that lemma is automatically satisfied. Replacing the condition that $|\lambda| \geq C r^{1/2}$ with $|\lambda| = \delta$, we have the bound
\begin{align*}
||K_1(\lambda)K(\lambda)^{-1}|| &\leq C( \xi + r^{1/2})\max\left\{ r^{-1/4}, \frac{1}{\xi X} \right\} \\
&\leq C r^{1/2} \max\left\{ r^{-1/4}, \frac{1}{\xi X} \right\} \\
&\leq C r^{1/2} \max\left\{ r^{-1/4}, 1, \frac{r^{-1/2}}{X} \right\} \\
&\leq C \max\left\{ r^{1/4}, \frac{1}{X} \right\} \\
&\leq C \max\left\{ r^{1/4}, \frac{1}{n|\log r| + |\log(b_1 \cdots b_{n-1})| } \right\} 
\end{align*}
where we used the expression for $X$ from Lemma \ref{reparam}. Since the bound for $C_1$ is stronger than this, for sufficiently small $r$, $I + C_1 + K_1(\lambda)K(\lambda)^{-1}$ is invertible. Let $C_3 = (I + C_1 + K_1(\lambda)K(\lambda)^{-1})$. As in the previous lemma, multiply the top row of \eqref{blockeq2} by $C_3$ to get \eqref{blockeq3}. Then finding a nontrivial soluton to \eqref{blockeq3} is equivalent to solving $E(\lambda) = 0$, where 
\begin{equation}
E(\lambda) = \det 
\begin{pmatrix}
K(\lambda) & C_3 D_1 \\
C_2 K(\lambda) + K_2(\lambda) & A - \lambda^2 MI + D_2
\end{pmatrix}
\end{equation}

As in the previous lemma, since $K(\lambda)$ and $A - \lambda^2 M I$ are invertible for $|\lambda| = \xi$, we use a standard determinant identity to write $E(\lambda)$ as 
\begin{align*}
E(\lambda)
&= \det(K(\lambda))\det(A - \lambda^2 MI)\det(I + R(\lambda))
\end{align*}
where
\[
R(\lambda) = 
(A - \lambda^2 MI)^{-1}(D_2 - (C_2 + K_2(\lambda)K(\lambda)^{-1})C_3 D_1)
\]

First, we bound $(D_2 - (C_2 + K_2(\lambda)K(\lambda)^{-1})C_3 D_1)$. Since $K_2(\lambda)$ is of the same form as $K_1(\lambda)$, we can use the bound for $K_1(\lambda)K(\lambda)^{-1}$ here.
\begin{align*}
||(D_2 &- (C_2 + K_2(\lambda)K(\lambda)^{-1})C_3 D_1)|| \\
&\leq C \left( (\xi + r^{1/2})^3 + (\xi + r^{1/2})^2 \left( r^{\tilde{\gamma}/2}(\xi + r^{1/2})^2 + \max\left\{ r^{1/4}, \frac{1}{n|\log r| + |\log(b_1 \cdots b_{n-1})| } \right\} \right) \right) \\
&\leq C \left( r^{3/2} + r \left( r^{1 + \tilde{\gamma}/2} + \max\left\{ r^{1/4}, \frac{1}{n|\log r| + |\log(b_1 \cdots b_{n-1})| } \right\} \right) \right) \\
&\leq C r \max\left\{ r^{1/4}, \frac{1}{n|\log r| + |\log(b_1 \cdots b_{n-1})| } \right\} 
\end{align*}

For a bound on $(A - \lambda^2 MI)^{-1}$, we use Lemma \ref{Ainvboundlemma}. The bound depends on $\xi$. If $\xi = \frac{\pi}{2X}$, we can go ahead and use the bound \eqref{Ainvbound1} from Lemma \ref{Ainvboundlemma}, which is
\[
||(A - \lambda^2 MI)^{-1}|| \leq C X^{n+1} r^{(n-1)/4}
\]

If $\xi = \frac{\tilde{\mu}_m r^{1/2}}{2}$, then we need to compute a new bound. For $\det(A - \lambda^2 MI)$, we can adapt the proof of Lemma \ref{detAboundlemma} to get
\[
|\det(A - \lambda^2 MI)| \leq C \xi^2 (\xi + r^{1/2})^{n-1} r^{(n-1)/2}
\]
Using this in the proof of Lemma \ref{Ainvboundlemma}, we get
\begin{align*}
||(A - \lambda^2 MI)^{-1}|| &\leq C \frac{(r + \xi^2)^{n-1}}{\xi^2 (\xi + r^{1/2})^{n-1} r^{(n-1)/2}} \\
&= C \frac{r^{n-1}}{r \: r^{(n-1)/2} r^{(n-1)/2}} \\
&= \frac{C}{r} 
\end{align*}

Combining all of these, we can get a bound on $R(\lambda)$. If $\xi = \frac{\pi}{2X}$, then
\begin{align*}
||R(\lambda)|| &\leq C X^{n+1} r^{(n-1)/4} r \max\left\{ r^{1/4}, \frac{1}{n|\log r| + |\log(b_1 \cdots b_{n-1})| } \right\} \\
&\leq C r^{1/2} (r^{1/4} X)^{n+1} \max\left\{ r^{1/4}, \frac{1}{n|\log r| + C_b } \right\} \\
&\leq C r^{1/2} \left(r^{1/4}(n|\log r| + |\log(b_1 \cdots b_{n-1})|)\right)^{n+1} \max\left\{ r^{1/4}, \frac{1}{n|\log r| + |\log(b_1 \cdots b_{n-1})| } \right\}
\end{align*}

This can be made arbitrarily small by taking $r$ sufficiently small. If $\xi = \frac{\tilde{\mu}_m r^{1/2}}{2}$, then 
\begin{align*}
||R(\lambda)|| &\leq \frac{C}{r} r \max\left\{ r^{1/4}, \frac{1}{n|\log r| + |\log(b_1 \cdots b_{n-1})|} \right\} \\
&= C \max\left\{ r^{1/4}, \frac{1}{n|\log r| + |\log(b_1 \cdots b_{n-1})|} \right\} 
\end{align*}
This can also be made arbitrarily small. Choose $r_3$ sufficiently small so that for $r \leq r_3$,
\[
||R(\lambda)||_{max} \leq \frac{1}{3n}
\] 
so that $\text{Tr}(R(\lambda)) \leq 1/3$. From a standard expansion of the determinant, 
\begin{align*}
\det(I + R(\lambda)) &= 1 + \text{Tr}(R(\lambda)) + \mathcal{O}\left(\frac{1}{9n^2} \right) \\
&= 1 + p
\end{align*}
where $0 < |p| < 1$. Thus we have
\begin{align*}
E(\lambda) &= \det(K(\lambda))\det(A - \lambda^2 MI)\det(I + R(\lambda)) \\
&= \det(K(\lambda))\det(A - \lambda^2 MI)(1 + p) \\
&= \det(K(\lambda))\det(A - \lambda^2 MI) + p \det(K(\lambda))\det(A - \lambda^2 MI)
\end{align*}

Since $|p| < 1$, by Rouch\'e's Theorem, $E(\lambda)$ and $\det(K(\lambda))\det(A - \lambda^2 MI)$ have the same number of zeros (counting multiplicty) inside the circle $|\lambda| = \xi$. By our choice of $\xi$, $\det(K(\lambda))\det(A - \lambda^2 MI)$ has exactly 3 zeros inside $|\lambda| = \xi$, all of which occur at $\lambda = 0$. We conclude that $\tilde{E}(\lambda)$ (and thus $E(\lambda)$) has exactly 3 zeros inside $|\lambda| = \xi$.\\
\end{proof}
\end{lemma}

\subsection{Proof of Theorem \ref{locateeigtheorem}}

For part (i), it follows from \eqref{Arelations} that $Q_{np}'(x)$ is an eigenfunction with eigenvalue 0 and $T_{np}(x)$ is a generalized eigenfunction with eigenvalue 0 corresponding to $Q_{np}'(x)$. By Hypothesis \ref{Melnikov2hyp}, this Jordan chain cannot continue. By Lemma \ref{varadjsolutions}, there is another eigenfunction $V^c(x)$ with eigenvalue 0; this eigenfunction is bounded but does not decay to 0. Although Lemma \ref{varadjsolutions} states this for $A(Q(x))$, i.e. the linearization about the primary pulse, the same argument holds for $A(Q_{np}(x))$, since it only depends on the rest state.

Part (ii) follows from Lemma \eqref{inteigslemma} and part (iii) follows from Lemma \ref{essspeclemma}. Part (iv) follows from Lemmas \ref{eigcount} and \ref{eigcount2}. Using these lemmas, a complete account of the eigenvalues inside $|\lambda| = \delta$ is as follows.
\begin{enumerate}
	\item At $\lambda = 0$, there is an eigenvalue with algebraic multiplicity 3. The eigenfunctions are those from part (i).
	\item There are $2n - 2$ interaction eigenvalues, which come in pairs $\pm \lambda$. Each pair is real or purely imaginary.
	\item There $2 k_M$ purely imaginary ``essential spectrum'' eigenvalues, which also come in pairs.
\end{enumerate}

\subsection{Proof of Theorem \ref{inteigsparity}}

Let $r_1$ and $b^*$ be as in Theorem \ref{unifperexist}. Let $\tilde{A}_0$ be the tri-diagonal, symmetric matrix 
\begin{align*}
\tilde{A}_0 &= \begin{pmatrix}
-\tilde{a}_0 & \tilde{a}_0 \\
\tilde{a}_0 & -\tilde{a}_0 - \tilde{a}_1 &  \tilde{a}_1 \\
& \tilde{a}_1 & -\tilde{a}_1 - \tilde{a}_2 &  \tilde{a}_2 \\
& & \ddots & & \ddots \\
& & & & & \tilde{a}_{n-2} & -\tilde{a}_{n-2} \\
\end{pmatrix}
\end{align*}
which is obtained from the matrix $\tilde{A}$ by taking $\tilde{a}_{n-1} = 0$. The matrix $\tilde{A}_0$ is symmetric, so its eigenvalues are real, and $(1, 1, \dots, 1)^T$ is an eigenvector of $\tilde{A}_0$ with eigenvalue 0. Let $0, \mu^0_1, \dots, \mu^0_{n-1}$ be the eigenvalues of $\tilde{A}_0$. Let $n_+$ be the number of positive $\tilde{a}_j$ and $n_i = n - n_+ - 1$ be the number of negative $\tilde{a}_j$. By Lemma 5.4 of \cite{Sandstede1998} (noting that $\tilde{A}_0$ is the matrix $-A_0$ in that lemma),
\begin{enumerate}[(i)]
\item $\tilde{A}_0$ has $n_+$ negative eigenvalues (counting multiplicity)
\item $\tilde{A}_0$ has $n_-$ positive eigenvalues (counting multiplicity)
\end{enumerate}

For $\tilde{a}_{n-1}$ small, $\tilde{A}$ is a small perturbation of $\tilde{A}_0$. Since characteristic polynomials are smooth functions of matrix entries, the eigenvalues of a matrix are also smooth functions of the matrix entries. In particular, the eigenvalues of $\tilde{A}$ depend smoothly on $\tilde{a}_{n-1}$, and as $\tilde{a}_{n-1}$ approaches 0, the eigenvalues of $\tilde{A}$ approach those of $\tilde{A}_0$. In particular, as $\tilde{a}_{n-1}$ is increased from 0, the eigenvalues of $\tilde{A}$ can only change sign by crossing through 0. Thus for sufficiently small $\tilde{a}_{n-1}$, the signs of the eigenvalues of $\tilde{A}$ are determined by the signs of the $n-1$ matrix elements $\tilde{a}_0, \dots, \tilde{a}_{n-2}$. We will now obtain this result in terms of $r$ and the baseline length parameters used to contruct the periodic $n-$pulse.

From the proof of Lemma \eqref{reparam}, $\tilde{a}_j$ is given by
\begin{align}\label{tildeaj2}
\tilde{a}_j
&= (-1)^{-\rho \log r / \pi} s_0 e^{\alpha \phi/\beta} \left( \beta b_j(r; m_j, \theta) \cos\left( -\rho \log b_j(r; m_j, \theta) \right) - \alpha b_j(r; m_j, \theta) \sin \left( -\rho \log b_j(r; m_j, \theta) \right) \right)
\end{align}
As $r \rightarrow 0$, $b_j(r; m_j, \theta) \rightarrow b_j^*(m_j, \theta)$ by Theorem \ref{perexist}. As $m_{n-1} \rightarrow \infty$, $b_j^*(m_j, \theta) \rightarrow b_j^0$ by Lemma \ref{thetaparamlemma}. Finally, since $b_{n-1}^0 = \exp\left(-\frac{m_{n-1} \pi}{\rho}\right)$, $b_{n-1}(r; m_{n-1}, \theta) \rightarrow 0$ (and so $\tilde{a}_{n-1} \rightarrow 0$) as $(r, m_{n-1}) \rightarrow (0, \infty)$. Thus we can find a positive integer $M_1$ and $\tilde{r}_1 \leq r_1$ such that for all $m_{n-1} \geq M_1$ and $r \leq \tilde{r}_1$, 
\begin{enumerate}[(i)]
	\item $\tilde{a}_{n-1}$ is sufficiently small so that the eigenvalues of $\tilde{A}$ have the same sign as those of $\tilde{A}_0$.
	\item For $j = 0, \dots, n-2$ and for all $\theta \in [-\arctan \rho, \pi - \arctan \rho)$,
	\begin{equation}\label{bjunif}
	b_j(r; m_j, \theta) = b_j^0 e^{ -\frac{1}{\rho} \theta^*_j(r, m_{n-1}) } = e^{ -\frac{1}{\rho}(m_j \pi + \theta^*_j(r, m_{n-1})) } 
	\end{equation}
	where $|\theta^*_j(r, m_{n-1})| < \arctan \rho$.
\end{enumerate}

We can now determine the signs of the $\tilde{a}_j$ for $j = 0, \dots, n-2$. Since
\begin{align*}
\cos\left( -\rho \log b_j(r; m_j, \theta) \right) 
&= \cos\left( -\rho \log e^{ -\frac{1}{\rho}(m_j \pi + \theta^*_j(r, m_{n-1})) } \right) \\
&= \cos\left( m_j \pi + \theta^*_j(r, m_{n-1})\right) \\
&= (-1)^{m_j} \cos \theta^*_j(r, m_{n-1})
\end{align*}
and
\begin{align*}
\sin\left( -\rho \log b_j(r; m_j, \theta) \right) 
&= \sin\left( m_j \pi + \theta^*_j(r, m_{n-1})\right) \\
&= (-1)^{m_j} \sin \theta^*_j(r, m_{n-1})
\end{align*}
upon substituting \eqref{bjunif} into equation \eqref{tildeaj2} we obtain
\begin{align*}
\tilde{a}_j 
&= (-1)^{-\rho \log r / \pi} (-1)^{m_j} s_0 e^{\alpha \phi/\beta} b_j^0 e^{ -\frac{1}{\rho} \theta^*_j(r, m_{n-1}) } \left( \beta \cos\theta^*_j(r, m_{n-1}) - \alpha \sin \theta^*_j(r, m_{n-1}) \right) \\
&= (-1)^{m + m_j} \left[ s_0 \alpha e^{\alpha \phi/\beta} b_j^0 e^{ -\frac{1}{\rho} \theta^*_j(r, m_{n-1}) } \cos\theta^*_j(r, m_{n-1}) \right] \left( \rho - \tan \theta^*_j(r, m_{n-1}) \right)
\end{align*}
where we have let $r \in \mathcal{R}$ as $r = e^{-\frac{1}{\rho}m \pi}$ for some nonnegative integer $m$. The term in brackets is always positive since $|\theta^*_j(r, m_{n-1})| < \arctan \rho$. Thus the sign of $\tilde{a}_j(0)$ is completely determined by the term $(-1)^{m + m_j}$, and we have
\begin{align*}
\tilde{a}_j(0) &> 0 && \text{if } m + m_j \text{ is even} \\
\tilde{a}_j(0) &< 0 && \text{if } m + m_j \text{ is odd}
\end{align*}
From this, it follows that if $m$ is even,
\begin{enumerate}[(i)]
\item $\tilde{A}_0$ has $n_{\text{even}}$ negative eigenvalues (counting multiplicity)
\item $\tilde{A}_0$ has $n_{\text{odd}}$ positive eigenvalues (counting multiplicity)
\end{enumerate}
This is reversed if $m$ is odd. The result follows from equation \eqref{inteigsformula} from Lemma \ref{inteigslemma}, where we note the dependence on the sign of $M$. The condition that $m_{n-1} \geq M_1$ is equivalent to $b_{n-1}^0 \leq \tilde{b}^*$ for some $\tilde{b}^* \leq b^*$.

\subsection{Interaction Eigenvalues: Specific Cases}

There are specific cases in which we can compute the eigenvalues of $\tilde{A}$ in terms of the $\tilde{a}_j$.

Recall that the $\tilde{a}_j$ are given by
\begin{align*}
\tilde{a}_j(r)
&= (-1)^{-\rho \log r / \pi} s_0 e^{\alpha \phi/\beta} \left( \beta b_j(r; m_j; \theta) \cos\left( -\rho \log b_j(r; m_j; \theta) \right) - \alpha b_j(r; m_j; \theta) \sin \left( -\rho \log b_j(r; m_j; \theta)  \right) \right) + \mathcal{O}(r^{\gamma/2\alpha})
\end{align*}

First, we consider the case when $n = 2$. In that case, the eigenvalues of $\tilde{A}$ are $\{0, \tilde{a} \}$, where
\begin{align*}
\tilde{a} = \tilde{a}_0 + \tilde{a}_1
\end{align*}
For a symmetric 2-periodic pulse, $m_0 = m_1 = 0$ (recall that one of them must be 0). Then by Theorem \ref{2pulsebifurcation}, for sufficiently small $r$ we have symmetric solutions with equal length parameters $b_0(\theta) = b_1(\theta) = e^{-\theta/\rho}$. These length parameters do not depend on $r$. Thus we have
\begin{align*}
\tilde{a}_0(r) = \tilde{a}_1(r)  
&= (-1)^{-\rho \log r / \pi} s_0 e^{\alpha \phi/\beta} e^{-\theta/\rho} \left( \beta \cos \theta - \alpha \sin \theta \right) + \mathcal{O}(r^{\gamma/2\alpha}) \\
&= (-1)^{-\rho \log r / \pi}  \frac{s_0 e^{\alpha \phi/\beta} }{\alpha}  e^{-\theta/\rho} \left( \rho \cos \theta - \sin \theta \right) + \mathcal{O}(r^{\gamma/2\alpha})
\end{align*}

For $r = 0$, $\tilde{a}(0) = 0$ at the pitchfork bifurcation point $\theta = p^*(0)$. 



*****

For small $r$, we should be able to argue that $\tilde{a}_j(r) = 0$ at the pitchfork bifurcation point $p_0(r)$. Thus there is a pitchfork bifurcation in the integration eigenvalues at $\theta = p_0(r)$, in which the interaction eigenvalues collide at the origin and switch from a real pair to a purely imaginary pair or vice versa. 


\iffulldocument\else
	\bibliographystyle{amsalpha}
	\bibliography{thesis.bib}
\fi

\end{document}
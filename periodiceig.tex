\documentclass[thesis.tex]{subfiles}

\begin{document}

\iffulldocument\else
	\chapter{KdV5}
\fi

\section{Proof of Theorem \ref{locateeigtheorem}}

Let $Q_n(x)$ be a periodic n-pulse with length parameters $\{ b_0(r), \dots, b_{n-1}(r)\}$, and assume all of the hypotheses in the statement of Theorem \ref{locateeigtheorem}. Let $r_0$ be sufficiently small so that \cref{nobubblecondgen} is satisfied. Since the LHS of \cref{nobubblecondgen} scales like $r_0^{1/2}$ and the RHS scales like $1/|\log r_0|$, such an $r_0$ can always be found. We will always take $r \leq r_0$.

The outline of the proof is as follows. To find the interaction eigenvalues, we will solve the first line of the block matrix equation \cref{blockeq} for $c$ and substitute the result into the second line. To find the essential spectrum eigenvalues, we will solve the second line of the block matrix equation \cref{blockeq} for $d$ and substitute the result into the first line. These steps require inverting the matrices $K(\lambda)$ and $A - \lambda^2 M I$; the condition \cref{nobubblecondgen} assures that this can be done and gives us sufficient bounds on their inverses. For the eigenvalue counts, we will use Rouch\'e's theorem on an appropriate circle around the origin in the complex plane.

\subsection{Rescaling and simplification}

Before we can solve the block matrix equation, we will reparameterize and simplify the problem. First, adopting the same scaling and parameterization as in Theorem \ref{perexist}. We will also simplify the remainder terms in a similar manner to what we did in \cref{theorem:2peigscase1}. Finally, we will change variables by making the substitution $\lambda = \nu^{-1}(\mu)$ to eliminate the $\nu(\lambda)$ terms. We do this in the next lemma.

% block matrix reparameterization
\begin{lemma}\label{reparam}
The block matrix equation from Theorem \ref{blockmatrixtheorem} reduces to
\begin{equation}\label{blockeq2}
\begin{pmatrix}
K(\lambda) + S_1 & S_2 \\
S_3 & r \tilde{A} - \mu^2 c^2 M I + S_4
\end{pmatrix}
\begin{pmatrix} c \\ d \end{pmatrix} = 0
\end{equation}
where $\mu = \nu(\lambda)$ and $\tilde{A} = r^{-1} A$. The remainder terms have bounds
\begin{align*}
S_1 &= \mathcal{O}(|\mu|(|\mu| + r^{\tau/2})) \\
S_2 &= \mathcal{O}(|\mu|(|\mu| + r^{1/2})) \\
S_3 &= \mathcal{O}(|\mu| + r^{1/2}) \\
S_4 &= \mathcal{O}((|\mu| + r^{\tau/2})(|\mu| + r^{1/2})) 
\end{align*}
where $0 < \tau < 1$ (FOR NOW).
\begin{proof}
First, we use \cref{lemma:expnubound} to simplify the remainder matrices. The condition \cref{nobubblecondgen} implies that there exists a constant $C_2$ such that $r^{1/2} X \leq C_2$. Since we will look for interaction eigenvalues of order $\mathcal{O}(r^{1/2})$ and essential spectrum eigenvalues on the imaginary axis, we can assume that $|\Re \lambda| \leq C_1 r^{1/2}$ for a constant $C_1$. Since the hypotheses of \cref{lemma:expnubound} are met, all the terms of the form $e^{\nu(\lambda)X_j}$ are bounded by a constant. Thus we can write the remainder matrices as
\begin{align*}
S_1 &= C_1 K(\lambda) + K_1(\lambda) + C_2 = \mathcal{O}(|\lambda|(|\lambda + e^{-\alpha X_m})) \\
S_2 &= c \lambda S(\lambda) + D_1 = \mathcal{O}(|\lambda|(|\lambda + e^{-\alpha_1 X_m})) \\
S_3 &= -c \lambda M^c \tilde{K}(\lambda) + C_3 K(\lambda) + K_2(\lambda) + C_4 = \mathcal{O}(|\lambda| + e^{-\alpha_1 X_m}) \\
S_4 &= D_2 = \mathcal{O}((|\lambda| + e^{-\alpha X_m})(|\lambda| + e^{-\alpha_1 X_m})^2)
\end{align*}
Making these substutitions, the block matrix equation \cref{blockeq} becomes
\begin{equation}\label{blockeqred}
\begin{pmatrix}
K(\lambda) + S_1 & S_2 \\
S_3 & A - \lambda^2 MI + S_4
\end{pmatrix}
\begin{pmatrix} c \\ d \end{pmatrix} = 0
\end{equation}

Next, let $\lambda = \nu^{-1}(\mu)$. As in \cref{sec:blockdet},
\begin{equation}\label{munusub1}
\nu^{-1}(\mu) = c \mu + \mathcal{O}(\mu^3)
\end{equation}
Making this substutition and absorbing the $\mathcal{O}(\mu^3)$ terms from \cref{munusub1} into the remainder matrices, the block matrix equation becomes
\begin{equation}\label{blockeqred1}
\begin{pmatrix}
K(\lambda) + S_1 & S_2 \\
S_3 & A - \mu^2 c^2 M I + S_4
\end{pmatrix}
\begin{pmatrix} c \\ d \end{pmatrix} = 0
\end{equation}

Finally, we rescale the system. From the scaling used in the existence problem,
\[
e^{-\alpha_1 X_m} = C r^{1/2} 
\]
Recall that $\alpha = \alpha_1 - \eta$, and let 
\[
\tau = \frac{\alpha}{\alpha_1} = \frac{\alpha_1 - \eta}{\alpha_1}
\]
Then we have
\begin{align*}
e^{-\alpha X_m} &= \left( e^{-\alpha_1 X_m} \right)^{\alpha/\alpha_1} \\
&= r^{\tau / 2}
\end{align*}
Making these substitutions, we obtain \cref{blockeq2} and the bounds in the statement of the lemma.
\end{proof}
\end{lemma}

Now that we have made the change of variables, let
\[
\tilde{\mu}_*^j(r) = \sqrt{\frac{\ell_j(r)}{M c^2}}
\]
and let $\mu_*^j(r) = r^{1/2} \tilde{\mu}_*^j(r)$. Our first task is to find the interaction eigenvalues, which we expect to find near $\pm \mu_*^j(r)$. To do this, we will consider $\mu$ with
\begin{equation}\label{mucondition}
\frac{1}{2} \min_{j = 1, \dots, n-1}\tilde{\mu}_*^j(0) r^{1/2} \leq |\mu| \leq 2 \max_{j = 1, \dots, n-1}\tilde{\mu}_*^j(0) r^{1/2} \leq \frac{1}{2X}
\end{equation}
where the last inequality comes from \cref{nobubblecondgen} and the choice of $r \leq r_0$.

\subsection{Characterization of $K$}

In this section, we will bound the inverse of $K(\mu)$, which is always nonsingular by the condition \cref{mucondition}. The first step is to prove the following general result about the determinant of a periodic, bi-diagonal matrix.

% bidiagonal determinant
\begin{lemma}\label{bidiag}
Let $A$ be the periodic bi-diagonal matrix
\begin{equation}
A = \begin{pmatrix}
a_1 & & & & & & b_n \\
b_1 & a_2 \\
& b_2 & a_3 \\
\vdots & & & \vdots & &&  \vdots \\
& & & & b_{n-2} & a_{n-1} \\
& & & & & b_{n-1} & a_n
\end{pmatrix}
\end{equation}
Then 
\begin{equation}
\det{A} = \prod_{k = 1}^n a_k + (-1)^n \prod_{k = 1}^{n-1} b_k
\end{equation}
\begin{proof}
Expanding by minors using the last column, we have
\begin{align*}
\det A &= a_n \det
\begin{pmatrix}
a_1 \\
b_1 & a_2 \\
& b_2 & a_3 \\
\vdots & & & & \vdots \\
& & & & b_{n-2} & a_{n-1}
\end{pmatrix}
+ (-1)^{n-1} \det
\begin{pmatrix}
b_1 & a_2 \\
& b_2 & a_3 \\
\vdots & & & & \vdots \\
& & & & & b_{n-2} & a_{n-1} \\
& & & & & & b_{n-1}
\end{pmatrix} \\
&= \prod_{k = 1}^n a_k + (-1)^{n-1} \prod_{k = 1}^n b_k
\end{align*}
since both of the matrices on the RHS are triangular.
\end{proof}
\end{lemma}

As a corollary, we can compute the determinant of $K(\mu)$.
% determinant of K(\mu)
\begin{corollary}\label{detKcorr}
For the matrix $K(\mu)$, 
\begin{enumerate}[(i)]
\item 
\begin{equation}\label{detK}
\det K(\mu) = e^{-\mu X} - e^{\mu X} = -2 \sinh (\mu X)
\end{equation}
where $X = X_0 + X_1 + \dots + X_{n-1}$ is half the length of the periodic domain. 
\item $\det K(\mu = 0$ if and only if 
\begin{align*}
\mu &= \frac{n \pi i}{X} && n \in Z.
\end{align*} 
\end{enumerate}
\begin{proof}
Since $K(\mu)$ is a periodic, bi-diagonal matrix, by Lemma \ref{bidiag} we have
\begin{align*}
\det K(\mu) &= \prod_{k = 0}^{n-1} e^{-\mu X_k} + (-1)^{n-1} \prod_{k = 1}^n (-e^{\mu X_k}) \\
&= e^{-\mu(X_0 + X_1 + \dots X_{n-1})} + (-1)^{n-1} (-1)^n e^{\mu(X_0 + X_1 + \dots X_{n-1})} \\
&= e^{-\mu X} - e^{\mu X} \\
&= -2 \sinh (\mu X)
\end{align*}
Part (ii) follows immediately.
\end{proof}
\end{corollary}

In the next lemma, we derive an expression for the inverse of $K(\mu)$ (when it is nonsingular).

% lemma : inverse of K(mu)
\begin{lemma}\label{Kinvlemma}
If $\det K(\mu) \neq 0$,
\begin{equation}\label{Klambdainv}
K(\mu)^{-1} = \frac{1}{\det K(\mu} \tilde{K}(\mu)
\end{equation}
where
\begin{align}\label{tildeK}
\tilde{K}&(\mu) = \\
&\begin{pmatrix}
e^{-\mu(X_2+\dots+X_{n-1}+X_0)} & e^{-\mu(-X_2-\dots-X_{n-1}-X_0)} &
e^{-\mu(X_2-\dots-X_{n-1}-X_0)} & \dots & e^{-\mu(X_2+\dots+X_{n-1}-X_0)}  \\ 
e^{-\mu(X_3+\dots+X_0-X_1)} & e^{-\mu(X_3+\dots+X_0+X_1)} &
e^{-\mu(-X_3-\dots-X_0-X_1)} & \dots & e^{-\mu(X_3+\dots-X_0-X_1)} \\ 
& \ddots & \ddots \\
e^{-\mu(-X_1-X_2 -\dots-X_{n-1})} & e^{-\mu(X_1-X_2 -\dots-X_{n-1})} &
e^{-\mu(X_1+X_2 -\dots-X_{n-1})} & \dots & e^{-\mu(X_1+X_2+\dots+X_{n-1})}  \nonumber 
\end{pmatrix}
\end{align}
and we have the bound
\begin{equation}\label{Klambdainvnorm}
||K(\mu)^{-1}|| \leq C \frac{e^{|\text{Re }\mu|X }}{| \det K(\mu) |}
\end{equation}
\begin{proof}
This can be verified directly. Note that each row is essentially a cyclic permutation of the previous row. Everything is shifted one place to the right, but a different index omitted in each row; for example, row $k$ omits index $k$ (this is taken $\text{mod } n$, which means that row $n$ omits the index 0). For row $j$, we can verify by matrix multiplication that
\begin{align*}
[K(\lambda)\tilde{K}(\lambda)]_{jj} &= e^{-\mu(X_0 + \dots + X_{n-1})} - e^{\mu(X_0 + \dots + X_{n-1})} = \det K(\lambda) \\
[K(\lambda)\tilde{K}(\lambda)]_{jk} &= 0 && j \neq k
\end{align*}
where $\det(K(\mu))$ is given in Corollary \ref{detKcorr}. The result immediately follows. Since every entry in $\tilde{K}(\lambda)$ is bounded by $e^{|\text{Re }\mu|X }$, we have the bound \cref{Klambdainvnorm}.
\end{proof}
\end{lemma}

We can now derive a bound for the inverse of $K(\mu)$.

\begin{lemma}\label{Kinvboundslemma}
For $\mu$ satisfying \cref{mucondition},
\begin{equation}\label{detKbound}
|\det K(\mu)| \geq C r^{1/2}X
\end{equation}
and 
\begin{equation}\label{Kinvbound}
||K(\mu)^{-1}|| \leq C \frac{r^{-1/2}}{X}
\end{equation}
\begin{proof}
Since $|\mu| \leq X/2$, $|\mu X| \leq 1/2$, thus we can expand $\sinh(\mu X)$ in a Taylor series about $0$ to get
\begin{align*}
\det K(\mu) &= \sinh(\mu X) = \mu X + \mathcal{O}(\mu X)^3
\end{align*}
Since \cref{mucondition} gives a lower bound for $\mu$, we can find a constant $C$ so that \cref{detKbound} holds. 

From Lemma \ref{Kinvlemma} and \cref{Kinvbound},
\begin{align*}
||K(\mu)^{-1}|| \leq C \frac{e^{|\text{Re }\mu|X }}{| \det K(\mu) |} \leq C \frac{r^{1/2}}{X} e^{|\text{Re }\mu|X }
\end{align*}
By \cref{mucondition}, $e^{|\text{Re }\mu|X } \leq e^{1/2} = C$, thus we have the bound \cref{Kinvbound}.
\end{proof}
\end{lemma}

\subsection{Interaction Eigenvalues}

We can now use the results from the previous section to find the interaction eigenvalues. We will always assume $\mu$ satisfies \cref{mucondition}. To do this, we take the scaling
\[
\mu = r^{1/2}\tilde{\mu}
\]
In the next lemma, we derive an equation we can solve to find the interaction eigenvalues.

% equation for d
\begin{lemma}\label{deqlemma}
For sufficiently small $r$, the interaction eigenvalues are the values of $\tilde{\mu}$ for which the equation
\begin{equation}\label{eqford}
(\tilde{A} - \tilde{\mu}^2 c^2 MI + \tilde{D}_3)d = 0
\end{equation}
has a nontrivial solution. The remainder term $\tilde{D}_3$ has bound
\begin{equation}\label{tildeD3bound}
||\tilde{D}_3|| \leq C \frac{1}{|\log r|} 
\end{equation}
\begin{proof}
First, we solve the top line of the block matrix equation \eqref{blockeq2} for $c$. Since $\mu$ satisfies \cref{mucondition}, $K(\mu)$ is invertible, and we can write the top line of \eqref{blockeq2} as
\begin{align*}
(I + S_1 K(\mu)^{-1} ) K(\mu) c = -S_2 d
\end{align*}
Taking $\mu = \mathcal{O}(r^{1/2})$ in the bounds from \cref{reparam} and using \cref{Kinvboundslemma}, we have the bound
\begin{align*}
\|I + S_1 K(\mu)^{-1}\| &\leq C \frac{r^{-1/2}}{X} r^{1/2}r^{\tau/2} = C \frac{r^{\tau/2}}{X}
\end{align*}
Since $X = \mathcal{O}(|\log r|)$, $r^{\tau/2}/X \rightarrow 0$ as $r \rightarrow 0$. Thus for sufficiently small $r$, $I + S_1 K(\mu)^{-1}$ is invertible. Let $T_1 = (I + S_1 K(\mu)^{-1})^{-1}$. Then we can solve for $c$ to get
\[
c = -K(\mu)^{-1} T_1 S_2 d
\]
Plugging this into the second line of \eqref{blockeq}, we get the equation for $d$
\begin{equation}\label{eqford2}
(r\tilde{A} - r \tilde{\mu}^2 c^2  M I + D_3)d = 0,
\end{equation}
where
\[
D_3 = S_4 - S_3 K(\mu)^{-1} T_1 S_2 
\]
is the remainder term, which has bound
\begin{align*}
\|D_3\| &\leq C \left( r^{1 + \tau/2} + \frac{r^{-1/2}}{X} r^{3/2} \right) \\
&= C r\left( r^{\tau/2} \frac{1}{X} \right)
\end{align*}
Dividing by $r$, equation \cref{eqford2} becomes
\[
(\tilde{A} - \tilde{\mu}^2 c^2 MI + \tilde{D}_3)d = 0.
\]
Since $X = \mathcal{O}(n |\log r|)$
\[
\|\tilde{D}_3\| \leq C \frac{1}{|\log r|} 
\]
\end{proof}
\end{lemma}

Because of our scaling, the parameter $r$ only occurs in the remainder term, which is what we want. In the next lemma, we solve \eqref{eqford} for the interaction eigenvalues.

% solve for int eigs
\begin{lemma}\label{inteigslemma}
Assume Hypothesis \ref{Adistincteigs}. Then there exists $r_1 \leq r_0$ such that for $r \in \mathcal{R}$ with $r \leq r_1$, there are $n - 1$ pairs of nonzero interaction eigenvalues
\begin{align*}
\lambda &= \pm \lambda^{\text{int}}_j(r) && j = 1, \dots, n-1
\end{align*}
where
\begin{align*}
\lambda^{\text{int}}_j(r) = \sqrt{\frac{\ell_j(0)}{M}}r^{1/2} + \mathcal{O}\left( \frac{r^{1/2}}{|\log r|}\right)
\end{align*}
These eigenvalue pairs are either real or purely imaginary, and the remainder term cannot move them off of the real or imaginary axis.

\begin{proof}
Equation \eqref{eqford} has a nontrivial solution if and only if
\[
\tilde{E}(\tilde{\mu}, r) = \det
\left( \tilde{A} - \tilde{\mu}^2 c^2 M I + \mathcal{O}\left(\frac{1}{|\log r| }\right) \right) = 0
\]
When $r = 0$, since the determinant is a polynomial, thus smooth in $r$, 
\begin{equation}\label{tildeE0}
\tilde{E}(\tilde{\lambda}, 0) = \det(\tilde{A}(0) - \tilde{\mu}^2 c^2 MI)
\end{equation}
The characteristic polynomial of $\tilde{A}(0)$ has roots $\{0,\ell_1(r), \dots, \ell_{n-1}(r) \}$. Thus we can write \cref{tildeE0} as
\begin{equation}\label{tildeE1}
\tilde{E}(\tilde{\lambda}, 0) = \tilde{\mu}^2 c^2 M \left( \tilde{\mu}^2 c^4 M^2 - \ell_1(0) \right )\cdots\left( \tilde{\mu}^2 c^2 M - \ell_{n-1}(0) \right )
\end{equation}
which we can factor and simplify to get
\begin{equation}\label{tildeE2}
\tilde{E}(\tilde{\mu}, 0) = M^n c^{2n} \tilde{\mu}^2
\left( \tilde{\mu} - \tilde{\mu}_*^1(0) \right)
\left( \tilde{\mu} + \tilde{\mu}_*^1(0) \right) \dots
\left( \tilde{\mu} - \tilde{\mu}_*^{n-1}(0) \right)
\left( \tilde{\mu} + \tilde{\mu}_*^{n-1}(0) \right)
\end{equation}

For $j = 1, \dots, n-1$, $\tilde{E}(\tilde{\mu}_*^j(0), 0) = 0$. Since we are assuming the eigenvalues of $\tilde{A}(0)$ are distinct, equation \cref{tildeE2} has no repeated roots other than $0$, thus for $j = 1, \dots, n-1$,
\[
\partial_{\tilde{\mu}} \tilde{E}(\tilde{\mu}_*^j(0), 0) \neq 0
\]
Thus there exists $r_1^j \leq r_0$ so that for $r \leq r_1^j$, we can use the implicit function theorem to solve for $\tilde{\mu}$ in terms of $r$ near $\tilde{\mu}_*^j(0)$. Letting $r_1$ be the minimum of the $r_1^j$, there are unique smooth functions $\tilde{\mu}^j(r)$ with $\tilde{\mu}^j(0) = \tilde{\mu}_*^j(0)$ such that for all $r \leq r_1$,
\[
\tilde{E}(\tilde{\mu}^j(r), r) = 0
\]
Expanding in a Taylor series,
\[
\tilde{\mu}^j(r) = \tilde{\mu}_*^j(0) + \mathcal{O}\left( \frac{1}{|\log r|} \right)
\]
Undoing the scaling and changing variables back to $\lambda$, we have found $n-1$ interaction eigenvalues located at
\begin{align*}
\lambda_j(r) = \pm \sqrt{\frac{\ell_j(0)}{M}} r^{1/2} + \mathcal{O}\left( \frac{r^{1/2}}{|\log r|} \right)
\end{align*}
By Hamiltonian symmetry, eigenvalues must come in quartets. Since the eigenvalues $\ell_j$ of $\tilde{A}$ are distinct, the only way we can satisfy Hamiltonian symmetry is if $\lambda_j(r)$ is either real or purely imaginary and there is a corresponding interaction eigenvalue $-\lambda_j(r)$. Thus the interaction eigenvalues are given by $\lambda = \pm \lambda^{\text{int}}_j(r)$, where
\begin{align*}
\lambda^{\text{int}}_j(r) = \sqrt{\frac{\ell_j(0)}{M}} r^{1/2} + \mathcal{O}\left( \frac{r^{1/2}}{|\log r|} \right)
\end{align*}
By Hamiltonian symmetry, these are real or purely imaginary.
\end{proof}
\end{lemma}

\subsection{Characterization of $A$ }

We will now find the nonzero essential spectrum eigenvalues, which we expect to occur near the points $m \pi i/X$ for integer $m$. Let $N$ be as in the statement of \cref{locateeigtheorem}. Since the smallest nonzero essential spectrum eigenvalue is close to $\pi i/X$ and we are not considering essential spectrum eigenvalues larger than $N \pi i/X$, in this section we will consider $\mu$ with
\begin{equation}\label{mucondess}
\frac{1}{X} \leq |\mu| \leq \frac{(N+1)\pi}{X}
\end{equation}

The first step is to invert the matrix $A - \mu^2 c^2 M I$. We will start by proving  a lower bound the determinant of $A - \mu^2 c^2 M I$. 

% lemma : bound on det
\begin{lemma}\label{detAboundlemma}
If $\mu$ satisfies \cref{mucondess},
\begin{equation}\label{detAbound}
|\det(A - \mu^2 c^2 M I)|
\geq C \frac{|\mu|^{n-1}}{X^{n+1}}
\end{equation}
\begin{proof}
$\det(A - \mu^2 c^2 MI)$ is the characteristic polynomial $p(t)$ of $A$ with $t = \mu^2 / M c^2$. Since the eigenvalues of $A$ are $\{0, r \ell_1(r), \dots, r \ell_j(r) \}$, we have, similar to \cref{tildeE2},
\begin{align*}
\det(A &- \mu^2 c^2 M I) = c^2 M \mu^2 \left(\mu^2 c^2 M - r \ell_1(r) \right)\cdots\left(\mu^2 c^2 M - r \ell_{n-1}(r) \right) \\
&= c^{2n} M^n \mu^2 \left( \mu - \sqrt{\frac{\ell_1(r)}{M c^2 }} r^{1/2} \right)\left( \mu + \sqrt{\frac{\ell_1(r)}{M c^2 }} r^{1/2} \right)\cdots\left( \mu - \sqrt{\frac{\ell_{n-1}(r)}{M c^2 }} r^{1/2} \right)\left( \mu + \sqrt{\frac{\ell_{n-1}(r)}{M c^2 }} r^{1/2} \right)
\end{align*}
By \cref{mucondess}, $n-1$ of the terms in $\det(A - \mu^2 c^2 M I)$ are bounded below by $1/2X$, the $|\mu|^2$ term is bounded below by $1/X^2$, and the remaining $n-1$ terms are bounded below by $\mu$. This gives us the lower bound \cref{detAbound}.
\end{proof}
\end{lemma}

In the next lemma, we derive a bound for $(A - \mu^2c^2 M I)^{-1}$.

% bound on inverse of A
\begin{lemma}\label{Ainvboundlemma}
For $\mu$ satisfying \cref{mucondess},
\begin{align}\label{Ainvbound1}
\|(A - \mu^2 c^2 M I)^{-1}\| &\leq C X^{n+1} |\mu|^{n-1}
\end{align}
\begin{proof}
The inverse of $(A - \mu^2 c^2 M I)^{-1}$ is given by the formula
\[
(A - \mu^2 c^2 M I)^{-1} = \frac{1}{\det(A - \mu^2 c^2 M I)}\text{Adj}(A - \mu^2 c^2 M I)
\]
where $\text{Adj}(A - \mu^2 c^2 M I)$ is the adjugate matrix (the transpose of the cofactor matrix). Since $A$ is $n \times n$, each entry in $\text{Adj}(A - \mu^2 c^2 M I)$ involves sums of products of $n-1$ of the elements of $A - \mu^2 c^2 M I$, each of which is $\mathcal{O}(r + |\mu|^2)$. Thus we have the estimate
\[
\| \text{Adj}(A - \mu^2 c^2 M I)\| = \mathcal{O}(r + |\mu|^2)^{n-1}
\]
Combining this with the bound on the determinant from \cref{detAboundlemma}, we have
\begin{align*}
\|(A - \mu^2 c^2 M I)^{-1}\| &\leq C \frac{X^{n+1}}{|\mu|^{n-1}} (r + |\mu|^2)^{n-1}
\end{align*}
By \cref{mucondess} and the last inequality of \cref{mucondition}, $|\mu| \geq C r^{1/2}$, thus this simplifies to
\begin{align*}
\|(A - \mu^2 c^2 M I)^{-1}\| &\leq C X^{n+1} |\mu|^{n-1}
\end{align*}
\end{proof}
\end{lemma}

\subsection{Essential spectrum eigenvalues}

To find the essential spectrum eigenvalues, we solve the second line of of the block matrix equation \eqref{blockeq2} for $d$ and plug it into the first line. We will always assume that \cref{mucondess} holds. In the next lemma, we derive an equation we can solve to obtain the essential spectrum eigenvalues.

% lemma : equation for c
\begin{lemma}\label{ceqlemma}
Let $N$ be as in the statement of the theorem. Then for sufficiently small $r$, the essential spectrum eigenvalues are the values of $\mu$ for which 
\begin{align}\label{eqforc}
(K(\mu) + C_5)c &= 0
\end{align}
has a nontrivial solution, where
\[
\|C_5\| \leq C \frac{1}{X^2}
\]
\begin{proof}
Since $|\mu| \geq 1/X$ by \cref{mucondess}, $A - \mu^2 c^2 MI$ is nonsingular, and we can write the bottom line of \eqref{blockeq2} as 
\begin{align}\label{blockeqbottom}
S_3 c + (A - \mu^2 c^2 MI)(I + (A - \mu^2 c^2 MI)^{-1} S_3))d = 0
\end{align}
Using the bounds from \cref{reparam} and \cref{Ainvboundlemma} and the second inequality in \cref{mucondess},
\begin{align*}
\|(A - \mu^2 c^2 MI)^{-1} S_3 \|
&\leq C X^{n+1} |\mu|^{n-1} |\mu|^3 \\
&\leq C X^{n+1}\frac{(N+1)^{n+2} \pi^{n+2}}{X^{n+2}} \\
&\leq \frac{C}{X}
\end{align*}
where the constant $C$ depends on our choice of $N$. Since $X = \mathcal{O}(|\log r|)$, for sufficiently small $r$, $I + (A - \mu^2 c^2 MI)^{-1} S_3)$ is invertible. Let
\[
T_2 = (I + (A - \mu^2 c^2 MI)^{-1} S_3)^{-1}
\]
Then we can solve for $d$ in the second line of the block matrix equation to get
\[
d = -T_2 (A - \mu^2 c^2 MI)^{-1}S_3 c.
\]
Substitute this into the top line of the block matrix equation \cref{blockeq2} to get
\begin{align*}
(K(\mu) + C_5) c = 0
\end{align*}
where 
\[
C_5 = (S_1 - S_2 T_2 (A - \mu^2 c^2 MI)^{-1}S_3
\]
Using the bounds from \cref{reparam} and \cref{Ainvboundlemma},
\begin{align*}
\| C_5 \| &\leq C \left( |\mu|^2 + |\mu|^2 X^{n+1} |\mu|^{n-1}|\mu|^3 \right) \\
&\leq C \left( |\mu|^2 + X^{n+1}|\mu|^{n+4} \right) \\
&\leq C \frac{1}{X^2}
\end{align*}
\end{proof}
\end{lemma}

In the next lemma, we will solve \cref{eqforc} to find the essential spectrum eigenvalues.

% find essential spectrum eigs
\begin{lemma}\label{essspeclemma}
Let $N$ be as in the statement of the theorem. Then There exists $r_2 \leq r_0$ such that for $r \in \mathcal{R}$ with $r \leq r_0$, there are $N$ pairs of essential spectrum eigenvalues at $\lambda = \pm \lambda_m^{\text{ess}}(r)$ for $m = 1, \dots, N$, where
\begin{align}
\lambda^{\text{ess}}_m(r) = c \frac{m \pi i }{X} + \mathcal{O}\left( \frac{1}{|\log r|^2} \right)
\end{align}
These are pure imaginary, and the remainder terms cannot move them off of the imaginary axis.

\begin{proof}
Choose any positive integer $m$ with $m \leq N$, and let
\begin{equation}\label{essspecmu}
\mu = \frac{m \pi i}{X} + \frac{h}{X}
\end{equation}
Let 
\[
K_m(h) =  K\left( \frac{m \pi i}{X} + \frac{h}{X} \right)
\]
which, from the definition of $K(\mu)$, no longer depends on $X$. By Corollary \ref{detKcorr},
\[
\det K_m(h) = -2\sinh(m \pi i + h)
\]
Substitute \cref{essspecmu} into \cref{eqforc} to get
\[
\left( K_m(h) + \mathcal{O}\left( {\frac{1}{|\log r|^2}} \right)\right)c = 0,
\]
where we used the fact that $X = \mathcal{O}(|\log r|)$. This has a nontrivial solution if and only if
\begin{equation}\label{defEmess}
E_m(h, r) = \det \left( K_m(h) + \mathcal{O}\left( {\frac{1}{|\log r|^2}} \right)\right) = 0
\end{equation}
Since the determinant is a polynomial, it is smooth in $r$. Since $K_m(h)$ does not depend on $r$,
\[
E_m(0, 0) = -2 \sinh(m \pi i) = 0
\]
and 
\[
\partial_h E_m(0, 0) = -2 \cosh(m \pi i) = -2(-1)^m \neq 0
\]
Thus there exists $r_2^m \leq r_0$ such for $r \leq r_2^m$, we can use the implicit function theorem to solve for $h$ in terms of $r$ near $h = 0$. Doing this for $m = 1, \dots, N$ and letting $r_2$ be the minimum of the $r_2^m$, there are unique smooth functions $h_m(r)$ with $h_m(r) = 0$ such that for all $r \leq r_2$,
\[
E_m(h_m(r), 0) = 0
\]
Undoing the scaling and expanding $h(r)$ in a Taylor series, we have found essential spectrum eigenvalues at
\begin{align}\label{muess}
\mu_m(r) &= \frac{m \pi i}{X} + \mathcal{O}\left(\frac{1}{X^2}\right) = \frac{m \pi i}{X} + \mathcal{O}\left(\frac{1}{|\log r|^2}\right) 
\end{align}
for $m = 1, \dots, N$. Changing variables back to $\lambda$ by taking $\lambda = \nu(\mu)$, there are essential spectrum eigenvalues at
\begin{align*}
\lambda_m(r) &= c \frac{m \pi i }{X} \left( 1 + \mathcal{O}\left( \frac{1}{|\log r|^2} \right)\right) + \mathcal{O}\left( \frac{1}{|\log r|^2} \right) && m = 1, \dots, N
\end{align*}
which simplifies to 
\begin{align*}
\lambda_m(r) &= c \frac{m \pi i }{X} + \mathcal{O}\left( \frac{1}{|\log r|^2} \right) && m = 1, \dots, N
\end{align*}
By Hamiltonian symmetry, these eigenvalues cannot come in quartets, thus they must be purely imaginary. Furthermore, there must also be an essential spectrum eigenvalue at $-\lambda_m(r)$. We conclude that there are $N$ pairs of nonzero essential spectrum eigenvalues at $\lambda = \pm \lambda_m^{\text{ess}}(r)$ for $m = 1, \dots, N$, where
\begin{align}
\lambda^{\text{ess}}_m(r) = c \frac{m \pi i }{X} + \mathcal{O}\left( \frac{1}{|\log r|^2} \right)
\end{align}
These are pure imaginary, and the remainder terms cannot move them off of the imaginary axis.
\end{proof}
\end{lemma}

\subsection{Eigenvalue counts}

Finally, we will perform two counts of the eigenvalues near 0 so that we can conclude that we have accounted for everything. First, we will count the eigenvalues in a large cicle around the origin in the complex plane.

\begin{lemma}\label{eigcount}
Let $r_1$ be as in part (i) of \cref{locateeigtheorem}. Let $r_2$ correspond to the integer $N+1$ in part (ii) of \cref{locateeigtheorem}. Then there exists $r_3 \leq \min\{r_1, r_2\}$ such that there are exactly $2n + 2 N + 1$ eigenvalues inside the circle $|\lambda| = c \xi$, where
\[
\xi = \frac{\pi}{X}\left( N + \frac{1}{2} \right)
\]

\begin{proof}
To do this, we again change variables by taking $\mu = \nu(\lambda)$. The circle of radius $\xi$ in the complex plane cuts (approximately) half way between $\mu_N(r)$ and $\mu_{N+1}(r)$, where $\mu_m(r)$ is defined in \cref{muess}. 

Since $\xi$ and the singular points $m \pi i/X$ of $K(\mu)$ scale exactly the same as $r$ varies, $\det(K(\mu)) = -2 \sinh \mu X$ will always have exactly $2N + 1$ zeros inside the circle of radius $\xi$. By \cref{mucondition}, all the zeros of $\det(A - \mu^2 c^2 M I)$ are inside the circle of radius $\xi$ for all $r \leq r_0$. We will use Rouch\'es theorem to show that there are no other solutions $\mu$ to the block matrix equation \cref{blockeq2} inside the circle.

Take $\mu$ on this circle, i.e. $|\mu| = \xi$. By our choice of $\mu$, $K(\lambda)$ is invertible and we can expand $\det K(\mu) = \sinh \mu X$ in a Taylor series about $N \pi i/X$ to get the lower bound $|\det K(\mu)| \geq C$. Following the proof of Lemma \ref{Kinvboundslemma}, this implies
\begin{equation}\label{Kinvboundxi}
\|K(\mu)^{-1}\| \leq C
\end{equation}
By \cref{mucondition}, the $2n-2$ nonzero roots of $\det(A - \mu^2 c^2 M I)$ are at least $1/2X$ from $\mu$, thus following the proof of Lemma \ref{Ainvboundlemma}, we have
\begin{equation}\label{Ainvboundxi}
\|(A - \mu^2 c^2 M I)^{-1}\| \leq X^{n+1} \xi^{n-1} \leq C X^2
\end{equation}
where the second inequality follows from our choice of $\xi$. 

Factoring $K(\mu)$ out of the top left of \cref{blockeq2}, rewrite the block matrix equation as
\begin{equation}\label{blockeq3}
\begin{pmatrix}
(I + S_1 K(\mu)^{-1})K(\mu) & S_2 \\
S_3 & A - \mu^2 c^2 MI + S_4
\end{pmatrix}
\begin{pmatrix} c \\ d \end{pmatrix} = 0
\end{equation}
From the proof of Lemma \ref{deqlemma}, $I + S_1 K(\mu)^{-1}$ is invertible with inverse $T_1$. Multiply the top row of \eqref{blockeq3} by $T_1$ to obtain the equivalent system
\begin{equation}\label{blockeq4}
\begin{pmatrix}
K(\mu) & T_1 S_2 \\
S_3 & A - \mu^2 c^2 MI + S_4
\end{pmatrix}
\begin{pmatrix} c \\ d \end{pmatrix} = 0
\end{equation}
which has a nontrivial solution if and only if
\begin{equation}\label{countdefE}
E(\mu) = \det 
\begin{pmatrix}
K(\mu) & T_1 S_2 \\
S_3 & A - \mu^2 c^2 MI + S_4
\end{pmatrix} = 0
\end{equation}

Using a standard determinant identity, since both $K(\mu)$ and $A - \mu^2 c^2 M I$ are invertible, we can write $E(\mu)$ as
\begin{align*}
E(\mu) &= \det(K(\mu))
\det ( A - \mu^2 c^2 MI + S_4 - S_3 K(\mu)^{-1}T_1 S_2 ) \\
&= \det(K(\mu))\det(A - \mu^2 MI)
\det ( I + (A - \mu^2 c^2 MI)^{-1}(S_4 - S_3 K(\mu)^{-1}T_1 S_2) ) \\
&= \det(K(\mu))\det(A - \mu^2 c^2 MI)\det(I + R(\mu))
\end{align*}
where
\[
R(\mu) = (A - \mu^2 c^2 MI)^{-1}(S_4 - S_3 K(
\mu)^{-1}T_1 S_2)
\]
Using the bounds \eqref{Ainvboundxi} and \eqref{Kinvboundxi} together with the bounds from Lemma \ref{reparam} and our choice of $\xi$, the remainder term $R(\mu)$ has bound
\begin{align*}
||R(\mu)|| \leq C X^2 \xi^3 \leq C \xi
\end{align*}

Let $R(\mu) = \xi \tilde{R}(\mu)$, where $\tilde{R}(\mu) = \mathcal{O}(1)$. From a standard expansion of the determinant, 
\begin{align*}
\det(I + R(\mu)) &= 1 + \xi \text{Tr}(\tilde{R}(\mu)) + \mathcal{O}(\xi^2) = 1 + \mathcal{O}(\xi)
\end{align*}
Since 
\[
\xi = \mathcal{O}\left( \frac{1}{|\log r|} \right)
\]
there exists $r_3 \leq \min\{r_1, r_2\}$ such that for $r \leq r_3$, 
\[
\det(I + R(\mu)) = 1 + \epsilon(r),
\]
where $|\epsilon(r)| < 1$. Thus we can write $E(\mu)$ as
\begin{equation}
E(\mu) = \det(K(\mu))\det(A - \mu^2c^2 MI) + \epsilon(r) \det(K(\mu))\det(A - \mu^2c^2 MI)
\end{equation}

Since $\epsilon(r) < 1$ and $|\mu| = \xi$, by Rouch\'e's theorem, $E(\mu)$ and $\det(K(\mu))\det(A - \mu^2 c^2 MI)$ have the same number of zeros (counting multiplicty) inside the circle of radius $\xi$. As discussed above for all $r \leq r_0$, $\det(K(\mu))\det(A - \mu^2 c^2 MI)$ has exactly $2n + 2N + 1$ zeros inside this circle. Changing variables back to $\lambda$, there are exaclty $2n + 2N + 1$ eigenvalues inside the circle $|\lambda| = c \xi$. 
\end{proof}
\end{lemma}

Finally, we will count the eigenvalues in a small ball around the 0.

\begin{lemma}\label{eigcount2}
Let 
\begin{equation}\label{xiradius}
\xi = \frac{1}{2}r^{1/2} \sqrt{\frac{R_{\min}}{M}}
\end{equation}
where
\[
R_{\min} = \min_{j = 1, \dots, n-1} |\ell_j(0) |
\]
Then there exists $r_4 \leq r_0$ such that for $r \leq r_4$, there are exactly 3 eigenvalues inside the circle of radius $\xi$ in the complex plane.

\begin{proof}
The argument is almost identical to the previous lemma. Since we are close to 0, we do not need to change variables to $\mu$. For this choice of $\xi$, the radius $\xi$ and the nonzero eigenvalues $r^{1/2}\ell_j(r)$ of $A$ scale together, thus there will always be exactly 2 zeros of $A - \lambda^2 M I$ inside the circle of radius $\xi$. Since $r \leq r_0$, by \cref{nobubblecondgen}, the only zero of $K(\lambda)$ inside the circle is the single root at $\lambda = 0$. Thus $K(\lambda)(A - \lambda^2 M I)$ has exactly three roots inside the circle. We will again use Rouch\'e's theorem so show that these are the only roots.

For $|\lambda| = \xi$, $K(\lambda)$ is invertible, and the inverse has lower bound
\[
\| K(\lambda)^{-1}\| \leq C \frac{r^{1/2}}{X}
\]
For $(A - \lambda^2 M I)$, when $|\lambda| = \xi$, $\lambda$ is at least $\xi$ from all $2n$ roots of $\det(A - \lambda^2 M I)$. Thus, following the proofs of Lemma \ref{detAboundlemma} and Lemma \ref{Ainvboundlemma}, we have
\[
\| (A - \lambda^2 M I)^{-1} \| \leq C \frac{ r^{2(n-1)}}{\xi^{2n}} 
\]
As in the previous lemma, for sufficiently small $r$ less than $r_0$, the block matrix equation \cref{blockeq2} has a nontrivial solution if and only if 
\begin{align*}
E(\lambda) &= \det(K(\lambda))\det(A - \lambda^2 MI)\det(I + R(\lambda)) = 0
\end{align*}
where the remainder term $R(\lambda)$ has bound
\begin{align*}
\| R(\lambda) \| &\leq C \| (A - \lambda^2 MI)^{-1}(S_4 - S_3 K(
\mu)^{-1} S_2) \| \\
&\leq C \left( \frac{r^{2(n-1)}}{\xi^{2n}} r \left( r^{\tau/2} + \frac{1}{|\log r| } \right) \right) \\
&\leq C r^{n-1}
\end{align*}
As in the previous lemma, we can choose $r_4 \leq r_0$ so that for all $r \leq r_4$, 
\[
\det(I + R(\lambda)) = 1 + \epsilon(r),
\]
where $|\epsilon(r)| < 1$. Thus we can write $E(\lambda)$ as
\begin{equation}
E(\lambda^2) = \det(K(\lambda^2))\det(A - \lambda^2 MI) + \epsilon(r) \det(K(\lambda^2))\det(A - \lambda^2 MI)
\end{equation}
As in the previous lemma, by Rouch\'e's Theorem, $E(\lambda)$ and $\det(K(\lambda))\det(A - \lambda^2 MI)$ have the same number of zeros (counting multiplicty) inside the circle $|\lambda| = \xi$. As discussed above, $\det(K(\lambda))\det(A - \lambda^2 MI)$, and thus $E(\lambda)$, has exactly 3 zeros inside the circle $|\lambda| = \xi$.
\end{proof}
\end{lemma}

\subsection{Proof of Theorem \ref{locateeigtheorem}}

For part (i), $\partial_x Q_n(x)$ is an eigenfunction with eigenvalue 0, $\partial_c Q_n(x)$ is a generalized eigenfunction with eigenvalue 0 corresponding to $\partial_x Q_n(x)$. By Hypothesis \ref{Melnikov2hyp}, this Jordan chain cannot continue.

\iffulldocument\else
	\bibliographystyle{amsalpha}
	\bibliography{thesis.bib}
\fi

\end{document}
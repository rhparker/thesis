\documentclass[thesis.tex]{subfiles}

\begin{document}

Consider the equations
\begin{align}
V'(x) = A(Q(x); \lambda) V(x) \label{Veqlambda} \\
W'(x) = -A(Q(x); \lambda)^* W(x) \label{Weqlambda}
\end{align}\
where $Q(x)$ is the primary pulse solution and $A(Q(x), \lambda) = A(Q(x)) + \lambda B$. When $\lambda = 0$, these are the variational and adjoint variational equations.

Recall that $A(0; \lambda)$ has a small eigenvalue $\nu(\lambda)$. Let $V_0(\lambda)$ and $W_0(\lambda)$ be the eigenvectors of $A(0)$ and $-A(0)^*$ corresponding to $\nu(\lambda)$. For convenience, we let $V_0 = V_0(0)$ and $W_0 = W_0(0)$.

In Lemma \ref{varsolutions}, we showed that when $\lambda = 0$, \eqref{Veqlambda} has a bounded solution $V^c(x)$ with
\[
V^c(x) \rightarrow V_0(0) \text{ as }|x| \rightarrow \infty
\]
which is symmetric with respect to the standard reversor operator $R$. In Lemma \ref{adjsolutions}, we showed that when $\lambda = 0$, \eqref{Weqlambda} has an exponentially decaying solution $\Psi(x)$ and a bounded solution $\Psi^c(x)$ with 
\[
\Psi^c(x) \rightarrow W_0 \text{ as }|x| \rightarrow \infty
\]

Using the Gap Lemma, we can find solutions $V^\pm(x; \lambda)$ to \eqref{Veqlambda} and $W^\pm(x; \lambda)$ to \eqref{Weqlambda} on $\R^\pm$ such that
\begin{align}
V^\pm(x; \lambda) &= V_0(\lambda)e^{\nu(\lambda)x} + V_R^\pm(x; \lambda) \label{Vpm} \\
W^\pm(x; \lambda) &= W_0(\lambda)e^{\nu(\lambda)x} + W_R^\pm(x; \lambda) \label{Vpm}
\end{align}
where for the remainder terms we have $V_R^\pm(x; \lambda), W_R^\pm(x; \lambda) = \mathcal{O}(e^{-\alpha |x|})$. We have also scaled things so that $\langle V_0(\lambda), W_0(\lambda) \rangle = 1$. These solutions are analytic in $\lambda$. They are also not unique. We would like to better characterize these solutions. 

\begin{enumerate}

\item Scale out $e^{\nu(\lambda)x}$. Write 
\begin{align*}
V^\pm(x; \lambda) &= e^{\nu(\lambda)x} (V_0(\lambda) + e^{-\nu(\lambda)x} V_R^\pm(x; \lambda)) \\
&= e^{\nu(\lambda)x}(V_0(\lambda) + \tilde{V}_R^\pm(x; \lambda))
\end{align*}
where $\tilde{V}_R^\pm(x; \lambda) = \mathcal{O}(e^{-(\alpha - \eta) |x|})$. Let
\begin{equation}\label{tildeV}
V^\pm(x; \lambda) = e^{\nu(\lambda)x}\tilde{V}^\pm(x; \lambda)
\end{equation}
Then we have
\[
\tilde{V}^\pm(x; \lambda) = V_0(\lambda) + \tilde{V}_R^\pm(x; \lambda)
\]

\item Derive an ODE which is solved by $\tilde{V}^\pm(x; \lambda)$. Differentiating equation \eqref{tildeV}, we obtain
\begin{align*}
\nu(\lambda) e^{\nu(\lambda)x}\tilde{V}^\pm(x; \lambda) 
+ e^{\nu(\lambda)x}[\tilde{V}^\pm(x; \lambda)]'
&= A(Q(x); \lambda) (e^{\nu(\lambda)x}\tilde{V}^\pm(x; \lambda))
\end{align*}
Dividing by $e^{\nu(\lambda)x}$ and rearranging, we get the ODE for $\tilde{V}^\pm(x; \lambda)$.
\begin{align*}
[\tilde{V}^\pm(x; \lambda)]'
&= [A(Q(x); \lambda) - \nu(\lambda)I]\tilde{V}^\pm(x; \lambda)
\end{align*}
which, for convenience, we will write as
\begin{align}\label{tildeVeq}
[\tilde{V}^\pm(x; \lambda)]'
&= \tilde{A}(Q(x); \lambda)\tilde{V}^\pm(x; \lambda)
\end{align}
where 
\begin{align*}
\tilde{A}(Q(x); \lambda) &= A(Q(x); \lambda) - \nu(\lambda)I \\
&= A(Q(x)) + \lambda B - \nu(\lambda)I
\end{align*}
We note that $\mu$ is an eigenvalue of the asympotitic matrix $A(0; \lambda)$ if and only if $\mu - \nu(\lambda)$ is an eigenvalue of $\tilde{A}(0; \lambda)$, i.e. all eigenvalues are shifted by $\nu(\lambda)$. Thus the asymptotic matrix $\tilde{A}(0; \lambda)$ has a simple eigenvalue at 0, and all other eigenvalues $\mu$ satisfy
\[
|\text{Re }\mu| \geq \alpha_0 - |\nu(\lambda)| \geq \alpha_0 - \eta = \alpha
\]

\item Taylor expand $\tilde{V}^\pm(x; \lambda)$. Since $\tilde{V}^\pm(x; \lambda)$ is still analytic in $\lambda$, and $\tilde{V}^\pm(x; 0) = V^c(x)$, we have the leading order expansion
\begin{align*}
\tilde{V}^\pm(x; \lambda) &= \tilde{V}^\pm(x; 0) + \lambda V_1^\pm(x; \lambda) \\
&= V^c(x) + \lambda V_1^\pm(x; \lambda)
\end{align*}

\item Derive an ODE which is solved by $V_1^\pm(x; \lambda)$.
Plug the expansion for $\tilde{V}^\pm(x; \lambda$ into equation \eqref{tildeVeq} to get
\begin{align*}
[V^c(x)]' + \lambda [V_1^\pm(x; \lambda)]'
&= \tilde{A}(Q(x); \lambda)(V^c(x) + \lambda V_1^\pm(x; \lambda)) \\
&= (A(Q(x)) + \lambda B - \nu(\lambda)I)V^c(x) + \lambda \tilde{A}(Q(x); \lambda) V_1^\pm(x; \lambda) \\
\lambda [V_1^\pm(x; \lambda)]' &= (\lambda B - \nu(\lambda)I)V^c(x) + \lambda \tilde{A}(Q(x); \lambda) V_1^\pm(x; \lambda)
\end{align*}
Substituting the expansion for $\nu(\lambda)$
\[
\nu(\lambda) = \frac{1}{c} \lambda + \mathcal{O}(|\lambda|^3),
\]
this becomes
\begin{align*}
\lambda [V_1^\pm(x; \lambda)]' &= \lambda \tilde{A}(Q(x); \lambda) V_1^\pm(x; \lambda) + \left(\lambda B - \frac{1}{c} \lambda I + \mathcal{O}(|\lambda|^3)I\right)V^c(x)
\end{align*}
Finally, divide by $\lambda$ to get the ODE for $V_1^\pm$
\begin{align}\label{V1eq}
[V_1^\pm(x; \lambda)]' &= \tilde{A}(Q(x); \lambda) V_1^\pm(x; \lambda) + \tilde{B} V^c(x) + \lambda^2 R(x)
\end{align}
where
\[
\tilde{B} = B - \frac{1}{c} I
\]
and $||R|| \leq C$. 

\item Compare the two expressions for $\tilde{V}^\pm(x; \lambda)$, i.e. 
\begin{align*}
V^c(x) + \lambda V_1^\pm(x; \lambda) &= V_0(\lambda) + \tilde{V}_R^\pm(x; \lambda)
\end{align*}
Taylor expand $V_0(\lambda)$ about $\lambda = 0$ to get
\[
V_0(\lambda) = V_0 + \lambda Z_0(\lambda)
\]
and recall that when $\lambda = 0$ we have
\[
V^c(x) = V_0 + V^c_R(x)
\]
where $V^c_R(x) = \mathcal{O}(e^{-\alpha |x|})$. Since
\begin{align*}
\tilde{V}^\pm(x; \lambda) = V_0(\lambda) + \tilde{V}_R^\pm(x; \lambda)
\end{align*}
we take $\lambda = 0$ to get
\begin{align*}
\tilde{V}^\pm(x; 0) = V_0 + \tilde{V}_R^\pm(x; 0) \\
V^c(x) = V_0 + \tilde{V}_R^\pm(x; 0) 
\end{align*}
from which we see that
\[
\tilde{V}_R^\pm(x; 0) = V^c_R(x)
\]
This allows us to Taylor expand $\tilde{V}_R^\pm(x; \lambda)$ about $\lambda = 0$ to get
\[
\tilde{V}_R^\pm(x; \lambda) = V^c_R(x) + \lambda V_2^\pm(x; 
\lambda)
\]
Substituting all of these in, we get
\begin{align*}
V_0 + V^c_R(x) + \lambda V_1^\pm(x; \lambda) &= V_0 + \lambda Z_0(\lambda) + V^c_R(x) + \lambda V_2^\pm(x; \lambda) 
\end{align*}
which simplifies to
\begin{equation}\label{V2eq}
V_1^\pm(x; \lambda) = Z_0(\lambda) + V_2^\pm(x; \lambda)
\end{equation}
where $V_2^\pm(x; \lambda) = \mathcal{O}(e^{-(\alpha - \eta)x}) $. So basically we know that $V_1^\pm(x; \lambda)$ is a constant plus something which decays exponentially.

\item Plug this in to get an equation for $V_2^\pm(x; \lambda)$. For what follows, we will take $\lambda$ fixed, so we suppress the dependence on $\lambda$ for convenience of notation. We write equation \eqref{V1eq} as
\begin{align*}
[V_1^\pm(x)]' &= \tilde{A}(Q(x)) V_1^\pm(x) + \tilde{B} V^c(x) + \lambda^2 R(x)
\end{align*}
Plugging in \eqref{V2eq}, we get
\begin{align*}
[Z_0(\lambda) + V_2^\pm(x)]' &= \tilde{A}(Q(x)) [Z_0(\lambda) + V_2^\pm(x)] + \tilde{B} V^c(x) + \lambda^2 R(x) \\
[V_2^\pm(x)]' &= \tilde{A}(Q(x)) V_2^\pm(x) + \tilde{A}(Q(x)) Z_0(\lambda) + \tilde{B} V^c(x) + \lambda^2 R(x) \\
\end{align*}
For convenience, let
\[
H(x) = \tilde{A}(Q(x)) Z_0(\lambda) + \tilde{B} V^c(x) + \lambda^2 R(x)
\]
Then we have
\[
[V_2^\pm(x)]' = \tilde{A}(Q(x)) V_2^\pm(x) + H(x)
\]
Let $\Phi(x,y)$ be the evolution operator for the ODE
\[
V' = \tilde{A}(Q(x))V
\]
which has an exponential trichotomy on $\R^+$ and $\R^-$ with a one-dimensional center subspace corresponding to the eigenvalue $\lambda = 0$ of the asymptotic matrix. Since the center subspace is one-dimensional, we can show that the evolution operator in that subspace is given by
\[
\Phi^c_\pm(x,y) = \langle \tilde{W}^\pm(x), \cdot \rangle \tilde{V}^\pm(x)
\]
Then we can write our system in integrated form as the following set of fixed point equations
\begin{equation*}
\begin{aligned}
V_2^-(x) &= \Phi_u^-(x, 0) b^- + \Phi_c^-(x, -L) c^- \\
&+ \int_0^x \Phi_u^-(x, y) H(y) dy + \int_{-\infty}^x \Phi_s^-(x, y) H(y) dy + \int_{-L}^x \Phi_c^-(x, y) H(y) dy \\ 
V_2^+(x) &= \Phi_s^+(x, 0) b^+ + \Phi_c^+(x, L)c^+ \\
&+ \int_0^x \Phi_s^+(x, y)H(y) dy
+ \int_{\infty}^x \Phi_u^+(x, y)H(y) dy + \int_{L}^x \Phi_c^+(x, y) H(y) dy
\end{aligned}
\end{equation*}
where $L$ is arbitrary (for now). Ideally we would like to take $L = \pm \infty$, which requires that the integrals converge. For the initial conditions we have
\begin{align*}
b^- &\in \text{ran}(P_u^-(0)) \\
b^+ &\in \text{ran}(P_s^+(0)) \\
c^\pm &\in E^c \\
\end{align*}
Substituting in the form of the center evolution,
\begin{equation*}
\begin{aligned}
V_2^-(x) &= \Phi_u^-(x, 0) b^- + \langle \tilde{W}^-(x), c^- \rangle \tilde{V}^-(x) \\
&+ \int_0^x \Phi_u^-(x, y) H(y) dy + \int_{-\infty}^x \Phi_s^-(x, y) H(y) dy + \int_{-L}^x \langle \tilde{W}^-(x), H(y) \rangle \tilde{V}^-(x) dy \\ 
V_2^+(x) &= \Phi_s^+(x, 0) b^+ +  \langle \tilde{W}^+(x), c^+ \rangle \tilde{V}^+(x) \\
&+ \int_0^x \Phi_s^+(x, y)H(y) dy
+ \int_{\infty}^x \Phi_u^+(x, y)H(y) dy + \int_{L}^x \langle \tilde{W}^+(x), H(y) \rangle \tilde{V}^+(x) dy
\end{aligned}
\end{equation*}

\end{enumerate}
\end{document}
\documentclass[thesis.tex]{subfiles}

\begin{document}

Consider the equations
\begin{align}
V'(x) = A(Q(x); \lambda) V(x) \label{Veqlambda} \\
W'(x) = -A(Q(x); \lambda)^* W(x) \label{Weqlambda}
\end{align}\
where $Q(x)$ is the primary pulse solution and $A(Q(x), \lambda) = A(Q(x)) + \lambda B$. When $\lambda = 0$, these are the variational and adjoint variational equations.

Recall that $A(0; \lambda)$ has a small eigenvalue $\nu(\lambda)$. Let $V_0(\lambda)$ and $W_0(\lambda)$ be the eigenvectors of $A(0)$ and $-A(0)^*$ corresponding to $\nu(\lambda)$. 

In Lemma \ref{varsolutions}, we showed that when $\lambda = 0$, \eqref{Veqlambda} has a bounded solution $V^c(x)$ with
\[
V^c(x) \rightarrow V_0(0) \text{ as }|x| \rightarrow \infty
\]
which is symmetric with respect to the standard reversor operator $R$. In Lemma \ref{adjsolutions}, we showed that when $\lambda = 0$, \eqref{Weqlambda} has an exponentially decaying solution $\Psi(x)$ and a bounded solution $\Psi^c(x)$ with 
\[
\Psi^c(x) \rightarrow W_0 \text{ as }|x| \rightarrow \infty
\]

Using the Gap Lemma, we can find solutions $V^\pm(x; \lambda)$ to \eqref{Veqlambda} and $W^\pm(x; \lambda)$ to \eqref{Weqlambda} on $\R^\pm$ such that
\begin{align*}
V^\pm(x; \lambda) &= V_0(\lambda)e^{\nu(\lambda)x} + \tilde{V}^\pm(x; \lambda) \\
W^\pm(x; \lambda) &= W_0(\lambda)e^{\nu(\lambda)x} + \tilde{W}^\pm(x; \lambda)
\end{align*}
where $\tilde{V}^\pm(x; \lambda), \tilde{W}^\pm(x; \lambda) = \mathcal{O}(e^{-\tilde{\alpha}|x|})$. Furthermore, these solutions are analytic in $\lambda$. For $\lambda$ small, $V^\pm(x; \lambda)$ is a small perturbation of $V^c(x)$, thus we write 
\[
V^\pm(x; \lambda) = V^c(x) + V_1^\pm(x; \lambda)
\]
Substituting this into \eqref{Veqlambda}, we get
\begin{align*}
(V^c)'(x) &+ (V_1^\pm)'(x; \lambda)
= (A(Q(x)) + \lambda B)(V^c(x) + V_1^\pm(x; \lambda)) \\
&= A (Q(x))V^c(x) + A(Q(x))V_1^\pm(x; \lambda) + \lambda B V^c(x) + \lambda B V_1^\pm(x; \lambda) \\
\end{align*}
Since $V^c(x)$ satisfies \eqref{Veqlambda} with $\lambda = 0$, this simplifies to the following ODE for $V_1^\pm(x; \lambda)$.
\begin{align*}
(V_1^\pm)'(x; \lambda)
= A(Q(x))V_1^\pm(x; \lambda) + \lambda B V_1^\pm(x; \lambda) + \lambda B V^c(x) \\
\end{align*}
Let $\Phi(x,y)$ be the evolution operator for \eqref{Veqlambda} with $\lambda = 0$. Then we can write this in integrated form as the following fixed point equations
\begin{equation*}
\begin{aligned}
V_1^-(x) &= \Phi_s^-(x, -L) a^- + \Phi_u^+(x, 0) b^- + \Phi_c^-(x, -L) c^- \\
&+ \lambda \int_0^x \Phi_u^-(x, y) [B V_1^-(y; \lambda) + B V^c(y)] dy + \lambda \int_{-L}^x \Phi_s^-(x, y)[B V_1^-(y; \lambda) + B V^c(y)] dy \\
&+ \lambda \int_{-L}^x \Phi_c^-(x, y) [B V_1^-(y; \lambda) + B V^c(y)] dy \\ 
V_1^+(x) &= \Phi_u^+(x, L) a^+ + \Phi_s^+(x, 0) b^+ + \Phi_c^+(x, L)c^+ \\
&+ \lambda \int_0^x \Phi_s^+(x, y)[B V_1^+(y; \lambda) + B V^c(y)] dy
+ \lambda \int_{L}^x \Phi_u^+(x, y)[B V_1^+(y; \lambda) + B V^c(y)] dy \\
&+ \lambda \int_{L}^x \Phi_c^+(x, y)[B V_1^+(y; \lambda) + B V^c(y)] dy
\end{aligned}
\end{equation*}
where $L$ is arbitrary and for the initial conditions we have
\begin{align*}
a^- &\in E^s \\
a^+ &\in E^u \\
b^- &\in \text{ran}(P_u^-(0)) = \R Q'(0) \oplus Y^- \\
b^+ &\in \text{ran}(P_s^+(0)) = \R Q'(0) \oplus Y^+ \\
c^\pm &\in E^c \\
\end{align*}
Since the center evolution occurs in a 1-dimensional space, we know its exact form.
\[
\Phi^c(x, y) u = V^c(x)\langle u, W^c(x)\rangle
\]
Substituting this in, we have
\begin{equation*}
\begin{aligned}
V_1^-(x) &= \Phi_s^-(x, -L) a^- + \Phi_u^-(x, 0) b^- + V^c(x)\langle c^-, W^c(-L)\rangle \\
&+ \lambda \int_0^x \Phi_u^-(x, y) [B V_1^-(y; \lambda) + B V^c(y)] dy + \lambda \int_{-L}^x \Phi_s^-(x, y)[B V_1^-(y; \lambda) + B V^c(y)] dy \\
&+ \lambda \int_{-L}^x V^c(x)\langle B V_1^-(y; \lambda) + B V^c(y), W^c(y) \rangle dy \\ 
V_1^+(x) &= \Phi_u^+(x, L) a^+ + \Phi_s^+(x, 0) b^+ + V^c(x)\langle c^+, W^c(L)\rangle \\
&+ \lambda \int_0^x \Phi_s^+(x, y)[B V_1^+(y; \lambda) + B V^c(y)] dy
+ \lambda \int_{L}^x \Phi_u^+(x, y)[B V_1^+(y; \lambda) + B V^c(y)] dy \\
&+ \lambda \int_{L}^x V^c(x) \langle B V_1^+(y; \lambda) + B V^c(y), W^c(y) \rangle  dy
\end{aligned}
\end{equation*}
Evaluating these at $x = 0$, we have
\begin{equation*}
\begin{aligned}
V_1^-(0) &= \Phi_s^-(0, -L) a^- + b^- + V^c(0)\langle c^-, W^c(-L)\rangle \\
&+ \lambda \int_{-L}^0 \Phi_s^-(0, y)[B V_1^-(y; \lambda) + B V^c(y)] dy \\
&+ \lambda \int_{-L}^0 V^c(0)\langle B V_1^-(y; \lambda) + B V^c(y), W^c(y) \rangle dy \\ 
V_1^+(0) &= \Phi_u^+(0, L) a^+ + b^+ + V^c(0)\langle c^+, W^c(L)\rangle \\
&+ \lambda \int_{L}^0 \Phi_u^+(0, y)[B V_1^+(y; \lambda) + B V^c(y)] dy \\
&+ \lambda \int_{L}^0 V^c(0) \langle B V_1^+(y; \lambda) + B V^c(y), W^c(y) \rangle  dy
\end{aligned}
\end{equation*}


\end{document}
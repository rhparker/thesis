\documentclass[thesis.tex]{subfiles}

\begin{document}

Consider the equations
\begin{align}
V'(x) = A(Q(x); \lambda) V(x) \label{Veqlambda} \\
W'(x) = -A(Q(x); \lambda)^* W(x) \label{Weqlambda}
\end{align}\
where $Q(x)$ is the primary pulse solution and $A(Q(x), \lambda) = A(Q(x)) + \lambda B$. When $\lambda = 0$, these are the variational and adjoint variational equations.

Recall that $A(0; \lambda)$ has a small eigenvalue $\nu(\lambda)$. Let $V_0(\lambda)$ and $W_0(\lambda)$ be the eigenvectors of $A(0)$ and $-A(0)^*$ corresponding to $\nu(\lambda)$. For convenience, we let $V_0 = V_0(0)$ and $W_0 = W_0(0)$.

In Lemma \ref{varsolutions}, we showed that when $\lambda = 0$, \eqref{Veqlambda} has a bounded solution $V^c(x)$ with
\[
V^c(x) \rightarrow V_0(0) \text{ as }|x| \rightarrow \infty
\]
which is symmetric with respect to the standard reversor operator $R$. In Lemma \ref{adjsolutions}, we showed that when $\lambda = 0$, \eqref{Weqlambda} has an exponentially decaying solution $\Psi(x)$ and a bounded solution $\Psi^c(x)$ with 
\[
\Psi^c(x) \rightarrow W_0 \text{ as }|x| \rightarrow \infty
\]

Using the Gap Lemma, we can find solutions $V^\pm(x; \lambda)$ to \eqref{Veqlambda} and $W^\pm(x; \lambda)$ to \eqref{Weqlambda} on $\R^\pm$ such that
\begin{align*}
V^\pm(x; \lambda) &= V_0(\lambda)e^{\nu(\lambda)x} + V_R^\pm(x; \lambda) \\
W^\pm(x; \lambda) &= W_0(\lambda)e^{\nu(\lambda)x} + W_R^\pm(x; \lambda)
\end{align*}
where for the remainder terms we have $V_R^\pm(x; \lambda), W_R^\pm(x; \lambda) = \mathcal{O}(e^{-\tilde{\alpha}|x|})$. These solutions are analytic in $\lambda$. They are also not unique.

We would like to better characterize these solutions. Let
\[
V^\pm(x; \lambda) = e^{\nu(\lambda)x}\tilde{V}^\pm(x; \lambda)
\]
Then we can write $\tilde{V}^\pm(x; \lambda)$ as
\[
\tilde{V}^\pm(x; \lambda) = V_0(\lambda) + \tilde{V}_R^\pm(x; \lambda)
\]
where $\tilde{V}^\pm(x; \lambda)$ is a remainder term which decays exponentially at a slightly slower rate (which we can make precise). We can do the same for $W$.

Differentiating this, we have
\begin{align*}
\nu(\lambda) e^{\nu(\lambda)x}\tilde{V}^\pm(x; \lambda) 
+ e^{\nu(\lambda)x}[\tilde{V}^\pm(x; \lambda)]'
&= A(Q(x); \lambda) (e^{\nu(\lambda)x}\tilde{V}^\pm(x; \lambda))
\end{align*}
Dividing by $e^{\nu(\lambda)x}$ and rearranging, we get the following ODE for $\tilde{V}^\pm(x; \lambda)$.
\begin{align*}
[\tilde{V}^\pm(x; \lambda)]'
&= (A(Q(x); \lambda) - \nu(\lambda)I)\tilde{V}^\pm(x; \lambda)
\end{align*}
which, for convenience, we will write as
\begin{align*}
[\tilde{V}^\pm(x; \lambda)]'
&= \tilde{A}(Q(x); \lambda)\tilde{V}^\pm(x; \lambda)
\end{align*}
where $\tilde{A}(Q(x); \lambda) = A(Q(x); \lambda) - \nu(\lambda)I)$. We note that $\mu$ is an eigenvalue of the asympotitic matrix $A(0; \lambda)$ if and only if $\mu - \nu(\lambda)$ is an eigenvalue of $\tilde{A}(0; \lambda)$, i.e. all eigenvalues are shifted by $\nu(\lambda)$. Thus the asymptotic matrix $\tilde{A}(0; \lambda)$ has a simple eigenvalue at 0, and all other eigenvalues $\mu$ satisfy
\[
|\text{Re }\mu| \geq \alpha_0 - |\nu(\lambda)| \geq \alpha_0 - \rho = \alpha
\]
$\tilde{V}^\pm(x; \lambda)$ is still analytic in $\lambda$, and $\tilde{V}^\pm(x; 0) = V^c(x)$, so we can take 
\[
\tilde{V}^\pm(x; \lambda) = V^c(x) + \lambda V_1^\pm(x; \lambda)
\]
Next, we plug this into the equation for $\tilde{V}^\pm(x; \lambda)$ to get an equation for $V_1(x; \lambda)$.
\begin{align*}
[V^c(x)]' + \lambda [V_1^\pm(x; \lambda)]'
&= \tilde{A}(Q(x); \lambda)(V^c(x) + \lambda V_1^\pm(x; \lambda)) \\
&= (A(Q(x)) + \lambda B - \nu(\lambda)I)V^c(x) + \lambda \tilde{A}(Q(x); \lambda) V_1^\pm(x; \lambda) \\
\lambda [V_1^\pm(x; \lambda)]' &= (\lambda B - \nu(\lambda)I)V^c(x) + \lambda \tilde{A}(Q(x); \lambda) V_1^\pm(x; \lambda)
\end{align*}

Substituting the expansion for $\nu(\lambda)$
\[
\nu(\lambda) = \frac{1}{c} \lambda + \mathcal{O}(|\lambda|^3)
\]
this becomes
\begin{align*}
\lambda [V_1^\pm(x; \lambda)]' &= \lambda \tilde{A}(Q(x); \lambda) V_1^\pm(x; \lambda) + \left(\lambda B - \frac{1}{c} \lambda I + \mathcal{O}(|\lambda|^3)I\right)V^c(x)
\end{align*}
Finally, divide by $\lambda$ to get
\begin{align*}
[V_1^\pm(x; \lambda)]' &= \tilde{A}(Q(x); \lambda) V_1^\pm(x; \lambda) + \tilde{B} V^c(x) + \lambda^2 R(x)
\end{align*}
where
\[
\tilde{B} = B - \frac{1}{c} I
\]
and $||R|| \leq C$. 

For what follows, we will take $\lambda$ fixed, so we suppress the dependence on $\lambda$, i.e. we write
\begin{align*}
[V_1^\pm(x)]' &= \tilde{A}(Q(x)) V_1^\pm(x) + \tilde{B} V^c(x) + \lambda^2 R(x)
\end{align*}
Let $\Phi(x,y)$ be the evolution operator for the ODE
\[
V' = \tilde{A}(Q(x))V
\]
For the center subspace, the evolution operator is given by
\[
\Phi^c_\pm(x,y) = \langle \tilde{W}^\pm(x), \cdot \rangle \tilde{V}^\pm(x)
\]

Then we can write our systemn in integrated form as the following set of fixed point equations
\begin{equation*}
\begin{aligned}
V_1^-(x) &= \Phi_u^+(x, 0) b^- + \Phi_c^-(x, -L) c^- \\
&+ \int_0^x \Phi_u^-(x, y) [\tilde{B} V^c(y) + \lambda^2 R(y)] dy + \int_{-\infty}^x \Phi_s^-(x, y)[\tilde{B} V^c(y) + \lambda^2 R(y)] dy \\
&+ \int_{-L}^x \Phi_c^-(x, y)[\tilde{B} V^c(y) + \lambda^2 R(y)] dy \\ 
V_1^+(x) &= \Phi_s^+(x, 0) b^+ + \Phi_c^+(x, L)c^+ \\
&+ \int_0^x \Phi_s^+(x, y)[\tilde{B} V^c(y) + \lambda^2 R(y)] dy
+ \int_{\infty}^x \Phi_u^+(x, y) [\tilde{B} V^c(y) + \lambda^2 R(y)] dy \\
&+ \int_{L}^x \Phi_c^+(x, y)[\tilde{B} V^c(y) + \lambda^2 R(y)] dy
\end{aligned}
\end{equation*}
where $L$ is arbitrary and for the initial conditions we have
\begin{align*}
a^- &\in E^s \\
a^+ &\in E^u \\
b^- &\in \text{ran}(P_u^-(0)) = \R Q'(0) \oplus Y^- \\
b^+ &\in \text{ran}(P_s^+(0)) = \R Q'(0) \oplus Y^+ \\
c^\pm &\in E^c \\
\end{align*}

Substituting in the form of the center evolution,
\begin{equation*}
\begin{aligned}
V_1^-(x) &= \Phi_u^+(x, 0) b^- + \langle \tilde{W}^-(x), c^- \rangle \tilde{V}^-(x) \\
&+ \int_0^x \Phi_u^-(x, y) [\tilde{B} V^c(y) + \lambda^2 R(y)] dy + \int_{-\infty}^x \Phi_s^-(x, y)[\tilde{B} V^c(y) + \lambda^2 R(y)] dy \\
&+ \int_{-L}^x \langle \tilde{W}^-(x), \tilde{B} V^c(y) + \lambda^2 R(y) \rangle \tilde{V}^-(x) dy \\ 
V_1^+(x) &= \Phi_s^+(x, 0) b^+ + \langle \tilde{W}^+(x), c^+ \rangle \tilde{V}^+(x) \\
&+ \int_0^x \Phi_s^+(x, y)[\tilde{B} V^c(y) + \lambda^2 R(y)] dy
+ \int_{\infty}^x \Phi_u^+(x, y) [\tilde{B} V^c(y) + \lambda^2 R(y)] dy \\
&+ \int_{L}^x \langle \tilde{W}^+(x), \tilde{B} V^c(y) + \lambda^2 R(y) \rangle \tilde{V}^+(x) dy
\end{aligned}
\end{equation*}


Since the center evolution occurs in a 1-dimensional space, we know its exact form.
\[
\Phi^c(x, y) u = V^c(x)\langle u, W^c(x)\rangle
\]
Substituting this in, we have
\begin{equation*}
\begin{aligned}
V_1^-(x) &= \Phi_s^-(x, -L) a^- + \Phi_u^-(x, 0) b^- + V^c(x)\langle c^-, W^c(-L)\rangle \\
&+ \lambda \int_0^x \Phi_u^-(x, y) [B V_1^-(y; \lambda) + B V^c(y)] dy + \lambda \int_{-L}^x \Phi_s^-(x, y)[B V_1^-(y; \lambda) + B V^c(y)] dy \\
&+ \lambda \int_{-L}^x V^c(x)\langle B V_1^-(y; \lambda) + B V^c(y), W^c(y) \rangle dy \\ 
V_1^+(x) &= \Phi_u^+(x, L) a^+ + \Phi_s^+(x, 0) b^+ + V^c(x)\langle c^+, W^c(L)\rangle \\
&+ \lambda \int_0^x \Phi_s^+(x, y)[B V_1^+(y; \lambda) + B V^c(y)] dy
+ \lambda \int_{L}^x \Phi_u^+(x, y)[B V_1^+(y; \lambda) + B V^c(y)] dy \\
&+ \lambda \int_{L}^x V^c(x) \langle B V_1^+(y; \lambda) + B V^c(y), W^c(y) \rangle  dy
\end{aligned}
\end{equation*}
Evaluating these at $x = 0$, we have
\begin{equation*}
\begin{aligned}
V_1^-(0) &= \Phi_s^-(0, -L) a^- + b^- + V^c(0)\langle c^-, W^c(-L)\rangle \\
&+ \lambda \int_{-L}^0 \Phi_s^-(0, y)[B V_1^-(y; \lambda) + B V^c(y)] dy \\
&+ \lambda \int_{-L}^0 V^c(0)\langle B V_1^-(y; \lambda) + B V^c(y), W^c(y) \rangle dy \\ 
V_1^+(0) &= \Phi_u^+(0, L) a^+ + b^+ + V^c(0)\langle c^+, W^c(L)\rangle \\
&+ \lambda \int_{L}^0 \Phi_u^+(0, y)[B V_1^+(y; \lambda) + B V^c(y)] dy \\
&+ \lambda \int_{L}^0 V^c(0) \langle B V_1^+(y; \lambda) + B V^c(y), W^c(y) \rangle  dy
\end{aligned}
\end{equation*}


\end{document}
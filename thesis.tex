\documentclass[11pt,reqno,oneside]{report}
% \usepackage[pdfborder={0 0 0.5 [3 2]}]{hyperref}%
\usepackage[left=1in,right=1in,top=1in,bottom=1.25in]{geometry}%

% brown university thesis style
\usepackage{setspace}
\usepackage[utf8]{inputenc}
\usepackage{buthesis}
\usepackage{mathtools}
\usepackage[usenames,dvipsnames]{xcolor}
\usepackage{mdwlist}
\usepackage{caption}
\usepackage{subcaption}

% AMS packages
\usepackage{amsmath}
\usepackage{amssymb}
\usepackage{amsthm}
\usepackage{graphicx}
\usepackage{enumerate}
\usepackage{float}
\usepackage{subfiles}

% my custom packages
\usepackage{macros}
\usepackage{packages}
\usepackage{style}

% \usepackage[backend=biber]{biblatex}
% \addbibresource{thesis.bib}

%    If you are using the author-year citation style:
%\usepackage{natbib}

% For use when working on individual chapters
%\includeonly{}

%    As set up here, all theorem-class objects will be numbered with
%    the same counter, starting with 1 at every new chapter; numbers
%    will have the form <chapter>.<theorem>.  This may be changed if
%    the author prefers.

\theoremstyle{plain}
\newtheorem{theorem}{Theorem}[chapter]
\newtheorem{corollary}[theorem]{Corollary}
\newtheorem{lemma}[theorem]{Lemma}

\theoremstyle{definition}
\newtheorem{definition}[theorem]{Definition}
\newtheorem{example}[theorem]{Example}

\theoremstyle{remark}
\newtheorem{remark}[theorem]{Remark}
\newtheorem{hypothesis}[theorem]{Hypothesis}

\numberwithin{section}{chapter}
\numberwithin{equation}{chapter}
\numberwithin{figure}{chapter}  

\DeclareMathOperator{\spn}{span}
\DeclareMathOperator{\ran}{ran}
\DeclareMathOperator{\dm}{dim}

\setcounter{secnumdepth}{4}
% \graphicspath{{figures/}}
\setcounter{tocdepth}{1}

% \makeindex

\pagestyle{plain}
% \setlength{\footskip}{48pt}

% conditional for full document
\newif\iffulldocument

%% Control the fonts and formatting used in the table of contents, list of
%% figures, and list of tables

%% Aesthetic spacing redefines that look nicer to me than the defaults.
\usepackage[titles]{tocloft}
\setlength{\cftbeforechapskip}{-1ex}
\setlength{\cftbeforesecskip}{-3.5ex}
\setlength{\cftbeforesubsecskip}{-3.5ex}
\setlength{\cftbeforetabskip}{-3.5ex}
\setlength{\cftbeforefigskip}{-3.5ex}

\dissertation

% Information about the document

\title{Nonlinear waves in the fifth-order Korteweg-de Vries equation}
\author{Ross Hamilton Parker}
\degree{Doctor of Philosophy}
\department{The Division of Applied Mathematics}
\previousdegrees{
    B.A., Bowdoin College; Brunswick, ME, 1998\\
    M.D., University of Pennsylvania School of Medicine; Philadelphia, PA, 2009\\
    M.A., CUNY Hunter College; New York, NY, 2013 \\
    M.Sc., Brown University; Providence, RI, 2015} 
\thesismonth{February} \thesisyear{2020}

\begin{document}

% \doublespacing % comment this for single-spacing (final version should be double-spaced)

%%
%% stuff specific to thesis, comment out for now
%%

\begin{preliminaries}

\maketitle

\copyrightpage

\begin{signature}
  \director{Bj\"{o}rn Sandstede, Ph.D., Advisor}
  \reader{John Mallet-Paret, Ph.D., Reader}
  \reader{Ryan Goh, Ph.D., Reader}
\end{signature}

\begin{vita}\label{thesis:vita}
  \section*{Education}

\begin{itemize}
	\item Brown University, Providence, RI, 2014 - 2019\\
	Ph.D. in Applied Mathematics, expected December 2019 \\
	M.Sc. in Applied Mathematics, 2015

	\item CUNY Hunter College, New York, NY, 2011-2013\\
	M.A. in pure mathematics, 2013

	\item University of Pennsylvania School of Medicine, Philadelphia, PA, 2004-2009 \\
	M.D., 2009
	
	\item Bowdoin College, Brunswick, ME, 1994-1998 \\
	B.S. in Music \emph{summa cum laude} with minor in chemistry, 1998\\
\end{itemize}

\section*{Honors and Awards}

\begin{itemize}
	\item 2019, SIAM Student Travel Award for SIAM Conference on Applications of Dynamical Systems
	\item 2009, Alpha Omega Alpha medical honor society, University of Pennsylvania school of medicine
	\item 1998, Phi Beta Kappa Undergraduate honor society, Bowdoin College
	\item 1998, Sue Winchell Burnett Senior Prize in Music, Bowdoin College
	\item 1996, Edwin Herbert Hall Sophomore Prize in Physics, Bowdoin College
	\item 1995, CRC First Year Prize in Chemistry, Bowdoin College
\end{itemize}

\section*{Teaching Experience}
\begin{itemize}
	\item Instructor, Statistical Inference I, Brown University, Summer 2016
	\item Teaching Assistant, Applied Ordinary Differential Equations, Brown University, Spring 2016
	\item Teaching Assistant, Statistical Inference I, Brown University, Fall 2015
\end{itemize}

\section*{Service and Professional Activities}

\begin{itemize}
	\item Sheridan Center for Teaching and Learning, Brown University
	\begin{itemize*}
		\item Course Design Seminar, Spring 2019
		\item Teaching Consultant Program, 2017 - 2018
		\item New TA orientation workshop leader, Fall 2017
		\item Teaching seminar: Reflective Teaching, 2015-2016
	\end{itemize*}

	\item Co-organizer, Brown/BU/UMass Joint Dynamical Systems and PDE Seminar, Fall 2019 - Spring 2020

	\item Co-rganizer, Graduate Student Seminar, Division of Applied Mathematics, Brown University, Fall 2018 - Spring 2019

	\item Mentor, Directed Reading Program, Division of Applied Mathematics, Brown University, Spring 2019

	\item Vice president, Brown University SIAM student chapter, Fall 2018 - Spring 2019

	\item Co-chair, Pinewoods Scottish Sessions, Royal Scottish Country Dance Society, Boston Branch. 2017 and 2018. 

\end{itemize}

\section*{Publications}

\begin{itemize}
	\item R. Parker, P. G. Kevrekidis, and B. Sandstede. \textit{Existence and spectral stability of multi-pulses in discrete Hamiltonian lattice systems}, arXiv e-prints 2019, 1910.11864.
	\item T. Kapitula, R. Parker, and B. Sandstede. \textit{A reformulated Krein matrix for star-even polynomial operators with applications}, arXiv e-prints 2019, 1909.06411.
\end{itemize}

\section*{Invited talks}

\begin{itemize}
	\item \emph{Spectral Stability of Periodic Multi-Pulses in the 5th Order KdV Equation} at SIAM Conference on Applications of Dynamical Systems, Snowbird, UT, 19-23 May, 2019
	\item \emph{Spectral Stability of Multi-pulses via the Krein Matrix} at IMACS Conference on Nonlinear Evolution Equations and Wave Phenomena, Athens, GA, 17-19 April, 2019
	\item \emph{Stability of Double Pulse Solutions to the 5th order KdV Equation} at Applied Mathematics Colloquium, University of Massachusetts, Amherst, MA, 13 Feb, 2018
	\item \emph{Stability of Double Pulse Solutions to the 5th order KdV Equation} at Brown/BU Joint Dynamics and PDE Seminar, Boston, MA, 30 Nov, 2017
\end{itemize}

\section*{Professional Associations}
\begin{itemize}
	\item American Mathematical Society (AMS)

	\item Society for Industrial and Applied Mathematics (SIAM)
\end{itemize}




\end{vita}

\begin{dedication}\label{thesis:dedication}
  % Dedication

In memory of John Loustau (d. 7 September, 2019), Professor of Mathematics at CUNY Hunter College, New York, NY.

\end{dedication}

\begin{acknowledgments}\label{thesis:acknowledgments}
  \begin{itemize}
\item My advisor Bj\"{o}rn Sandstede, for his guidance and support, and for not letting me give up.
\item Ryan Goh and John Mallet-Paret, for serving on my thesis committee.
\item My colleagues in the Division of Applied Mathematics, especially the \#Bjornlings.
\item All the students I have taught at Brown and from whom I have learned so much, especially Katie Wu and the first year APMA cohorts.
\item Ed Barth, Rick Boucher, Pat Chen, Shannah Green, Anne Hooper, Otto Liebmann, Joanna Read, Sheri Silva, and J.D. Swanson for helping me to stay sane through yoga, martial arts, and music.
\item The Rhode Island Sacred Harp and Royal Scottish Country Dance Society communities.
\item My Pinewoods Scottish Sessions co-chair Laura DeCesare. \#mischiefmanaged.
\item My partner Sarah Stefanski. \#allinclusive.
\end{itemize}

\end{acknowledgments}

% Comment out the abstract for final submitted version of the dissertation
% \begin{abstract}\label{thesis:abstract}
%   \thispagestyle{empty}
%   \subfile{abstract}
%   \thispagestyle{empty}
% \end{abstract}

\begin{spacing}{1} % single-spacing for these
  \tableofcontents
  \clearpage{\pagestyle{empty}\cleardoublepage}

  \footnotesize
  \fontsize{11.5pt}{12.5pt}\selectfont
  \listoftables
  \clearpage{\pagestyle{empty}\cleardoublepage}

  \listoffigures
  \clearpage{\pagestyle{empty}\cleardoublepage}
  \normalsize
\end{spacing}

\end{preliminaries}

%\pagestyle{myheadings}
\pagestyle{plain}

\fulldocumenttrue

\chapter{Introduction}\label{chapter:intro}
\subfile{intro}

\chapter{The fifth-order KdV equation}\label{chapter:KdV5}
\subfile{kdv5}

\chapter{Numerical results}\label{chapter:KdV5numerics}
\subfile{kdv5numerics}

\chapter{Generalization}\label{chapter:kdv5general}
\subfile{kdv5general}
\subfile{primarypulse}

\chapter{Multi-pulse solutions}\label{chapter:kdv5homoclinic}
\subfile{kdv5homoclinic}

\chapter{Periodic multi-pulse solutions }\label{chapter:kdv5periodic}
\subfile{periodic}

\chapter{Multi-pulses in other systems}\label{chapter:other}
\subfile{other}

\chapter{Conclusions and future directions}
\subfile{conclusions}

\appendix

\chapter{Proof of existence results for periodic multi-pulses}\label{perexistproof}
\subfile{periodicexist}

\chapter{Proof of Theorem \ref{blockmatrixtheorem}}\label{blockmatrixproof}
\subfile{blockmatrixsetup}
\subfile{blockmatrixinv}

\chapter{Proof of eigenvalue results for periodic 2-pulses}\label{per2pproof}
\subfile{periodiceig2p}

%    Include appendix "chapters" here.
% \include{}

%    Bibliographies can be prepared with BibTeX using amsplain,
%    amsalpha, or (for "historical" overviews) natbib style.
\begin{spacing}{1} 
\bibliographystyle{amsalpha}
\bibliography{thesis.bib}
\end{spacing}
% \printbibliography

%    See note above about multiple indexes.
% \printindex

\end{document}

\documentclass[thesis.tex]{subfiles}

\begin{document}

\iffulldocument\else
	\chapter{KdV5}
\fi

\section{Inversion}

We will now solve the system \cref{eigsystem}. The proof follows the outline of \cite[Theorem 2]{Sandstede1998}, with the main differences being the presence of a center subspace and the fact that we are on a periodic domain. Define the spaces
\begin{align*}
V_Z &= \bigoplus_{i=0}^{n-1} C_b([-X_{i-1},0]:\C^{2m+1}) \oplus C_b([0,X_i]:\C^{2m+1})  \\
V_a &= \bigoplus_{i=0}^{n-1} E^u(\lambda) \oplus E^s(\lambda) \oplus E^c(\lambda) \\
V_b &= \bigoplus_{i=0}^{n-1} E^u(0) \oplus E^s(0) \\
V_c &= \bigoplus_{i=0}^{n-1} \C \\
V_d &= \bigoplus_{i=0}^{n-1} \C \\
V_\lambda &= B_\delta(0) \subset \C
\end{align*}
where the subscripts are all taken $\Mod n$, since we are on a periodic domain. All the product spaces are endowed with the maximum norm, e.g. for $V_b$, 
\[
|b| = \max(|b_0^-|, \dots, |b_{n-1}^-|, |b_0^+|, \dots, |b_{n-1}^+|)
\]

As in \cite{Sandstede1998}, we will write \eqref{systemZ} in integrated form as a set of fixed point equations. Using the variation of constants formula and splitting the evolution up into the evolution on the stable, unstable, and center eigenspaces of $A(\lambda)$, we write \eqref{systemZ} as the set of fixed point equations
\begin{equation}\label{Zfpeq}
\begin{aligned}
Z&_i^-(x) = \Phi^s(x, -X_{i-1}; \lambda) a_{i-1}^- + \Phi^u(x, 0; \lambda) b_i^- - \Phi^c(x, -X_{i-1}; \lambda) a_{i-1}^c V_0(\lambda) \\
&+ \int_0^x \Phi^u(x, y; \lambda)P_-(y; \lambda)^{-1} [ G_i^-(y) P_-(y; \lambda)Z_i^-(y) + c_{i-1} e^{\nu(\lambda)X_{i-1}} G_i^-(y) V^-(y; \lambda) + \lambda^2 d_i \tilde{H}_i^-(y)] dy \\
&+ \int_{-X_{i-1}}^x \Phi^s(x, y; \lambda) P_-(y; \lambda)^{-1}[ G_i^-(y) P_-(y; \lambda)Z_i^-(y) + c_{i-1} e^{\nu(\lambda)X_{i-1}} G_i^-(y) V^-(y; \lambda) + \lambda^2 d_i \tilde{H}_i^-(y)] dy \\
&+ \int_{-X_{i-1}}^x \Phi^c(x, y; \lambda) P_-(y; \lambda)^{-1}[ G_i^-(y) P_-(y; \lambda)Z_i^-(y) + c_{i-1} e^{\nu(\lambda)X_{i-1}} G_i^-(y) V^-(y; \lambda) + \lambda^2 d_i \tilde{H}_i^-(y)] dy  \\ 
Z&_i^+(x) = \Phi^u(x, X_i; \lambda) a_i^+ + \Phi^s(x, 0; \lambda) b_i^+ + \Phi^c(x, X_i; \lambda) a_i^c V_0(\lambda) \\
&+ \int_0^x \Phi^s(x, y; \lambda) P_+(y; \lambda)^{-1} [ G_i^+(y) P_+(y; \lambda) Z_i^+(y) + c_i e^{-\nu(\lambda)X_i} G_i^-(y)V^+(y; \lambda) + \lambda^2 d_i \tilde{H}_i^+(y)] dy \\
&+ \int_{X_i}^x \Phi^u(x, y; \lambda) P_+(y; \lambda)^{-1} [ G_i^+(y) P_+(y; \lambda) Z_i^+(y) + c_i e^{-\nu(\lambda)X_i} G_i^-(y)V^+(y; \lambda) + \lambda^2 d_i \tilde{H}_i^+(y)] dy \\
&+ \int_{X_i}^x \Phi^c(x, y; \lambda) P_+(y; \lambda)^{-1} [ G_i^+(y) P_+(y; \lambda) Z_i^+(y) + c_i e^{-\nu(\lambda)X_i} G_i^-(y)V^+(y; \lambda) + \lambda^2 d_i \tilde{H}_i^+(y)] dy \\
\end{aligned}
\end{equation}
where $i = 0, \dots, n-1$. Since the center subspace is the span of the single vector $V_0(\lambda)$, $a_i^c \in \C$. As in \cite{Sandstede1998}, we will solve the eigenvalue problem in a series of inversion steps. First, we will solve \cref{systemZ} for $Z_i^\pm$. To do this, we rewrite \eqref{Zfpeq} as
\begin{equation}\label{L1L2eq}
[I - L_1(\lambda)]Z = L_2(\lambda)(a,b,c,d)Z
\end{equation}
where $L_1(\lambda)$ is the linear operator consisting of the terms from the RHS of \eqref{Zfpeq} involving $Z_i^\pm$ and $L_2(\lambda)(a,b,c,d)$ is the linear operator consisting of the terms from the RHS of \eqref{Zfpeq} which do not involve $Z_i^\pm$. Specifically,
\begin{align*}
[L_1&(\lambda)_i^- Z](x) 
= \int_0^x \Phi^u(x, y; \lambda)P_-(y; \lambda)^{-1} G_i^-(y) P_-(y; \lambda) Z_i^-(y) dy \\
&+ \int_{-X_{i-1}}^x \Phi^s(x, y; \lambda) P_-(y; \lambda)^{-1} G_i^-(y) P_-(y; \lambda) Z_i^-(y) dy \\
&+ \int_{-X_{i-1}}^x \Phi^c(x, y; \lambda) P_-(y; \lambda)^{-1} G_i^-(y) P_-(y; \lambda) Z_i^-(y) dy  \\ 
[L_1&(\lambda)_i^+ Z](x) = \int_0^x \Phi^s(x, y; \lambda) P_+(y; \lambda)^{-1} G_i^+(y) P_+(y; \lambda) Z_i^+(y) dy \\
&+ \int_{X_i}^x \Phi^u(x, y; \lambda) P_+(y; \lambda)^{-1} G_i^+(y) P_+(y; \lambda) Z_i^+(y) dy \\
&+ \int_{X_i}^x \Phi^c(x, y; \lambda) P_+(y; \lambda)^{-1} G_i^+(y) P_+(y; \lambda) Z_i^+(y) dy \\
\end{align*}
and
\begin{align*}
[L_2&(\lambda)_i^-(a,b,c,d)](x) = \Phi^s(x, -X_{i-1}; \lambda) a_{i-1}^- + \Phi^u(x, 0; \lambda) b_i^- - \Phi^c(x, -X_{i-1}; \lambda) a_{i-1}^c V_0(\lambda) \\
&+ \int_0^x \Phi^u(x, y; \lambda) P_-(y; \lambda)^{-1} [ c_{i-1} e^{\nu(\lambda)X_{i-1}} G_i^-(y)V^-(y; \lambda) + \lambda^2 d_i \tilde{H}_i^-(y)] dy \\
&+ \int_{-X_{i-1}}^x \Phi^s(x, y; \lambda) P_-(y; \lambda)^{-1} [c_{i-1} e^{\nu(\lambda)X_{i-1}}G_i^-(y)V^-(y; \lambda) + \lambda^2 d_i \tilde{H}_i^-(y)] dy \\
&+ \int_{-X_{i-1}}^x \Phi^c(x, y; \lambda) P_-(y; \lambda)^{-1}[c_{i-1} e^{\nu(\lambda)X_{i-1}} G_i^-(y)V^-(y; \lambda) + \lambda^2 d_i \tilde{H}_i^-(y)] dy \\
[L_2&(\lambda)_i^+(a,b,c,d)](x) = \Phi^u(x, X_i; \lambda) a_i^+ + \Phi^s(x, 0; \lambda) b_i^+ + \Phi^c(x, X_i; \lambda) a_i^c V_0(\lambda) \\
&+ \int_0^x \Phi^s(x, y; \lambda) P_+(y; \lambda)^{-1} [c_i e^{-\nu(\lambda)X_i} G_i^+(y)V^+(y; \lambda) + \lambda^2 d_i \tilde{H}_i^+(y)] dy \\
&+ \int_{X_i}^x \Phi^u(x, y; \lambda) P_+(y; \lambda)^{-1} [c_i e^{-\nu(\lambda)X_i} G_i^+(y)V^+(y; \lambda) + \lambda^2 d_i \tilde{H}_i^+(y)] dy  \\
&+ \int_{X_i}^x \Phi^c(x, y; \lambda) P_+(y; \lambda)^{-1} [c_i e^{-\nu(\lambda)X_i} G_i^+(y)V^+(y; \lambda) + \lambda^2 d_i \tilde{H}_i^+(y)] dy 
\end{align*}

We obtain bounds for $L_1$ and $L_2$ in the next two lemmas.

\begin{lemma}\label{L1boundlemma}
For the linear operator $L_1$, we have uniform bound
\begin{equation}\label{L1uniformbound}
\|L_1(\lambda)Z\| \leq C e^{-\alpha X^*}\|Z\|
\end{equation}
\begin{proof}
The bound on $L_1$ will depend on the integral involving the center subspace, since there is potential growth in that subspace. For the ``minus'' pieces, using the bound for $G_i^\pm(x)$ from Lemma \ref{stabestimateslemma}
\begin{align*}
\Bigg| \int_{-X_{i-1}}^x &\Phi^c(x, y; \lambda) P_-(y; \lambda)^{-1} G_i^-(y) P_-(y; \lambda) Z_i^-(y) dy \Bigg| \\ 
&\leq C \| Z_i^-\|\int_{-X_{i-1}}^x e^{\nu(\lambda)(x - y)} \left( e^{-\alpha X_{i-1}} e^{-\alpha(X_{i-1} + y) } + e^{-2 \alpha X_i} e^{\alpha y} \right) dy \\
&\leq C \| Z_i^-\|\left( \int_{-X_{i-1}}^x \left|e^{\nu(\lambda)(x - y)}\right| e^{-\alpha X_{i-1}} e^{-\alpha(X_{i-1} + y) } dy + \int_{-X_{i-1}}^x e^{-\tilde{\alpha} y} e^{-2\alpha X_i} e^{\alpha y} dy \right) \\
\end{align*}
By \cref{lemma:expnubound}, $\left|e^{\nu(\lambda)(x - y)}\right| \leq R_N$, which is a constant depending on $N$. Thus we have
\begin{align*}
\Bigg| \int_{-X_{i-1}}^x &\Phi^c(x, y; \lambda) P_-(y; \lambda)^{-1} G_i^-(y) P_-(y; \lambda) Z_i^-(y) dy \Bigg| \\ 
&\leq C \| Z_i^-\|\left( \int_{-X_{i-1}}^x e^{-\alpha X_{i-1}} e^{-\alpha(X_{i-1} + y) } dy + \int_{-X_{i-1}}^x e^{-2\alpha X_i} e^{(\alpha - \tilde{\alpha}) y} dy \right) \\
&\leq C e^{-\alpha X^*} \| Z_i^-\| 
\end{align*}
The bound for the ``plus'' pieces is similar, which gives us the uniform bound \cref{L1uniformbound}.
\end{proof}
\end{lemma}

We note that the constant $C$ in Lemma \ref{L1boundlemma} depends on the initial choice of $N$. This will be the case for all the constants $C$ in future lemmas.

\begin{lemma}\label{L2boundlemma}
For the linear operator $L_2$, we have uniform bound
\begin{equation}\label{L2bound}
\| L_2(\lambda) (a,b,c,d) \| \leq C\left(|a| + |b| + e^{-\alpha X^*} |c| + e^{-\alpha X^*}|\lambda|^2|d|\right)
\end{equation}
\begin{proof}
For the ``minus'' pieces, since $\Phi^c(x, -X_{i-1}; \lambda) = e^{\nu(\lambda)(x + X_{i-1})}$, we can use \cref{lemma:expnubound} to get the bound
\begin{align*}
\| \Phi^c(x, -X_{i-1}; \lambda) a_{i-1}^c V_0(\lambda) \| \leq C |e^{-\nu(\lambda)X_{i-1}}|a_{i-1}^c| \leq C |a_{i-1}^c| \leq C |a|
\end{align*}
For the remaining terms which do not involve integrals,
\[
\| \Phi^s(x, -X_{i-1}; \lambda) a_{i-1}^- + \Phi^u(x, 0; \lambda) b_i^- \| \leq C(|a| + |b|)
\]
For the integral terms, as in \cref{L1boundlemma}, the bound is determined by the integrals involving the center subspace. The integrals involving $H_i^\pm$ has the same bounds as the integrals in Lemma \ref{L1boundlemma}, since the same estimates hold for $\tilde{H}_i^-(y)$ as for $G_i^-(y)$ by Lemma \ref{stabestimateslemma}. For the integrals involving $G_i^\pm(x) V^\pm(x; \lambda)$, we recall from Lemma \ref{lemma:Vpm} that
\[
V^\pm(x) = e^{\nu(\lambda)x}(V_0(\lambda) + \mathcal{O}(e^{-\alpha |x| }).
\]
Using the bound for $G_i^\pm(x)$ from Lemma \ref{stabestimateslemma}
\begin{align*}
\Bigg| c_{i-1} &e^{\nu(\lambda)X_{i-1}} \int_{-X_{i-1}}^x \Phi^c(x, y; \lambda) P_-(y; \lambda)^{-1} G_i^-(y) V^-(y) dy \Bigg| \\ 
&\leq C |c| \left| e^{\nu(\lambda)X_{i-1}} \right| \int_{-X_{i-1}}^x \left| e^{\nu(\lambda)(x - y)} \right| \left|e^{\nu(\lambda)(y)}\right| \left( e^{-\alpha X_{i-1}} e^{-\alpha(X_{i-1} + y) } + e^{-2 \alpha X_i} e^{\alpha y} \right) dy
\end{align*}
Using \cref{lemma:expnubound}, 
\[
\left| e^{\nu(\lambda)X_{i-1}} \right| \left| e^{\nu(\lambda)(x - y)} \right| \left|e^{\nu(\lambda)(y)}\right| \leq R_N^3,
\]
which is a constant depending on $N$. Following the same procedure as in Lemma \ref{L1boundlemma}, we thus obtain the bound
\begin{align}\label{intGVpmbound}
\Bigg| c_{i-1} e^{\nu(\lambda)X_{i-1}} \int_{-X_{i-1}}^x &\Phi^c(x, y; \lambda) P_-(y; \lambda)^{-1} G_i^-(y) V^-(y) dy \Bigg| \leq C e^{-\alpha X^*} |c| 
\end{align}
The bound for the ``plus'' piece is similar. Putting all of this together, we obtain the uniform bound \cref{L2bound}.
\end{proof}
\end{lemma}

We can now solve equation \eqref{systemZ} for $Z$, which we do in the following lemma.

\begin{lemma}\label{Zinv0}
There exists an operator $Z_1: V_\lambda \times V_a \times V_b \times V_c \times V_d \rightarrow V_z$ such that $Z = Z_1(\lambda)(a,b,c,d)$ solves \eqref{systemZ}. The operator $Z_1$ is analytic in $\lambda$, linear in $(a,b,c,d)$, and has uniform bound
\begin{equation}\label{Z1bound}
\| Z_1(\lambda)(a,b,c,d) \| \leq C\left(|a| + |b| + e^{-\alpha X^*} |c| + e^{-\alpha X^*}|\lambda|^2|d|\right)
\end{equation}
\begin{proof}
Using the bound on $L_1$ from Lemma \ref{L1boundlemma}, 
\[
\|L_1(\lambda)Z\| \leq C e^{-\alpha X^*}\|Z\|
\]
Since $e^{-\alpha X^*} = C r^{1/2} < \delta$, we have
\[
\|L_1(\lambda)Z\| \leq C \delta \|Z\|.
\]
Thus, decreasing $\delta$ if necessary, $I - L_1(\lambda)$ is invertible on $V_Z$. The inverse $(I - L_1(\lambda))^{-1}$ is analytic in $\lambda$, thus we have the solution
\begin{equation}\label{L1L2eqinv}
Z = (I - L_1(\lambda)^{-1}L_2(\lambda)(a,b,c,d)
\end{equation}
The bound \eqref{Z1bound} follows from the bound \eqref{L2bound} from Lemma \ref{L2boundlemma}.
\end{proof}
\end{lemma}

In the next lemma, we solve equation \eqref{systemmiddle}, which are the matching conditions at $\pm X_i$.
\begin{align*}
P_i^+(X_i; \lambda) Z_i^+(X_i) - P_{i+1}^-(-X_i; \lambda) Z_{i+1}^-(-X_i) &= D_i d + C_i c && i = 0, \dots, n-1
\end{align*}

% first inversion lemma : match at \pm X_i
\begin{lemma}\label{Zinv1}
There exists operators
\begin{align*}
A_1: &V_\lambda \times V_b \times V_c \times V_d \rightarrow V_a \\
Z_2: &V_\lambda \times V_b \times V_c \times V_d \rightarrow V_Z
\end{align*}
such that 
\[
(a, Z) = (A_1(\lambda)(b, c, d), Z_2(\lambda)(b,c,d))
\]
solves \eqref{systemZ} and \eqref{systemmiddle} for any $(b, c, d)$ and $\lambda$. These operator are analytic in $\lambda$ and linear in $(b, c, d)$. Bounds for $A_1$ and $Z_1$ are given by
\begin{align}\label{A1bound}
|A_1(\lambda)(b, c, d)| \leq C \left( e^{-\tilde{\alpha} X^*} |b|  + e^{-\alpha X^*}|\lambda|^2 |d| + |C_i||c| + |D_i||d| \right)
\end{align} 
and
\begin{equation}\label{Z2bound}
\| Z_2(\lambda)(b,c,d) \| \leq C\left(|b| + |C_i||c| + |D_i||d| + e^{-\alpha X^*}|\lambda|^2|d|\right)
\end{equation}

\begin{proof}
At $\pm X_i$, the fixed point equations \eqref{Zfpeq} are
\begin{align*}
Z_{i+1}^-&(-X_i) = a_i^- + \Phi^u(-X_i, 0; \lambda) b_{i+1}^- - a_i^c V_0(\lambda) \\ 
&+ \int_0^{-X_i} \Phi^u(-X_i, y; \lambda) P_-(y; \lambda)^{-1} \big[ G_{i+1}^-(y) P_-(y; \lambda)Z_{i+1}^-(y) \\
&\qquad+ c_i e^{\nu(\lambda)X_i} G_{i+1}^-(y) V^-(x; \lambda) + \lambda^2 d_{i+1} \tilde{H}_{i+1}^-(y)\big] dy \\
Z_i^+&(X_i) = a_i^+ + \Phi^s(X_i, 0; \lambda) b_i^+ + a_i^c V_0(\lambda) \\
&+ \int_0^{X_i} \Phi^s(X_i, y; \lambda) P_+(y; \lambda)^{-1} \big[ G_i^+(y) P_+(y; \lambda) Z_i^+(y) \\
&\qquad+ c_i e^{-\nu(\lambda)X_i} G_i^-(y) V^+(y; \lambda) + \lambda^2 d_i \tilde{H}_i^+(y)\big] dy
\end{align*}
where we used, for example, the fact that $a_i^- \in E^s(\lambda)$ and $\Phi^s(-X_{i-1}, -X_{i-1}; \lambda)$ is the identity on $E^s(\lambda)$. Applying the appropriate conjugation operators, subtracting, and using equation \eqref{projTheta}, we obtain the equation 
\begin{equation}\label{Dideq0}
\begin{aligned}
D_i &d + C_i c = (I + \Theta_+(X_i; \lambda))a_i^+ + (I + \Theta_+(X_i; \lambda))a_i^c V_0(\lambda) + P_+(X_i; \lambda)\Phi^s(X_i, 0; \lambda) b_i^+ \\
&+ P_+(X_i; \lambda) \int_0^{X_i} \Phi^s(X_i, y; \lambda) P_+(y; \lambda)^{-1}\big[ G_i^+(y) P_+(y; \lambda) Z_i^+(y) \\
&\qquad+ c_i e^{-\nu(\lambda)X_i} G_i^-(y) V^+(y; \lambda) + \lambda^2 d_i \tilde{H}_i^+(y)\big] dy \\
&- (I + \Theta_-(-X_i; \lambda))a_i^- + (I + \Theta_-(-X_i; \lambda))a_i^c V_0(\lambda) - P_-(-X_i; \lambda)\Phi^u(-X_i, 0; \lambda) b_{i+1}^- \\ 
&- P_-(-X_i; \lambda) \int_0^{-X_i} \Phi^u(-X_i, y; \lambda) P_-(y; \lambda)^{-1}\big[ G_{i+1}^-(y) P_-(y; \lambda)Z_{i+1}^-(y) \\
&\qquad+ c_i e^{\nu(\lambda)X_i} G_{i+1}^-(y) V^-(y; \lambda) + \lambda^2 d_{i+1} \tilde{H}_{i+1}^-(y)\big] dy
\end{aligned}
\end{equation}
which simplifies to
\begin{equation}\label{Didexpansion}
\begin{aligned}
D_i &d + C_i c= a_i^+ - a_i^- + 2 a_i^c V_0(\lambda) \\
&+ \Theta_+(X_i; \lambda)a_i^+ - \Theta_-(-X_i; \lambda)a_i^- + (\Theta_+(X_i; \lambda) + \Theta_-(-X_i; \lambda)) a_i^c V_0(\lambda)\\
&+ P_+(X_i; \lambda)\Phi^s(X_i, 0; \lambda) b_i^+ - P_-(-X_i; \lambda)\Phi^u(-X_i, 0; \lambda) b_{i+1}^- \\
&+ P_+(X_i; \lambda) \int_0^{X_i} \Phi^s(X_i, y; \lambda) P_+(y; \lambda)^{-1}\big[ G_i^+(y) P_+(y; \lambda) Z_i^+(y) \\
&\qquad+ c_i e^{-\nu(\lambda)X_i} G_i^-(y) V^+(y; \lambda) + \lambda^2 d_i \tilde{H}_i^+(y)\big] dy \\ 
&- P_-(-X_i; \lambda) \int_0^{-X_i} \Phi^u(-X_i, y; \lambda) P_-(y; \lambda)^{-1}\big[ G_{i+1}^-(y) P_-(y; \lambda)Z_{i+1}^-(y) \\
&\qquad+ c_i e^{\nu(\lambda)X_i} G_{i+1}^-(y) V^-(y; \lambda) + \lambda^2 d_{i+1} \tilde{H}_{i+1}^-(y)\big] dy
\end{aligned}
\end{equation}
This is of the form
\begin{align}\label{Dideq1}
D_i d + C_i c &= a_i^+ - a_i^- + 2 a_i^c V_0(\lambda) + L_3(\lambda)_i(a, b, c, d)
\end{align}
where the linear operator $L_3(\lambda)_i(a, b, c, d)$ is defined by the RHS of \cref{Didexpansion}. $L_3(\lambda)_i(a, b, c, d)$ is linear in $(a,b,c,d)$ and analytic in $\lambda$. To get a bound on $L_3$, we will bound the individual terms. 
\begin{enumerate}
\item For the terms involving $a_i^\pm$, we use \eqref{conjthetadecay} to get
\[
|\Theta_+(X_i; \lambda)a_i^+ - \Theta_-(-X_i; \lambda)a_i^-| \leq C e^{-\alpha X^*}|a|
\]
\item For the terms involving $a_i^c$, we use \eqref{conjthetadecay} to get
\[
|(\Theta_+(X_i; \lambda) + \Theta_-(-X_i; \lambda))a_i^c V_0(\lambda)| \leq 
C e^{-\alpha X^*} |a_i^c| \leq C e^{-\alpha X^*}|a|
\]
\item For the terms involving $b$, we use the exponential dichotomy bounds \eqref{Zevolbounds} to get
\[
| P_+(X_i; \lambda)\Phi^s(X_i, 0; \lambda) b_i^+ - P_-(-X_i; \lambda) \Phi^u(-X_i, 0; \lambda) b_{i+1}^-| \leq C e^{-\tilde{\alpha} X^*} |b|
\]
\item For the integral terms involving $Z$, we use the bound on $Z_1$ from Lemma \ref{Zinv0} to get
\begin{align*}
&\left|
P^+(X_i; \beta_i^+, \lambda) \int_0^{X_i} \Phi^s(X_i, y; \lambda) P_+(y; \lambda)^{-1} G_i^+(y) P_+(y; \lambda) Z_i^+(y) dy \right| \\
&\leq C \int_0^{X_i} e^{-\tilde{\alpha}(X_i - y)} |G_i^+(y)| |Z_i^+(y)| dy \\
&\leq C \left(|a| + |b| + e^{-\alpha X^*} |c| + e^{-\alpha X^*}|\lambda|^2|d|\right) \int_0^{X_i} e^{-\tilde{\alpha}(X_i - y)} \left(e^{-\alpha(X_i - y)}e^{-\alpha X_i} + e^{-2 \alpha X_{i-1}} e^{-\alpha y} \right) dy \\
&\leq C e^{-\alpha X^*} \left(|a| + |b| + e^{-\alpha X^*} |c| + e^{-\alpha X^*}|\lambda|^2|d|\right)
\end{align*}
The other integral is similar.

\item For the integral terms involving $\tilde{H}_i^\pm$, by \eqref{Zevolbounds} and the estimates from Lemma \ref{stabestimateslemma}, the integral has the same bounds as the integral involving $G_i^\pm$ that we just estimated. Thus we have
\begin{align*}
&\left|
P_+(X_i; \lambda) \int_0^{X_i} \Phi^s(X_i, y; \lambda) P_+(X_i; \lambda)^{-1} \tilde{H}_i^+(y) dy \right| \leq C e^{-\alpha X^*} 
\end{align*}
The other integral is similar.

\item The integral terms involving $c_{i-1} e^{\nu(\lambda) X_{i-1}} G_i^-(x) V^-(x; \lambda)$ and $c_i e^{-\nu(\lambda) X_i} G_i^+(x) V^+(x; \lambda)$ are bounded by $C e^{-\alpha X^*} |c|$ as in \cref{intGVpmbound}. 
\end{enumerate}

Putting all of these together, we have the following bound for $L_3$.
\begin{equation}\label{L3bound}
|L_3(\lambda)(a, b, c, d)| \leq C \left( e^{-\alpha X^*} (|a| + |c| + |\lambda|^2 |d|) + e^{-\tilde{\alpha} X^*} |b| \right)
\end{equation}
As in \cref{Zinv0}, $e^{-\alpha X^*} < \delta$, thus \cref{L3bound} becomes
\begin{equation*}
|L_3(\lambda)(a, b, c, d)| \leq C \left( \delta |a| + e^{-\alpha X^*}(|c| + |\lambda|^2 |d|) + e^{-\tilde{\alpha} X^*} |b| \right)
\end{equation*}
Let 
\[
J_1: \bigoplus_{j=1}^n (E^s(\lambda) \times E^u(
\lambda) \times E^c(\lambda) ) \rightarrow \bigoplus_{j=1}^n \rightarrow \C^{2m+1}
\]
be defined by $(J_1)_i(a_i^+, a_i^-, 2 a_i^c V_0(\lambda)) = a_i^+ - a_i^- + 2 a_i^c V_0(\lambda)$. The map $J_i$ is a linear isomorphism since 
\[
E^s(\lambda) \oplus E^u(\lambda) \oplus E^c(\lambda) = \C^{2m+1}.
\]
Consider the map
\[
S_1(a) = J_1(a) + L_3(\lambda)(a, 0, 0, 0) = J_1\left( I + J_1^{-1} L_3(\lambda)(a, 0, 0, 0)\right)
\]
Decreasing $\delta$ if necessary, $\|J_1^{-1} L_3(\lambda)(a, 0, 0, 0)\| < 1$, thus the operator $S_1(a)$ is invertible. We can then solve for $a$ to get
\[
a = A_1(\lambda)(b, c, d) = S_i^{-1}\left(-D d - C_i c + L_3(\lambda)(0, b, c, d)\right)
\]
Using the bounds on $L_3$, $A_1$ has bound given by \cref{A1bound}. We then plug this into $Z_1$ from Lemma \ref{Zinv0} to get the operator $Z_2(\lambda)(b,c,d)$ which has bound given by \cref{Z1bound}.
\end{proof}
\end{lemma}

Next, we derive expressions for $a_i^\pm$ and $a_i^c$ which we will use in evaluating the jump conditions. First, we have the following expressions for $a_i^\pm$.

\begin{lemma}\label{lemma:aipm}
For the initial conditions $a_i^\pm$, we have the expressions
\begin{equation}\label{aipmexp1}
\begin{aligned}
a_i^+ &= P_+(X_i; \lambda)^{-1} \left( P_0^u(\lambda) D_i d + A_2(\lambda)_i^+(b, c, d) \right) \\
a_i^- &= -P_-(-X_i; \lambda)^{-1} \left( P_0^s(\lambda) D_i d + A_2(\lambda)_i^-(b, c, d) \right)
\end{aligned}
\end{equation}
$A_2$ is linear in $(b, c, d)$, and has bound
\begin{align}
|A_2(\lambda)_i(b, c, d)|
&\leq C \left(e^{-\tilde{\alpha} X^*}|b| + e^{-\alpha X^*}(|c| + |\lambda|^2|d| + |D_i||d|) \right) \label{A2bound}
\end{align}

\begin{proof}
We apply the projections $P_0^{s/u}(\lambda)$ on the eigenspaces $E^{s/u}(\lambda)$ to \eqref{Dideq1}. Using the bound $|C_i c| \leq e^{-\alpha X^*}|c|$ from Lemma \ref{stabestimateslemma}, we obtain the expressions
\begin{align*}
a_i^+ &= P_0^u(\lambda) D_i d + A_2(\lambda)_i^+(b, c, d) \\
a_i^- &= -P_0^s(\lambda) D_i d + A_2(\lambda)_i^-(b, c, d) \\
\end{align*}
The bound on the remainder term $A_2(\lambda)(b, c, d)$ is found by substituting the bound for $A_1$ into the bound for $L_3$ and simplifying to get
\begin{align*}
|A_2&(\lambda)_i(b, c, d)| \leq C \left(e^{-\tilde{\alpha} X^*}|b| + e^{-\alpha X^*}(|c| + |\lambda|^2|d| + |D_i||d|) \right)
\end{align*} 

Anticipating what we will need when we evaluate the jump expressions, we will modify the expressions for $a_i^+$ and $a_i^-$ to involve the conjugation operators. Using the conjugation operator $P_+(X_i; \lambda)$, write $a_i^+$ as
\begin{align*}
a_i^+ &= P_+(X_i; \lambda)a_i^+ + (I - P_+(X_i; \lambda))a_i^+ \\
&= P_+(X_i; \lambda)a_i^+ - \Theta_+(X_i; \lambda))a_i^+
\end{align*}
Rearranging this and substituting the expression above for $a_i^+$, we get
\begin{align*}
P_+(&X_i; \lambda) a_i^+ = P_0^u(\lambda) D_i d + A_2(\lambda)_i^+(b, c, d) + \Theta_+(X_i; \lambda)a_i^+.
\end{align*}
Using the bound $A_1$ and the estimate \eqref{Thetadecay}, the last term on the RHS has the same bound as $A_2$. Incorporating that term into $A_2(\lambda)_i^+(b, c, d)$, we have
\begin{align*}
P_+(X_i; \lambda)a_i^+ &= P_0^u(\lambda) D_i d + A_2(\lambda)_i^+(b, c, d),
\end{align*}
where the bound on $A_2(\lambda)_i^+(b, c, d)$ is unchanged. Finally, apply the operator $P_+(X_i; \lambda)^{-1}$ on the left on both sides and solve for $a_i^+$ and get the first equation in \cref{aipmexp1}. Repeating the same procedure on $a_i^-$ using the conjugation operator $P_-(-X_i; \lambda)$ which gives us the second equation in \cref{aipmexp1}.
\end{proof}
\end{lemma}

Before we derive an expression for $a_i^c$, we will evaluate an inner product term which will appear in that expression.

\begin{lemma}\label{lemma:WprimeIP}
For the following inner product, we have
\begin{equation}\label{WprimeIP}
\langle W_0'(0), Q'(X_j) \rangle = -q(X_j) + \mathcal{O}(e^{-2\alpha X_j})
\end{equation}
where $q(x)$ is the first component of the primary pulse $Q(x)$ and $W_0'(0)$ is defined in \cref{nulambdalemma}.
\begin{proof}
Using the expression \cref{W0prime} for $W_0'(0)$ from \cref{nulambdalemma} and $Q'(x) = (\partial_x q(x), \partial_x^2 q(x), \dots, \partial_x^{2m}(x), 0)^T$, we can compute
\begin{align}\label{WprimeIP1}
\langle W_0'(0), Q'(X_j) \rangle = 
\frac{1}{c} \left( -c_3 \partial_x^2 q(X_j) -c_5 \partial_x^4 q(X_j) -\dots -c_{2m-1} \partial_x^{2m-2} q(X_j) + \partial_x^{2m} q(X_j) \right)
\end{align}
where the coefficients $c_j$ are defined in \cref{fpartials0}. Using \cref{Eprimeuform}, an equilibrium solution $u(x)$ to \cref{eqODE} satisfies the equation
\[
\partial_x^{2m} u- f(u, \partial_x u, \dots, \partial_x^{2m-1}u ) + cu = 0
\]
Expanding this in a Taylor series about the zero solution $u = 0$, we extract the linear part of $f$  to get
\[
\partial_x^{2m} u - \sum_{k=1}^{2m} f_{u_k}(0) \partial_x^{k-1} + cu + g(u, \partial_x u, \dots, \partial_x^{2m-1}u ) = 0
\]
where $g$ is at least quadratic in all of its arguments. Substituting the coefficients $c_j$ from \cref{fpartials0}, this becomes
\begin{equation*}
\partial_x^{2m} u - c_3 \partial_x^2 u - c_5 \partial_x^4 u - \dots - c_{2m-1} \partial_x^{2m-2}u + cu + g(u, \partial_x u, \dots, \partial_x^{2m-1}u ) = 0
\end{equation*}
which we rearrange to get 
\begin{equation}\label{lineqODE1}
\partial_x^{2m} u - c_3 \partial_x^2 u - c_5 \partial_x^4 u - \dots - c_{2m-1} \partial_x^{2m-2}u = -cu - g(u, \partial_x u, \dots, \partial_x^{2m-1}u )
\end{equation}
Since $q(x)$ solves \cref{lineqODE1}, equation \cref{WprimeIP1} becomes
\begin{align*}
\langle W_0'(0), Q'(X_j) \rangle &= 
\frac{1}{c} \left( -cq(X_j) - g(q(X_j), \partial_x q(X_j), \dots, \partial_x^{2m-1}q(X_j) \right) \\
&= -q(X_j) + \mathcal{O}(e^{-2\alpha X_j})
\end{align*}
since $g$ is as least quadratic in all of its and $q(x)$ and all of its derivatives are exponentially localized.
\end{proof}
\end{lemma}

In the next lemma, we derive an expression for $a_i^c$.

\begin{lemma}\label{lemma:ac1}
For the center initial conditions $a_i^c \in \C$, we have the expression
\begin{align}\label{tildeciexp1}
a_i^c &= \lambda q(X_i) (d_{i+1} - d_i ) + \tilde{A}_2(\lambda)_i(b, c, d)
\end{align}
$\tilde{A}_2$ is linear in $(b, c, d)$ and has bound
\begin{align}\label{tildeA2bound}
|\tilde{A}_2(\lambda)(b, c, d) \leq C |\lambda| ( e^{-\alpha X^*} + |\lambda|) \left( e^{-\alpha X^*}  |b| + |c| + e^{-\alpha X^*} |d| \right)
\end{align}

\begin{proof}
First, we look at the stable projection $\tilde{P}^s_+(X_i; \lambda) = P_+(X_i; \lambda) \Phi^s_+(X_i, X_i; \lambda) P_+^{-1}(X_i; \lambda)$. Since the range of the complement projection is not uniquely specified, we are free to choose any complement of $\ran \tilde{P}^s_+(X_i; \lambda)$ 



Apply the projection $P^c(\lambda)$ to the matching condition $P_i^+(X_i; \lambda) Z_i^+(X_i) - P_{i+1}^-(-X_i; \lambda) Z_{i+1}^-(-X_i) = D_i d + C_i c$ to get
\begin{equation}\label{Dideq0c}
\begin{aligned}
P^c(&\lambda)(D_i d + C_i c) = P^c(\lambda)( P_+(X_i; \lambda) a_i^c V_0(\lambda) + P_-(-X_i; \lambda)a_i^c V_0(\lambda)) + P^c(\lambda) \tilde{L}_3(\lambda)_i(a,b,c,d),
\end{aligned}
\end{equation}
where
\begin{equation}\label{tildeL3}
\begin{aligned}
\tilde{L}_3(&\lambda)_i(a,b,c,d) = P_+(X_i; \lambda)a_i^+ + P_+(X_i; \lambda)\Phi^s(X_i, 0; \lambda) b_i^+ \\
&+ P_+(X_i; \lambda) \int_0^{X_i} \Phi^s(X_i, y; \lambda) P_+(y; \lambda)^{-1}\big[ G_i^+(y) P_+(y; \lambda) Z_i^+(y) \\
&\qquad+ c_i e^{-\nu(\lambda)X_i} G_i^-(y) V^+(y; \lambda) + \lambda^2 d_i \tilde{H}_i^+(y)\big] dy \\
&- P_-(-X_i; \lambda)a_i^-  - P_-(-X_i; \lambda)\Phi^u(-X_i, 0; \lambda) b_{i+1}^- \\ 
&- P_-(-X_i; \lambda) \int_0^{-X_i} \Phi^u(-X_i, y; \lambda) P_-(y; \lambda)^{-1}\big[ G_{i+1}^-(y) P_-(y; \lambda)Z_{i+1}^-(y) \\
&\qquad+ c_i e^{\nu(\lambda)X_i} G_{i+1}^-(y) V^-(y; \lambda) + \lambda^2 d_{i+1} \tilde{H}_{i+1}^-(y)\big] dy
\end{aligned}
\end{equation}
To evaluate $P^c(\lambda)D_i d$, we use the expression for $P^c(\lambda)$ and the Taylor expansion of $W_0(\lambda)$ from \cref{nulambdalemma}. Since $\langle W_0, D_i d \rangle = 0$,
\begin{align*}
P^c(&\lambda)D_i d = \langle W_0(\lambda), D_i d \rangle + \mathcal{O}(e^{-\alpha X_i} |\lambda|^2 |d| ) \\
&= \langle W_0 + \overline{\lambda} W_0'(0) + \mathcal{O}(\overline{\lambda}^2), D_i d \rangle + \mathcal{O}(e^{-\alpha X_i} |\lambda|^2 |d| ) \\
&= \langle W_0, D_i d \rangle + \lambda \langle W_0'(0), (Q'(X_i) + Q'(-X_i))(d_{i+1} - d_i ) \rangle + \mathcal{O}(e^{-\alpha X_i} |\lambda|(e^{\alpha X_i} + |\lambda|)|d|) \\
&= \lambda \langle W_0'(0), Q'(X_i) + Q'(-X_i) \rangle (d_{i+1} - d_i ) + \mathcal{O}(e^{-\alpha X_i} |\lambda|(e^{\alpha X^*} + |\lambda|)|d|) \\
&= \lambda ( \langle W_0'(0), Q'(X_i) \rangle + \langle W_0'(0), -R Q'(X_i)\rangle )(d_{i+1} - d_i ) \rangle + \mathcal{O}(e^{-\alpha X_i} |\lambda|(e^{\alpha X^*} + |\lambda|)|d|)\\
&= \lambda \left( \langle W_0'(0), Q'(X_i) \rangle + \langle -R W_0'(0), Q'(X_i)\rangle \right)(d_{i+1} - d_i ) \rangle + \mathcal{O}(e^{-\alpha X_i} |\lambda|(e^{\alpha X^*} + |\lambda|)|d|) \\
&= 2 \lambda \langle W_0'(0), Q'(X_i) \rangle (d_{i+1} - d_i ) \rangle + \mathcal{O}(e^{-\alpha X^*} |\lambda|(e^{\alpha X^*} + |\lambda|)|d|)
\end{align*}
where in the last line we used the symmetry relation $W'(0) = -R W'(0)$ from Lemma \ref{nulambdalemma}. Using Lemma \ref{lemma:WprimeIP} to evaluate the inner product, this becomes
\begin{equation}\label{PcDid2}
P^c(\lambda)D_i d = -2 \lambda q(X_i) (d_{i+1} - d_i ) + \mathcal{O}(e^{-\alpha X^*} |\lambda|(e^{\alpha X^*} + |\lambda|)|d|)
\end{equation}

For $P^c(\lambda)C_i c$, we first expand $C_i c$ using the expressions for $V^\pm(\pm x; \lambda)$ from \cref{lemma:Vpm} and the expansions of the conjugation operators from the conjugation lemma to get
\begin{align*}
C_i c &= c_i \left( e^{\nu(\lambda) X_i} V^-(-X_i; \lambda) - e^{-\nu(\lambda) X_i} V^+(X_i; \lambda) \right) \\
&= c_i \left( e^{\nu(\lambda) X_i} e^{\nu(\lambda)X_i} P_-(-X_i; \lambda) V_0(\lambda) - e^{-\nu(\lambda) X_i} e^{\nu(\lambda)X_i} P_+(X_i; \lambda)V_0(\lambda) \right) \\
&= c_i \left( P_-(-X_i; \lambda) - P_+(X_i; \lambda) \right) V_0(\lambda) \\
&= c_i \left( R \Theta_+(X_i; \lambda) R - \Theta_+(X_i; \lambda) \right) V_0(\lambda),
\end{align*}
where in the last line we used \cref{defPminus} and the expansion \cref{conjlemmaP} for the conjugation operator. Expanding $\Theta_+(X_i; \lambda)$ and $V_0(\lambda)$ in a Taylor series about $\lambda = 0$, this becomes
\begin{align*}
C_i c &= c_i \left( R \Theta_+(X_i; 0) R - \Theta_+(X_i; 0) + \mathcal{O}(|\lambda|e^{-\alpha X^*}) \right) (V_0 + \mathcal{O}(|\lambda|)) \\
&= c_i \left( R \Theta_+(X_i; 0) R - \Theta_+(X_i; 0) \right) V_0 + \mathcal{O}(|\lambda|e^{-\alpha X^*})
\end{align*}
Applying the expression for $P^c(\lambda)$ from \cref{nulambdalemma} to this,
\begin{align*}
P^c(\lambda)&C_i c = \langle W_0(\lambda), C_i c \rangle + \mathcal{O}(|\lambda|^2 |c| ) \\
&= c_i \langle W_0 + \mathcal{O}(|\overline{\lambda}|),(R \Theta_+(X_i; 0) R - \Theta_+(X_i; 0) )( V_0 + \mathcal{O}(|\lambda|)) \rangle + \mathcal{O}(|\lambda|(e^{-\alpha X^*} + |\lambda|)|c|) \\
&= c_i \left( \langle R W_0, \Theta_+(X_i; 0) R V_0 \rangle - \langle W_0, \Theta_+(X_i; 0), V_0 \rangle \right) + \mathcal{O}(|\lambda|(e^{-\alpha X^*} + |\lambda|)|c|) \\
&= \mathcal{O}(|\lambda|(e^{-\alpha X^*} + |\lambda|)|c|) 
\end{align*}
since $R W_0 = W_0$ and $RV_0 = V_0$.

For the terms involving $a_i^c$, we use \cref{W0projlemma}(iii) to get
\begin{align*}
P_+(X_i; \lambda) a_i^c V_0(\lambda) + P_-(-X_i; \lambda)a_i^c V_0(\lambda) 
&= a_i^c( P_+(X_i, 0) + P_-(-X_i, 0) ) V_0(\lambda) + \mathcal{O}(|\lambda|e^{-\alpha|x|})
\end{align*}
Applying the expression for $P^c(\lambda)$ from \cref{nulambdalemma} and expanding both $W_0(\lambda)$ and $V_0(\lambda)$ in a Taylor series,
\begin{align*}
P^c(&\lambda)( P_+(X_i; \lambda) a_i^c V_0(\lambda) + P_-(-X_i; \lambda)a_i^c V_0(\lambda) ) \\
&= a_i^c\langle W_0 + \overline{\lambda}W_0'(0) + \mathcal{O}(|\overline{\lambda}|^2), ( P_+(X_i, 0) + R P_+(-X_i, 0) R )(V_0 + \lambda V_0'(0) + \mathcal{O}(|\lambda|^2)) \rangle + \mathcal{O}(|\lambda|e^{-\alpha X_i }|a|) \\
&= a_i^c\langle W_0 + \overline{\lambda}W_0'(0) + \mathcal{O}(|\overline{\lambda}|^2), ( 2 I + \mathcal{O}(e^{-\alpha X_i}) )(V_0 + \lambda V_0'(0) + \mathcal{O}(|\lambda|^2)) \rangle + \mathcal{O}(|\lambda|e^{-\alpha X_i }|a|) \\
&= 2 a_i^c \left( \langle W_0, V_0 \rangle + \lambda \langle W_0'(0), V_0 \rangle 
+ \lambda \langle W_0, V_0'(0) \rangle \right) + \mathcal{O}(|\lambda|(e^{-\alpha X_i } + |\lambda|)|a|) \\
&= 2 a_i^c + \mathcal{O}(|\lambda|(e^{-\alpha X_i } + |\lambda|)|a|) 
\end{align*}
Substituting the bound $A_1$ from \cref{Zinv1}, we have
\begin{align*}
P^c(&\lambda)( P_+(X_i; \lambda) a_i^c V_0(\lambda) + P_-(-X_i; \lambda)a_i^c V_0(\lambda) ) \\
&= 2 a_i^c + \mathcal{O}\left(|\lambda|(e^{-\alpha X_i } + |\lambda|)\left(e^{-\tilde{\alpha} X^*}|b| + e^{-\alpha X^*}(|c| + |\lambda|^2|d| + |D_i||d|) \right)\right) 
\end{align*}

Finally, we evaluate $P^c(\lambda)\tilde{L}_3(\lambda)_i(a,b,c,d)$. To do this, we use \cref{W0projlemma}(viii) together with the bounds we obtained for the terms of $L_3(\lambda)_i(a,b,c,d)$ from \cref{Zinv1} to get 
\begin{equation*}
|P^c(\lambda)\tilde{L}_3(\lambda)_i(a,b,c,d)| \leq C |\lambda|( e^{-\alpha X_i} + |\lambda|) \left( |a| + e^{-\alpha X^*} (|c| + |\lambda|^2 |d|) + e^{-\tilde{\alpha} X^*} |b| \right)
\end{equation*}
Substituting in the bound for $A_1$ from \cref{Zinv1} and simplifying, 
\begin{equation*}
|P^c(\lambda)\tilde{L}_3(\lambda)_i(a,b,c,d)| \leq C |\lambda| e^{-\alpha X^*} ( e^{-\alpha X_i} + |\lambda|) \left( |b| + |c| + |d| \right)
\end{equation*}
Combining all of these and dividing by 2, we get 
\begin{equation*}
a_i^c = -\lambda q(X_i) (d_{i+1} - d_i ) + \tilde{A}_2(\lambda)_i(b, c, d)
\end{equation*}
where we have bound
\begin{align*}
|\tilde{A}_2(\lambda)(b, c, d) \leq C |\lambda| ( e^{-\alpha X^*} + |\lambda|) \left( e^{-\alpha X^*}  |b| + |c| + e^{-\alpha X^*} |d| \right)
\end{align*}
\end{proof}
\end{lemma}

In the next lemma, we solve for the conditions at $x = 0$, which are
\begin{equation}\label{centercond}
\begin{aligned}
P_\pm(0; \lambda) Z_i^\pm(0) &\in \oplus Y^+ \oplus Y^- \oplus \C \Psi(0) \oplus \C \Psi^c(0) \\
P_+(0; \lambda) Z_i^+(0) - P_-(0; \lambda) Z_i^-(0) &\in \C \Psi(0) \oplus \C \Psi^c(0)
\end{aligned}
\end{equation}
Recall that we have the decomposition
\begin{equation}\label{DSdecomp}
\C^{2m+1} = \C Q'(0) \oplus Y^+ \oplus Y^- \oplus \C \Psi(0) \oplus \C \Psi^c(0)
\end{equation}
Thus \eqref{centercond} is equivalent to the three projections
\begin{equation}\label{centercond2}
\begin{aligned}
P(\C Q'(0) ) P_-(0; \lambda) Z_i^-(0) &= 0 \\
P(\C Q'(0) ) P_+(0; \lambda) Z_i^+(0) &= 0 \\
P(Y_i^+ \oplus Y_i^-) ( P_+(0; \lambda) Z_i^+(0) - P_-(0; \lambda) Z_i^-(0) ) &= 0
\end{aligned}
\end{equation}
where the kernel of each projection is the remaining spaces in the direct sum decomposition \eqref{DSdecomp}. We do not need to include $\C Q'(0)$ in the third equation of \eqref{centercond2} since we eliminated any component in $\C Q'(0)$ in the first two equations.

% second inversion lemma
\begin{lemma}\label{Zinv2}
There exist operators
\begin{align*}
B_1: &V_\lambda \times V_c \times V_d \rightarrow V_b \\
A_3: &V_\lambda \times V_c \times V_d \rightarrow V_a \\
Z_3: &V_\lambda \times V_c \times V_d \rightarrow V_Z
\end{align*}
such that $( a, b, Z ) = ( A_3(\lambda)(c, d), B_1(\lambda)(c, d), Z_3(\lambda)(c, d) )$ solves \eqref{systemZ}, \eqref{systemmiddle}, \eqref{systemcenter1}, and \eqref{systemcenter2} for any $(c, d)$. These operators are analytic in $\lambda$ and linear in $(c, d)$. Bounds for $B_1$, $A_3$, and $Z_3$ are given by
\begin{align}
|B_1(\lambda)(c, d)| &\leq C\Big( (|\lambda| + e^{-\alpha X^*})|c| + (|\lambda| + e^{-\alpha X^*})^2 |d| \Big) \label{B1bound} \\
|A_3(\lambda)(b, c, d)| &\leq C \left( e^{-\alpha X^*} |c| + |\lambda|^2 |d| + |D_i||d| \right) \label{A3bound} \\
\| Z_3(\lambda)(a,b,c,d) \| &\leq C\left( e^{-\alpha X^*} |c| + |\lambda|^2|d| + |D_i||d| \right) \label{Z3bound}
\end{align} 
In addition, we can write
\begin{align*}
a_i^+ &= P_i^+(X_i; \lambda)^{-1} \left( P_0^u(\lambda) D_i d + A_4(\lambda)_i^+(c, d) \right) \\
a_i^- &= -P_-(-X_i; \lambda)^{-1} \left( P_0^s(\lambda) D_i d + A_4(\lambda)_i^-(c, d) \right) \\
a_i^c &= -\lambda q(X_i)(d_{i+1} - d_i ) + \tilde{A}_4(\lambda)_i(c, d) 
\end{align*}
where $A_4$ and $\tilde{A}_4$ are linear in $(c, d)$, and have bounds
\begin{align}
|A_4(\lambda)(b, c, d)|
&\leq C \left( (|\lambda| + e^{-\alpha X^*})|c| + (|\lambda| + e^{-\alpha X^*})^2 |d| ) \right)  \label{A4bound} \\
|\tilde{A}_4(\lambda)(b, c, d)| &\leq C |\lambda| ( e^{-\alpha X^*} + |\lambda|) \left( |c| + e^{-\alpha X^*} |d| \right) \label{tildeA4bound}
\end{align}

\begin{proof}
At $x = 0$, the fixed point equations \eqref{Zfpeq} become
\begin{align*}
Z_i^-&(0) = \Phi^s(0, -X_{i-1}; \lambda) a_{i-1}^- + \Phi^u(0, 0; \lambda) b_i^- -\Phi^c(0, -X_{i-1}; \lambda) a_{i-1}^c V_0(\lambda) \\
&+ \int_{-X_{i-1}}^0 \Phi^s(0, y; \lambda)P_-(y; \lambda)^{-1} [ G_i^-(y) P_-(y; \lambda) Z_i^-(y) + c_{i-1} e^{\nu(\lambda)X_{i-1}} G_i^-(y) V^-(y; \lambda) + \lambda^2 d_i \tilde{H}_i^-(y)] dy \\
&+ \int_{-X_{i-1}}^0 \Phi^c(0, y; \lambda) P_-(y; \lambda)^{-1} [ G_i^-(y) P_-(y; \lambda) Z_i^-(y) + c_{i-1} e^{\nu(\lambda)X_{i-1}} G_i^-(y) V^-(y; \lambda) + \lambda^2 d_i \tilde{H}_i^-(y)] dy
\end{align*}
and
\begin{align*}
Z_i^+&(0) = \Phi^u(0, X_i; \lambda) a_i^+ + \Phi^s(0, 0; \lambda) b_i^+ + \Phi^c(0, X_i; \lambda) a_i^c V_0(\lambda) \\
&+ \int_{X_i}^0 \Phi^u(0, y; \lambda) P_+(y; \lambda)^{-1} [ G_i^+(y) P_+(y; \lambda) Z_i^+(y) + c_i e^{-\nu(\lambda)X_i} G_i^-(y) V^+(y; \lambda) + \lambda^2 d_i \tilde{H}_i^+(y)] dy \\
&+ \int_{X_i}^0 \Phi^c(0, y; \lambda) P_+(y; \lambda)^{-1} [ G_i^+(y) P_+(y; \lambda) Z_i^+(y) + c_i e^{-\nu(\lambda)X_i} G_i^-(y) V^+(y; \lambda) + \lambda^2 d_i \tilde{H}_i^+(y)] dy \\
\end{align*}
First, recall that at $Q(0)$, the tangent spaces to the stable and unstable manifold are given by
\begin{align*}
T_{Q(0)} W^u(0) &= \R Q'(0) \oplus Y^- \\
T_{Q(0)} W^s(0) &= \R Q'(0) \oplus Y^+
\end{align*}
Thus we have
\begin{align*}
P^-(0)^{-1} Q'(0) &= V^- \in E^u(0) \\
P^+(0)^{-1} Q'(0) &= V^+ \in E^s(0)
\end{align*}
Let
\begin{align*}
E^u(0) &= \C V^- \oplus E^- \\
E^s(0) &= \C V^+ \oplus E^+ \\
\end{align*}
Then we have
\begin{align*}
P^-(0)^{-1} Y^- = E^- \\
P^+(0)^{-1} Y^+ = E^+ \\
\end{align*}
We will use this to decompose $b_i^\pm$ uniquely as $b_i^\pm = x_i^\pm + y_i^\pm$, where $x_i^\pm \in \C V^\pm$ and $y_i^\pm \in E^\pm$. Since the evolution operators $\Phi(0, 0; \lambda)$ involve $\lambda$ but the projections we will be taking are for $\lambda = 0$, we will substitute
\begin{align*}
P_0^{s/u/c}(\lambda) &= P_0^{s/u/c}(0) + (P_0^{s/u/c}(\lambda) - P_0^{s/u/c}(0)) 
\end{align*}
Using these together with equation \eqref{centerevol} for the evolution $\Phi^c$ on $E^c(\lambda)$, we have
\begin{align*}
Z_i^-&(0) = \Phi^s(0, -X_{i-1}; \lambda) a_{i-1}^- + x_i^- + y_i^- + (P_0^u(\lambda) - P_0^u(0))b_i^- \\
&- P_0^c(0) e^{\nu(\lambda) X_{i-1}} a_{i-1}^c V_0(\lambda) + (P_0^c(\lambda) - P_0^c(0)) e^{\nu(\lambda) X_{i-1}} a_{i-1}^c V_0(\lambda) \\
&+ \int_{-X_{i-1}}^0 \Phi^s(0, y; \lambda) P_-(y; \lambda)^{-1} [ G_i^-(y) P_-(y; \lambda) Z_i^-(y) + c_{i-1} e^{\nu(\lambda)X_{i-1}} G_i^-(y) V^-(y; \lambda) + \lambda^2 d_i \tilde{H}_i^-(y)] dy \\
&+ \int_{-X_{i-1}}^0 \Phi^c(0, y; \lambda) P_-(y; \lambda)^{-1} [ G_i^-(y) P_-(y; \lambda) Z_i^-(y) + c_{i-1} e^{\nu(\lambda)X_{i-1}} G_i^-(y) V^-(y; \lambda) + \lambda^2 d_i \tilde{H}_i^-(y)] dy
\end{align*}
and
\begin{align*} 
Z_i^+&(0) = \Phi^u(0, X_i; \lambda) a_i^+ + x_i^+ + y_i^+ + (P_0^s(\lambda) - P_0^s(0)) b_i^+ \\
&+ P_0^c(0) e^{-\nu(\lambda)X_i} a_i^c V_0(\lambda) + (P_0^c(\lambda) - P_0^c(0)) e^{-\nu(\lambda)X_i} a_i^c V_0(\lambda) \\
&+ \int_{X_i}^0 \Phi^u(0, y; \lambda) P_+(y; \lambda)^{-1}[ G_i^+(y) P_+(y; \lambda) Z_i^+(y) + c_i e^{-\nu(\lambda)X_i} G_i^-(y) V^+(y; \lambda) + \lambda^2 d_i \tilde{H}_i^+(y)] dy \\
&+ \int_{X_i}^0 \Phi^c(0, y; \lambda) P_+(y; \lambda)^{-1}[ G_i^+(y) P_+(y; \lambda) Z_i^+(y) + c_i e^{-\nu(\lambda)X_i} G_i^-(y) V^+(y; \lambda) + \lambda^2 d_i \tilde{H}_i^+(y)] dy 
\end{align*}
Finally, we apply the conjugation operators $P_\pm(0; \lambda)$ in \eqref{centercond2}. For the terms involving $b$, and $a_i^c$, we substitute
\[
P_\pm(0; \lambda) = P_\pm(0; 0) + (P_\pm(0; \lambda) - P_\pm(0; 0))
\]
to obtain the expressions
\begin{align*}
P_-&(0; \lambda) Z_i^-(0) = P_-(0; 0)( x_i^- + y_i^- - P_0^c(0) e^{\nu(\lambda) X_{i-1}} a_{i-1}^c V_0(\lambda) ) \\
&+ P_-(0; \lambda) \Phi^s(0, -X_{i-1}; \lambda) a_{i-1}^- + (P_-(0; \lambda) - P_-(0; 0))b_i^- + P_-(0; \lambda)(P_0^u(\lambda) - P_0^u(0))b_i^- \\
&- (P_-(0; \lambda) - P_-(0; 0)) P_0^c(0) e^{\nu(\lambda) X_{i-1}} a_{i-1}^c V_0(\lambda) + P_-(0; \lambda) (P_0^c(\lambda) - P_0^c(0)) e^{\nu(\lambda) X_{i-1}} a_{i-1}^c V_0(\lambda) \\
&+ P_-(0; \lambda) \int_{-X_{i-1}}^0 \Phi^s(0, y; \lambda) P_-(y; \lambda)^{-1} \big[ G_i^-(y) P_-(y; \lambda) Z_i^-(y) \\
&\qquad+ c_{i-1} e^{\nu(\lambda)X_{i-1}} G_i^-(y) V^-(y; \lambda) + \lambda^2 d_i \tilde{H}_i^-(y)\big] dy \\
&+ P_-(0; \lambda) \int_{-X_{i-1}}^0 \Phi^c(0, y; \lambda) P_-(y; \lambda)^{-1} \big[ G_i^-(y) P_-(y; \lambda) Z_i^-(y) \\
&\qquad+ c_{i-1} e^{\nu(\lambda)X_{i-1}} G_i^-(y) V^-(y; \lambda) + \lambda^2 d_i \tilde{H}_i^-(y)\big] dy 
\end{align*}
and
\begin{align*}
P_+&(0; \lambda) Z_i^+(0) = P_+(0; 0)( x_i^+ + y_i^+ + P_0^c(0) e^{-\nu(\lambda)X_i} a_i^c V_0(\lambda) )\\
&+ P_+(0; \lambda) \Phi^u(0, X_i; \lambda) a_i^+ + (P_+(0; \lambda) - P_+(0; 0)) b_i^+ + P_+(0; \lambda) (P_0^s(\lambda) - P_0^s(0)) b_i^+ \\
&+ (P_+(0; \lambda) - P_+(0; 0))P_0^c(0) e^{-\nu(\lambda)X_i} a_i^c V_0(\lambda) + P_+(0; \lambda) (P_0^c(\lambda) - P_0^c(0)) e^{-\nu(\lambda)X_i} a_i^c V_0(\lambda) \\
&+ P_+(0; \lambda) \int_{X_i}^0 \Phi^u(0, y; \lambda) P_+(y; \lambda)^{-1}\big[ G_i^+(y) P_+(y; \lambda) Z_i^+(y) \\
&\qquad+ c_i e^{-\nu(\lambda)X_i} G_i^-(y) V^+(y; \lambda) + \lambda^2 d_i \tilde{H}_i^+(y)\big] dy \\
&+ P_+(0; \lambda) \int_{X_i}^0 \Phi^c(0, y; \lambda) P_+(y; \lambda)^{-1}\big[ G_i^+(y) P_+(y; \lambda) Z_i^+(y) \\
&\qquad + c_i e^{-\nu(\lambda)X_i} G_i^-(y) V^+(y; \lambda) + \lambda^2 d_i \tilde{H}_i^+(y)\big] dy \\
\end{align*}
With this setup, the projections on $Q'(0)$ and $Y^+ \oplus Y^-$ will either eliminate or act as the identity on the terms in the first lines of $P_i^-(0; \lambda) Z_i^-(0)$ and $P_i^+(0; \lambda) Z_i^+(0)$. Applying the projections in \eqref{centercond2}, we obtain an expression of the form
\begin{equation}\label{projxy}
\begin{pmatrix}x_i^- \\ x_i^+ \\ 
y_i^+ - y_i^- \end{pmatrix} + L_4(\lambda)_i(b, c, d) = 0
\end{equation}
To get a bound on $L_4$, we will bound the individual terms involved. 

\begin{enumerate}
\item For the $a_i^\pm$ terms, we use the expressions from Lemma \ref{lemma:aipm}, the bounds for $A_2$, and the exponential dichotomy bounds to get
\begin{align*}
|P_-(0; \lambda) \Phi^s(0, -X_{i-1}; \lambda) a_{i-1}^-| 
&\leq C e^{-\tilde{\alpha} X^*} \left( e^{-\tilde{\alpha} X^*} |b|  + e^{-\alpha X^*}(|c| + |\lambda|^2 |d|) + |D_i||d| \right)\\
|P_+(0; \lambda) \Phi^u(0, X_i; \lambda) a_i^+| 
&\leq C e^{-\tilde{\alpha} X^*} \left( e^{-\tilde{\alpha} X^*} |b|  + e^{-\alpha X^*}(|c| + |\lambda|^2 |d|) + |D_i||d| \right)
\end{align*}

\item For the $a_i^c$ terms, we use the expression from Lemma \ref{lemma:ac1} together with \cref{lemma:expnubound} and the fact that the conjugation operators and eigenprojections are smooth in $\lambda$ to get
\begin{align*}
|(P_+&(0; \lambda) - P_+(0; 0))P_0^c(0) e^{-\nu(\lambda)X_i} a_i^c V(0)
\lambda) + P_+(0; \lambda) (P_0^c(\lambda) - P_0^c(0)) e^{-\nu(\lambda)X_i} a_i^c V(0)
\lambda)| \\
&\leq C |\lambda|^2 ( e^{-\alpha X^*} + |\lambda|) \left( e^{-\alpha X^*}  |b| + |c| + |d| \right)
\end{align*}
The term involving $a_{i-1}^c V_0(\lambda)$ is similar. 

\item For the remainder terms involving $b_i$, since the conjugation operators $P_\pm(x; \lambda)$ and eigenprojections $P_0^{s/u}(\lambda)$ are smooth in $\lambda$, we have
\begin{align*}
|(P_-(0; \lambda) &- P_-(0; 0))b_i^- + P_-(0; \lambda)(P_0^u(\lambda) - P_0^u(0))b_i^-| \leq C |\lambda| |b|
\end{align*}
The $b_i^+$ term is similar.

\item The bound on the integral terms is determined by the integrals involving the center subspace, since the other integrals will have stronger bounds. The integrals involving $Z$ are the same as those in \cref{L1boundlemma}, where we take $x = 0$. Thus they are bounded in the same way. Using the bound $Z_2$ for $| Z_i^+(y)|$, we have 
\begin{align*}
&\left| P_i^+(0; \lambda) \int_{X_i}^0 \Phi^c(0, y; \lambda) P_+(y; \lambda)^{-1} G_i^+(y) P_+(y; \lambda) Z_i^+(y) dy \right| \\
&\qquad \leq C e^{-\alpha X^*} \left(|b| + |c| + |D_i||d| + e^{-\alpha X^*}|\lambda|^2|d|\right)
\end{align*}
The other integral term is similar.

\item The integrals involving $\tilde{H}_i^\pm$ are the same as those in \cref{L2boundlemma}, where we take $x = 0$. Thus we have
\begin{align*}
\left| \lambda^2 d_i P_-(0; \lambda) \int_{-X_{i-1}}^0 \Phi^c(0, y; \lambda) P_-(y; \lambda)^{-1} \tilde{H}_i^-(y) dy \right| &\leq C |\lambda|^2 |d| 
\end{align*}
We could obtain a better bound, but it is not necessary. The other term is similar.

\item The integrals involving $G_i^\pm(x) V^\pm(x)$ are the same as those in \cref{L2boundlemma} with $x = 0$, and are thus bounded by $C e^{-\alpha X^*} |c|$.
\end{enumerate}

Combining all of these individual bounds, recalling that $\alpha = 2 \tilde{\alpha}$, using the estimate $|D_i| \leq C e^{-\alpha X^*}$, and simplifying, we obtain the uniform bound for $L_4(\lambda)(b, c, d)$.
\begin{align*}
L_4(\lambda)(b, c, d) \leq 
C\Big( (|\lambda| + e^{-\alpha X^*})(|b| + |c|) + (|\lambda| + e^{-\alpha X^*})^2 |d|  \Big) 
\end{align*}
Since $|\lambda| < \delta$ and $e^{-\alpha X^*} < \delta$, this becomes
\begin{align*}
L_4(\lambda)(b, c, d) \leq 
C\Big( \delta |b| + (|\lambda| + e^{-\alpha X^*})|c| + (|\lambda| + e^{-\alpha X^*})^2 |d| \Big) 
\end{align*}
which is uniform in $|b|$. Define the map
\[
J_2: \left( \bigoplus_{j=1}^n \C V^+ \oplus \C V^- \right) \oplus
\left( \bigoplus_{j=1}^n E^+ \oplus E^- \right) 
\rightarrow \bigoplus_{j=1}^n \C V^+ \oplus \C V^- \oplus (E^+ \oplus E^-)
\]
by 
\[
J_2( (x_i^+, x_i^-),(y_i^+, y_i^-))_i = ( x_i^+, x_i^-, y_i^+ - y_i^- )
\]
Since $\C^{2m} = E^s(0) \oplus E^u(0) = \C V^+ \oplus \C V^- \oplus (E^+ \oplus E^-)$, $J_2$ is an isomorphism. Using this as the fact that $b_i = (x_i^- + y_i^-, x_i^+ + y_i^+)$, we can write \eqref{projxy} as
\begin{equation}\label{projxy2}
J_2( (x_i^+, x_i^-),(y_i^+, y_i^-))_i 
+ L_4(\lambda)_i(b_i, 0, 0) + L_4(\lambda)_i(0, c, d) = 0
\end{equation}
Consider the map
\begin{align*}
S_2(b)_i &= J_2( (x_i^+, x_i^-),(y_i^+, y_i^-))_i 
+ L_4(\lambda)_i(b_i, 0, 0) 
\end{align*}
Substituting this in \eqref{projxy2}, we have
\begin{align*}
S_2(b) &= -L_4(\lambda)(0, c, d)
\end{align*}

Decreasing $\delta$ if necessary, the operator $S_2(b)$ is invertible. Thus we can solve for $b$ to get
\begin{align}
b = B_1(\lambda)(c,d) 
= -S_2^{-1} L_4(\lambda)(0, c, d)
\end{align}
The bound on $B_1$ is given by the bound on $L_4$.
\begin{align*}
|B_1(\lambda)(c, d)| \leq C\Big( (|\lambda| + e^{-\alpha X^*})|c| + (|\lambda| + e^{-\alpha X^*})^2 |d| \Big) 
\end{align*}
We now plug $B_1$ into all of the previous bounds, using in addition the bound on $C_i c$ from Lemma \ref{stabestimateslemma}. Plugging $B_1$ into the bound \eqref{A1bound} for $A_1$, we get $A_3$ with bound given by \cref{A3bound}. Plugging $B_1$ into the bound $Z_2$, we get $Z_3$ with bound given by \cref{Z3bound}. Plugging $B_1$ into the bound \eqref{A2bound} for $A_2$, we get $A_4$ with bound \cref{A4bound}. Finally, we plug in $B_1$ into the bound for $\tilde{A}_2$ to get $\tilde{A}_4$ with bound \cref{tildeA4bound}.
\end{proof}
\end{lemma}

\section{Jump Conditions}

Up to this point, we have solved uniquely for $Z$, $a$, $b$, and $g$. We will not be able to obtain a unique solution for $c$ and $d$. We will instead compute jump conditions in the direction of $\Psi(0)$ and $\Psi^c(0)$. Obtaining a nontrivial solution for $c$ and $d$ will be equivalent to these jump conditions being 0. First, we compute the jump in the direction of $\Psi^c(0)$. 

% jump lemma : center adjoint
\begin{lemma}\label{jumpcenteradj}
The jumps in the direction of $\Psi^c(0)$ are given by
\begin{equation}\label{xic}
\begin{aligned}
\xi^c_i &= e^{-\nu(\lambda)X_i} c_i - e^{\nu(\lambda)X_i} c_{i-1} - \frac{1}{2}\lambda \tilde{M} \left( e^{-\nu(\lambda)X_i}c_i + e^{\nu(\lambda)X_i}c_{i-1}\right)\\
&- \lambda q(X_i) (d_{i+1} - d_i ) - \lambda q(X_{i-1}) (d_i - d_{i-1} )
- \lambda^2 \tilde{M}^c d_i + R^c(\lambda)_i(c, d)
\end{aligned}
\end{equation}
where
\begin{align*}
\tilde{M}^c &= \int_{-\infty}^\infty \partial_c q(y) dy \\
\tilde{M} &= \int_{-\infty}^{\infty} \left(v^c(y) - \frac{1}{c}\right) dy
\end{align*}
The remainder term $R^c_i(c, d)$ is analytic in $\lambda$ and linear in $(c, d)$ and has bound \cref{centerR} given below in the proof. The jump conditions can be written in matrix form as
\begin{equation}\label{matrixjumpc}
\left( K(\lambda) -\frac{1}{2} \lambda \tilde{M} K_1(\lambda) + C_1 \right) c + (-\lambda A_1 - \lambda^2 \tilde{M}^c I + D_1) d = 0.
\end{equation}

$K(\lambda)$ is the matrix
\begin{align*}
K(\lambda) =  
\begin{pmatrix}
e^{-\nu(\lambda)X_1} & & & & & -e^{\nu(\lambda)X_0} \\
-e^{\nu(\lambda)X_1} & e^{-\nu(\lambda)X_2} \\
& -e^{\nu(\lambda)X_2} & e^{-\nu(\lambda)X_3} \\
  & & \ddots & && \\
& & & & -e^{\nu(\lambda)X_{n-1}} & e^{-\nu(\lambda)X_0}
\end{pmatrix},
\end{align*}
$K_1(\lambda)$ is the matrix
\begin{align*}
K_1(\lambda) =  
\begin{pmatrix}
e^{-\nu(\lambda)X_1} & & & & & e^{\nu(\lambda)X_0} \\
e^{\nu(\lambda)X_1} & e^{-\nu(\lambda)X_2} \\
& e^{\nu(\lambda)X_2} & e^{-\nu(\lambda)X_3} \\
 & & \ddots & &&   \\
& & & & e^{\nu(\lambda)X_{n-1}} & e^{\nu(\lambda)X_0}
\end{pmatrix},
\end{align*}
and $A_1$ is the matrix
\begin{align*}
A_1 &= \begin{pmatrix}
q(X_0) - q(X_1) & q(X_1) &&& -q(X_0) \\
-q(X_1) & q(X_1) - q(X_2) & q(X_2) \\
& -q(X_2) & q(X_2) - q(X_3) & q(X_3) \\ && \ddots \\
\\
q(X_0) &&& -q(X_{n-1}) & q(X_{n-1}) - q(X_0) 
\end{pmatrix},
\end{align*}
where $q(x)$ is the first component of the primary pulse solution $Q(x)$. The remainder matrices have uniform bounds
\begin{align}\label{centerjumprem}
|C_1| &\leq C |\lambda|(|\lambda| + r^{1/2}) \\
|D_1| &\leq C |\lambda|(|\lambda| + r^{1/2})^2
\end{align}

\begin{proof}
We want to solve the center jump condition, which we recall is given by
\[
\xi_i^c = 
\langle \Psi^c(0), P_+(0; \lambda) Z_i^+(0) - P_-(0; \lambda) Z_i^-(0) + c_i e^{-\nu(\lambda)X_i}V^+(0; \lambda) - c_{i-1} e^{\nu(\lambda)X_i} V^-(0; \lambda) \rangle 
\]
From \cref{lemma:VpmPsiIP},
\begin{align}\label{centerVjump}
\langle \Psi^c(0), &c_i e^{-\nu(\lambda)X_i}V^+(0; \lambda) - c_{i-1} e^{\nu(\lambda)X_i} V^-(0; \lambda) \rangle \\
&= e^{-\nu(\lambda)X_i}c_i  - e^{\nu(\lambda)X_i}c_{i-1} 
-\frac{1}{2}\lambda \tilde{M} \left( e^{-\nu(\lambda)X_i}c_i + e^{\nu(\lambda)X_i}c_{i-1}\right)+ \mathcal{O}(|\lambda|^2 |c| ),
\end{align}
where we used Lemma \ref{lemma:expnubound} to bound $e^{\pm\nu(\lambda)X_i}$ in the remainder term by a constant. The terms $P_i^\pm(0; \lambda) Z_i^\pm(0)$ are given by
\begin{align*}
P_-&(0; \lambda) Z_i^-(0) = P_-(0; 0) b_i^- - P_-(0; \lambda) e^{\nu(\lambda) X_{i-1}} a_{i-1}^c V_0(\lambda) \\
&+ P_-(0; \lambda) \Phi^s(0, -X_{i-1}; \lambda) a_{i-1}^- + (P_-(0; \lambda) - P_-(0; 0))b_i^- + P_-(0; \lambda)(P_0^u(\lambda) - P_0^u(0))b_i^- \\
&+ P_-(0; \lambda) \int_{-X_{i-1}}^0 \Phi^s(0, y; \lambda) P_-(y; \lambda)^{-1} \big[ G_i^-(y) P_-(y; \lambda) Z_i^-(y) \\
&\qquad+ c_{i-1} e^{\nu(\lambda)X_{i-1}} G_i^-(y) V^-(y; \lambda) + \lambda^2 d_i \tilde{H}_i^-(y)\big] dy \\
&+ P_-(0; \lambda) \int_{-X_{i-1}}^0 \Phi^c(0, y; \lambda) P_-(y; \lambda)^{-1} \big[ G_i^-(y) P_-(y; \lambda) Z_i^-(y) \\
&\qquad+ c_{i-1} e^{\nu(\lambda)X_{i-1}} G_i^-(y) V^-(y; \lambda) + \lambda^2 d_i \tilde{H}_i^-(y)\big] dy
\end{align*}
and
\begin{align*}
P_+&(0; \lambda) Z_i^+(0) = P_+(0; 0) b_i^+ + P_+(0; \lambda) e^{-\nu(\lambda)X_i} a_i^c V_0(\lambda) \\
&+ P_+(0; \lambda) \Phi^u(0, X_i; \lambda) a_i^+ + (P_+(0; \lambda) - P_+(0; 0)) b_i^+ + P_+(0; \lambda) (P_0^s(\lambda) - P_0^s(0)) b_i^+ \\
&+ P_+(0; \lambda) \int_{X_i}^0 \Phi^u(0, y; \lambda) P_+(y; \lambda)^{-1} \big[ G_i^+(y) P_+(y; \lambda) Z_i^+(y) \\
&\qquad+ c_i e^{-\nu(\lambda)X_i} G_i^-(y) V^+(y; \lambda) + \lambda^2 d_i \tilde{H}_i^+(y)\big] dy \\
&+ P_+(0; \lambda) \int_{X_i}^0 \Phi^c(0, y; \lambda) P_+(y; \lambda)^{-1} \big[ G_i^+(y) P_+(y; \lambda) Z_i^+(y) \\
&\qquad+ c_i e^{-\nu(\lambda)X_i} G_i^-(y) V^+(y; \lambda) + \lambda^2 d_i \tilde{H}_i^+(y)\big] dy 
\end{align*}

The leading order terms involve $a_i^c$ and the center integral. 
\begin{enumerate}

\item For the terms involving $a_i^c$, using the expression from Lemma \ref{lemma:ac1}, \cref{lemma:VpmPsiIP}, and \cref{lemma:expnubound} we have
\begin{align*}
\langle W_0, &P_+(0; \lambda) e^{-\nu(\lambda)X_i} a_i^c V_0(\lambda) \rangle = -e^{-\nu(\lambda)X_i} a_i^c \langle W_0, V^+(0; \lambda) \rangle \\
&= e^{-\nu(\lambda)X_i} \lambda q(X_i) (d_{i+1} - d_i ) + \mathcal{O}\left( |\lambda| (|\lambda| + e^{-\alpha X^*})|c| + |\lambda| e^{-\alpha X^*}(|\lambda| + e^{-\alpha X^*}) |d|) \right)
\end{align*}
Expanding the exponential in a Taylor series,
\begin{align*}
e^{-\nu(\lambda)X_i} 
&= 1 - \nu(\lambda)X_i + \mathcal{O}( \nu(\lambda)^2 X_i^2 e^{-\nu(\lambda)X_i} ) \\
&= 1 - \frac{1}{c}\lambda X_i + \mathcal{O}(|\lambda|^3 X_i + |\lambda|^2 X_i^2 e^{-\nu(\lambda)X_i}) \\
&= 1 + \mathcal{O}(|\lambda|)
\end{align*}
since $X_i \leq X = \mathcal{O}(|\log r|)$. Substituting this in,
\begin{align*}
\langle W_0, &P_+(0; \lambda) e^{-\nu(\lambda)X_i} a_i^c V_0(\lambda) \rangle = -e^{-\nu(\lambda)X_i} a_i^c \langle W_0, V^+(0; \lambda) \rangle \\
&= -\lambda q(X_i) (d_{i+1} - d_i ) + \mathcal{O}\left( |\lambda| (|\lambda| + e^{-\alpha X^*})|c| + |\lambda| e^{-\alpha X^*}(|\lambda| + e^{-\alpha X^*}) |d|) \right)
\end{align*}
Similarly, 
\begin{align*}
\langle \Psi^c(0), &P_-(0; \lambda) e^{\nu(\lambda) X_{i-1}} a_{i-1}^c \rangle \\
&= -\lambda q(X_{i-1}) (d_i - d_{i-1} ) +\mathcal{O}\left( |\lambda|(|\lambda| + e^{-\alpha X^*})|c| + |\lambda| e^{-\alpha X^*}(|\lambda| + e^{-\alpha X^*}) |d|) \right)
\end{align*}

\item For the center integral involving $\tilde{H}_i^\pm$, 
\begin{align*}
\langle W_0, &P_-(0; \lambda) \lambda^2 d_i \int_{-X_{i-1}}^0 \Phi^c(0, y; \lambda) P_-(y; \lambda)^{-1} \tilde{H}_i^-(y) dy \rangle \\
&= \lambda^2 d_i  \langle W_0, \int_{-X_{i-1}}^0 e^{-\nu(\lambda) y} P_-(0; \lambda) P^c(\lambda) P_-(y; \lambda)^{-1} B \partial_c Q(y) dy \rangle  + \mathcal{O}(|\lambda|^2 e^{-\alpha X^*} |d|) \\
&= \lambda^2 d_i  \langle W_0, \int_{-X_{i-1}}^0 e^{-\nu(\lambda) y} \tilde{P}^c_-(0) B \partial_c Q(y) dy \rangle + \mathcal{O}(|\lambda|^2( e^{-\alpha X^*} + |\lambda|) |d|)\\
&= \lambda^2 d_i  \int_{-X_{i-1}}^0 e^{-\nu(\lambda) y} \langle W_0, B \partial_c Q(y) \rangle \langle W_0, V^c(0) \rangle dy + \mathcal{O}(|\lambda|^2( e^{-\alpha X^*} + |\lambda|) |d|) \\
&= \lambda^2 d_i \int_{-X_{i-1}}^0 e^{-\nu(\lambda) y} \partial_c q(y) dy + \mathcal{O}(|\lambda|^2( e^{-\alpha X^*} + |\lambda|) |d|)
\end{align*}
To evaluate the integral, we expand the exponential in a Taylor series to get
\begin{align*}
e^{-\nu(\lambda)y} 
&= 1 - \nu(\lambda)y + \mathcal{O}( \nu(\lambda)^2 y^2 e^{-\nu(\lambda)y} ) \\
&= 1 - \frac{1}{c}\lambda y + \mathcal{O}(|\lambda|^3 y + |\lambda|^2 y^2 e^{-\nu(\lambda)y})
\end{align*}
When we evaluate the integral, we note that since $\partial_c q(y)$ is exponentially localized with decay constant $\alpha$, and $|\nu(\lambda)| < \alpha$,
\begin{align*}
\int_{-X_{i-1}}^0 y \partial_c q(y) dy = C
\end{align*}
and
\begin{align*}
\int_{-X_{i-1}}^0 y^2 e^{-\nu(\lambda)y}\partial_c q(y) dy = C
\end{align*}
Thus we have
\begin{align*}
\int_{-X_{i-1}}^0 e^{-\nu(\lambda) y} \partial_c q(y) dy &= \int_{-X_{i-1}}^0 \partial_c q(y)dy + \mathcal{O}(|\lambda|) \\
&= \int_{-\infty}^0 \partial_c q(y)dy + \mathcal{O}(|\lambda| + e^{-\alpha X^*}) 
\end{align*}
Combining everything,
\begin{align*}
\langle W_0, &P_-(0; \lambda) \lambda^2 d_i \int_{-X_{i-1}}^0 \Phi^c(0, y; \lambda) P_-(y; \lambda)^{-1} \tilde{H}_i^-(y) dy \rangle \\
&= \lambda^2 d_i \int_{-\infty}^0 \partial_c q(y)dy + \mathcal{O}(|\lambda|^2( |\lambda| + e^{-\alpha X^*})|d|) 
\end{align*}
For the other integral term, we similarly have
\begin{align*}
\langle W_0, &P_+(0; \lambda) \int_{X_i}^0 \Phi^c(0, y; \lambda) \lambda^2 d_i P_+(y; \lambda)^{-1} \tilde{H}_i^+(y) dy \rangle = \lambda^2 d_i \int_{\infty}^0 \partial_c q(y)dy + \mathcal{O}(|\lambda|^2( |\lambda| + e^{-\alpha X^*})) \\
&= -\lambda^2 d_i \int_0^{\infty} \partial_c q(y)dy + \mathcal{O}(|\lambda|^2( |\lambda| + e^{-\alpha X^*})|d|)
\end{align*}
\end{enumerate}

The rest of the terms are higher order.
\begin{enumerate}

\item For the terms involving $a_i^\pm$, we use the expressions from Lemma \ref{Zinv2} together with Lemma \ref{W0projlemma}(iv) and the exponential dichotomy evolution bounds to get
\begin{align*}
|P_-(0; \lambda) \Phi^s(0, -X_{i-1}; \lambda) a_{i-1}^-| &\leq C |\lambda| e^{-\tilde{\alpha} X_{i-1}} \left( (|\lambda| + e^{-\alpha X^*})|c| + |\lambda|^2 |d| + |D_i||d| \right) \\
|P_+(0; \lambda) \Phi^u(0, X_i; \lambda) a_i^+| &\leq C |\lambda| e^{-\tilde{\alpha} X_i}\left( (|\lambda| + e^{-\alpha X^*})|c| + |\lambda|^2 |d| + |D_i||d| \right) 
\end{align*} 

\item For the terms involving $b$, the terms $P_-(0; 0) b_i^-$ and $P_+(0, 0)b_i^+$ are eliminated outright when we take the inner product with $W_0$. For the other terms, we use the estimate for $B_1$ from Lemma \ref{Zinv2}.
\begin{align*}
|(P_-&(0; \lambda) - P_-(0; 0))b_i^- + P_-(0; \lambda)(P_0^u(\lambda) - P_0^u(0))b_i^-| \\
&\leq C |\lambda|\Big( (|\lambda| + e^{-\alpha X^*})|c| + (|\lambda| + e^{-\alpha X^*})^2 |d| \Big)
\end{align*}

\item We will do the center integrals first. For the integrals involving $Z$, when $\lambda = 0$, the integrands are eliminated by the center projection. 
\begin{align*}
P_+(0; \lambda) &\Phi^c(0, y; \lambda) P_+(y; \lambda)^{-1} G_i^+(y) P_+(y; \lambda) Z_i^+(y) \\
&= P_+(0; 0) \Phi^c(0, y; 0) P_+(y; 0)^{-1} G_i^+(y) P_+(y; \lambda) Z_i^+(y) + \mathcal{O}(|e^{-\nu(\lambda) y}||\lambda||G_i^+(y) Z_i^+(y)|) \\
&= e^{-\nu(\lambda) y} P_+(0; 0) P_+(y; 0)^{-1} G_i^+(y) P_+(y; \lambda) Z_i^+(y) + \mathcal{O}(|e^{-\nu(\lambda) y}||\lambda||G_i^+(y) Z_i^+(y)|) \\
&= e^{-\nu(\lambda) y} G_i^+(y) P_+(y; \lambda) Z_i^+(y) + \mathcal{O}(|e^{-\nu(\lambda) y}||\lambda||G_i^+(y) Z_i^+(y)|)
\end{align*}
Since the bottom row of $G_i^\pm(y)$ is all zeros for all $y$, 
\[
\langle W_0, e^{-\nu(\lambda) y} G_i^+(y) P_+(y; \lambda) Z_i^+(y) \rangle = 0
\]
Using this together with the estimate $Z_3$ and the estimate of the integral from Lemma \ref{L1boundlemma}, we have
\begin{align*}
&\left| \langle W_0, P_+(0; \lambda) \int_{X_i}^0 \Phi^c(0, y; \lambda) P_+(y; \lambda)^{-1} G_i^+(y) P_+(y; \lambda) Z_i^+(y) dy \rangle \right| \\
&\qquad \leq C |\lambda| e^{-\alpha X^*}\Big(|c| + e^{-\alpha X^*}\left(|D_i||d| + |\lambda|^2|d|\right) \Big)
\end{align*}
The other integral is similar. The center integral involving $G_i^\pm(x) V^\pm(x)$ is bounded by $C e^{-\alpha X^*}|c|$ as in Lemma \ref{L2boundlemma}.

\item Finally, we look at the noncenter integrals. For the integral involving $Z$, we first evaluate 
\begin{align*}
P_-(0; &\lambda) \Phi^s(0, y; \lambda) P_-(y; \lambda)^{-1} G_i^-(y) P_-(y; \lambda)Z_i^-(y) \\
&= P_-(0; 0) \Phi^s(0, y; 0) P_-(y; 0)^{-1}G_i^-(y) P_-(y; \lambda)Z_i^-(y) + \mathcal{O}(|\lambda| G_i^-(y) Z_i^-(y)) \\
&= \tilde{P}_-^s(0)G_i^-(y) P_-(y; \lambda)Z_i^-(y) + \mathcal{O}(|\lambda| G_i^-(y) Z_i^-(y))
\end{align*}
From Lemma \ref{centerprojlemma},
\[
\langle W_0, \tilde{P}_-^s(0)G_i^-(y) P_-(y; \lambda)Z_i^-(y) \rangle = 0
\]
The remaining integral has the same bound as for the center integral involving $Z$. For the noncenter integral involving $\tilde{H}_i^\pm$, 
\begin{align*}
\langle W_0, 
&P_+(0; \lambda) \int_{X_i}^0 \Phi^u(0, y; \lambda) \lambda^2 d_i P_+(y; \lambda)^{-1} \tilde{H}_i^+(y) dy\\
&= \lambda^2 d_i \langle W_0, \int_{X_i}^0 P_+(0; 0) \Phi^u(0, y; 0) P_+(y; 0)^{-1} B \partial_c Q(y) dy \rangle + \mathcal{O}(|\lambda|^3 |d| ) \\
&= \lambda^2 d_i \int_{X_i}^0 \langle W_0, \tilde{\Phi}^u(0, y)  B \partial_c Q(y) \rangle dy + \mathcal{O}(|\lambda|^3 |d| ) \\
&= \mathcal{O}(|\lambda|^3 |d| ) 
\end{align*}
The other integral is similar. The bound for the noncenter integral involving $G_i^\pm(x) V^\pm(x)$ is stronger than the bound for the center integral involving $G_i^\pm(x) V^\pm(x)$, thus is subsumed by that bound.
\end{enumerate}

We put this all together to obtain the center jump expressions, which are given by
\begin{align*}
\xi^c_i &= e^{-\nu(\lambda)X_i} c_i - e^{\nu(\lambda)X_i} c_{i-1} - \frac{1}{2}\lambda \tilde{M} \left( e^{-\nu(\lambda)X_i}c_i + e^{\nu(\lambda)X_i}c_{i-1}\right)\\
&- \lambda q(X_i) (d_{i+1} - d_i ) - \lambda q(X_{i-1}) (d_i - d_{i-1} )
- \lambda^2 \tilde{M}^c d_i + R^c(\lambda)_i(c, d)
\end{align*}
where
\begin{align*}
\tilde{M}^c &= \int_{-\infty}^\infty \partial_c q(y) dy \\
\tilde{M} &= \int_{-\infty}^{\infty} \left(v^c(y) - \frac{1}{c}\right) dy
\end{align*}
and the remainder term $R^c_i(c, d)$ has uniform bound
\begin{equation}\label{centerR}
\begin{aligned}
\|R^c&(c, d)\| \leq C \Big( |\lambda|(|\lambda| + e^{-\alpha X^*})|c| + |\lambda| (|\lambda| + e^{-\alpha X^*})^2 |d| \Big)
\end{aligned}
\end{equation}

Finally, we write this in matrix form. The leading order terms involving $c$ are given by $K(\lambda)c -\frac{1}{2} \lambda \tilde{M} K_1(\lambda) c$, where $K(\lambda)$ and $K_1(\lambda)$ are the periodic, bi-diagonal matrices
\begin{align*}
K(\lambda) =  
\begin{pmatrix}
e^{-\nu(\lambda)X_1} & & & & & -e^{\nu(\lambda)X_0} \\
-e^{\nu(\lambda)X_1} & e^{-\nu(\lambda)X_2} \\
& -e^{\nu(\lambda)X_2} & e^{-\nu(\lambda)X_3} \\
  & & \ddots & && \\
& & & & -e^{\nu(\lambda)X_{n-1}} & e^{-\nu(\lambda)X_0}
\end{pmatrix},
\end{align*}
\begin{align*}
K_1(\lambda) =  
\begin{pmatrix}
e^{-\nu(\lambda)X_1} & & & & & e^{\nu(\lambda)X_0} \\
e^{\nu(\lambda)X_1} & e^{-\nu(\lambda)X_2} \\
& e^{\nu(\lambda)X_2} & e^{-\nu(\lambda)X_3} \\
 & & \ddots & &&   \\
& & & & e^{\nu(\lambda)X_{n-1}} & e^{\nu(\lambda)X_0}
\end{pmatrix},
\end{align*}
and $c = (c_1, \dots, c_{n-1}, c_0)^T$. The 

For the leading order terms involving $d$, the term involving $\tilde{M}^c$ is $-\lambda^2 \tilde{M}^c I$. The other terms involving $d$ are given by $-\lambda A_1 d$, where $d = (d_1, \dots, d_{n-1}, d_0)^T$ and 
\begin{align*}
A_1 &= \begin{pmatrix}
q(X_0) - q(X_1) & q(X_1) &&& -q(X_0) \\
-q(X_1) & q(X_1) - q(X_2) & q(X_2) \\
& -q(X_2) & q(X_2) - q(X_3) & q(X_3) \\ && \ddots \\
\\
q(X_0) &&& -q(X_{n-1}) & q(X_{n-1}) - q(X_0) 
\end{pmatrix},
\end{align*}
where $q(x)$ is the first component of the primary pulse $Q(x)$.

Let $C_1$ and $D_1$ be the matrices we get from the remainder terms involving $c$ and $d$. Since $e^{-\alpha X^*} = C r^{1/2}$, $C_1$ and $D_1$ have the bounds in \cref{centerjumprem}. Similarly, using \cref{lemma:expnubound} to bound the terms of the form $e^{\pm \nu(\lambda)X_j}$, $K_2(\lambda)$ has the bound in \cref{centerjumprem}, which is uniform in $\lambda$. Combining everything, the center jump condition can be written in matrix form as \cref{matrixjumpc}.
\end{proof}
\end{lemma}

Finally, we compute the jump in the direction of $\Psi(0)$.

% lemma : jump in decaying adjoint direction
\begin{lemma}\label{jumpadj}
The jumps in the direction of $\Psi(0)$ are given by
\begin{equation}\label{jumpPsi0}
\begin{aligned}
\xi_i = \langle \Psi&(X_i), Q'(-X_i) \rangle (d_{i+1} - d_i ) + \langle \Psi(-X_{i-1}), Q'(X_{i-1}) \rangle (d_i - d_{i-1} ) \\
&-\frac{1}{2}\lambda M^c\left( e^{-\nu(\lambda)X_i}c_i + e^{\nu(\lambda)X_i}c_{i-1}\right)
- \lambda^2 d_i M + R_i(\lambda)(c, d)
\end{aligned}
\end{equation}
$M$ is the higher order Melnikov integral
\begin{equation}\label{defM2}
M = \int_{-\infty}^\infty \langle \Psi(y), B H(y) \rangle dy
\end{equation}
and $M^c$ is the center Melnikov integral defined in \cref{lemma:VderivIPs}. The remainder term $R(\lambda)(c, d)$ is analytic in $\lambda$, linear in $(c, d)$, and has bound given by \cref{noncenterR} in the proof below. The jump conditions can be written in matrix form as
\begin{equation}\label{Psimatrix}
\left(-\frac{1}{2} \lambda M^c K_1(\lambda) + C_2 \right)c + (A - \lambda^2 M I + D_2)d = 0.
\end{equation}
The matrix $A$ is given by
\begin{align*}
A &= \begin{pmatrix}
-a_0 -a_1 & a_0 + a_1  \\
-a_0 + a_1 & -a_0 - a_1
\end{pmatrix} && n = 2 \\
A &= \begin{pmatrix}
-a_{n-1} - a_0 & a_0 & & & \dots & a_{n-1}\\
a_0 & -a_0 - a_1 &  a_1   \\
& a_1 & -a_1 - a_2 &  a_2 \\
& & \vdots & & \vdots \\
a_{n-1} & & & & a_{n-2} & -a_{n-2} - a_{n-1} \\
\end{pmatrix} && n > 2
\end{align*}
with
\begin{align*}
a_i &= \langle \Psi(X_i), Q'(-X_i) \rangle \\
\end{align*}
The matrix $K_1(\lambda)$ is given by
\begin{align*}
K_1(\lambda) =  
\begin{pmatrix}
e^{-\nu(\lambda)X_1} & & & & & e^{\nu(\lambda)X_0} \\
e^{\nu(\lambda)X_1} & e^{-\nu(\lambda)X_2} \\
& e^{\nu(\lambda)X_2} & e^{-\nu(\lambda)X_3} \\
 & & \ddots & &&   \\
& & & & e^{\nu(\lambda)X_{n-1}} & e^{\nu(\lambda)X_0}
\end{pmatrix}
\end{align*}
The remainder matrices have uniform bounds
\begin{align}\label{adjjumprem}
|C_2| &\leq C (|\lambda| + r^{1/2})^2 \\
|D_2| &\leq C (|\lambda| + r^{1/2})^3 
\end{align}

\begin{proof}
We want to solve the decaying adjoint jump condition, which we recall is given by
\[
\xi_i = 
\langle \Psi(0), P_+(0; \lambda) Z_i^+(0) - P_-(0; \lambda) Z_i^-(0) + c_i e^{-\nu(\lambda)X_i}V^+(0; \lambda) - c_{i-1} e^{\nu(\lambda)X_i} V^-(0; \lambda) \rangle 
\]
From \cref{lemma:VpmPsiIP},
\begin{align}\label{PsiVpmjump}
\langle \Psi^c(0), c_i e^{-\nu(\lambda)X_i}V^+(0; \lambda) - c_{i-1} e^{\nu(\lambda)X_i} V^-(0; \lambda)  \rangle &= -\frac{1}{2}\lambda M^c\left( e^{-\nu(\lambda)X_i}c_i + e^{\nu(\lambda)X_i}c_{i-1}\right) + \mathcal{O}(|\lambda|^2 |c| ),
\end{align}
where we used Lemma \ref{lemma:expnubound} to bound $e^{\pm\nu(\lambda)X_i}$ in the remainder term by a constant. The terms $P_\pm(0; \lambda) Z_i^\pm(0)$ are given by
\begin{align*}
P_-&(0; \lambda) Z_i^-(0) = P_-(0; 0) b_i^- - P_-(0; \lambda) e^{\nu(\lambda) X_{i-1}} a^c_{i-1}V_0(\lambda) \\
&+ P_-(0; \lambda) \Phi^s(0, -X_{i-1}; \lambda) a_{i-1}^- + (P_-(0; \lambda) - P_-(0; 0))b_i^- + P_-(0; \lambda)(P_0^u(\lambda) - P_0^u(0))b_i^- \\
&+ P_-(0; \lambda) \int_{-X_{i-1}}^0 \Phi^s(0, y; \lambda) P_-(y; \lambda)^{-1} \big[ G_i^-(y) P_-(y; \lambda) Z_i^-(y) \\
&\qquad+ c_{i-1} e^{\nu(\lambda)X_{i-1}} G_i^-(y) V^-(y; \lambda) + \lambda^2 d_i \tilde{H}_i^-(y)\big]dy \\
&+ P_-(0; \lambda) \int_{-X_{i-1}}^0 \Phi^c(0, y; \lambda) P_-(y; \lambda)^{-1} \big[ G_i^-(y) P_-(y; \lambda) Z_i^-(y) \\
&\qquad+ c_{i-1} e^{\nu(\lambda)X_{i-1}} G_i^-(y) V^-(y; \lambda) + \lambda^2 d_i \tilde{H}_i^-(y)\big] dy
\end{align*}
and
\begin{align*}
P_+&(0; \lambda) Z_i^+(0) = P_+(0; 0) b_i^+ + P_+(0; \lambda) e^{-\nu(\lambda)X_i} a^c_i V_0(\lambda) \\
&+ P_+(0; \lambda) \Phi^u(0, X_i; \lambda) a_i^+ + (P_+(0; \lambda) - P_+(0; 0)) b_i^+ + P_+(0; \lambda) (P_0^s(\lambda) - P_0^s(0)) b_i^+ \\
&+ P_+(0; \lambda) \int_{X_i}^0 \Phi^u(0, y; \lambda) P_+(y; \lambda)^{-1} \big[ G_i^+(y) P_+(y; \lambda) Z_i^+(y) \\
&\qquad+ c_i e^{-\nu(\lambda)X_i} G_i^-(y) V^+(y; \lambda) + \lambda^2 d_i \tilde{H}_i^+(y)\big] dy \\
&+ P_+(0; \lambda) \int_{X_i}^0 \Phi^c(0, y; \lambda) P_+(y; \lambda)^{-1} \big[ G_i^+(y) P_+(y; \lambda) Z_i^+(y) \\
&\qquad+ c_i e^{-\nu(\lambda)X_i} G_i^-(y) V^+(y; \lambda) + \lambda^2 d_i \tilde{H}_i^+(y)\big] dy \\
\end{align*}
As in Lemma \ref{jumpcenteradj}, we begin by computing the leading order terms.

\begin{enumerate}
\item The non-center integral will give us the higher order Melnikov integral. For the piece on $\R^-$ we use the uniform estimate $\tilde{H}_i^-(y) = H(y) + \mathcal{O}(e^{-\alpha X^*})$ from Lemma \ref{stabestimateslemma} to get
\begin{align*}
&\langle \Psi(0), P_-(0; \lambda) \int_{-X_{i-1}}^0 \Phi^s(0, y; \lambda) P_-(y; \lambda)^{-1} \tilde{H}_i^-(y) dy \rangle \\
&= \int_{-X_{i-1}}^0 \langle \Psi(0), P_-(0; 0), \Phi^s(0, y; 0) P_-(y; 0)^{-1} H(y) \rangle dy + \mathcal{O}(|\lambda| + {e^{-\alpha X^*}}) \\
&= \int_{-X_{i-1}}^0 \langle \Psi(0), \tilde{\Phi}^s(0, y; 0) H(y) \rangle dy + \mathcal{O}(|\lambda| + {e^{-\alpha X^*}}) \\
&= \int_{-X_{i-1}}^0 \langle \tilde{P}_-^s(0)^*\Psi(0), \tilde{\Phi}(0, y; 0) H(y) \rangle dy + \mathcal{O}(|\lambda| + {e^{-\alpha X^*}})
\end{align*}
By \cref{lemma:trichadjoint}, $\tilde{P}_-^s(0)^*\Psi(0) = \Psi(0)$, thus this becomes
\begin{align*}
&\langle \Psi(0), P_-(0; \lambda) \int_{-X_{i-1}}^0 \Phi^s(0, y; \lambda) P_-(y; \lambda)^{-1} \tilde{H}_i^-(y) dy \rangle \\
&= \int_{-X_{i-1}}^0 \langle \tilde{\Phi}(0, y; 0) ^* \Psi(0), H(y) \rangle dy + \mathcal{O}(|\lambda| + {e^{-\alpha X^*}}) \\
&= \int_{-X_{i-1}}^0 \langle \Psi(y), H(y) \rangle dy + \mathcal{O}(|\lambda| + {e^{-\alpha X^*}})
\end{align*}
Since 
\[
\left| \int_{-\infty}^{-X_{i-1}} \langle \Psi(y), H(y) \rangle dy \right| \leq C e^{-\alpha X_{i-1}},
\]
\begin{align*}
&\langle \Psi(0), P_-(0; \lambda) \int_{-X_{i-1}}^0 \Phi^s(0, y; \lambda) P_-(y; \lambda)^{-1} \tilde{H}_i^-(y) dy \rangle \\
&= \int_{-\infty}^0 \langle \Psi(y), H(y) \rangle dy + \mathcal{O}(|\lambda| + {e^{-\alpha X^*}})
\end{align*}
The piece on $\R^+$ is similar, and gives us the other half of the Melnikov integral.

\item For the terms involving $a_i^\pm$, we use the expressions from Lemma \ref{Zinv2}. For the term involving $a_{i-1}^-$, we have
\begin{align*}
\langle \Psi(0), &P_-(0; \lambda) \Phi^s(0, -X_{i-1}; \lambda) a_{i-1}^- \rangle \\
&= -\langle \Psi(0), P_-(0; \lambda) \Phi^s(0, -X_{i-1}; \lambda) P_-(-X_{i-1}; \lambda)^{-1} \left( P_0^s(\lambda) D_{i-1} d + A_4(\lambda)_{i-1}^-(c, d) \right) \rangle \\
&= -\langle \Psi(0), \tilde{\Phi}^s(0, -X_{i-1}; 0)P_0^s(0) D_{i-1} d \rangle 
+\mathcal{O}\left( |\lambda|e^{-\alpha X_{i-1}}(e^{-\alpha X_{i-1}}|d| + |A_4(\lambda)_{i-1}^-(c, d)|) \right) \\
&= -\langle \Psi(-X_{i-1}), P_0^s(0) D_{i-1} d \rangle +\mathcal{O}\left( |\lambda|e^{-\alpha X_{i-1}}(e^{-\alpha X_{i-1}}|d| + |A_4(\lambda)_{i-1}^-(c, d)|) \right)
\end{align*}
where in the last line we used \cref{lemma:trichadjoint}. For the leading order term, we substitute the expression for $D_{i-1}d$ from Lemma \ref{stabestimateslemma} to get 
\begin{align*}
\langle \Psi(-&X_{i-1}), P_0^s(0) D_{i-1} d \rangle \\
&= \langle \Psi(-X_{i-1}), P_0^s(0)( Q'(-X_{i-1}) + Q'(X_{i-1})) \rangle (d_i - d_{i-1} ) + \mathcal{O}(e^{-2 \alpha X^*}(|\lambda| + e^{-\alpha X^*})|d|) \\
\end{align*}
Since
\begin{align*}
\langle \Psi(-X_{i-1}), P_0^s(0) Q'(X_{i-1})\rangle
&= \langle \Psi(-X_{i-1}), \tilde{P}^s(X_{i-1}) Q'(X_{i-1})\rangle + \mathcal{O}(e^{-3 \alpha X_{i-1}}) \\
&= \langle \Psi(-X_{i-1}), Q'(X_{i-1})\rangle + \mathcal{O}(e^{-3\alpha X^*})
\end{align*}
and
\begin{align*}
\langle \Psi(-X_{i-1}), P_0^s(0) Q'(-X_{i-1})\rangle
&= \langle \Psi(-X_{i-1}), \tilde{P}^s(-X_{i-1}) Q'(-X_{i-1})\rangle + \mathcal{O}(e^{-3 \alpha X_{i-1}}) \\
&= \langle \Psi(-X_{i-1}), Q'(-X_{i-1})\rangle + \mathcal{O}(e^{-3 \alpha X_{i-1}}) \\
&= \mathcal{O}(e^{-3 \alpha X^*})
\end{align*}
the leading order term becomes
\begin{align*}
\langle &\Psi(-X_{i-1}), P_0^s(0) D_i d \rangle = \langle \Psi(-X_{i-1}), Q'(X_{i-1}) \rangle (d_i - d_{i-1} ) + \mathcal{O}(e^{-2 \alpha X^*}(|\lambda| + e^{-\alpha X^*})|d|) \\
\end{align*}
Combining all of these, collecting remainder terms, and using the bound for $A_4$ from Lemma \ref{Zinv2}, we have for the $a_{i-1}^-$ term,
\begin{align*}
\langle \Psi(0), &P_-(0; \lambda) \Phi^s(0, -X_{i-1}; \lambda) a_{i-1}^- \rangle = -\langle \Psi(-X_{i-1}), Q'(X_{i-1}) \rangle (d_i - d_{i-1} ) \\
&+ \mathcal{O} \left( |\lambda|e^{-\alpha X^*}(|\lambda| + e^{-\alpha X^*})|c| + e^{-2 \alpha X^*} (|\lambda| + e^{-\alpha X^*}) |d| ) \right)
\end{align*}
Similarly, for the $a_i^+$ term, we have
\begin{align*}
\langle \Psi(0), &P_+(0; \lambda) \Phi^u(0, X_i; \lambda) a_i^+ \rangle = \langle \Psi(X_i), Q'(-X_i) \rangle (d_{i+1} - d_i ) \\
&+ \mathcal{O} \left( |\lambda|e^{-\alpha X^*}(|\lambda| + e^{-\alpha X^*})|c| + e^{-2 \alpha X^*} (|\lambda| + e^{-\alpha X^*}) |d| ) \right)
\end{align*}

\end{enumerate}

The remaining terms will be higher order. As in the previous lemma, we will bound these in turn.

\begin{enumerate}

\item For the terms involving $a_i^c$, using the expression from Lemma \ref{Zinv2} for $a_i^c$ together with \cref{lemma:VpmPsiIP}, and using Lemma \ref{lemma:expnubound} to bound terms of the form $e^{\pm \nu(\lambda)X_i}$ by a constant, we have
\begin{align*}
|\langle \Psi(0), P_+(0; \lambda) e^{-\nu(\lambda)X_i} a_i^c \rangle| &\leq C |\lambda|^2 (|\lambda| + e^{-\alpha X^*}) \left( |c| + |d| \right)
\end{align*}
The other term is similar.

\item For the terms involving $b$, we first note that the terms $P_-(0) b_i^-$ and $P_+(0)b_i^+$ will vanish when we take the inner product with $\Psi(0)$. For the remaining terms, we substitute the estimate for $B_1$ from Lemma \ref{Zinv2}.
\begin{align*}
&|\langle \Psi(0), (P_-(0; \lambda) - P_-(0; 0))b_i^- + P_-(0; \lambda)(P_0^u(\lambda) - P_0^u(0))b_i^- \rangle | \\
&\leq C |\lambda| \Big( (|\lambda| + e^{-\alpha X^*})|c| + (|\lambda| + e^{-\alpha X^*})^2 |d| \Big)
\end{align*}

\item For the center integral terms, as in Lemma \ref{jumpcenteradj}, the integrands are both zero when $\lambda = 0$. Thus these integral terms have the same bounds as in the previous lemma. For the center integral terms involving $Z$, we use the bound $Z_3$ for $Z$ to get
\begin{align*}
&\left| \langle \Psi(0), P_+(0; \lambda) \int_{X_i}^0 \Phi^c(0, y; \lambda) P_+(y; \lambda)^{-1} G_i^+(y) P_+(y; \lambda) Z_i^+(y) dy \rangle \right| \\
&\leq C |\lambda| e^{-\alpha X^*}\left(|c| + |D_i||d| + |\lambda|^2|d|\right)
\end{align*}
The other integral is similar. For the center integrals involving $\tilde{H}_i^\pm$, as in the previous lemma we have
\begin{align*}
&\left| \langle \Psi(0), P_-(0; \lambda) \lambda^2 d_i \int_{-X_{i-1}}^0 \Phi^c(0, y; \lambda) P_-(y; \lambda) \tilde{H}_i^-(y) dy \rangle \right| \\
&\leq C |\lambda|^3 |d| \int_{-X_{i-1}}^0 e^{\alpha y} e^{-\tilde{\alpha} y} dy \leq C |\lambda|^3 |d| 
\end{align*}
The other integral is similar. When we take the inner product with $\Psi(0)$, the center integral term involving $G_i^\pm(x) V^\pm(x) c_i$ is bounded by $C|\lambda|e^{-\alpha X^*}|c|$. 

\item Finally, we evaluate the noncenter integral involving $Z$. As in \cite[p. 448]{Sandstede1998}, since $DF(0)$ is hyperbolic, we can increase the decay constant $\alpha$ to $\alpha_0$ for $\Psi(x)$ to get the decay rate
\[
|\Psi(x)| \leq C e^{-\alpha_0 |x|}
\]
The price to pay is that the constant $C$ is larger, which does not matter. Using this, the integral then becomes
\begin{align*}
&\left| \int_{X_i}^0 \langle \Psi(y), G_i^+(y) P_+(y; \lambda) Z_i^+(y) dy \rangle \right| \\
&\qquad \leq C \| Z_i^+\|  \left( \int_0^{X_i} e^{-\alpha_0 y} e^{-\alpha X_i} e^{-\alpha(X_i - y)} dy + \int_0^{X_i} e^{-2 \alpha X_{i-1}} e^{-\alpha y} \right) dy \\
&\qquad =C \| Z_i^+\|  \left( e^{-2 \alpha X_i} \int_0^{X_i} e^{-(\alpha_0 - \alpha) y} dy + \int_0^{X_i} e^{-2 \alpha X_{i-1}} e^{-\alpha y} \right) dy \\
&\qquad \leq C \| Z_i^+\| e^{-2 \alpha X^*}
\end{align*}
Using the bound $Z_3$, this becomes
\begin{align*}
\left| \int_{X_i}^0 \langle \Psi(y), G_i^+(y) P_+(y; \lambda) Z_i^+(y) dy \rangle \right|\leq C \left( e^{-2 \alpha X^* } |c| + e^{-\alpha X^* }(e^{-\alpha X^*} + |\lambda|)^2 |d| \right)
\end{align*}
The other integral is similar. The bound on the noncenter integral term involving $G_i^\pm(x) V^\pm(x) c_i$ is the same as the corresponding center integral term.
\end{enumerate}

Putting this all together, we have the jump expressions
\begin{align*}
\xi_i = \langle \Psi&(X_i), Q'(-X_i) \rangle (d_{i+1} - d_i ) + \langle \Psi(-X_{i-1}), Q'(X_{i-1}) \rangle (d_i - d_{i-1} ) \\
&-\frac{1}{2}\lambda M^c\left( e^{-\nu(\lambda)X_i}c_i + e^{\nu(\lambda)X_i}c_{i-1}\right)
- \lambda^2 d_i M + R_i(\lambda)(c, d)
\end{align*}
where $M$ is the higher order Melnikov integral
\[
M = \int_{-\infty}^\infty \langle \Psi(y), H(y) \rangle dy,
\]
$M^c$ is the center Melnikov integral defined in \cref{lemma:VderivIPs}, and the remainder term bound
\begin{equation}\label{noncenterR}
|R(\lambda)(c, d)| \leq C \left( (e^{-\alpha X^*} + |\lambda|)^2 |c| + (e^{-\alpha X^*} + |\lambda|)^3 |d| \right)
\end{equation}

As in Lemma \ref{jumpcenteradj}, we will write these jump expressions in matrix form. The leading order terms involving $c$ are given by the matrix $-\frac{1}{2}\lambda M^c K_1(\lambda)c$, where $K_1(\lambda)$ is the periodic, bi-diagonal matrix
\begin{align*}
K_1(\lambda) =  
\begin{pmatrix}
e^{-\nu(\lambda)X_1} & & & & & e^{\nu(\lambda)X_0} \\
e^{\nu(\lambda)X_1} & e^{-\nu(\lambda)X_2} \\
& e^{\nu(\lambda)X_2} & e^{-\nu(\lambda)X_3} \\
 & & \ddots & &&   \\
& & & & e^{\nu(\lambda)X_{n-1}} & e^{\nu(\lambda)X_0}
\end{pmatrix}
\end{align*}
We note that $K_1(\lambda)$ is the same matrix as $K(\lambda)$ in the previous lemma with all terms positive.

For the terms involving $d$. Let
\[
a_i = \langle \Psi(X_i), Q'(-X_i) \rangle 
\]
By reversibility, $Q(-x) = R Q(x)$ implies $Q'(-x) = -R Q'(x)$. From Lemma \ref{varadjsolutions}, $\Psi(x) = R \Psi(-x)$. Thus, since $R^2 = I$ and $R$ is self-adjoint, we have
\begin{align*}
\langle \Psi(-X_i), Q'(X_i) \rangle &= \langle \Psi(-X_i), R^2 Q'(X_i) \rangle \\
&= -\langle R \Psi(-X_i), Q'(-X_i) \rangle \\
&= -\langle \Psi(X_i), Q'(-X_i) \rangle \\
&= -a_i
\end{align*}
Thus the leading order terms involving $d$ are given in matrix form by $(A - \lambda^2 M I + D_2)d$, where the $A$ is given by
\begin{align*}
A &= \begin{pmatrix}
-a_{n-1} - a_0 & a_0 & & & \dots & a_{n-1}\\
a_0 & -a_0 - a_1 &  a_1 \\
& a_1 & -a_1 - a_2 &  a_2 \\
& & \vdots & & \vdots \\
a_{n-1} & & & & a_{n-2} & -a_{n-2} - a_{n-1} \\
\end{pmatrix}
\end{align*}
Let $C_2$ and $D_2$ be the matrices we get from the remainder terms involving $c$ and $d$. Combining everything above gives us the matrix form \cref{Psimatrix}. Since $e^{-\alpha X^*} = C r^{1/2}$, $C_2$ and $D_2$ have the bounds in \cref{adjjumprem}.
\end{proof}
\end{lemma}

\section{Proof of Theorem \ref{blockmatrixtheorem}}

Theorem \ref{blockmatrixtheorem} combines the jump matrix formulas from Lemma \ref{jumpcenteradj} and Lemma \ref{jumpadj} into a single block matrix. Using the definitions of $H(y)$ and $B$ and equation \cref{psicomponents} for $\Psi(y)$, we can compute the higher order Melnikov integral.
\begin{align*}
M = \int_{-\infty}^\infty \langle \Psi(y), H(y) \rangle dy = \int_{-\infty}^\infty q(y) \partial_c q(y) dy
\end{align*}

\section{Characterization of matrix $A$}

Before we conclude this section, we will prove some results based on \cref{blockmatrixtheorem}. In this section, we will characterize the matrix $A$ in the lower left block of the block matrix \cref{blockeq}. 

\begin{lemma}\label{lemma:ajparam}
For $r \leq r_*$, the terms $a_j$ in the block matrix equation \cref{blockeq} have the form
\[
a_j = r \tilde{a}^j(r),
\]
where
\begin{equation}\label{ajparam}
\tilde{a}_j(r) = -p_0 e^{\alpha \phi/\beta_0} b_j \left( \beta \cos\left(-\rho \log b_j \right) - \alpha \sin \left(-\rho \log b_j \right) \right) + \mathcal{O}(r^{\gamma/2\alpha}),
\end{equation}
$p_0 > 0$, and
\begin{equation}\label{bj2}
b_j = e^{-2 \alpha (X_j - X^*)}
\end{equation}

\begin{proof}
Using \cite[Lemma 6.1]{Sandstede1998} and making the same modifications as we did in Lemma \ref{IPform},
\begin{align*}\label{IPpsiQprime}
\langle \Psi(-x), Q'(x) \rangle
&= p_0 e^{-2 \alpha x}\left( \beta_0 \cos(2 \beta_0 x + \phi) - \alpha \sin(2 \beta_0 x + \phi)\right) + \mathcal{O}(e^{-(2 \alpha + \gamma) x}
\end{align*}
where $p_0 > 0$ and $\gamma$ are the same as in Lemma \ref{IPform}. From the proof of Lemma \ref{jumpadj}, $\langle \Psi(X_j), Q'(-X_j) \rangle = - \langle \Psi(-X_j), Q'(X_j) \rangle$. Using equation \cref{Xjscale} for $X_j$, we have
\begin{align*}
a_j(r) &= \langle \Psi(X_j), Q'(-X_j) \rangle = -\Psi(-X_j), Q'(X_j) \rangle \\
&= -s_0 e^{-2 \alpha X_j}\left( \beta_0 \cos(2 \beta_0 X_j + \phi) - \alpha \sin(2 \beta_0 X_j + \phi)\right) + \mathcal{O}(e^{-(2 \alpha + \gamma_0) X_j}) \\
&= -s_0 e^{\alpha \phi/\beta_0} r b_j \left( \beta_0 \cos\left( -\frac{\beta_0}{\alpha} \log(b_j r) \right) - \alpha \sin \left( -\frac{\beta_0}{\alpha} \log(b_j r) \right) \right) + \mathcal{O}(r^{1+\gamma/2\alpha} b_j^{1 + \gamma/2\alpha}) \\
&= -s_0 e^{\alpha \phi/\beta_0} r b_j \left( \beta_0 \cos\left( -\rho \log(b_j r) \right) - \alpha \sin \left( -\rho \log(b_j r) \right) \right) + \mathcal{O}(r^{1+\gamma/2\alpha})
\end{align*}
since $b_j \in (0, 1]$. Since $r \in \mathcal{R}$, $r = \exp\left(-\frac{2 m \pi}{\rho}\right)$ for some nonnegative integer $m$, this becomes 
\begin{align*}
a_j(r) &= -s_0 e^{\alpha \phi/\beta_0} r b_j \left( \beta_0 \cos\left( 2 m \pi -\rho \log b_j \right) - \alpha \sin \left( 2 m \pi -\rho \log b_j \right) \right) + \mathcal{O}(r^{1+\gamma/2\alpha}) \\
&= -s_0 e^{\alpha \phi/\beta_0} r b_j \left( \beta_0 \cos\left(-\rho \log b_j \right) - \alpha \sin \left(-\rho \log b_j \right) \right) + \mathcal{O}(r^{1+\gamma/2\alpha})
\end{align*}
The result follows from letting $a_j(r) = r \tilde{a}_j(r)$.
\end{proof}
\end{lemma}

We can now characterize the matrix $A$.

\begin{lemma}\label{lemma:tildeA}
For $r \leq r_*$, the matrix $A$ in \cref{blockeq} has the form
\begin{equation}\label{deftildeA}
A = r \tilde{A}(r)
\end{equation}
where $\tilde{A}(r) = \tilde{A}(r; m_0, \dots, m_{n-1}, \theta)$ is symmetric and continuous in $r$. $\tilde{A}(r)$ has an eigenvalue of 0 corresponding to eigenvector $(1, 1, \dots, 1)^T$. The remaining eigenvalues $\ell_1(r), \dots, \ell_{n-1}(r) \}$ of $\tilde{A}(r)$ are real and are continuous in $r$.
\begin{proof}
By \cref{lemma:ajparam}, we can scale out $r$ from each of the terms $a_j = r \tilde{a}_j(r)$ in \cref{blockeq} to obtain \cref{deftildeA}. Since all of the $\tilde{a}_j(r)$ are continuous in $r$, $\tilde{A}$ is as well. Since $A$ is symmetric, $\tilde{A}(r)$ is also symmetric, thus its eigenvalues are real. The kernel eigenvector can be verified directly since all rows of $\tilde{A}$ sum to 0.
\end{proof}
\end{lemma}

\section{Center eigenfunction}

Recall that when $\lambda = 0$, equation \cref{PDEeigsystemper2} has a solution $\partial_x Q_n(x)$. 
In this section, we will show that under certain conditions, there is another solution to equation \cref{PDEeigsystemper2} when $\lambda = 0$.

\begin{lemma}\label{lemma:centereigenfn1}
If the kernel of the operator $\calL(q_n)$ on $H^{2m}_{\text{per}}[-X,X]$ is simple, there exists a function $V_n^c(x)$ which satisfies
\begin{align*}
[V_n^c]'(x) &= A(Q_n(x))V_n^c(x) \\
V_n^c(-X) &= V_n^c(X)
\end{align*}
\begin{proof}
Consider the equation
\begin{equation}\label{Lqn1}
\calL(q_n) v = 1
\end{equation}
posed on $H^{2m}_{\text{per}}[-X,X]$. Since both $q_n$ and the constant function $1$ are in $H^{2m}_{\text{per}}[-X,X]$, this is well-defined. The operator $\calL(q_n)$ is self-adjoint, and $\partial_x q_n \in \ker \calL(q_n)$. Integrating by parts and noting that the boundary term cancels due to the periodic boundary conditions,
\[
\langle \partial_x q_n(x), 1 \rangle_{H^{2m}_{\text{per}}[-X,X]} = 0
\]
Assuming that the kernel of $\calL(q_n)$ is simple, it follows from follows from the Fredholm alternative that \cref{Lqn1} has a solution $v(x) \in {H^{2m}_{\text{per}}[-X,X]}$. Let
\[
V_n^c = \left( v(x), \partial_x v(x), \dots, \partial_x^{2m-1}(x), 1 \right)
\]
Then $V_n^c(-X) = V_n^c(X)$, and from the definition of $A(Q_n(x))$, $[V_n^c]'(x) = A(Q_n(x))V_n^c(x)$.
\end{proof}
\end{lemma}

For the conclusion of \cref{lemma:centereigenfn1} to hold, we need a condition for when $\calL{q_n}$ has a simple kernel. Using a spatial dynamics formulation, the eigenvalue problem $\calL{q_n}v = \lambda v$ is equivalent to the first order system of ODEs 
\begin{equation}\label{Lsystem1}
\begin{aligned}
V'(x) &= DF(Q_n(x))V(x) + \lambda B V(x) \\
V(-X) &= V(X)
\end{aligned}
\end{equation}
where $V(x) \in C^0(\R, R^{2m})$. When $\lambda = 0$, $\partial_x Q(x)$ is a solution. In the next lemma, we give a condition for which this is the only solution.

\begin{lemma}\label{Lsystemeigs}
If the eigenvalues of $\tilde{A}(0)$ are distinct, then for sufficiently small $r$, the only solution to \cref{Lsystem1} when $\lambda = 0$ is $\partial_x Q(x)$.
\begin{proof}
Since the matrix $DF(0)$ is hyperbolic, solutions to \cref{Lsystem1} can be found using the method in \cite{Sandstede1998} adapted to periodic boundary conditions. Briefly, this works as follows. The setup is exactly the same as in \cite{Sandstede1998}, except periodic boundary conditions require one more matching condition at the tails of the pulses to ``close the loop''. Following the proof of \cite[Theorem 2]{Sandstede1998}, an eigenfunction $V(x)$ exists for sufficiently small $\lambda$ if and only if the $n$ jump conditions
\begin{align*}
\xi_i = \langle \Psi(X_i), &\partial_x Q(-X_i) \rangle (d_{i+1} - d_i) + \Psi(-X_{i-1}), \partial_x Q(X_{i-1}) \rangle (d_i - d_{i-1}) \\
&- d_i \int_{-\infty}^\infty \langle \Psi(y) B \partial_x Q(y) \rangle dy + R(\lambda)_i d = 0
\end{align*}
are satisfied, where $\tilde{\Psi}(x) = \nabla H(Q(x))$ is the unique bounded solution to the adjoint variational equation \cref{adjvareq1} and the remainder term has bound
\[
|R(\lambda)d| \leq C \left( e^{-3 \alpha X^*} + |\lambda|(e^{-\alpha X^*} + |\lambda| ) \right)
\]
This is analogous to \cite[(3.56)]{Sandstede1998}. By Remark \ref{remark:psinotation}, the inner product $\langle \Psi(x), \partial_x Q(y) \rangle$ are the same for the existence and the stability problem, therefore the jump conditions can be written in matrix form as in \cite[Theorem 2]{Sandstede1998} as
\begin{equation}\label{Ljumps}
E(\lambda) = \det S(\lambda) = \det( A - M_1 \lambda I + R(\lambda)) = 0
\end{equation}
where $A$ the same matrix as in \cref{blockmatrixtheorem} and
\[
M_1 = 
\int_{-\infty}^\infty \langle \tilde{\Psi}(y) B \partial_x Q(y) \rangle dy =  
\int_{-\infty}^\infty [\partial_y q(y)]^2 dy > 0
\]
By \cref{lemma:tildeA}, $A = r \tilde{A}(r)$, where $\tilde{A}(r)$ is continuous in $r$ and has an eigenvalue at 0 for all $r$. If the eigenvalues of $\tilde{A}(0)$ are distinct, $\tilde{A}(0)$ has a simple eigenvalue at 0. Following the proof of \cite[Theorem 3]{Sandstede1998}, for sufficiently small $r$, equation \cref{Ljumps} has a simple zero at 0. Since this zero must correspond to $\partial_x Q_n(x)$, the result follows.
\end{proof}
\end{lemma}

We combine the two previous lemmas to get the following result.

\begin{lemma}\label{lemma:centereigenfn}
If the eigenvalues of $\tilde{A}(0)$ are distinct, then for sufficiently small $r$ there exists a function $V_n^c(x)$ which satisfies \cref{PDEeigsystemper3} for $\lambda = 0$.
\end{lemma}

\iffulldocument\else
	\bibliographystyle{amsalpha}
	\bibliography{thesis.bib}
\fi

\end{document}
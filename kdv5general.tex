\documentclass[thesis.tex]{subfiles}

\begin{document}

\iffulldocument\else
	\chapter{Generalization}
\fi

In this chapter, we write our problem in a general form, for which KdV5 will be a special case. First, we formulate the problem as a Hamiltonian PDE with certain symmetries. We then look for traveling wave solutions, which satisfy a related nonlinear in a moving reference frame. In particular, we are interested in pulses, which are localized traveling waves. To do this, we shift to a spatial dynamics perspective, in which pulses are homoclinic orbits. We discuss the existence of a primary pulse solution, which is a single, symmetric homoclinic orbit. We will use this framework to  consider the existence of multi-pulses in \cref{chapter:kdv5homoclinic} and periodic multi-pulses in \cref{chapter:kdv5periodic}. Finally, we set up the spectral problem, which we will treat separately for the $n-$homoclinic case in \cref{chapter:kdv5periodic} and the $n-$periodic case in \cref{chapter:kdv5perspec}. All proofs will be deferred until the end of this chapter.

\section{Hamiltonian PDE}\label{sec:HamPDE}

First, we set up a Hamiltonian PDE which is reversible and translation invariant. This analysis follows \cite{Grillakis1987}. Let $X = H^{2m}(\R)$ and $Y = L^2(\R)$. Consider the PDE
\begin{equation}\label{genPDE}
u_t = \partial_x \calE'(u)
\end{equation}
where $u \in X$ and $\calE(u): X \subset Y \rightarrow \R$ is a smooth functional representing the energy of the system. We take the following hypothesis regarding the energy $\calE(u)$.

\begin{hypothesis}\label{Ehyp}
The energy $\calE(u)$ has the following properties
\begin{enumerate}[(i)]
\item $\calE(0) = 0$ and $\calE'(0) = 0$.
\item $\calE(u)$ is reversible, i.e. $\calE(u) = \calE(\rho(u))$, where $\rho$ is the operator on $X$ defined by $[\rho(u)](x) = u(-x)$.
\item $\calE(u)$ is translation invariant, i.e. $\calE(T(s)u) = \calE(u)$ for all $s \in \R$, where $\{T(s) : s \in \R \}$ is the one parameter group of unitary operators on $X$ defined by $[T(s)]u(\cdot) = u(\cdot - s)$.
\item $\calE'(u): X \rightarrow X$ is a differential operator of the form
\begin{equation}\label{Eprimeuform}
\calE'(u) = \partial_x^{2m}u - f(u, \partial_x u, \dots, \partial_x^{2m-1} u),
\end{equation}
where $f: \R^{2m} \rightarrow \R$ is smooth.
\end{enumerate}
\end{hypothesis}

\noi\cref{Ehyp}(iv) holds in applications such as KdV5, and ensures that we can write the equilibrium equation $\partial_x \calE'(u) = 0$ as a first order system.

Differentiating the reversibility relation $\calE(u) = \calE(\rho(u))$ with respect to $u$, we have
\[
\calE'(u) = \rho^*( \calE'(\rho(u) ) = \rho( \calE'(\rho(u) ),
\]
since $\rho$ is self-adjoint. This implies that the linear part of $\calE'(u)$ only involves even-order derivatives of $u$ with respect to $x$. Differentiating the symmetry relation $\calE(T(s)u) = \calE(u)$ with respect to $u$, we have
\begin{equation}\label{Eprimesymm}
\calE'(u) = T(s)^* \calE'(T(s)u) = T(-s) \calE'(T(s)u)
\end{equation}
and
\begin{equation}\label{EHessiansymm}
\calE''(u) = T(s)^* \calE''(T(s)u) T(s).
\end{equation}
Finally, differentiating the symmetry relation $\calE(T(s)u) = \calE(u)$ with respect to $s$ at $s = 0$, 
\begin{align*}
0 = \langle \calE'(u), T'(s) u \rangle|_{s = 0}
= \langle \calE'(u), T'(0) u \rangle
= \langle \calE'(u), \partial_x u \rangle
\end{align*}
for all $u \in X$, since $T'(0) = \partial_x$ is the infinitesimal generator of the translation group $T(s)$. The energy $E(u)$ is conserved along solutions $u$ to \cref{genPDE}, since
\[
\frac{d}{dt}\calE(u) = \langle \calE'(u), u_t \rangle = \langle \calE'(u), \partial_x \calE'(u) \rangle = 0,
\]
where we used the fact that the operator $\partial_x$ is skew-symmetric. In addition, there is a second conserved quantity $V: L^2(\R) \rightarrow \R$, given by
\begin{equation}\label{defV}
V(u) = -\frac{1}{2} \int_{-\infty}^\infty u^2 dx,
\end{equation}
which represents charge in some applications. $V(u)$ is also conserved along solutions $u$ to \cref{genPDE}, since
\begin{align*}
\frac{d}{dt}V(u) &= \langle V'(u), u_t \rangle
= \langle -u, \partial_x \calE'(u) \rangle 
= \langle \partial_x u, \calE'(u) \rangle = 0,
\end{align*}
where we used $V'(u) = -u$ in the second equality.

As in \cite{Grillakis1987}, we will study bound states of \cref{genPDE}, which are solutions of the form
\begin{equation}
u(x, t) = T(ct)\phi(x) = \phi(x - ct),
\end{equation}
where $\phi \in X$. Since $T(s)$ is the translation group, these bound states are traveling waves; we will refer to these states as traveling waves from here on. If $\phi$ satisfies the equilibrium equation
\begin{equation}\label{eqODE1}
\calE'(\phi) = c V'(\phi),
\end{equation}
then $T(ct)\phi(x)$ is a traveling wave, since 
\begin{align*}
\frac{d}{dt}T(ct)\phi &= c T'(ct)\phi 
= c T'(0)T(ct)\phi \\
&= \partial_x T(ct) c V'(\phi)
= \partial_x T(ct) \calE'(\phi) \\
&= \partial_x T(ct) T(ct)^* \calE'(T(ct)\phi) \\
&= \partial_x \calE'(T(ct)\phi) 
\end{align*}
Since $V'(\phi) = -\phi$, we can write the equilibrium equation as
\begin{equation}\label{eqODE}
\calE'(\phi) + c \phi = 0.
\end{equation}
Without loss of generality, we will assume that $\calE'(\phi)$ does not contain any terms of the form $b\phi$ for $b$ constant, since those are accounted for by the $c \phi$ term in \cref{eqODE}.

We take the following hypothesis concerning existence of traveling waves. This hypothesis is similar to \cite[Assumption 2]{Grillakis1987}. In the next section, we will give a condition under which this hypothesis is satisfied. 
\begin{hypothesis}\label{cintervalhyp}
There exists an open interval $(c_1, c_2) \subset \R$ and a $C^1$ map $c \mapsto \phi_c$ such that for every $c \in (c_1, c_2)$, $\phi_c$ is a traveling wave solution for \cref{genPDE}, i.e. 
\[
\calE'(\phi_c) + c \phi_c = 0.
\]
Furthermore, $\partial_x \phi_c \neq 0$.
\end{hypothesis}

We are interested in the stability of these traveling wave solutions. As a first step in stability analysis, we will look at the spectrum of the linearization of the PDE \cref{genPDE} about a traveling wave. For $c \in (c_1, c_2)$, let $\phi_c$ be a solution to \cref{eqODE}, which is a traveling wave solution to \cref{genPDE}. Then the linearization of the PDE \cref{genPDE} about $\phi_c$ is the linear operator $\partial_x \calL(\phi_c)$ defined by
\begin{equation}\label{PDElinearization}
\partial_x \calL(\phi_c) = 
\partial_x (\calE''(\phi_c) + c )
\end{equation}
where $\calE''(\phi_c)$ is the Hessian of the energy $\calE(\phi_c)$. Both $\calE''(\phi_c)$ and $\calL(\phi_c)$ are self-adjoint. Since $\phi_c$ is a traveling wave, using \cref{Eprimesymm} and \cref{eqODE}, we have
\begin{align}\label{TeqODE}
\calE'(T(s)\phi_c) + c T(s)\phi_c = T(s)[\calE'(\phi_c) + c \phi_c] = 0.
\end{align}
Differentiating \cref{TeqODE} with respect to $s$ at $s = 0$, this becomes
\begin{align*}
0 &= \calE''(T(0)\phi_c)T'(0)\phi_c + c T'(0)\phi_c \\
&= (\calE''(\phi_c) + c ) \partial_x \phi_c \\
&= \calL(\phi_c) \partial_x \phi_c
\end{align*}
Thus $\partial_x \phi_c$ is an eigenfunction of $\calL(\phi_c)$ with eigenvalue 0. Differentiating one more time, $\partial_x \phi_c$ is also an eigenfunction of $\partial_x \calL(\phi_c)$ with eigenvalue 0.

Differentiating \cref{eqODE} with respect to $c$, which we can do because the mapping $c \mapsto \phi_c$ in Hypothesis \ref{cintervalhyp} is $C^1$, we have
\begin{align*}
\calE''(\phi_c)\partial_c \phi_c + c \partial_c \phi_c + \phi_c = 0,
\end{align*}
which we can rearrange to get 
\begin{align*}
(\calE''(\phi_c) + c )(-\partial_c \phi_c) = \phi_c.
\end{align*}
Differentiating both sides, we have
\begin{align*}
\partial_x \calL(\phi_c) (-\partial_c \phi_c) 
= \partial_x (\calE''(\phi_c) + c )(-\partial_c \phi_c) = \partial_x\phi_c.
\end{align*}
Thus $-\partial_c \phi_c$ is a generalized eigenfunction of $\partial_x \calL(\phi_c)$ corresponding to eigenvalue 0.

\section{Localized traveling waves}\label{sec:travelingwaves}

We will look now look at traveling wave solutions to the PDE \cref{genPDE}, which are solutions to equilibrium equation \cref{eqODE}. In particular, we are interested in traveling waves which are exponentially localized. We will use a spatial dynamics approach. From this viewpoint, an exponentially localized traveling wave is a homoclinic orbit connecting a rest state to itself. To do this, we write \cref{eqODE} as a first order system of ODEs in the standard way. Let $U = (u, \partial_x u, \dots, \partial_x^{2m-1} u)^T \in \R^{2m}$. Then by \cref{Ehyp}(iv), equation \cref{eqODE} is equivalent to the first order system
\begin{equation}\label{genODE}
U'(x) = F(U(x); c),
\end{equation}
where $F: \R^{2m} \times \R \rightarrow \R^{2m}$ is given by
\begin{equation}\label{defF}
F(u_1, u_2, \dots, u_{2m}; c) = 
\begin{pmatrix}
u_2 \\ u_3 \\ \vdots \\ f(u_1, u_2, \dots, u_{2m}) - c u_1
\end{pmatrix}.
\end{equation}
$F$ is smooth, and $F(0; c) = 0$ for all $c$. Reversibility implies that
\begin{equation}\label{genODErev}
F(RU; c) = -RF(U; c),
\end{equation}
where $R:\R^{2m} \rightarrow \R^{2m}$ is the standard reversor operator defined by
\begin{equation}\label{reverserR2m}
R(u_1, u_2, \dots, u_{2m-1}, u_{2m}) = (u_1, -u_2, \dots, u_{2m-1}, -u_{2m}).
\end{equation}
In particular, if $U(x)$ is a solution to \cref{genODE}, so is $RU(-x)$. Taking the derivative of \cref{genODErev} with respect to $U$, it follows that 
\begin{equation}\label{genODErevDF}
D F(RU; c) = -RDF(U; c)R.
\end{equation}
From the last component of \eqref{genODErev}, $f(RU) = f(U)$. It also follows from reversibility that $Df(RU) = RDf(U)$, which implies
\begin{equation}\label{frev}
\begin{aligned}
\partial_{u_j} f(R U) &= (-1)^{j+1} \partial_{u_j} f(U) \\
\partial^2_{u_j u_k} f(R U) &= (-1)^{j+k} \partial^2_{u_j u_k} F(U)
\end{aligned}
\end{equation}
For $U = 0$,
\begin{align}\label{fpartials0}
\partial_{u_j} f(0) &= \begin{cases}
0 & j \text{ even}\\
0 & j = 1 \\
c_j & j \text{ odd, } j > 1
\end{cases}
\end{align}
where the $c_j$ are constants. Again, we can assume without without loss of generality that $c_0 = \partial_{u_1} f(0) = 0$, since that is accounted for by the $c u_1$ term in \cref{defF}.

Next, we assume that \cref{genODE} is a conservative system.

\begin{hypothesis}\label{Hhyp}
There exists a smooth function $H: \R^{2m} \times \R \rightarrow \R$, denoted $H(U; c)$, such that 
\begin{enumerate}[(i)]
\item $H(0; c) = 0$ for all $c$
\item $\nabla_U H(U; c) = 0$ if and only if $F(U; c) = 0$
\item For all $U \in \R^{2m}$ and all $c$,
\begin{equation}
\langle \nabla_U H(U; c), F(U; c) \rangle = 0
\end{equation}
\end{enumerate}
\end{hypothesis}

\noi It follows from \cref{Hhyp} that $H$ is conserved along solutions $U(x)$ to \cref{genODE}. If $U(x)$ is such a solution, then
\[
\frac{d}{dx}H(U(x); c) = \langle \nabla H(U(x); c), U'(x) \rangle
= \langle \nabla H(U(x); c), F(U(x); c) \rangle = 0
\]

Since $F(0; c) = 0$ for all $c$, the rest state $U = 0$ is an equilibrium of \cref{genODE} for all $c$. The next hypothesis addresses the hyperbolicity of this equilibrium. We note that although the eigenvalue pattern described in \cref{hypeqhyp} is not necessary for the existence of a homoclinic orbit, it is a sufficient condition for the existence of multi-pulse solutions.

\begin{hypothesis}\label{hypeqhyp}
For a specific $c_0 \in \R$ with $c_0 > 0$, $U = 0$ is a hyperbolic equilibrium of \cref{genODE}, i.e. $DF(0; c_0)$ has no eigenvalues with real part 0. Furthermore, the spectrum of $DF(0; c_0)$ contains a quartet of simple eigenvalues $\pm \alpha_0 \pm \beta_0 i$, $\alpha_0, \beta_0 > 0$, and for any other eigenvalue $\nu$ of $DF(0; c_0)$, $|\text{Re }\nu| > \alpha_0$.
\end{hypothesis}

Let $W^s(0; c_0)$ and $W^u(0; c_0)$ be the stable and unstable manifolds of the equilibrium at 0, and let $E_0^s(c_0)$ and $E_0^u(c_0)$ be the stable and unstable eigenspaces of $DF(0; c_0)$. By reversibility, these spaces have dimension $m$. The stable and unstable manifolds are smooth since $F$ is smooth.

\section{Existence of pulses}

\subsection{Primary pulse solution}\label{sec:primarypulse}

We now address the existence of a primary pulse solution, which is a symmetric homoclinic orbit connecting the rest state equilibrium to itself. In general, the existence of such a solution is unknown, although in specific cases such as KdV5 we do have an existence result. Thus we will take the existence of a primary pulse solution as a hypothesis.

\begin{hypothesis}\label{Qexistshyp}
For the same $c_0$ as in \cref{hypeqhyp}, there exists a homoclinic orbit solution $Q(x; c_0) \in W^s(0) \cap W^u(0) \subset H^{-1}(0)$ to \cref{genODE}. In addition,
\begin{enumerate}[(i)]
\item $Q(0; c_0) \neq 0$
\item $\nabla_U H(Q(0; c_0); c_0) \neq 0$
% \item $\partial_c H(Q(0; c_0); c_0) \neq 0$
\item $Q(x; c_0)$ is symmetric with respect to the reversor operator \cref{reverserR2m}, i.e. $Q(-x; c_0) = R Q(x; c_0)$
\end{enumerate}
\end{hypothesis}

\noi Since we obtained \cref{genODE} by putting \cref{eqODE} into a first order system in the standard way, 
\begin{equation}\label{Qqrelation}
Q(x; c_0) = (q(x; c_0), \partial_x q(x; c_0), \dots, \partial_x^{2m-1}q(x; c_0))^T,
\end{equation}
where $q(x; c_0)$ is an even function and is an exponentially localized traveling wave solution solution to \cref{genPDE}. 

Next, we address the intersection of the stable and unstable manifolds $W^s(0; c_0)$ and $W^u(0; c_0)$. By \cref{Hhyp}, $W^s(0; c_0), W^u(0; c_0) \subset H^{-1}(0; c_0)$, where $H^{-1}(0; c_0)$ is the 0-level set of $H$. We have the following lemma concerning $H^{-1}(0; c_0)$.

\begin{lemma}\label{manifoldinH0}
There exists $\delta > 0$ such that for $c \in (c_0 - \delta, c_0 + \delta)$, the zero level set $H^{-1}(0; c)$ contains a smooth $(2m-1)$-dimensional manifold $M(c)$; $M(c_0)$ contains $Q(0; c_0)$.
\end{lemma}

We then take the following hypothesis regarding the intersection of the stable and unstable manifolds in $H^{-1}(0; c_0)$.

\begin{hypothesis}\label{H0transversehyp}
The stable manifold $W^s(0; c_0)$ and the unstable manifold $W^u(0; c_0)$ intersect transversely in $H^{-1}(0; c_0)$ at $Q(0; c_0)$.
\end{hypothesis}

\noi By \cref{H0transversehyp} and \cref{manifoldinH0}, 
\[
\dim (T_{Q(0; c_0)}W^s(0; c_0) + T_{Q(0; c_0)}W^u(0; c_0)) = \dim M(c_0) = 2m-1 
\]
Since $\dim T_{Q(0; c_0)}W^s(0; c_0) = m$ and $\dim T_{Q(0; c_0)}W^s(0; c_0) = m$, by a dimension-counting argument, $\dim T_{Q(0; c_0)}W^s(0; c_0) \cap T_{Q(0; c_0)}W^u(0; c_0) = 1$. Since $Q(x; c_0) \subset W^s(0; c_0) \cap W^u(0; c_0)$, 
\[
Q'(0; c_0) \in T_{Q(0; c_0)}W^s(0; c_0) \cap T_{Q(0; c_0)}W^u(0; c_0),
\]
thus we have the nondegeneracy condition
\begin{equation}\label{nondegencond}
T_{Q(0; c_0)}W^s(0; c_0) \cap T_{Q(0; c_0)}W^u(0; c_0) = \R Q'(0; c_0)
\end{equation}
 
Using \cref{nondegencond}, we decompose the tangent spaces of the stable and unstable manifolds at $Q(0; c_0)$. Since the two tangent spaces have a one-dimensional intersection spanned by $Q'(0; c_0)$, we can decompose them as 
\begin{equation}\label{TQ0decomp1}
\begin{aligned}
T_{Q(0; c_0)}W^u(0; c_0) &= \R Q'(0; c_0) \oplus Y^-(c_0) \\
T_{Q(0; c_0)}W^s(0; c_0) &= \R Q'(0; c_0) \oplus Y^+(c_0)
\end{aligned}
\end{equation}

We can now prove the existence of homoclinic orbits $Q(x; c)$ for $c$ near $c_0$ and show that $Q(x; c)$ is differentiable in $c$.

\begin{theorem}\label{transverseint}
Assume \cref{Hhyp} and \cref{H0transversehyp}. Then there exists $\delta > 0$ with the following property. For $c \in (c_0 - \delta, c_0 + \delta)$, the stable and unstable manifolds $W^s(0; c)$ and $W^u(0; c)$ have a one-dimensional transverse intersection in $H^{-1}(0; c)$. This intersection yields a homoclinic orbit $Q(x; c)$. The map $c \rightarrow Q(x; c)$ is smooth, and the derivative $\partial_c Q(x; c)$ is exponentially localized. Specifially, for any $\epsilon > 0$, there exists $\delta_1 > 0$ with $\delta_1 < \delta$ such that for $c \in (c_0 - \delta_1, c_0 + \delta_1)$,
\begin{equation}\label{Qcbound}
|\partial_c Q(x; c)| \leq C e^{-(\alpha_0 - \epsilon)|x|}
\end{equation}
\end{theorem}
 
\noi \cref{transverseint} implies that \cref{cintervalhyp} is satisfied. From this point forward, we will fix $c = c_0$ For simplicity, we will use $c$ in place of $c_0$ and will omit the dependence on $c$ in our notation.

\subsection{Variational equation}\label{sec:vareq1}

The variational equation is the linearization of \cref{genODE} about the primary pulse solution $Q(x)$ and will be an important tool in our analysis. The variational and adjoint variational equations associated with \cref{genODE} are
\begin{align}
V' = DF(Q(x)) V \label{vareq1} \\
W' = -DF(Q(x))^* W \label{adjvareq1},
\end{align}
where $DF(Q(x))$ is the $2m \times 2m$ matrix
\begin{equation}\label{defDF}
DF(Q(x)) = 
\begin{pmatrix}
0 & 1 & 0 & \dots & 0 & 0 \\
0 & 0 & 1 & \dots & 0 & 0 \\
& && \ddots \\
0 & 0 & 0 & \dots & 1 & 0 \\
0 & 0 & 0 & \dots & 0 & 1 \\
\partial_{u_1}f(Q(x)) - c & \partial_{u_2}f(Q(x)) & \partial_{u_3}f(Q(x)) & \dots & \partial_{u_{2m-1}}f(Q(x)) & \partial_{u_{2m}}f(Q(x))
\end{pmatrix}
\end{equation}
It follows from \cref{nondegencond} that $Q'(x)$ is the unique bounded solution to \cref{vareq1}, and that there exists a unique bounded solution $\Psi(x)$ to \cref{adjvareq1}. (In both cases, uniqueness is up to scalar multiples). Since we have a conserved quantity $H$, we know the exact form of $\Psi(x)$, which is given in the following lemma.

\begin{lemma}\label{psiform}
The solution $\Psi(x)$ is given by $\Psi(x) = \nabla H(Q(x))$. In addition, $\Psi(x)$ is symmetric with respect to the standard reversor operator $R$, i.e. $\Psi(-x) = R \Psi(x)$, and the last component of $\Psi(x)$ is $q'(x)$.
\end{lemma}

The last lemma of the section collects a few important results about solutions to the variational and adjoint variational equations.

\begin{lemma}\label{eigadjoint}
Consider the linear ODE $V' = A(x)V$ and the corresponding adjoint equation $W' = -A(x)^* W$, where $A$ is an $n \times n$ matrix depending on $x$. Then the following are true.
\begin{enumerate}[(i)]
\item $\dfrac{d}{dx}\langle V(x), W(x) \rangle = 0$, thus the inner product is constant in $x$.
\item If $W(x)$ is bounded and $V(x) \rightarrow 0$ as $x \rightarrow \infty$ or $V(x) \rightarrow -\infty$, then $\langle V(x), W(x) \rangle = 0$ for all $x \in \R$. The same holds if we reverse the roles of $W$ and $V$.
\item If $\Phi(y, x)$ is the evolution operator for $V'(x) = A(x)V(x)$, then $\Phi(x, y)^*$ is the evolution operator for the adjoint equation $W'(y) = -A(y)^* W(y)$.
\end{enumerate}
\end{lemma}

\noi From \cref{eigadjoint}(ii), $\Psi(0) \perp \R Q'(0) \oplus Y^+ \oplus Y^-$, thus we can decompose $\R^{2m}$ as
\begin{equation}\label{R2mdecomp}
\R^{2m} = \R Q'(0) \oplus Y^+ \oplus Y^- \oplus \R \Psi(0).
\end{equation}

\section{Spectrum of pulses}\label{sec:genspectrum}

In this section, we set up the the problem that we will use to find the eigenvalues associated with an equilibrium solution to \cref{genPDE}. Let $\phi$ be an exponentially localized traveling wave solution to \cref{genPDE}. We will study the spectrum of the linearized operator $\partial_x \calL(\phi) = \partial_x(\calE''(\phi) + c)$. First we note that the PDE eigenvalue problem
\begin{equation}\label{genPDEeig}
\partial_x \calL(\phi) v = \lambda v
\end{equation}
is equivalent to the system
\begin{equation}\label{genPDEeig2}
\begin{aligned}
\calL(\phi) v &= k \\
\partial_x k &= \lambda v
\end{aligned}
\end{equation}
It will turn out that \cref{genPDEeig2} is much more convenient for analysis.

As in \cite{Sandstede1998} and the previous section, we will take a spatial dynamics approach. To do this, we write \cref{genPDEeig2} as a first order system of ODEs. Let $U(x) = (\phi(x), \partial_x \phi(x, \dots, \partial_x^{2m-1}\phi(x), 0)^T \in \R^{2m+1}$ and let $V(x) = (v(x), \partial_x v(x), \dots, \partial_x^{2m-1} v(x), k)$. Then from \cref{Eprimeuform} and \cref{PDElinearization}, equation \cref{genPDEeig2} is equivalent to the first order system
\begin{equation}\label{PDEeigsystem}
V'(x) = A(U(x))V(x) + \lambda B V(x),
\end{equation}
where $A(U(x))$ is the $(2m+1)\times(2m+1)$ matrix
\begin{equation}\label{defAphi}
A(U(x)) = 
\begin{pmatrix}
0 & 1 & 0 & \dots & 0 & 0 & 0 \\
0 & 0 & 1 & \dots & 0 & 0 & 0\\
&  && \ddots \\
0 & 0 & 0 & \dots & 1 & 0 & 0 \\
0 & 0 & 0 & \dots & 0 & 1 & 0 \\
\partial_{u_1}f(U(x)) - c & \partial_{u_2}f(U(x)) & \partial_{u_3}f(U(x)) & \dots & \partial_{u_{2m-1}}f(U(x)) & \partial_{u_{2m}}f(U(x)) & 1 \\
0 & 0 & 0 & \dots & 0 & 0 & 0
\end{pmatrix}
\end{equation}
and $B$ is the $(2m+1) \times (2m+1)$ constant coefficient matrix
\begin{equation}\label{DefB}
B = \begin{pmatrix}0 & 0 & 0 & 0 & 0 \\0 & 0 & 0 & 0 & 0 \\  & 
\vdots & & \vdots & \\0 & 0 & 0 & 0 & 0 \\1 & 0 & 0 & 0 & 0 \end{pmatrix}.
\end{equation}
We note that the upper left $2m \times 2m$ block of $A(U(x))$ is $DF(U(x))$, which is given by \cref{defDF}. 

Since $\phi$ is exponentially localized, so is $U(x)$, thus $A(U(x))$ is exponentially asymptotic to the constant coefficient matrix $A(0)$. The following lemma describes the eigenvalues and kernel of $A(0)$.

\begin{lemma}\label{eigA0lemma}
The following hold concerning the eigenvalues of $A(0)$.
\begin{enumerate}[(i)]
\item $A(0)$ has a simple eigenvalue at 0 and a quartet of eigenvalues at $\pm \alpha_0 \pm \beta_0 i$. For any other eigenvalue $\nu$ of $A(0)$, $|\text{Re }\nu| > \alpha_0$.
\item The kernel of $A(0)$ is spanned by $V_0$ and the kernel of $A^*(0)$ is spanned by $W_0$, where
\begin{align}
V_0 &= (1/c, 0, \dots, 0, 1)^T \label{V0} \\
W_0 &= (0, 0, \dots, 0, 1)^T \label{W0}
\end{align}
and $\langle V_0, W_0 \rangle = 1$.
\item The projection on the kernel of $A(0)$ is given by
\begin{equation}\label{projkernelA0}
P_{\ker A(0)} = \langle W_0, \cdot \rangle V_0
\end{equation}
\end{enumerate} 
\end{lemma}

\noi Since $A(0)$ is not hyperbolic, the results of \cite{Sandstede1998} do not apply. For multi-pulse solutions, we will get around this problem in \cref{chapter:kdv5homoclinic} by using an exponentially weighted space; this make the analysis very similar to that in \cite{Sandstede1998} but only provides limited results. In \cref{chapter:kdv5perspec}, we will look at the spectrum of periodic multi-pulse solutions. The analysis is more difficult, since it involves a one-dimensional center manifold, but we will be able to obtain more results. The remainder of this section is setup material which is common to both cases.

Let $W^s(0)$, $W^u(0)$, and $W^c(0)$ be the stable, unstable, and center manifolds of the equilibrium at 0. By reversibility, $\dim W^s(0) = m$ and $\dim W^u(0) = m$. The center manifold $W^c(0)$ is one-dimensional.

Let $\tilde{Q}(x)$ be the primary pulse solution from \cref{sec:primarypulse}, and define $Q(x) \in R^{2m+1}$ by $Q(x) = (\tilde{Q}(x), 0)$. The variational and adjoint variational equations associated with \cref{PDEeigsystem} are
\begin{align}
V'(x) = A(Q(x)) V(x) \label{vareq2} \\
W'(x) = -A(Q(x))^* W(x) \label{adjvareq2}
\end{align}
where we reuse the same notation $V$ and $W$ for convenience. Since $(\partial_x \calL(q))\partial_x q = 0$ and $(\partial_x \calL(q))(-\partial_c q) = \partial_x q$, it follows that 
\begin{align}
(\partial_x Q(x))' &= A(Q(x))(\partial_x Q(x)) \label{Qprimevarsol} \\
(\partial_c Q(x))' &= A(Q(x))(\partial_c Q(x)) + B(\partial_x Q(x))\label{Qcvarsol}
\end{align}
Thus $\partial_x Q(x)$ is an exponentially localized solution to the variational equation \eqref{vareq2}.

Since $Q(x)$ is exponentially localized, it cannot involve the center manifold $W^c(0)$, thus it must lie in the intersection of the stable and unstable manifolds. In particular, this implies that $\R Q'(0) \subset T_{Q(0)}W^s(0) \cap T_{Q(0)}W^u(0)$. It follows from \cref{H0transversehyp} that these are actually equal, which we state in the following lemma.

\begin{lemma}\label{nondegenlemma}
We have the nondegeneracy condition
\begin{equation}\label{nondegen2}
T_{Q(0)}W^s(0) \cap T_{Q(0)}W^u(0) = \R Q'(0)
\end{equation}
\end{lemma}

\noi Using the nondegeneracy condition \eqref{nondegen2}, we can decompose the tangent spaces of the stable and unstable manifolds at $Q(0)$ as
\begin{align*}
T_{Q(0)}W^s(0) &= \R Q'(0) \oplus Y^+ \\
T_{Q(0)}W^u(0) &= \R Q'(0) \oplus Y^- \\
\end{align*}
Since $\dim \R Q'(0) \oplus Y^+ \oplus Y^- = 2m-1$, there we need two more directions to span $\R^{2m+1}$. To find these, we will use the following lemma regarding solutions to \eqref{vareq2} and \eqref{adjvareq2}.

\begin{lemma}\label{varadjsolutions}
We have the following solutions to the variational equation and adjoint variational equation.
\begin{enumerate}
	\item There exists a bounded solution $V^c(x)$ to \eqref{vareq2} such that 
	\begin{equation}
	V^c(x) \rightarrow V_0 \text{ as }|x| \rightarrow \infty
	\end{equation}
	$V^c$ is symmetric with respect to the standard reversor operator $R$, i.e. $V^c(-x) = R V^c(x)$. Furthermore, $V^c = (\tilde{V}^c, 1)$, where $\tilde{V}^c$ solves $\tilde{V}^c(x)' = DF(Q(x)) \tilde{V}^c(x) + (0, \dots, 0, 1)^T$.

	\item There exist two linearly independent, bounded solutions $\Psi(x)$ and $\Psi^c(x)$ to \eqref{adjvareq2}.
	\begin{enumerate}[(i)]
	\item $\Psi(x)$ is the exponentially localized solution
	\begin{equation}\label{psicomponents}
	\Psi(x) = (\nabla H(Q(x)), q(x)).
	\end{equation}
	$\Psi(x)$ is symmetric with respect to the standard reversor operator $R$, i.e. $\Psi(-x) = R \Psi(x)$. 
	\item $\Psi^c(x)$ is the constant solution
	\begin{equation}
	\Psi^c(x) = W_0
	\end{equation}
	\end{enumerate}
	Any other bounded solution to \eqref{adjvareq2} is a linear combination of these. Finally, $\Psi(0), \Psi^c(0) \perp \R Q'(0) \oplus Y^+ \oplus Y^-$.
\end{enumerate}
\end{lemma}

\begin{remark}\label{remark:computeVc}
If $v^c(x)$ is the first component of $V^c(x)$, $v^c$ is a formal solution to $\calL(q) v_c = 1$. Since
\[
V^c(x) = (v^c(x), \partial_x v^c(x), \dots, \partial_x^{2m-1}v^c(x), 1 ),
\]
this provides a convenient way of computing $V^c(x)$ numerically. We also note that since $v^c(x)$ is not in $H^{2m}(\R)$, $v^c(x)$ is not an eigenfunction of \cref{genPDEeig}.
\end{remark}

\begin{remark}\label{remark:psinotation}
We note that for convenience we have used the notation $Q(x)$ and $\Psi(x)$ in both \cref{sec:primarypulse} and \cref{sec:genspectrum} to denote the primary pulse and the localized solution to the adjoint variational equation. Since we will be considering the existence and the stability problem separately, this should not cause confusion. In addition, the inner products $\langle \Psi(x), Q(y) \rangle$ and $\langle \Psi(x), Q'(y) \rangle$ are the same.
\end{remark}

Using Lemma \ref{nondegenlemma} and Lemma \ref{varadjsolutions}, we can decompose $\R^{2m+1}$ as  
\begin{equation}
\R^{2m+1} = \R \Psi(0) \oplus \R \Psi^c(0) \oplus \R Q'(0) \oplus Y^+ \oplus Y^-
\end{equation}
where $\R \Psi(0) \oplus \R \Psi^c(0) \perp \R Q'(0) \oplus Y^+ \oplus Y^-$.

Before we conclude this section, we will make one final hypothesis concerning a higher order Melnikov integral. First, we note that the standard Melnikov integral is 0, i.e. 
\begin{equation}\label{M1}
M_1 = \int_{-\infty}^\infty \langle \Psi(x), B Q'(x) \rangle =
\int_{-\infty}^\infty q(x) \: \partial_x q(x)  = 0
\end{equation}
since the last component of $\Psi(x)$ is $q(x)$ by Lemma \ref{varadjsolutions}, and $q(x)$ is an even function. Since we have a Hamiltonian system, this is the expected result. We take the following hypothesis regarding a higher order Melnikov integral.
\begin{hypothesis}\label{Melnikov2hyp}
The following higher order Melnikov integral is nonzero.
\begin{equation}\label{M2}
M = \int_{-\infty}^\infty \langle \Psi(x), B Q_c(x) \rangle dx =
\int_{-\infty}^\infty q(x) q_c(x) dx \neq 0
\end{equation}
\end{hypothesis}
In applications such as KdV5, numerical analysis suggests that \cref{Melnikov2hyp} is satisfied with $M > 0$.

\iffulldocument\else
	\bibliographystyle{amsalpha}
	\bibliography{thesis.bib}
\fi

\end{document}
\documentclass[thesis.tex]{subfiles}

\begin{document}

\iffulldocument\else
	\chapter{KdV5}
\fi

In this chapter, we look at the spectral stability of the periodic multi-pulses which we constructed in \cref{chapter:kdv5periodic}. To do this, we will start with the framework in \cref{sec:genspectrum}. For a periodic multi-pulse $Q_n(x)$, we will use Lin's method to construct eigenfunctions as piecewise perturbations of the kernel eigenfunctions $\partial_x Q_n(x)$ and $\partial_c Q_n(x)$. As in \cite{Sandstede1998}, we will reduce the problem of finding eigenvalues to evaluating the determinant of a matrix. In this case, since there is a center direction, we will have $2n \times 2n$ block matrix where the diagonal blocks represent, to leading order, interaction eigenvalues and essential spectrum eigenvalues. We will then use this block matrix equation to actually locate the eigenvalues, first for the periodic 2-pulse and then for general periodic multi-pulses.

\section{Setup of problem}

Let $Q_n(x)$ be a periodic $n-$pulse solution constructed according to Theorem \ref{perexist}. As in \cref{chapter:kdv5periodic}, we will use Lin's method to construct an eigenfunction $V(x)$ as a small perturbation of a piecewise linear combination of the kernel eigenfunctions $\partial_x Q_n(x)$ and $\partial_c Q_n(x)$. For $X_m = \min\{X_0, \dots, X_{n-1} \}$ sufficiently large and $\lambda$ sufficiently small, Lin's method provides a unique function $V(x)$ which solves \eqref{PDEeig2} but which generically has $n$ jumps. Since we are not using an exponential weight, equation \cref{PDEeigsystem} has a center direction, which will imply that these $n$ jumps $V(x)$ can be the two-dimensional space spanned by $\Psi(0)$ and $\Psi^c(0)$. This gives us $2n$ jump conditions. Finding the eigenvalues amounts to solving the $2n$ jump conditions.
 
The first step is to rewrite the eigenvalue problem \cref{PDEeigsystem} as
\begin{equation}\label{PDEeig3}
V' = A(Q_n(x); \lambda)V 
\end{equation} 
where 
\begin{equation}
A(Q_n(x); \lambda) = A(Q_n(x)) + \lambda B
\end{equation}
Let $A(\lambda) = A(0; \lambda)$. For $\lambda = 0$, the asymptotic matrix $A(0)$ has a simple eigenvalue at 0. The next lemma states that for small $\lambda$, $A(0)$ has a simple eigenvalue $\nu(\lambda)$ near 0.

% nu(lambda) lemma

\begin{lemma}\label{nulambdalemma}
There exists $\delta_0 > 0$ such that for $|\lambda| < \delta_0$, the $A(\lambda)$ has a simple eigenvalue $\nu(\lambda)$. $\nu(\lambda)$ is smooth in $\lambda$, $\nu(0) = 0$, $\nu'(0) = 1/c$, and for $|\lambda| < \delta_0$,
\begin{equation}\label{nulambda}
\nu(\lambda) = \frac{1}{c} \lambda + \mathcal{O}(|\lambda|^3)
\end{equation}
In addition,
\begin{enumerate}[(i)]
\item $\nu(-\lambda) = -\nu(\lambda)$, i.e. $\nu(\lambda)$ is an odd function.
\item If $\lambda$ is pure imaginary, $\nu(\lambda)$ is also pure imaginary.
\item The corresponding eigenvector to $\nu(\lambda)$ is $V_0(\lambda)$ which is smooth in $\lambda$ and has Taylor expansion
\begin{equation}\label{V0expansion}
V_0(\lambda) = V_0 + \lambda V_0'(0) + \mathcal{O}(\lambda^2),
\end{equation}
where $V_0'(0) = (0, 1/c^2, 0, \dots, 0)^T$. Furthermore, $V_0(-\lambda) = R V_0(\lambda)$, where $R$ is the standard reversor operator.
\end{enumerate}

Similarly, the matrix $-A(\lambda)^*$ has an eigenvalue $-\overline{\nu(\lambda)}$ with corresponding eigenvector $W_0(\overline{\lambda})$, both of which are smooth in $\overline{\lambda}$. $W_0(\overline{\lambda})$ has Taylor expansion
\begin{equation}\label{W0expansion}
W_0(\lambda) = W_0 + \overline{\lambda} W'(0) + \mathcal{O}(\overline{\lambda}^2),
\end{equation}
where 
\begin{equation}\label{W0prime}
W_0'(0) = \frac{1}{c} \left( 0, -c_3, 0, -c_5, 0, \dots, 0, -c_{2m-1}, 0, 1, 0\right)^T
\end{equation}
and symmetry property $W_0(-\overline{\lambda}) = R W_0(\overline{\lambda})$.
\end{lemma}

The constant-coefficient ODE $Z(x)' = A(\lambda)Z(x)$ has a solution $Z(x) = V_0(\lambda)e^{\nu(\lambda)x}$. In the next lemma, we show that \cref{PDEeig3} has solutions on $\R^+$ and $\R^-$ which approach $V_0(\lambda)e^{\nu(\lambda)x}$ as $x \rightarrow \pm \infty$.

\begin{lemma}\label{lemma:Vpm}
For sufficiently small $|\lambda|$ and any $\alpha_1 < \alpha_0$, the equation $V' = A(Q_n(x); \lambda)V$ has solutions $V^\pm(x; \lambda)$ on $\R^\pm$ which are given by
\begin{align}\label{Vpmlambda}
V^\pm(x; \lambda) &= e^{\nu(\lambda)x}(V_0(\lambda) + V_1^\pm(x; \lambda)),
\end{align}
where
\begin{equation}\label{Vpmdecay}
|V_1^\pm(x; \lambda)| \leq C e^{-\alpha_1 |x|}
\end{equation}
and we have the symmetry relationship
\begin{equation}\label{Vpmsymmetry}
V^-(x; \lambda) = R V^+(-x; -\lambda).
\end{equation}
$V^\pm(x; 0) = V^c(x)$, which is defined in \cref{varadjsolutions}. Finally, $V^\pm(x; \lambda)$ has Taylor expansion
\begin{equation}
V^\pm(x; \lambda) = V^c(x) + \partial_\lambda V^\pm(x; 0) + \mathcal{O}(|\lambda|^2)
\end{equation}
where $\partial_\lambda V^-(x; 0) = -R \partial_\lambda V^+(x; 0)$ and $\partial_\lambda V^+(x; 0)$ solves the equation
\[
Z'(x) = A(Q(x),0) Z(x) + B V^c(x)
\]
on $\R^+$.
\end{lemma}

\begin{remark}\label{remark:computeVc}
If $z(x)$ is the first component of $\partial_\lambda V^+(x; 0)$, then $z(x)$ is a formal solution to $\partial_x \calL(q) z(x) = v^c(x)$ on $\R^+$, where $v^c(x)$ is the first component of $V^c(x)$. Since
\[
\partial_\lambda V^+(x; 0) = (z(x), \partial_x z(x), \dots, \partial_x^{2m-1}z^c(x), q^c(x) ),
\]
this provides a method way of computing $V^c(x)$ numerically, as long as the appropriate boundary conditions are used at 0.
\end{remark}

\section{Block matrix theorem}

We can now state the main theorem of this chapter which provides a condition for $\lambda$ to be an eigenvalue of \eqref{PDEeig3}. This theorem is analagous to \cite[Theorem 2]{Sandstede1998}, where the matrix $A$ replaced by a block matrix.

% block matrix theorem
\begin{theorem}\label{blockmatrixtheorem}
Assume Hypotheses \ref{Ehyp}, \ref{Hhyp}, \ref{hypeqhyp}, \ref{Qexistshyp}, and \ref{H0transversehyp}. Let $Q_n(x)$ be a periodic $n-$pulse solution constructed according to Theorem \ref{perexist} with [pulse lengths $X_0, \dots, X_{n-1}$. Then there exists $\delta > 0$ with the following property. There exists a bounded, nonzero solution $V$ of \eqref{PDEeig3} for $|\lambda| < \delta$ if and only if the $(2n \times 2n)$ block matrix equation
\begin{equation}\label{blockeq}
\begin{pmatrix}
K(\lambda) + C_1 K(\lambda) + K_1(\lambda) + C_2 & \lambda \tilde{A} + D_1 \\
\lambda p_1 \tilde{K}(\lambda) + C_3 K(\lambda) + K_2(\lambda) + C_4 & A - \lambda^2 MI + D_2
\end{pmatrix}
\begin{pmatrix} c \\ d \end{pmatrix} = 0
\end{equation}
has a nontrivial solution. The individual terms in the block matrix are as follows.

\begin{enumerate}
\item $K(\lambda)$ is the banded matrix
\begin{equation}
K(\lambda) = 
\begin{pmatrix}
e^{-\nu(\lambda)X_1} & & & & & -e^{\nu(\lambda)X_0} \\
-e^{\nu(\lambda)X_1} & e^{-\nu(\lambda)X_2} \\
& -e^{\nu(\lambda)X_2} & e^{-\nu(\lambda)X_3} \\
& \ddots & \ddots & &&  \\
& & & & -e^{\nu(\lambda)X_{n-1}} & e^{-\nu(\lambda)X_0} 
\end{pmatrix}
\end{equation}
where $\nu(\lambda)$ is defined in Lemma \ref{nulambdalemma}. $\tilde{K}(\lambda)$ is the same matrix with all terms positive.

\item $A$ is the symmetric banded matrix
\begin{align}\label{Asymm}
A &= \begin{pmatrix}
-a_0 -a_1 & a_0 + a_1 \\
a_0 + a_1 & -a_0 - a_1
\end{pmatrix} && n = 2 \\
A &= \begin{pmatrix}
-a_{n-1} - a_0 & a_0 & & &  & a_{n-1}\\
a_0 & -a_0 - a_1 &  a_1 \\
& a_1 & -a_1 - a_2 &  a_2 \\
& \ddots & \ddots & \ddots \\
a_{n-1} & & & & a_{n-2} & -a_{n-2} - a_{n-1} \\
\end{pmatrix} && n > 2 \nonumber
\end{align}
where
\begin{align*}
a_i &= \langle \Psi(X_i), Q'(-X_i) \rangle
\end{align*}

\item $\tilde{A}$ is the matrix
\begin{align*}
\tilde{A} &= \begin{pmatrix}
-e^{-\nu(\lambda)X_1} k_1 & e^{-\nu(\lambda)X_1} k_1 \\
& -e^{-\nu(\lambda)X_2} k_2 & e^{-\nu(\lambda)X_2} k_2 \\
& \ddots \\
e^{-\nu(\lambda)X_0} k_0 & &  & -e^{-\nu(\lambda)X_0} k_0 & 
\end{pmatrix}
\end{align*}
where 
\begin{equation*}
k_i = 2 \langle W_0'(0), Q'(X_i) \rangle
\end{equation*}
and $W'(0)$ is defined by \cref{W0prime} in \cref{nulambdalemma}. $\tilde{A}$ has uniform bound
\begin{align*}
\tilde{A} &= \mathcal{O}( e^{-\alpha X_m})
\end{align*}

\item $M$ is the higher order Melnikov integral
\begin{align*}
M &= \int_{-\infty}^\infty \Psi(y) t(y) dy \\
\end{align*}
and
\begin{align*}
p_1 = \langle \Psi(0), \partial_\lambda V^+(0; 0) \rangle
\end{align*}
where $V^+(0; 0)$ is defined in \cref{lemma:Vpm}.

\item The remaining terms are remainder matrices which are analytic in $\lambda$ and have uniform bounds
\begin{align*}
C_1 &= \mathcal{O}(|\lambda|e^{-\alpha X_m}(|\lambda| + e^{-\alpha X_m})) \\
C_2 &= \mathcal{O}(|\lambda|e^{-\alpha X_m}) \\
C_3 &= \mathcal{O} (|\lambda| + e^{-\alpha X_m})^2) \\
C_4 &= \mathcal{O}((|\lambda| + e^{-\alpha X_m})^2) \\
D_1 &= \mathcal{O}(|\lambda|(|\lambda| + e^{-\alpha X_m})^2) \\
D_2 &= \mathcal{O}((|\lambda| + e^{-\alpha X_m})^3)
\end{align*}
for $\alpha$ slightly smaller than $\alpha_0$ (which is defined precisely in the proof), and $X_m = \min\{X_0, \dots, X_{n-1}\}$.

\item $K_1(\lambda)$ and $K_2(\lambda)$ are obtained from $K(\lambda)$ by multiplying each nonzero entry by $\mathcal{O}(|\lambda|(|\lambda| + e^{-\alpha X_m}))$ and $\mathcal{O}((|\lambda| + e^{-\alpha X_m})^2)$ (respectively). They are defined precisely in the proofs of Lemmas \ref{jumpcenteradj} and \ref{jumpadj}
\end{enumerate}
\end{theorem}

We expect to find eigenvalues near where the two leading order blocks, $A - \lambda^2 MI$ and $K(\lambda)$, are singular. As in \cref{chapter:kdv5homoclinic}, points where $A - \lambda^2 MI$ is singular give us the interaction eigenvalues as well as two kernel eigenvalues. $K(\lambda)$ is singular for a discrete set of points on the imaginary axis. These are the periodic analogue to the essential spectrum in the $n$-homoclinic case. Although they are techically still point spectrum, we will refer to them as essential spectrum eigenvalues to distinguish them from the interaction eigenvalues.

We also note that although the expressions for the remainder matrices in \cref{blockmatrixtheorem} seem tediously detailed, this will be useful for the proof of results in a later section.

\section{Interaction eigenvalues of periodic 2-pulses}

We will now use Theorem \ref{blockmatrixtheorem} to locate the interaction eigenvalues of \eqref{PDEeig3}. First we consider the case of the periodic 2-pulse. Although the computation is tedious, we can compute the determinant of \cref{blockeq} exactly. 

WE WILL STATE (AND PROVE AT THE END) THE FOLLOWING THEOREMS HERE.
\begin{enumerate}
\item Find interaction eigenvalues if not in a Krein bubble.
\item Find essential spectrum eigenvalues if not in a Krein bubble.
\item Find the Krein bubble
\item Demonstrate the eigenvalue bifurcation for periodic 2-pulses with equal period and show that it follows the pitchfork bifurcation structure from the previous chapter.
\end{enumerate}

\section{Eigenvalues of periodic multi-pulses}

In this section, we generalize the results to periodic multi-pulses. Since computing the determinant of \cref{blockeq} is unfeasible for $n \geq 3$, this will require a different approach. First, we note that the matrix $A$ from Theorem \ref{blockmatrixtheorem} is symmetric, thus its eigenvalues are real. $A$ has an eigenvalues of 0 corresponding to eigenvector $(1, 1, \dots, 1)^T$. In the next hypothesis, we assume that the eigenvalues of $A$ distinct. 

\begin{hypothesis}\label{Adistincteigs}
The eigenvalues of $A$ are given by $(0, \mu_1, \dots, \mu_{n-1})$, all of which are distinct.
\end{hypothesis}

\noi This is not true in general. For example, if $n \geq 3$ and all the terms $a_i$ in $A$ are identical, $A$ is a circulant matrix, and it is not hard to show that $A$ will have some eigenvalues of algebraic multiplicity 2.

We expect to find eigenvalues near where the leading order matrices $K(\lambda)$ and $A - \lambda^2 MI$ are singular. This happens at the following values of $\lambda$. 
\begin{itemize}
	\item $A - \lambda^2 M I$ is singular for $\lambda = 0$ (algebraic multiplicity 2) and $\lambda \in \{ \pm \sqrt{\mu_1/M}, \dots, \pm \sqrt{\mu_{n-1}/M}\}$. The interaction eigenvalues will be located near these points.	In terms of the scaling parameter $r$, the interaction eigenvalues will be order $\mathcal{O}(r^{1/2})$.

	\item $K(\lambda)$ is singular at $\lambda = \pm \lambda^K(X,k)$ for integer $k$, where $\lambda(X, 0) = 0$ (algebraic multiplicity 1) and
	\begin{align}\label{lambdaXkapprox}
	\lambda^K(X,k) \approx c \frac{k \pi i }{X} 
	\end{align}
	The essential spectrum eigenvalues will be close to these points. Using equation \cref{Xdomain}, in terms of the scaling parameter $r$, $\lambda^K(X,k) = \mathcal{O}(1/\log|r|)$. Since \cref{blockmatrixtheorem} only holds for $|\lambda| < \delta$, we will restrict our analysis to $|\lambda^K(X,k)| < \delta$.
\end{itemize}  

For the analysis to work, we need to ensure that nonzero singular points of the two leading order matrices are sufficiently isolated from each other. To that end, we take the following definition.

\begin{definition}\label{epsilonballs}
A periodic $n-$pulse $Q_n$ parameterized as in Theorem \ref{perexist} satisfies the \emph{$\epsilon-$ball condition} if the two sets of points 
\begin{itemize}
\item $S_1 = \{ \pm \sqrt{\mu_1/M}, \dots, \pm \sqrt{\mu_{n-1}/M} \}$
\item $S_2 = \{ \lambda^K(X,k) : k \in \Z \text{ and } |\lambda^K(X,k)| < \delta \}$
\end{itemize}
are separated by at least $\epsilon$, where
\[
\epsilon = \frac{r^{1/2}}{X^{1/2}} = \mathcal{O} \left( \frac{r^{1/2}}{|\log r| } \right)
\]
\end{definition}

The following lemma states that this condition can always be satisfied by taking $r$ sufficiently small

\begin{lemma}\label{epsilonballlemma}
Choose baseline length parameters $b^0 = \{ b_0^0, \dots, b_{n-1}^0 \}$ and phase parameter $\theta$ and use them to construct a periodic multi-pulse according to \cref{perexist}. Then there exists $r(b^0, \theta) \leq r_0$ such that for $r \leq r(b, \theta)$, the $\epsilon-$ball condition is satisfied.
\end{lemma} 

Although we can always choose sufficiently small $r$ so that the $\epsilon$-ball condition is satisfied, this definition. The proof of Lemma \ref{epsilonballlemma} is constructive and guarantees that for all sufficiently small $r$, any purely imaginary interaction eigenvalues will be smaller in magnitude than all nonzero essential spectrum eigenvalues. In the previous section, we discussed Krein bubbles in the periodic 2-pulse, which occur when an essential spectrum eigenvalue passes through an interaction eigenvalue. We would like our theory to handle the case where such a passage has occurred.

With this taken care of, we have the following theorem, which we can to locate the eigenvalues of \eqref{PDEeig3} near the origin.

% eigenvalue location theorem

\begin{theorem}\label{locateeigtheorem}
Assume Hypotheses \ref{Ehyp}, \ref{Hhyp}, \ref{hypeqhyp}, \ref{Qexistshyp}, \ref{H0transversehyp}, \ref{Melnikov2hyp}, and \ref{Adistincteigs}. Let $Q_n(x)$ be a periodic $n-$pulse solution constructed according to Theorem \ref{perexist} with scaling parameter $r \leq r_0$. Let $\delta > 0$ be defined as in Theorem \ref{blockmatrixtheorem}. Then the following are true.

\begin{enumerate}[(i)]

\item There is an eigenvalue at 0 with (at minimum) geometric multiplicity 2 and algebraic multiplicity 3. The corresponding eigenfunctions are the kernel eigenfunction $\partial_x Q_n(x)$ from translation invariance; its generalized kernel eigenfunction $\partial_C Q_n(x)$; and a third kernel eigenfunction $V_n^c(x)$ which is bounded but does not decay exponentially.

\item There exists $r_1 \leq r_0$ such that for every $r \leq r_1$ for which the $\epsilon-$ball condition is satisfied, there are $n - 1$ pairs of interaction eigenvalues given by $\lambda = \pm \lambda^{\text{int}}_j(r)$ for $j = 1, \dots, n-1$, where
\begin{align*}
\lambda^{\text{int}}_j(r) = \sqrt{\frac{\mu_j}{M}} + \mathcal{O}(r^{3/4})
\end{align*}
These interaction eigenvalue pairs are either real or purely imaginary, and by Hamiltonian symmetry, the remainder term cannot move them off of the real or imaginary axis.

\item There exists $r_2 \leq r_1$ such that for every $r \leq r_2$ for which the $\epsilon-$ball condition is satisfied, there are pairs of purely imaginary essential spectrum eigenvalues given by $\lambda = \pm \lambda^{ess}(X,k; r)$ for every positive integer $k$ with $\frac{c \pi k}{X} < \delta$, where
\begin{equation}\label{lambdaess}
\lambda^{ess}(X, k; r) = c \frac{k \pi i }{X} \left( 1 + \mathcal{O}\left( \frac{1}{X} \right)\right) + \mathcal{O}\left( \frac{r^{1/2}}{X} \right)
\end{equation}
In terms of the parameters $r$ and the $b_j$, these are located at approximately
\begin{equation}\label{lambdaessr}
\lambda^{ess}(k; r) = C \frac{k \pi i }{n |\log r| + |\log (b_0 b_1 \cdots b_{n-1})|}  \left( 1 + \mathcal{O}\left( \frac{1}{n |\log r|} \right)\right) + \mathcal{O}\left( \frac{r^{1/2}}{n |\log r|} \right)
\end{equation}
The remainder terms cannot move these off of the imaginary axis.

\item For sufficiently small $r$, we have the following two eigenvalue counts.
\begin{itemize}
	\item There exists a small radius $\xi$ (which excludes the interaction eigenvalues and essential spectrum eigenvalues) such that there are exactly 3 eigenvalues inside the circle of radius $\xi$ in the complex plane. These must be the three eigenvalues from part (i).

	\item There are exactly $2n + 2 k_M + 1$ eigenvalues inside the circle of radius $\tilde{\delta}$ (which may be slightly smaller than $\delta$) in the complex plane, where $k_M$ is the largest positive integer $k$ such that $\lambda^K(k,X) < \tilde{\delta}$. 
\end{itemize}

If the $\epsilon-$ball condition is satisfied, there are no eigenvalues inside the circle of radius $\tilde{\delta}$ other than the ones already accounted for.
\end{enumerate}
\end{theorem}

THIS PROOF NEEDS TO BE REVISED SINCE I CHANGED THE EPSILON BALL AS A RESULT OF THE KREIN BUBBLE NUMERICS.

From \cref{locateeigtheorem}(ii), the interaction eigenvalue pattern is determined by the signs of the eigenvalues $\mu_j$. In general, there is no easy way to determine these signs. However, if we take one of the baseline length parameters $b_j^0$ to be small compared to the others, the matrix $A$ is approximately tri-diagonal, in which case we can use a similar argument to \cite[Theorem 3(iv)]{Sandstede1998} to determine the interaction eigenvalue pattern. Since we are on a periodic domain, we can without loss of generality take $b_{n-1}^0$ to be the small parameter. Recall from \cref{perexist} that the other baseline length parameters are given by
\begin{align*}
b_j^0 &= e^{-\frac{1}{\rho}m_j \pi} && j = 0, \dots, n-2
\end{align*}
for nonnegative integers $m_j$. The following theorem gives conditions under which the interaction eigenvalue pattern is determines by the parity of the integers $m_0, \dots, m_{n-2}$.

\begin{theorem}\label{inteigsparity}
Assume Hypotheses \ref{Ehyp}, \ref{Hhyp}, \ref{hypeqhyp}, \ref{Qexistshyp}, \ref{H0transversehyp}, \ref{Melnikov2hyp}, and \ref{Adistincteigs}. Let $r_1$ and $b^*$ be as in Theorem \ref{unifperexist}. Choose
\begin{itemize}
\item An integer $n \geq 2$ 
\item A sequence of $n-1$ baseline length parameters $b_0^0, \dots, b_{n-2}^0$, where 
\[
b_j^0 = \exp\left(-\frac{1}{\rho}m_j \pi\right) \in \mathcal{B}
\]
and the $m_j$ are nonnegative integers.
\item Any phase parameter $\theta$.
\end{itemize}

Then there exists $\tilde{r_1} \leq r_1$ and $\tilde{b}^* \leq b^*$ such that for any $r \leq \tilde{r}_1$ and $b_{n-1}^0 \leq \tilde{b}^*$, we have the following result. 

Let $n_{\text{even}}$ be the number of even $\{m_0, \dots, m_{n-2}\}$ and $n_{\text{odd}}$ be the number of odd $\{m_0, \dots, m_{n-2}\}$. 

\begin{itemize}
\item There are $n_{\text{odd}}$ pairs of purely imaginary interaction eigenvalues.
\item There are $n_{\text{odd}}$ pairs of real of interaction eigenvalues.
\end{itemize}
If $M < 0$, these are reversed.
\end{theorem} 

We note that since we took $b_{n-1}^0$, the parity of $m_{n-1}$ is irrelevant to the eigenvalue pattern.

\iffulldocument\else
	\bibliographystyle{amsalpha}
	\bibliography{thesis.bib}
\fi

\end{document}
\documentclass[thesis.tex]{subfiles}

\begin{document}

\iffulldocument\else
	\chapter{KdV5 with periodic boundary conditions}
\fi

\section{Proof of Primary Pulse Theorems}

\subsection{Proof of Lemma \ref{manifoldinH0}}

Let $U = (u_1, \dots, u_{2m})$ and $Q(x; c_0) = (q_1(x), \dots, q_{2m}(x))$. We wish to write the zero-level set $H^{-1}(0; c)$ as a graph near $Q(0; c_0)$. We will do this using the IFT.

By Hypothesis \ref{Qexistshyp}, $H(q_1(0), \dots, q_{2m}(0); c_0) = 0$ and $\nabla_U H(Q(0; c_0); c_0) \neq 0$. This implies that one of the $2m$ components of $\nabla_U H(Q(0; c_0); c_0)$ is nonzero. Renumbering the components of $Q(x)$ if necessary, we can without loss of generality take the first component of $\nabla_U H(Q(0); c_0)$ to be nonzero, i.e. $\partial_{u_1}H(q_1(0), \dots, q_{2m}(0); c_0) \neq 0$. We can use the IFT to solve $H(u_1, u_2 \dots, u_{2m}; c_0) = 0$ for $u_1$ in terms of $(u_2, \dots, u_{2m})$ and $c_0$ near $Q(0)$ and $c_0$.

Specifically, there exist open neighborhooods $V_1$ of $q_1(0)$ and $V_2$ of $(q_2(0), \dots, q_{2m}(0))$, $\delta > 0$, and a unique smooth function $g: V_2 \times (c_0 - \delta, c_0 + \delta) \rightarrow V_1$ such that $g(q_2(0), \dots, q_{2m}(0); c_0) = q_1(0)$ and for all $U \in V_2$ and $c \in (c_0 - \delta, c_0 + \delta)$, $H(g(U; c),U; c) = 0$. Our desired manifolds are then given by
\begin{equation}\label{defMc}
M(c) = \{ (g(U; c), U; c) : U \in V_2 \}
\end{equation}
for $c \in (c_0 - \delta, c_0 + \delta)$. These manifold are $(2m-1)-$dimensional, are smooth since $H$ is smooth, and are contained in $H^{-1}(0; c_0)$. Taking $U = (q_2(0), \dots, q_{2m}(0))$ and $c = c_0$ in \eqref{defMc}, $M(c_0)$ contains $Q(0; c_0)$.

\subsection{Proof of Theorem \ref{transverseint}}

First, we show that homoclinic orbits $Q(x; c)$ exist for $c$ near $c_0$.

\begin{lemma}
There exists $\delta > 0$ such that for $c \in (c_0 - \delta, c_0 + \delta)$, the stable and unstable manifolds $W^s(0; c)$ and $W^u(0; c)$ have a one-dimensional transverse intersection in $H^{-1}(0; c)$ which is a homoclinic orbit $Q(x; c)$. The map $c \rightarrow Q(x; c)$ is smooth.

\begin{proof}
This is a straightforward consequence of Lemma \ref{manifoldinH0}, the transverse intersection of $W^s(0; c_0)$ and $W^u(0; c_0)$ in $M(c_0) \subset H^{-1}(0; c_0)$ from Hypothesis \ref{H0transversehyp}, and the smoothness of $F$.

Let $\delta$ be as in Lemma \ref{manifoldinH0}, and for convenience, let $Q_0 = Q(0; c_0)$. Since we will be working entirely in the $(2m-1)$-dimensional manifolds $M(c)$, we will choose a local coordinate system to work in. To do this, we will write $M(c)$ as a graph over the tangent space of $M(c_0)$ in the following way. Since $W^s(0; c_0)$ and $W^u(0; c_0)$ in $M(c_0)$ intersect transversely in $M(c_0)$, and we have shown that this intersection is one-dimensional, we can write the three tangent spaces at $Q_0$ as follows.
\begin{align*}
T_{Q_0}W^u(0; c_0) &= V \oplus V^u \\
T_{Q_0}W^s(0; c_0) &= V \oplus V^s \\
T_{Q_0}M(c_0) &= T_{Q_0}W^u(0; c_0) 
+ T_{Q_0}W^s(0; c_0) = V \oplus V^u \oplus V^s
\end{align*}

Since $F(U; c)$ is smooth in $U$ and $c$, the stable and unstable manifolds $W^s(0; c)$ and $W^u(0; c)$ are also smooth in $c$ by the stable manifold theorem and smooth dependence of solutions to \eqref{genODE} on parameters. Thus for $c$ sufficiently close to $c_0$, $W^s(0; c)$ and $W^u(0; c)$ will both intersect $M(c)$. 

First, we write $M(c_0)$ as a graph over its own tangent space. There exists $r > 0$ such that for $|v|, |v_u|, |v_s| < r$ and $c \in (c_0 - \delta, c_0 + \delta)$ (shrinking $\delta$ if needed), 
\begin{align*}
M(c) = Q_0 + v + v_u + v^s + h(v, v_u, v_s; c)
\end{align*}
where $h: V \times V^u \times V^s \times (c_0 - \delta, c_0 + \delta) \rightarrow \R^{2m}$ is smooth, $h(0,0,0; c_0) = 0$ and $D_{(v, v_u, v_s)} h(0, 0, 0; c_0) = 0$.

Using this, we can write $W^s(0; c)$ and $W^u(0; c)$ as graphs over the their tangent spaces. Shrinking $r$ and $\delta$ if needed, there exist smooth functions $h^u: V \times V^u \times \R \rightarrow V^s$ and $h^s: V \times V^s \times \R V^u$ such that 
\begin{equation}\label{Wparam1}
\begin{aligned}
W^u(0; c) &= Q_0 + \{ v + v_u + h^u(v, v_u; c) + h(v, v_u, h^u(v, v_u; c); c): |v|, |v_u| \leq r, c \in (c_0 - \delta, c_0 + \delta) \} \\
W^s(0; c) &= Q_0 + \{ v + h^s(v, v_s; c) + v_s + h(v, h^s(v, v_s; c), v_s; c): |v|, |v_s| \leq r, c \in (c_0 - \delta, c_0 + \delta)  \}
\end{aligned}
\end{equation}
where
\begin{align*}
h^s(0, 0; c_0) &= h^u(0, 0; c_0) = 0 \\
D_{(v, v_u)} h^u(0, 0; c_0) &= D_{(v, v_s)}  h^s(0, 0; c_0) = 0
\end{align*}

Let $\Sigma$ be the hyperplane plane passing through $Q_0$ with no component in $V^c$, i.e.
\[
\Sigma = Q_0 + V^s + V^u
\]
To obtain a unique intersection point, we will consider the intersection of the stable and unstable manifolds to $\Sigma$. Since the restriction maps $(v, v_s) \mapsto v_s$ and $(v, y^-) \rightarrow y^-$ are smooth, $W^u(0; c) \cap \Sigma$ and $W^s(0; c) \cap \Sigma$ are smooth $(m-1)-$dimensional manifolds parameterized by
\begin{equation}\label{Wparam2}
\begin{aligned}
W^u(0; c) \cap \Sigma &= Q_0 + \{ v_u + h^u(0, v_u; c) + h(0, v_u, h^u(0, v_u; c); c): |v_u| \leq r, c \in (c_0 - \delta, c_0 + \delta) \} \\
W^s(0; c) \cap \Sigma &= Q_0 + \{ h^s(0, v_s; c) + v_s + h(0, h^s(0, v_s; c), v_s; c): |v_s| \leq r, c \in (c_0 - \delta, c_0 + \delta) \}
\end{aligned}
\end{equation}

We wish to show that for $c$ close to $c_0$, $W^s(0; c) \cap \Sigma$ and $W^u(0; c) \cap \Sigma$ have a unique point of intersection. Looking at \eqref{Wparam2}, it suffices to solve $K(v_u, v_s; c) = 0$, where $K: V^u \times V^s \times \R \rightarrow V^u \times V^s$ is defined by
\begin{align*}
K(v_u, v_s; c) = (v_u - h^s(0, v_s; c), v_s - h^u(0, v_u; c))
\end{align*}
Since $h^s(0, 0; c_0) = h^u(0, 0; c_0) = 0$, $K(0, 0, c_0) = 0$. Since $D_{v_u} h^u(0, 0; c_0) = D_{v_s} h^s(0, 0; c_0) = 0$, 
$D_{(v_u, v_s)}K(0, 0; c_0) = I_{2m-2}$, which is invertible. Using the IFT, there exists an open neighborhood $W$ of $(0, 0) \in V^u \times V^s$, $\tilde{\delta}$ with $0 < \tilde{\delta} < \delta$, and a unique smooth function $g: (c_0 - \tilde{\delta}, c_0 + \tilde{\delta}) \rightarrow W$ such that $g(c_0) = (0, 0)$ and $K(g(c), c) &= 0$ for all $c \in (c_0 - \tilde{\delta}, c_0 + \tilde{\delta})$.

Let $g(c) = (g_u(c), g_s(c)) \in V^u \times V^s$. Then for $c \in (c_0 - \tilde{\delta}, c_0 + \tilde{\delta})$, the unique intersection point of $W^s(0; c) \cap \Sigma$ and $W^u(0; c) \cap \Sigma$ is given by the smooth function
\[
P(c) = Q_0 + g_u(c) + g_s(c) + h(0, g_u(c), g_s(c); c)
\]
Using $P(c)$ as the initial condition at $x = 0$, we obtain a unique solution $Q(x; c)$ to \eqref{genODE}. Since $P(c)$ is in both $W^s(0; c)$ and $W^u(0; c)$, $Q(x; c)$ is a homoclinic orbit. Since $P(c)$ and $F(U; c)$ are smooth in $c$, the map $c \rightarrow Q(x; c)$ is smooth.
\end{proof}
\end{lemma}



\iffulldocument\else
	\bibliographystyle{amsalpha}
	\bibliography{thesis.bib}
\fi

\end{document}
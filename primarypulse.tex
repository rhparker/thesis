\documentclass[thesis.tex]{subfiles}

\begin{document}

\iffulldocument\else
	\chapter{KdV5 with periodic boundary conditions}
\fi

\section{Proof of Primary Pulse Theorems}

\subsection{Proof of Lemma \ref{manifoldinH0}}

Let $U = (u_1, \dots, u_{2m})$ and $Q(x; c_0) = (q_1(x), \dots, q_{2m}(x))$. We wish to write the zero-level set $H^{-1}(0; c)$ as a graph near $Q(0; c_0)$. We will do this using the IFT.

By Hypothesis \ref{Qexistshyp}, $H(q_1(0), \dots, q_{2m}(0); c_0) = 0$ and $\nabla_U H(Q(0; c_0); c_0) \neq 0$. This implies that one of the $2m$ components of $\nabla_U H(Q(0; c_0); c_0)$ is nonzero. Renumbering the components of $Q(x)$ if necessary, we can without loss of generality take the first component of $\nabla_U H(Q(0); c_0)$ to be nonzero, i.e. $\partial_{u_1}H(q_1(0), \dots, q_{2m}(0); c_0) \neq 0$. We can use the IFT to solve $H(u_1, u_2 \dots, u_{2m}; c_0) = 0$ for $u_1$ in terms of $(u_2, \dots, u_{2m})$ and $c_0$ near $Q(0)$ and $c_0$.

Specifically, there exist open neighborhooods $V_1$ of $q_1(0)$ and $V_2$ of $(q_2(0), \dots, q_{2m}(0))$, $\delta > 0$, and a unique smooth function $g: V_2 \times (c_0 - \delta, c_0 + \delta) \rightarrow V_1$ such that $g(q_2(0), \dots, q_{2m}(0); c_0) = q_1(0)$ and for all $U \in V_2$ and $c \in (c_0 - \delta, c_0 + \delta)$, $H(g(U; c),U; c) = 0$. Our desired manifolds are then given by
\begin{equation}\label{defMc}
M(c) = \{ (g(U; c), U; c) : U \in V_2 \}
\end{equation}
for $c \in (c_0 - \delta, c_0 + \delta)$. These manifold are $(2m-1)-$dimensional, are smooth since $H$ is smooth, and are contained in $H^{-1}(0; c_0)$. Taking $U = (q_2(0), \dots, q_{2m}(0))$ and $c = c_0$ in \eqref{defMc}, $M(c_0)$ contains $Q(0; c_0)$.

\subsection{Proof of Theorem \ref{transverseint}}

First, we show that homoclinic orbits $Q(x; c)$ exist for $c$ near $c_0$.

\begin{lemma}\label{Qcexistslemma}
There exists $\delta > 0$ such that for $c \in (c_0 - \delta, c_0 + \delta)$, the stable and unstable manifolds $W^s(0; c)$ and $W^u(0; c)$ have a one-dimensional transverse intersection in $H^{-1}(0; c)$ which is a homoclinic orbit $Q(x; c)$. The map $c \rightarrow Q(x; c)$ is smooth.

\begin{proof}
This is a straightforward consequence of Lemma \ref{manifoldinH0}, the transverse intersection of $W^s(0; c_0)$ and $W^u(0; c_0)$ in $M(c_0) \subset H^{-1}(0; c_0)$ from Hypothesis \ref{H0transversehyp}, and the smoothness of $F$.

Let $\delta$ be as in Lemma \ref{manifoldinH0}, and for convenience, let $Q_0 = Q(0; c_0)$. Since we will be working entirely in the $(2m-1)$-dimensional manifolds $M(c)$, we will choose a local coordinate system to work in. To do this, we will write $M(c)$ as a graph over the tangent space of $M(c_0)$ in the following way. Since $W^s(0; c_0)$ and $W^u(0; c_0)$ in $M(c_0)$ intersect transversely in $M(c_0)$, and we have shown that this intersection is one-dimensional, we can write the three tangent spaces at $Q_0$ as follows.
\begin{align*}
T_{Q_0}W^u(0; c_0) &= V \oplus V^u \\
T_{Q_0}W^s(0; c_0) &= V \oplus V^s \\
T_{Q_0}M(c_0) &= T_{Q_0}W^u(0; c_0) 
+ T_{Q_0}W^s(0; c_0) = V \oplus V^u \oplus V^s
\end{align*}

Since $F(U; c)$ is smooth in $U$ and $c$, the stable and unstable manifolds $W^s(0; c)$ and $W^u(0; c)$ are also smooth in $c$ by the stable manifold theorem and smooth dependence of solutions to \eqref{genODE} on parameters. Thus for $c$ sufficiently close to $c_0$, $W^s(0; c)$ and $W^u(0; c)$ will both intersect $M(c)$. 

First, we write $M(c_0)$ as a graph over its own tangent space. There exists $r > 0$ such that for $|v|, |v_u|, |v_s| < r$ and $c \in (c_0 - \delta, c_0 + \delta)$ (shrinking $\delta$ if needed), 
\begin{align*}
M(c) = Q_0 + v + v_u + v^s + h(v, v_u, v_s; c)
\end{align*}
where $h: V \times V^u \times V^s \times (c_0 - \delta, c_0 + \delta) \rightarrow \R^{2m}$ is smooth, $h(0,0,0; c_0) = 0$ and $D_{(v, v_u, v_s)} h(0, 0, 0; c_0) = 0$.

Using this, we can write $W^s(0; c)$ and $W^u(0; c)$ as graphs over the their tangent spaces. Shrinking $r$ and $\delta$ if needed, there exist smooth functions $h^u: V \times V^u \times \R \rightarrow V^s$ and $h^s: V \times V^s \times \R V^u$ such that 
\begin{equation}\label{Wparam1}
\begin{aligned}
W^u(0; c) &= Q_0 + \{ v + v_u + h^u(v, v_u; c) + h(v, v_u, h^u(v, v_u; c); c): |v|, |v_u| \leq r, c \in (c_0 - \delta, c_0 + \delta) \} \\
W^s(0; c) &= Q_0 + \{ v + h^s(v, v_s; c) + v_s + h(v, h^s(v, v_s; c), v_s; c): |v|, |v_s| \leq r, c \in (c_0 - \delta, c_0 + \delta)  \}
\end{aligned}
\end{equation}
where
\begin{align*}
h^s(0, 0; c_0) &= h^u(0, 0; c_0) = 0 \\
D_{(v, v_u)} h^u(0, 0; c_0) &= D_{(v, v_s)}  h^s(0, 0; c_0) = 0
\end{align*}

Let $\Sigma$ be the hyperplane plane passing through $Q_0$ with no component in $V^c$, i.e.
\[
\Sigma = Q_0 + V^s + V^u
\]
To obtain a unique intersection point, we will consider the intersection of the stable and unstable manifolds to $\Sigma$. Since the restriction maps $(v, v_s) \mapsto v_s$ and $(v, y^-) \rightarrow y^-$ are smooth, $W^u(0; c) \cap \Sigma$ and $W^s(0; c) \cap \Sigma$ are smooth $(m-1)-$dimensional manifolds parameterized by
\begin{equation}\label{Wparam2}
\begin{aligned}
W^u(0; c) \cap \Sigma &= Q_0 + \{ v_u + h^u(0, v_u; c) + h(0, v_u, h^u(0, v_u; c); c): |v_u| \leq r, c \in (c_0 - \delta, c_0 + \delta) \} \\
W^s(0; c) \cap \Sigma &= Q_0 + \{ h^s(0, v_s; c) + v_s + h(0, h^s(0, v_s; c), v_s; c): |v_s| \leq r, c \in (c_0 - \delta, c_0 + \delta) \}
\end{aligned}
\end{equation}

We wish to show that for $c$ close to $c_0$, $W^s(0; c) \cap \Sigma$ and $W^u(0; c) \cap \Sigma$ have a unique point of intersection. Looking at \eqref{Wparam2}, it suffices to solve $K(v_u, v_s; c) = 0$, where $K: V^u \times V^s \times \R \rightarrow V^u \times V^s$ is defined by
\begin{align*}
K(v_u, v_s; c) = (v_u - h^s(0, v_s; c), v_s - h^u(0, v_u; c))
\end{align*}
Since $h^s(0, 0; c_0) = h^u(0, 0; c_0) = 0$, $K(0, 0, c_0) = 0$. Since $D_{v_u} h^u(0, 0; c_0) = D_{v_s} h^s(0, 0; c_0) = 0$, 
$D_{(v_u, v_s)}K(0, 0; c_0) = I_{2m-2}$, which is invertible. Using the IFT, there exists an open neighborhood $W$ of $(0, 0) \in V^u \times V^s$, $\tilde{\delta}$ with $0 < \tilde{\delta} < \delta$, and a unique smooth function $g: (c_0 - \tilde{\delta}, c_0 + \tilde{\delta}) \rightarrow W$ such that $g(c_0) = (0, 0)$ and $K(g(c), c) = 0$ for all $c \in (c_0 - \tilde{\delta}, c_0 + \tilde{\delta})$.

Let $g(c) = (g_u(c), g_s(c)) \in V^u \times V^s$. Then for $c \in (c_0 - \tilde{\delta}, c_0 + \tilde{\delta})$, the unique intersection point of $W^s(0; c) \cap \Sigma$ and $W^u(0; c) \cap \Sigma$ is given by the smooth function
\[
P(c) = Q_0 + g_u(c) + g_s(c) + h(0, g_u(c), g_s(c); c)
\]
Using $P(c)$ as the initial condition at $x = 0$, we obtain a unique solution $Q(x; c)$ to \eqref{genODE}. Since $P(c)$ is in both $W^s(0; c)$ and $W^u(0; c)$, $Q(x; c)$ is a homoclinic orbit. Since $P(c)$ and $F(U; c)$ are smooth in $c$, the map $c \rightarrow Q(x; c)$ is smooth.
\end{proof}
\end{lemma}

Next, we will show that $Q_c(x)$ is exponentially localized. We will show this on $\R^+$; the argument for $\R^+$ is identical. To do this, we will first show that $Q_c(x)$ is a fixed point of a map on an exponentially weighted space. We will then show that the map is differentiable in $c$.

Let $\delta$ be as in Lemma \ref{Qcexistslemma}. By Hypothesis \ref{hypeqhyp}, $DF(0; c_0)$ is hyperbolic and $|\Re \nu| \geq \alpha_0$ for all eigenvalues of $DF(0; c_0)$. Since $F$ is smooth, the eigenvalues of $DF(0; c_0)$ are smooth functions of $c$. Choose any $\epsilon > 0$ such that $\alpha_0 - 2 \epsilon > 0$. Then we can find $\delta_1$ with $0 < \delta_1 \leq \delta$ such that for all $c \in (c_0 - \delta_1, c_0 + \delta_1)$, $|\Re \nu| \geq \alpha_0 - \epsilon$ for all eigenvalues of $DF(0; c)$.

For convenience, let $A(c) = DF(0; c)$. Let $P^u(c)$ and $P^s(c)$ be the projections on the unstable and stable eigenspaces of $A(c)$. Then we have the following bounds for the matrix exponential $e^{A(c)x}$.
\begin{equation}\label{eAcbounds}
\begin{aligned}
||e^{A(c)x}P^s(c)|| &\leq Ke^{-(\alpha_0 - \epsilon) x} && x \geq 0\\
||e^{A(c)x}P^u(c)|| &\leq Ke^{(\alpha_0 - \epsilon) x} && x \leq 0
\end{aligned}
\end{equation}

Let $\eta = \alpha_0 - 2 \epsilon$, and define the exponentially weighted norm
\[
||G||_\eta = \sup_{x \in [0, \infty)} |e^{\eta x} G(x)|
\]
and the exponentially weighted space
\begin{equation}\label{defXeta}
X_\eta = \{ G \in C^0([0, \infty), \R^n) : ||G||_\eta < \infty \}
\end{equation}
$X_\eta$ is known to be a Banach space.

% lemma: contraction map
\begin{lemma}\label{Hcontractionlemma}
There exists $\delta_1$ with $0 < \delta_1 \leq \delta$ with the following property. For all $c \in (c_0 - \delta_1, c_0 + \delta_1)$, there exists $x_0(c) \geq 0$ such that $Q(x + x_0(c); c)$ is the unique fixed point of the map $H: D \times B_1 \times B_2 \rightarrow D$ define by
\begin{equation}\label{defH}
[H(U, c, a)](x) = e^{A(c)x} P^s(c) a + \int_\infty^x e^{A(c)(x - y)}P^u(c) N(U(y))dy + \int_0^x e^{A(c)(x - y)}P^s(c) N(U(y))dy
\end{equation}
with initial condition $a = Q(x_0(c); c)$. The spaces $B_1$, $B_2$, and $D$ are defined below in the proof.

\begin{proof}
First, we separate $U(x)' = F(U(x); c)$ into linear and nonlinear parts by expanding in a Taylor series about $U = 0$. Since $F(0) = 0$, this gives us the ODE
\[
U(x)' = DF(0; c)U(x) + N(U(x)) = A(c)U(x) + N(U(x))
\] 
where $N(0) = 0$ and $DN(0) = 0$ since $N$ is purely nonlinear. Since we obtained \eqref{genODE} by writing \eqref{eqODE} as a first order system, only the linear part $A(c)$ depends on $c$.
Suppose that $|U(x)| \leq \rho$ for $x \geq 0$, where $\rho$ will be chosen later. Then, following the proof of the stable manifold theorem, we can write $U(x)$ in integrated form as 
\[
U(x) = e^{A(c)x} P^s(c) a + \int_\infty^x e^{A(c)(x - y)}P^u(c) N(U(y))dy + \int_0^x e^{A(c)(x - y)}P^s(c) N(U(y))dy
\]
where $a$ is the initial condition. Define the following spaces
\begin{align*}
B_1 &= (c_0 - \delta_1, c_0 + \delta_1) \\
B_2 &= \{ a \in \R^n : |P^s(c) a| \leq \rho/2K \text{ for all } c \in B_1\} \\
D &= \{ u \in X_\eta : ||u||_\eta \leq \rho \}
\end{align*}
and define the map $H: D \times B_1 \times B_2 \rightarrow X_\eta$ by
\begin{equation}\label{defH}
[H(U, c, a)](x) = e^{A(c)x} P^s(c) a + \int_\infty^x e^{A(c)(x - y)}P^u(c) N(U(y))dy + \int_0^x e^{A(c)(x - y)}P^s(c) N(U(y))dy
\end{equation}

First, we will show that $H$ is well-defined, i.e. it maps into $X_\eta$. Since $F$ is smooth, $N$ is Lipschitz in a neighborhood of 0. Thus since $N(0) = 0$ and $DN(0) = 0$, there exists a Lipschitz constant $L(\rho)$ with $L(\rho) \rightarrow 0$ as $\rho \rightarrow 0$ such that 
\[
|N(U)| \leq L(\rho)|U|
\]
whenever $|U| \leq \rho$. In particular, if $||U(x)||_\eta \leq \rho$, $||U(x)||_\eta \leq \rho$ as well. Thus $|e^{\eta x} N(U(x))| \leq L(\rho)|e^{\eta x}U(x)|$ for all $x \in \R^+$, from which it follows that 
\[
|N(U)|_\eta \leq L(\rho)|U|_\eta
\]
Similarly, it follows from the mean value inequality that for $U, V$ with $||U||_\eta, ||V||_\eta \leq \rho$
\[
||N(U) - N(V)||_\eta \leq L(\rho)(||U||_\eta - ||V||_\eta) 
\]

We consider each of the terms on the RHS of \eqref{defH} individually. For the first term, since $a \in B_1$, we have
\begin{align*}
|e^{\eta x} e^{A(c)x } P^s(c) a | &\leq K e^{\eta x} e^{-(\alpha_0 - \epsilon) x} | P^s(c) a |\\
&\leq K \frac{\rho}{2 K} e^{-\epsilon x}\\
&\leq \frac{\rho}{2}
\end{align*}
For the first integral term,
\begin{align*}
\left| e^{\eta x} \int_\infty^x e^{A(c)(x - y)}P^u(c) N(U(y))dy \right| &= \left| \int_\infty^x e^{\eta x} e^{A(c)(x - y)}P^u(c) N(U(y))dy \right|\\
&\leq \int_x^\infty K e^{\eta x}e^{(\alpha_0 - \epsilon)(x - y)}|N(U(y))|dy \\
&= K \int_x^\infty e^{\eta (x - y)}e^{(\alpha_0 - \epsilon)(x - y)} | e^{\eta y} N(U(y))|dy \\
&\leq K \int_x^\infty e^{(2 \alpha_0 - 3 \epsilon)(x - y)} || N(U)||_\eta dy \\
&\leq K L(\rho) ||U||_\eta \int_x^\infty e^{(2 \alpha_0 - 3 \epsilon)(x - y)} dy \\
&= \frac{ K L(\rho) }{2 \alpha_0 - 3 \epsilon} ||U||_\eta \\
&\leq \frac{ K L(\rho) }{2 \alpha_0 - 3 \epsilon} \rho
\end{align*}
where we used the fact that $U \in D$, thus $||U||_\eta \leq \rho$. For the second integral term,
\begin{align*}
\left| e^{\eta x} \int_0^x e^{A(c)(x - y)}P^s(c) N(U(y)) \right| &= \left| \int_0^x e^{\eta x} e^{A(c)(x - y)}P^s(c) N(U(y)) dy \right|\\
&\leq \int_0^x K e^{\eta x}e^{-(\alpha_0 - \epsilon)(x -y)}|N(U(y))|dy \\
&= K \int_0^x e^{\eta (x - y)}e^{-(\alpha_0 - \epsilon)(x - y)}| e^{\eta y} N(U(y))|dy \\
&\leq K \int_0^x e^{-\epsilon(x - y)} || N(U)||_\eta dy \\
&\leq K L(\rho) ||U||_\eta \int_0^x e^{-\epsilon(x - y)} dy \\
&= K L(\rho) ||U||_\eta \frac{1 - e^{-\epsilon x} }{\epsilon} \\
&\leq \frac{K L(\rho)}{\epsilon} \rho
\end{align*}

Putting all of this together and taking the supremum over $x \in \R^+$, we have the bound
\begin{equation*}
||H(U, c, a)](x)||_\eta \leq \frac{\rho}{2} + K L(\rho) \left( \frac{1}{2 \alpha_0 - 3 \epsilon} + \frac{1}{\epsilon} \right) \rho
\end{equation*}
Choose $\rho$ sufficiently small so that 
\[
K L(\rho) \left( \frac{1}{2 \alpha_0 - 3 \epsilon} + \frac{1}{\epsilon} \right) \leq \frac{1}{2}
\]
Then 
\[
||H(U, c, a)](x)||_\eta \leq \rho
\]
which not only implies $H: D \times B_1 \times B_2 \rightarrow X_\eta$ but also $H: D \times B_1 \times B_2 \rightarrow D$. For $U, V \in D$, following what we did above, we have
\begin{align*}
| &e^{\eta x} ( H(U, c, a) - H(V, c, a) ) | \\
&= \left| \int_\infty^x e^{\eta x} e^{A(c)(x - y)}P^u(c) [N(U(y)) - N(V(y))]dy + \int_0^x e^{\eta x} e^{A(c)(x - y)}P^s(c)[N(U(y)- N(V(y))]dy \right| \\
&\leq K L(\rho) \left( \int_x^\infty e^{(2 \alpha - 3 \epsilon)(x-y)}||U - V||_\eta dy + \int_0^x e^{-\epsilon(x-y)}||U - V||_\eta dy \right) \\
&= K L(\rho) \left( \frac{1}{2 \alpha_0 - 3 \epsilon} + \frac{1}{\epsilon} \right) ||U - V||_\eta  \\
&\leq \frac{1}{2} ||U - V||_\eta 
\end{align*}
Thus $H$ is a uniform contraction. By the uniform contraction mapping principle, there is a unique map $G: B_1 \times B_2 \rightarrow D$ such that $H(G(c, a), c, a) = G(c, a)$ for all $c \in B_1$ and $a \in B_2$. The maps $G$ and $F$ have the same smoothness in the parameters $c$. 

Since $Q(x; c)$ is exponentially localized, there exists $x_0(c) \geq 0$ such that $|e^{\eta x} Q(x; c)| \leq \rho$ for all $x \geq x_0(c)$. Let $a(c) = Q(x_0(c); c)$. Since $F$ is smooth, $Q(x + x_0(c); c)$ is the unique solution on $\R^+$ to \eqref{genODE} with initial condition $a(c)$. Since $||Q(x + x_0(c); c)||_\eta \leq \rho$, $Q(x + x_0(c); c)$ is also a fixed point of \eqref{defH}. Thus by uniqueness from the contraction mapping principle, 
\[
G(c, a(c)) = Q(x + x_0(c); c)
\]
\end{proof}
\end{lemma}

All that remains is to show that $H(U, c, a)$ is differentiable in $c$, which will imply that $G(c, a)$ is as well. First, we prove the following lemma, which is a product rule for Frechet derivatives. 

\begin{lemma}\label{frechetproductlemma}
Let $F(c): \R \rightarrow X_\eta$, where $X_\eta$ is the weighted exponential space defined in \eqref{defXeta}. Suppose $F$ is Frechet differentiable at $c$ with, with Frechet derivative $L$. Then the following are true.

\begin{enumerate}[(i)]
\item Let $T(c)$ be an $n \times n$ matrix which is smooth in $c$. Then $T(c)F(c): \R \rightarrow X_\eta$ has Frechet derivative $T'(c)F(c) + T(c)L$ at $c$.

\item Let $S(c)$ be an $n \times n$ matrix which is smooth in $c$. Then $F(c)S(c): \R \rightarrow X_\eta$ has Frechet derivative $L S(c) + F(c)S'(c)$ at $c$.

\item Let $S(c) \in GL(n, \R)$ be a family of smooth invertible matrices paramaterized by $c$, where $S(c)$ is smooth and invertible in an open interval around $c$. Then $S^{-1}(c)F(c)S: \R \rightarrow X_\eta$ has Frechet derivative $(S^{-1})'(c) F(c) S(c) + S^{-1}(c) L S (c) + S^{-1}(c) F(c) S'(c)$ at $c$.
\end{enumerate}

\begin{proof}
Let $||F(c)||_\eta = M$. For (i), $T(c)F(c): \R \rightarrow X_\eta$ since for $x \geq 0$
\begin{align*}
|e^{\eta x} T(c)[F(c)](x)| \leq ||T(c)||\:|e^{\eta x} [F(c)](x)| 
= ||T(c)|| \:||F(c)||_\eta < \infty
\end{align*}
Using the definition of the Frechet derivative,
\begin{align*}
\lim_{h \rightarrow 0}&\frac{||T(c+h)F(c+h) - T(c)F(c) - (T'(c)F(c) + T(c)L)h||_\eta}{|h|} \\
&= \lim_{h \rightarrow 0}\frac{||T(c+h)F(c+h) - T(c)F(c+h) + T(c)F(c+h) - T(c)F(c) - (T'(c)F(c) + T(c)L)h||_\eta}{|h|} \\
&\leq \lim_{h \rightarrow 0}\frac{||T(c+h)F(c+h) - T(c)F(c+h) - T'(c)F(c) h ||_\eta}{|h|} \\
&\:\:+ \lim_{h \rightarrow 0}\frac{||T(c)F(c+h) - T(c)F(c) - T(c)L)h||_\eta}{|h|}
\end{align*}
We will evaluate the two limits on the RHS separately. For the second limit,
\begin{align*}
\lim_{h \rightarrow 0}&\frac{||T(c)F(c+h) - T(c)F(c) - T(c)L)h||_\eta}{|h|} \\
&= |T(c)| \lim_{h \rightarrow 0}\frac{||F(c+h) - F(c) - L)h||_\eta}{|h|} \\
&= 0
\end{align*}
since $|T(c)|$ is a constant and $F(c)$ has Frechet derivative $L$. For the first limit,
\begin{align*}
\lim_{h \rightarrow 0}&\frac{||T(c+h)F(c+h) - T(c)F(c+h) - T'(c)F(c) h ||_\eta}{|h|} \\
&= \lim_{h \rightarrow 0}\frac{||T(c+h)F(c+h) - T(c)F(c+h) - T'(c)F(c+h)h + T'(c)F(c+h)h - T'(c)F(c) h ||_\eta}{|h|} \\
&\leq \lim_{h \rightarrow 0}\frac{||T(c+h)F(c+h) - T(c)F(c+h) - T'(c)F(c+h)h||_\eta}{|h|} + \lim_{h \rightarrow 0} \frac{||T'(c)F(c+h)h - T'(c)F(c) h ||_\eta}{|h|} \\
&\leq \lim_{h \rightarrow 0} \left| \frac{T(c+h) - T(c)}{h} - T'(c) \right| ||F(c+h)||_\eta + |T'(c)| \lim_{h \rightarrow 0} ||F(c+h) - F(c) ||_\eta \\
&= 0 \cdot M + |T'(c)| \cdot 0 \\
&= 0
\end{align*}
where we used the continuity $F: \R \rightarrow X_\eta$, the continuity of norms, and the differentiability of $T(c)$. Thus we have proved (i).

For (ii), the proof is similar. For (iii), take $T(c) = S^{-1}(c)$ in (i), then use (ii).
\end{proof}
\end{lemma}

Since $F(U; c)$ is smooth, $A(c)$, $P^s(c)$, and $P^u(c)$ are differentiable in $c$. Next, we prove a result about the Frechet derivative of $e^{A(c)x} P^s(c)$.

\begin{lemma}
For $c \in (c_0 - \delta_1, c_0 + \delta_1)$, the map $c \rightarrow e^{A(c)x} P^s(c)$ is Frechet differentiable from $\R$ to $X_\eta$.
\begin{proof}
To do this, we will put the matrix $A(c)$ into Jordan canonical form. Let $S^{-1}(c)A(c)S(c) = \Lambda(c)$, where $\Lambda(c)$ is block diagonal consisting of Jordan blocks. Since $A(c)$ is smooth in $c$, $S(c)$ and $\Lambda(c)$ are as well. $S^{-1}(c)P^s(c)S(c) = P_1$, where $P_1$ is the matrix which projects onto the stable eigenspace of $\Lambda(c)$. Since $A(c)$ is hyperbolic, the dimensions of the eigenspaces will be the same for $c \in (c_0 - \delta_1, c_0 + \delta_1)$; thus since $\Lambda(c)$ is block diagonal, $P_1$ is constant coefficient and does not depend on $c$. Thus we have
\begin{align*}
S^{-1}(c) e^{A(c)x} P^s(c) S(c) &=  
S^{-1}(c) e^{A(c)x} S(c) S^{-1}(c) P^s(c) S(c) \\
&= e^{\Lambda(c)x}P_1 
\end{align*}
from which it follows that
\begin{align*}
e^{A(c)x} P^s(c) 
&= S(c) e^{\Lambda(c)x} P_1 S^{-1}(c)
\end{align*}
By Lemma \ref{frechetproductlemma}, it suffices to show that $e^{\Lambda(c)x} P_1$ is Frechet differentiable. Since matrices commute with their own eigenprojections, we will use the relation
\begin{equation}\label{P1commutes}
e^{\Lambda(c)x} P_1 = P_1 e^{\Lambda(c)x}
\end{equation}
which holds for all $c \in (c_0 - \delta_1, c_0 + \delta_1)$. 

We will take
\begin{equation}\label{Frechetansatz1}
e^{\Lambda(c)x} P_1 \Lambda'(c)x
\end{equation}
as our ansatz for the Frechet derivative of $e^{\Lambda(c)x} P_1$. Since $\Lambda(c)$ is smooth in $c$, we can expand it as a Taylor series around $c$ to get
\[
\Lambda(c+h) = \Lambda(c) + \Lambda'(c)h + R(h)h
\]
where $R(h) \rightarrow 0$ as $h \rightarrow 0$. Using this together with the definition of the Frechet derivative and \eqref{P1commutes},
\begin{align*}
&\frac{e^{\eta x} |e^{\Lambda(c+h)x}P_1 - e^{\Lambda(c)x}P_1 - e^{\Lambda(c)x} P_1 \Lambda'(c)x h|}{|h|} \\
&= \frac{e^{\eta x}}{|h|}| P_1|\left| e^{(\Lambda(c) + \Lambda'(c)h + R(h)h)x} - e^{\Lambda(c)x} - e^{\Lambda(c)x} P_1 \Lambda'(c)h x \right| \\
&= \frac{e^{\eta x}}{|h|}| P_1 e^{\Lambda(c)x}|\left| e^{(\Lambda'(c)h + R(h)h)x} - I - \Lambda'(c)h x \right| \\
&= \frac{1}{|h|} |e^{(\alpha_0 - \epsilon)x} e^{\Lambda(c)x} P_1| e^{-\epsilon x} \left| e^{\Lambda'(c)h + R(h)h} - I - \Lambda'(c)hx \right| \\
&\leq \frac{C}{|h|} e^{-\epsilon x} \left| e^{\Lambda'(c)hx}e^{R(h)hx} - I - \Lambda'(c)hx \right| \\ 
\end{align*}
where we used the fact that $e^{(\alpha_0 - \epsilon)x} e^{\Lambda(c)x} P_1$ is uniformly bounded, which follows from \eqref{eAcbounds}. Using the series expansion for the matrix exponentials, we have
\begin{align*}
e^{\Lambda'(c)hx}&e^{R(h)hx} - I - \Lambda'(c)hx \\
&= e^{\Lambda'(c)hx} \left(I + \sum_{n=1}^\infty \frac{(R(h)hx)^n}{n!} \right) - I - \Lambda'(c)hx \\
&= I + \Lambda'(c)hx + \sum_{n=2}^\infty \frac{(\Lambda'(c)hx)^n}{n!} + e^{\Lambda'(c)hx} \sum_{n=1}^\infty \frac{(R(h)hx)^n}{n!} \right) - I - \Lambda'(c)hx \\
&= \sum_{n=2}^\infty \frac{(\Lambda'(c)hx)^n}{n!} + e^{\Lambda'(c)hx} \sum_{n=1}^\infty \frac{(R(h)hx)^n}{n!} \right) \\
&= (\Lambda'(c)hx)^2 \sum_{n=0}^\infty \frac{(\Lambda'(c)hx)^n}{(n+2)!} + e^{\Lambda'(c)hx} R(h)hx \sum_{n=0}^\infty \frac{(R(h)hx)^n}{(n+1)!} \right) 
\end{align*}
Since $\Lambda'(c)$ is a constant and is smooth in $c$ and $R(h) \rightarrow 0$ as $h \rightarrow 0$, choose $h \leq 1$ sufficiently small so that for all $c \in (c_0 - \delta_1, c_0 + \delta_1)$, 
\[
|\Lambda'(c)h|, |R(h)h| \leq \frac{\epsilon}{4}
\]
Recalling that $x \geq 0$, we have
\begin{align*}
&\left| e^{\Lambda'(c)hx}e^{R(h)hx} - I - \Lambda'(c)hx \right| \\
&\leq (|\Lambda'(c)h|x)^2 \sum_{n=0}^\infty \frac{(|\Lambda'(c)h|x)^n}{n!} + e^{|\Lambda'(c)h|x} |R(h)h|x \sum_{n=0}^\infty \frac{(|R(h)h|x)^n}{n!} \right) \\
&\leq (|\Lambda'(c)h|x)^2 e^{|\Lambda'(c)h|x} + e^{|\Lambda'(c)h|x} |R(h)h|x e^{|R(h)h|x} \\
&= C \left(|h|^2 x^2 \exp\left(\frac{\epsilon}{4}x\right) 
+ |h| |R(h)|x \exp\left(\frac{\epsilon}{2}x\right) \right)
\end{align*}
Substituting this in, we have
\begin{align*}
&\frac{e^{\eta x} |e^{\Lambda(c+h)x}P_1 - e^{\Lambda(c)x}P_1 - e^{\Lambda(c)x} P_1 \Lambda'(c)x h|}{|h|} \\
&\leq \frac{C}{|h|} e^{-\epsilon x} \left(|h|^2 x^2 \exp\left(\frac{\epsilon}{4}x\right) 
+ |h| |R(h)|x \exp\left(\frac{\epsilon}{2}x\right) \right) \\
&\leq C \left( |h| x^2 \exp\left(-\frac{3\epsilon}{4}x\right) + |R(h)| x \exp\left(\frac{\epsilon}{2}x\right) \right) \\
&= C (|h| + |R(h)|)
\end{align*}
Sending $h \rightarrow 0$, since this is uniform in $x$, we conclude that
\begin{align*}
\lim_{h \rightarrow 0}
\frac{||e^{\Lambda(c+h)x}P_1 - e^{\Lambda(c)x}P_1 - e^{\Lambda(c)x} P_1 \Lambda'(c)x h||_\eta}{|h|} = 0
\end{align*}
\end{proof}
\end{lemma}

\begin{lemma}\label{HFrechetlemma}
The map $H: D \times B_1 \times B_2 \rightarrow D$ defined in Lemma \ref{Hcontractionlemma} is Frechet differentiable in $c$.
\begin{proof}
Since the only dependence on $c$ is via the term $e^{A(c)x}$, this is not difficult. First, 

Next, we look at the Frechet derivative of $e^{A(c)x} P^s(c)$. 



\end{proof}
\end{lemma}


\iffulldocument\else
	\bibliographystyle{amsalpha}
	\bibliography{thesis.bib}
\fi

\end{document}